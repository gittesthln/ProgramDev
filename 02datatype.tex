%-*- coding: utf-8 -*-
\chapter{JavaScriptのデータとプログラミングの基礎}
\section{データーの型}
JavaScript の方には大きく分けて\KeyItem{プリミティブ型}と\KeyItem{非プリ
ミティブ型}の2種類がある。変数や値の型を知りたいときは\ElmJ{typeof} 演算子を使う。
\subsection{プリミティブデータ型}
表\ref{primitivedata}はJavaScriptにおける\KeyItem{プリミティブデータ型}の一覧であ
る。
 \begin{table}[ht]
  \caption{プリミティブデータ型}\label{primitivedata}
\begin{center}
\begin{tabular}{|c|m{33zw}|}\hline
 型&\multicolumn{1}{c|}{説明} \\\hline
 \ElmJ{Number} & 64ビット浮動小数点形式だけ\\ \hline
 \ElmJ{String} & \KeyItem{文字列型}、文字のデータ型はない。ダブルク
     オート(\Verb+"+)か%" 
     シングルクオート(\Verb+'+)で囲む。ES2015では文字列内に式
     の結果を埋め込むことができるテンプレートリテラル形式(バッククオート
     \Verb+`+ではさむ\footnotemark{})が追加された。\\ \hline
 \ElmJ{Boolean}& \ElmJ{true} か \ElmJ{false} の値のみ\\ \hline
 \ElmJ{undefined} & 変数の値が定義されていないことを示す\\ \hline
 \ElmJ{null}& \ElmJ{null}という値しか取ることができない特別なオブジェク
     ト\\ \hline
\end{tabular}
\end{center}
 \end{table}
\footnotetext{キーボードの\Verb+@+の上にある文字}
\subsubsection{Number型}
数を表現する方法(数値リテラル)としては次のものがある。
\begin{itemize}
 \item{\bfseries \KeyItem{整数リテラル}} 10進整数は通常通りの形式である。
      \KeyItem{16進数}を表す場合は
	      先頭に\Verb+0x+ か \Verb+0X+ をつける。先頭が\Verb+0+でそ
	      のあとに\Verb+x+または\Verb+X+が来ない場合は\KeyItem{8進数}と解釈
      される場合がある
      %\footnote{Opera, Internet Exploler, FireFox, Chrome では8進数と解
      %釈されるようである。}
      。\Strict ではこの形式はエラーとなる。

      2進数は\Verb+0b+ で、8進数は\Verb+0o+で始まるリテラルが定義されて
      いる。
 \item{\bfseries \KeyItem{浮動小数点リテラル}} 整数部、そのあとに必要ならば小数点、小数部そ
       のあとに指数部がある形式である。
\end{itemize}
\KeyItem{Number型}には次のような特別な Number が定義され
ている。
\begin{itemize}
 \item {\bfseries \ElmJ{Infinity}}無限大を表す読み出し可能な変数である。
       オーバーフローした場合や \Verb+1/0+ など演算の結果としてこの値が設定さ
       れる。
 \item {\bfseries \ElmJ{NaN}} Not a Number の略である。式が計算ができなかっ
       た場合の結果を表す読み出し可能な変数である。
       文字列を数値に変換できない場合や \Verb+0/0+ などの結果としてこの値が設定さ
       れる。
\end{itemize}
\subsubsection{String型}
文字列は\Verb+"+ または \Verb+'+ ではさむ。%"
1文字だけの文字の型はない。テンプレートリテラルは文字列
の中に式の値を埋め込むことができる。テンプレートリテラルはバッククオート
(\Verb+`+)ではさむ。埋め込む式は\texttt{\$\{2+3\}}のように記述する。
\begin{Verbatim}
`2+3=${2+3}`  => "2+3=5"
\end{Verbatim}
文字列に関する演算子としては、2つの文字列をつなげる\KeyItem{連接演算子}
がある。JavaScript では\ElmJ{+}を用いる。詳しくは\ref{operator}を参照の
こと。

\KeyItem{文字列}に関する情報(プロパティ)や操作(メソッド)として次の
ものがある\footnote{この表にある正規表現については後日解説をする}。
\begin{table}[ht]
 \caption{文字列のメソッドとプロパティ}\label{methodStrig}
\begin{center}\ \\[-0.015\textheight]
 \begin{tabular}{|c|m{28zw}|}\hline
 メンバー&\multicolumn{1}{c|}{説明} \\\hline
  \ElmJP{length} &文字列の長さ\\ \hline
\ElmJF{indexOf}{needle,start}& 与えられた文字列内に\Verb+needle+が
      初めて現れる位置を返す。\Verb+start+の引数があれば、その位
      置以降から調べる。見つからない場合は$-1$を返す。\\\hline
\ElmJF{lastIndexOf}{needle,start}& 与えられた文字列内に\Verb+needle+が
      一番最後に現れる位置を返す。\Verb+start+の引数があれば、指定された
      位置以前から調べる。見つからない場合は$-1$を返す。\\\hline
  \ElmJF{split}{separator,limit}&\Verb+separator+の文字列(ま
      たは正規表現)で与え
      られた文字列を分割し、配列で返す。\Verb+separator+の部分は返されない。
      2番目の引数はオプションで、分割する最大数を返す。\\ \hline
  \ElmJF{substring}{start,end}&与えられた文字列の\Verb+start+から
      \Verb+end+の前の位置までの部分文字列を返す。\\ \hline
  \ElmJF{slice}{start,end}&与えられた文字列の\Verb+start+から
      \Verb+end+の前の位置までの部分文字列を返す。値が負のときは文字列
      末尾から数えた位置を表す。\\ \hline
\end{tabular}
\end{center}
\end{table}
\input prob02-01string.tex
\subsubsection{Bool型}
\ElmJ{true} と \ElmJ{false} の2つの値をとる。この2つは予約語である。論理
式の結果としてこれらの値が設定されたり、論理値が必要なところでこれらの値
に設定される。詳しくは\ref{operator}を参照のこと。
\subsubsection{\protect\ElmJ{undefined}と\protect\ElmJ{null}}
\ElmJ{undefined}は値が存在しないことを示す読み出し可能な変数である。
変数が宣言されたのに値が設定されていないときの初期値や、戻り値が定義され
ていない関数の戻り値はこの値になる。
%\subsubsection{\protect\ElmJ{null}}

\Verb+typeof null+ の値が \Verb+"object"+であることを示すように、オブジェ
クトが存在しないことを示す特別なオブジェクト値(であると同時にオブジェク
トでもある)である。
\subsection{非プリミティブ型}
プリミティブ型以外のデータ型は「オブジェクト」がある。JavaScriptでは
次のようなオブジェクトが定義されている。\texttt{()}内はオブジェク
ト(を作成するコンストラクタ)の名称である。
\begin{itemize}
 \item 配列(\ElmJ{Array})
 \item 数学(\ElmJ{Math})
 \item 日時(\ElmJ{Date})
 \item 正規表現(\ElmJ{RegExp})
 \item マップ(\ElmJ{Map})とウィークマップ(\ElmJ{WeakMap})
 \item セット(\ElmJ{Set})とウィークセット(\ElmJ{WeakSet})
\end{itemize}
これらのデータ型のうち、マップ、セット、ウィークセット以外は後の章で解説する。 
\section{変数}
JavaScriptの変数の特徴は次の通りである。
\begin{itemize}
 \item 変数名はアルファベットまたは\Verb+_+(アンダースコア)で始まる英数
       字または\Verb+_+ が続く文字列
 \item 大文字と小文字は区別される。
 \item 変数の宣言は \texttt{let} で行う。\texttt{var}でもできるが、問題
       があるのでこの授業では使用しない。
 \item 宣言時に初期化ができる。\ES で導入された\texttt{const}による変数
       の宣言では初期化時の値の変更ができない。
 \item 非\Strict のときは変数は宣言をしなくても使用できる。初期化してい
       ない変数を演算の対象として使用するとエラーが起こる。詳しくは後述する。
 \item 変数に保存するデータの型には制限がない。途中で変更することもでき
       る。
\end{itemize}
\section{配列}
\subsection{配列の宣言と初期化}
配列を使うためには、変数を配列で初期化する必要がある。変数の宣言と同時に
行ってもよい。
\begin{Verbatim}
let a = [];      //空の配列の定義
let b = [1,2,3]; //要素が3つある配列
\end{Verbatim}
\Verb+a+ は空の配列で、\Verb+b+ は
\Verb+b[0]=1,b[1]=2,b[2]=3+ となる。
次のことに注意する必要がある。
\begin{itemize}
 \item 配列の各要素のデータの型は同じでなくてもよい。
 \item 実行時に配列の大きさを自由に変えられる。
 \item 配列の要素に配列を置くことができる。
\begin{Verbatim}
let a=[1,[2,3,4],"a"];
\end{Verbatim}
 \item 配列の変数の代入は参照である。
\begin{Verbatim}
>let a = [0,1,2,3,4];
>let b = a;  //別の変数への代入
>b;
(5) [0, 1, 2, 3, 4]  // a と同じ内容が表示
>b[3]= b[3]*10;      // 4番目の要素を10倍 [0, 1, 2, 30, 4]となる
30
>a;
(5) [0, 1, 2, 30, 4] // a も b と同じ値の要素を持つ
\end{Verbatim}
       配列の代入が参照なので片方の要素を変化させると、両方とも変わるこ
       とが分かる。
 \item 配列の要素の一部をまとめて別の変数に代入する分割代入が\ES で導
       入された。分割代入では代入演算子\texttt{=}の左辺を配列の形で記述
       する。

       代入\Verb+[a,,b,...c] = [0,1,2,3,4,5,6,7];+
の結果は次のとおりである。
\begin{Verbatim}
a ==> 0
b ==> 2
c ==> [3, 4, 5, 6, 7]
\end{Verbatim}
       \begin{itemize}
        \item  変数\Verb+a+には0番目の、\Verb+b+には3番目の、\Verb+c+に
               は4番目以降の配列の要素が代入される。
        \item \Verb+a+ と \Verb+b+ の間に\Verb+,,+がある
       ので2番目の要素は代入されない。
        \item 4番目以降の要素はまとめて変数\Verb+c+に代入されている。

       \end{itemize}

 配列が関数の戻り値のとき、戻り値の必要なところだけ利用す
       るために便利な機能である。

       なお、ここで使用している\Verb+...+は展開演算子と呼ばれ、配列に対
       して要素を並べたものを表す。
\end{itemize}
\input prob02-02asign.tex
\subsection{配列のメソッド}
表\ref{arrayPropMethod}は配列のメソッドやプロパティをまとめたものである。
\begin{table}[ht]
\caption{配列のメソッドとプロパティ}\label{arrayPropMethod}
%\begin{center}
 \begin{tabular}{|c|m{28zw}|}\hline
 メンバー&\multicolumn{1}{c|}{説明} \\\hline
  \Verb+length+ &配列の要素の数\\ \hline
  \Verb+join(separator)+& 配列を文字列に変換する。\Verb+separator+はオプ
      ションの引数で、省略された場合はカンマ\Verb+,+である。\\ \hline
  \Verb+pop()+& 配列の最後の要素を削除し、その値を返す。
%      配列をスタックとして利用できる。
      \\ \hline
  \Verb+push(i1,i2,...)+& 引数で渡された要素を配列の最後に付け加える。
%      配列をスタックやキューとして利用できる。
      \\ \hline
  \Verb+shift()+&配列の最初の要素を削除し、その値を返す。
%      配列をキューと して利用できる。
      \\ \hline
  \Verb+unshift()+&配列の最初の要素を削除し、その値を返す。
%XS      配列をキューと して利用できる。
      \\ \hline
  \Verb+slice(start,end)+&\Verb+start+から\Verb+end+の前の位置にある要素を取
      り出した配列を返す。元の配列は変化しない。\\ \hline
  \Verb+splice(start,No,i1,i2,...)+&\Verb+start+の位置から\Verb+No+の要素を取
      り除き、その位置に\Verb+i1,i2,...+以下の要素を付け加える。元の配列
      を変更する。\\ \hline
  \Verb+indexOf(value,start)+&配列の要素で\texttt{value}に等しい
      (\texttt{===})ものを探し、その位置を返す。見つからない場合は
      \texttt{-1}を返す。
      オプションの引数\texttt{start}はそのインデックス以降から探す。
      \\ \hline
  \Verb+lastIndexOf(value,start)+&配列の要素で\texttt{value}に等しいもの
      を後ろから探し、その位置を返す。見つからない場合は\texttt{-1}を返す。
      オプションの引数\texttt{start}はそのインデックス以前から探す。
      \\ \hline
\end{tabular}
%\end{center}
\end{table}

配列のメソッドとしてはこのほかに、引数に処理をする関数を与えるものがある。それらについ
ては関数のところで解説をする。
\input prob02-03.tex
\section{演算子}\label{operator}
\subsection{代入、四則演算}
数に対してはC言語と同様の演算子が使用できる。ただし、次のことに注意する
こと。
\begin{itemize}
 \item \Verb-+-演算子は文字列の連接にも使用できる。\Verb-+-演算子は左右のオペ
ランドがNumberのときだけ、和をとる。一方が文字列の
場合は数を文字列に直して、文字列として連接を行う。
 \item そのほかの演算子(\Verb+-*/+)については文字列を数に変換してから数
       として計算する。
 \item 文字列全体が数にならない場合には変換の結果が\Verb+NaN+になる。
\begin{Verbatim}
 2+3  => 5
 "2"+3 => "23"
 "2"-0 +3 => 5
 "2"*3    => 6
 "2"*"3"  => 6
 "f" *2   => NaN
 "0xf"*2  => 30
\end{Verbatim}
% "0o10"*2 => 16
 \item 整数を整数で割った場合、割り切れなければ小数となる。
       小数部分を切り捨てたいときは\Verb+Math.floor()+を用いる。
\begin{Verbatim}
1/3 => 0.3333333333333333
Math.floor(1/3) =>0
\end{Verbatim}
\end{itemize}
\subsection{論理演算子と比較演算子}
\subsubsection{論理演算子}
Boolean 型に対して使用できる演算子は次の3つである。
\begin{itemize}
 \item \Verb+!+ 論理否定
 \item \Verb+&&+ 論理積
 \item \Verb+||+ 論理和
\end{itemize}
論理演算子をBoolean 型でない値を与えると
次の値は \Verb+false+ とみなされる。
\begin{itemize}
 \item 空文字列 \Verb+""+
 \item \Verb+null+
 \item \Verb+undefined+
 \item 数字の $0$
 \item 数値の \Verb+NaN+
 \item Boolean の\Verb+false+
\end{itemize}
論理和や論理積では左のオペランドの結果により、式の値が決まる場合は右のオ
ペランドの評価は行われない。たとえば、論理和の場合、左の値が
\Verb+true+であれば右のオペランドの評価が行われない。次の例では変数
\Verb+a+ の値は $1$ のままである。 
\begin{Verbatim}
let a=1; true ||(a=3); 
\end{Verbatim}
\subsubsection{比較演算子}
比較演算子は比較の結果、Boolean の値を返す演算である。C言語と同様の比較
演算子が使用できる。\texttt{>,>=,<,<=}などがある。等しいことを比較するためには
\Verb+==+(等価比較演算子)のほかに 型変換を伴わない等価比較演算子
\Verb+===+ がある。等価比較演算子\texttt{==}は必要に応じて型変換を行う。
特別な事情がない限り、\Verb+==+や\Verb+!=+は使用しないこと。
\begin{Verbatim}
0 == "0" => true
0 === "0" => false
\end{Verbatim}
同様に非等価比較演算子\texttt{!=}にも型変換を伴わない非等価演算子
\texttt{!==}がある。

\Verb+NaN === NaN+ の結果は \Verb+false+ である。値が \Verb+NaN+ で
あるかを調べるには関数\ElmJ{isNaN()}を用いる。
\section{制御構造}
JavaScript ではC言語などと同様に\ElmJ{if}文、\ElmJ{for}文、\ElmJ{while}
文、\ElmJ{switch}文がある。使い方は同様なのでここでは説明をしない。また、
Java やC++のように
\ElmJ{for}文の初期化のところに現れる変数を\ElmJ{let}を用いて宣言すること
もできる。このの宣言された変数に関するJavaScript特有の注意は次回解説をする。
\section{組み込みオブジェクト}
\subsection{組み込み関数}
JavaScript に組み込まれている代表的な関数としては次のものがある。
\begin{itemize}
 \item \Verb+parseInt(string,radix)+ \Verb+string+(文字列)と
       \Verb+radix+(基数、省略したときは$10$)をとり、先頭から見て正しい
       整数表現のところまで整数に変換する。
 \item \ElmJ{parseFloat(string)} \Verb+string+(文字列)
       をとり、先頭から見て正しい
       浮動小数点表現のところまで浮動小数に変換する。
 \item \Verb+isNaN(N)+ \Verb+N+ が数であれば \Verb+true+、そうでなければ
       \Verb+false+ を返す。
 \item \Verb+isFinite(N)+ \Verb+N+ が数値または数値に変換できる値でかつ
       \Verb+Infinity+ または \Verb+-Infinity+ でないときに \Verb+true+、
       そうでないとき、\Verb+false+ を返す。
 \item \ElmJ{encodeURIComponent(string)} \texttt{string}を URLエンコー
			 ドする。ブラウザの日本語を含む字句検索の結果のアドレス
			 バーに\%で始まる文字列がURLエンコードした結果である。 
 \item \ElmJ{decodeURIComponent(string)}
			 \ElmJ{encodeURIComponent(string)}の逆の操作をする。
 \item \ElmJ{encodeURI(string)} \texttt{string} をURIエンコーディングす
			 る。この関数はプロトコル部分などはエンコードしない。
 \item \ElmJ{decodeURI(string)} \ElmJ{encodeURI(string)} の逆の操作をす
			 る。
\end{itemize}
%このほかにも関数があるが、省略する。
\subsection{\protect\texttt{Math}オブジェクト}
\ElmJ{Math}オブジェクトには数学的な定数の定義(円周率など)や三角関数など
の関数が定義されている。表\ref{JSMath}はそれらのプロパティや関数をまとめ
たものである。%\newpage%\vspace{-1cm}
\input 00Math.tex
いくつか注意をしておく。
\begin{itemize}
 \item 配列の要素の最大値や最小値を求めるために\Verb+Math.max+ や
       \Verb+Math.min+を直接使用できない。展開演算子\Verb+...+を用いると
       計算ができる。次の実行例を参照のこと。
\begin{Verbatim}
>a = [3,5,23,1,4];
(5) [3, 5, 23, 1, 4]
>Math.max(a);
NaN
>Math.max(...a);
23
\end{Verbatim}
 \item EcmaScript 2016 ではべき乗の演算子\texttt{**}が導入されたので
\texttt{Math.pow()}は使わなくてもよい。
\end{itemize}
詳しい解説は[\ref{ES2016},第16章]にある。
%\newpage
%\clearpage
\subsection{\protect\texttt{Dateオブジェクト}}
日付や時間に関するデータを扱うコンストラクタ関数である。
引数の数や型によりいくつかの異なる日付と日時のオブジェクトが作成できる。
\input 00Date.tex
\begin{itemize}
 \item 引数なしの場合は現在の日付と日時を用いて作成
\begin{Verbatim}
>new Date()
Thu Oct 01 2015 13:17:46 GMT+0900 (東京 (標準時))
\end{Verbatim}
 \item 引数が数値の場合は世界標準時(UTC)において1970年1月1日0時からの経
       過時間をミリ秒単位で作成
 \item 引数が文字列の場合は日時を示す文字列に基づいて作成(次の例を参照)
 \item 引数の数が2つ以上7つの場合は年、月、日、時、分、秒、ミリ秒の順が
       指定されたとして作成。ない引数は0と解釈される。

       また、月の値は\texttt{0} が1月を表す。
\begin{Verbatim}
>new Date(2015,10,2,10,10,20,500);
Mon Nov 02 2015 10:10:20 GMT+0900 (東京 (標準時))
\end{Verbatim}

上記の例では月の値\texttt{10}となっているので11
月(\Verb+Nov+)に設定されている。

       引数の値が範囲を外れる(日に対して0を指定するなど)
       場合でも、正しく作成される。たとえば日の値が0のときは指定した月の
       1日の1日前、つまり前の月の最終日と解釈される。
\end{itemize}
%\newpage
\input prob02-04Date.tex
%
\clearpage
\section{第2回目復習課題}
%
\Answertrue
%
\Answerfalse
\begin{Prob}\upshape
\input prob02-05Check.tex
\end{Prob}
