\input devHeadDarmstadt.tex
%\newcommand{\ElmJ}{\texttt}
\title{ソフトウェア開発\\第2回目授業}
\author{平野 照比古}
\institute{}
\date{2018/10/5}
\begin{document}
\frame{\maketitle}
%\frame{\tableofcontents}
\section{前回のレポートから}\subsection{}
\begin{frame}[containsverbatim]
 \frametitle{レポートの形式(再掲)}
 \begin{itemize}
  \item 一番上に復習問題の用紙を置き、全体をステープラで止める。表紙は不
        要。
  \item 問題の結果をそのまま手書きでもよい。
  \item キャプチャや画像を印刷したときは、関係する説明文や考察も同じ用紙
        に記述すること。
  \item ブラウザを全画面表示してものをキャプチャしないこと。アクティブウィ
        ンドウだけのキャプチャすること。
  \item 必要な結果が入る範囲でウィンドウの表示のサイズを決めること。また、
        キャプチャ内の文字が読める程度にすること。
  \item ルーブリック評価は点数も含めて必ず自己評価をすること。採点結果と
        の差がなくなるようにするためである。
 \end{itemize}
\end{frame}
\begin{frame}[containsverbatim]
 \frametitle{テンプレートリテラルについて}
\begin{itemize}
 \item 問2における出力が間違っているものが多数見受けられた。
 \item 出力結果が数になる。
 \item テンプレートリテラルはテキストにあるようにバッククオート(\Verb+@+
       の上の文字)で囲む。
 \item バッククオート、シングルクオート、ダブルクオートを区別すること。
\end{itemize} 
\end{frame}
\begin{frame}[containsverbatim]
 \frametitle{変数のスコープについて}
\begin{itemize}
 \item 定義された変数が有効である範囲を変数のスコープという。
 \item 関数内で宣言された変数は関数内でしか有効ではない。
 \item したがって、復習問題のリストで、\texttt{foo();}こコンソールから実
       行すると、戻ってきた時点で変数は消えている。したがって、そこで変
       数\texttt{i}を参照すると「変数がない」というエラーが返る。
 \item 宣言がないと変数はグローバルで定義される。したがって、変数
       \texttt{i}の宣言をなくすと、\texttt{foo();}の実行後に変数の値が
       参照できる。
 \item 変数のスコープは第3回の授業でもう一度解説をする。({\bfseries 重要})
\end{itemize} 
\end{frame}
\begin{frame}[containsverbatim]
 \frametitle{\texttt{CDATA}セクションについて}
 ブラウザのHTML文書の解釈の手順を理解しておくこと
 \begin{itemize}
  \item ブラウザは要素(タグ)に基づいて文書の構造を解析する。
  \item \Verb+<+と\Verb+>+をもとに要素の構成を調べる。要素名については関
        知しない。
  \item したがって\Verb+<script>+要素内の比較演算子の\Verb+<+も要素の開
        始ととらえる。
  \item そのように解釈されないために書いてある文字をそのまま取り扱うよう
        にするための範囲を定めたのが\texttt{CDATA}セクション
  \item 構造の解釈が終わると要素の解釈が行われる。
  \item その際に、\Verb+<script>+要素内に書かれた\texttt{CDATA}セクショ
        ンの記述はJavaScriptの文法では正しくないのでその前に
        JavaScriptのコメントとする必要がある。
 \end{itemize}
\end{frame}
\section{JavaScriptが取り扱うデータ}
\subsection{データーの型}
\begin{frame}[containsverbatim]
 \frametitle{プリミティブデータ型}
\begin{tabular}{|c|m{20zw}|}\hline
 型&\multicolumn{1}{c|}{説明} \\\hline
 Number & 浮動小数点数だけ\\ \hline
 String & 文字列型、1文字だけのデータ型はない。ダブルクオート(\Verb+"+)
     か%"
     シングルクオート(\Verb+'+)で囲む。\\ \hline
 Boolean& \Verb+true+ か \Verb+false+ の値のみ\\ \hline
 \Verb+undefined+ & 変数の値が定義されていないことを示す\\ \hline
 \Verb+null+& \Verb+null+という値しか取ることができない特別なオブジェク
     ト\\ \hline
\end{tabular}

変数や値の型を知りたいときは\Verb+typeof+ 演算子を使う。
\end{frame}
\begin{frame}[containsverbatim]
 \frametitle{Number型}
JavaScriptで扱う数は64ビット浮動小数点形式
\begin{itemize}
 \item{\bfseries 整数リテラル} 10進整数は通常通りの形式である。16進数を表す場合は
	      先頭に\Verb+0x+ か \Verb+0X+ をつける。\Verb+0+で始まりそ
	      のあとに\Verb+x+または\Verb+X+が来ない場合には8進数と解釈
      される場合があるので注意が必要である。

      \Strict ではこの形式はエラーとなる。
 \item{\bfseries 浮動小数点リテラル} 整数部、そのあとに必要ならば小数点、小数部そ
       のあとに指数部がある形式である。
\end{itemize}
\end{frame}
\begin{frame}[containsverbatim]
 \frametitle{特別なNumber}
\begin{itemize}
 \item {\bfseries \Verb+Infinity+}無限大を表す読み出し可能な変数である。
       オーバーフローした場合や \Verb+1/0+などの結果としてこの値が設定さ
       れる。
 \item {\bfseries \Verb+NaN+} Not a Number の略である。計算ができなかった場合
       表す読み出し可能な変数である。
       文字列を数値に変換できない場合や \Verb+0/0+ などの結果としてこの値が設定さ
       れる。
\end{itemize}
 計算の途中でこれらの値が得られてもプログラムの実行は中断されない。
\end{frame}
\begin{frame}[containsverbatim]
 \frametitle{String型}
文字列に関する情報や操作には次のようなものがある。
\begin{center}
  \begin{tabular}{|c|m{14zw}|}\hline
 メンバー&\multicolumn{1}{c|}{説明} \\\hline
  \Verb+length+ &文字列の長さ\\ \hline
\Verb+indexOf(needle,start)+& \Verb+needle+が与えられた文字列内にあ
      ればその位置を返す。\Verb+start+の引数がある場合には、指定された位
      置以降から調べる。見つからない場合は$-1$を返す。\\\hline
\Verb+lastIndexOf(needle,start)+& \Verb+needle+が与えられた文字列内にあ
      ればそれが一番最後に現れる位置を返す。\Verb+start+の引数がある場合
      には、指定された位置以前から調べる。見つからない場合は$-1$を返す。\\\hline
  \end{tabular}
\end{center}
\end{frame}
\begin{frame}[containsverbatim]
 \frametitle{String型}
\begin{center}
 \begin{tabular}{|c|m{14zw}|}\hline
 メンバー&\multicolumn{1}{c|}{説明} \\\hline
  \Verb+split(separator,limit)+&\Verb+separator+で与えられた文字列で与え
      られた文字列を分けて配列で返す。セパレーターの部分は返されない。
      2番目の引数はオプションで分割する最大数を返す。\\ \hline
  \Verb+substring(start,end)+&与えられた文字列の\Verb+start+から
      \Verb+end+の位置までの部分文字列を返す。\\ \hline
\end{tabular}
\end{center}
\end{frame}
\begin{frame}[containsverbatim]
 \frametitle{Bool型}
 \begin{itemize}
  \item \Verb+true+ と \Verb+false+ の2つの値をとる。
  \item この2つは予約語
  \item 論理式の結果としてこれらの値が設定されたり、論理値が必要なところ
        でこれらの値に設定される。
 \end{itemize}
\end{frame}
\begin{frame}[containsverbatim]
 \frametitle{\texttt{undefined}}
 \begin{itemize}
  \item 値が存在しないことを示す読み出し可能な変数
  \item 変数が宣言されたのに値が設定されていない場合などはこの値に初期化
  \item 関数に戻り値を指定しないと気の値
 \end{itemize}
\end{frame}
\begin{frame}[containsverbatim]
 \frametitle{\texttt{null}}
\Verb+typeof null+ の値が \Verb+"object"+であることを示すように、オブジェ
クトが存在しないことを示す特別なオブジェクト値(であると同時にオブジェク
トでもある)である。
\end{frame}
\begin{frame}[containsverbatim]
\frametitle{非プリミティブデータ}
プリミティブ型以外のデータ型は「オブジェクト」(\texttt{()}内はオブジェク
トを作成するコンストラクタ)の名称
\begin{itemize}
 \item 配列(\ElmJ{Array})
 \item 数学(\ElmJ{Math})
 \item 日時(\ElmJ{Date})
 \item 正規表現(\ElmJ{RegExp})
 \item マップ(\ElmJ{Map})とウィークマップ(\ElmJ{WeakMap})
 \item セット(\ElmJ{Set})とウィークセット(\ElmJ{WeakSet})
\end{itemize}
 これらのデータ型のうち、マップ(\ElmJ{Map})、セット(\ElmJ{Set})、ウィークセッ
 ト(\ElmJ{WeakSet})以外は後で解説
\end{frame}
\subsection{変数}
\begin{frame}[containsverbatim]
 \frametitle{JavaScriptにおける変数}
\begin{itemize}
 \item 変数名はアルファベットまたは\Verb+_+(アンダースコア)で始まる英数
       字または\Verb+_+で始まる文字列
 \item 大文字と小文字は区別される。
 \item 変数の宣言は \texttt{let} または \texttt{const} で行う。
 \item \texttt{const}による変数の宣言では初期化時の値の変更ができない。
 \item 従来の宣言方法の\texttt{var}は問題があるのでこの授業では使用しない。
 \item 宣言時に初期化ができる。
 \item 非\Strict のときは変数は宣言をしなくても使用できる。初期化してい
       ない変数を使うとエラーが起こる。詳しくは後述する。
 \item 変数に保存するデータの型には制限がない。途中で変更することもでき
       る。
\end{itemize}
\end{frame}
\section{配列}
\subsection{配列の宣言と初期化}
\begin{frame}[containsverbatim]
 \frametitle{配列の宣言と初期化}
配列を使うためには、変数を配列で初期化する必要がある。変数の宣言と同時に
行ってもよい。
\begin{Verbatim}
let a = [];
let b = [1,2,3];
\end{Verbatim}
 \Verb+a+ は空の配列で初期化\\
 \Verb+b+ は\Verb+b[0]=1,b[1]=2,b[2]=3+ となる配列で初期化
\end{frame}
\begin{frame}[containsverbatim]
 \frametitle{配列に関する注意(1)}
\begin{itemize}
 \item 配列の各要素のデータの型は同じでなくてもよい。
 \item 実行時に配列の大きさを自由に変えられる。
 \item 配列の要素に配列を置くことができる。
\begin{Verbatim}
let a=[1,[2,3,4],"a"];
\end{Verbatim}
\end{itemize}
\end{frame}
\begin{frame}[containsverbatim]
 \frametitle{配列に関する注意(2)}
配列の変数の代入は参照
\begin{Verbatim}
>let a = [0,1,2,3,4];
>let b = a;  //別の変数への代入
>b;
(5) [0, 1, 2, 3, 4]  // a と同じ内容が表示
>b[3]= b[3]*10;      // 4番目の要素を10倍 [0, 1, 2, 30, 4]となる
30
>a;
(5) [0, 1, 2, 30, 4] // a も b と同じ値の要素を持つ
\end{Verbatim}
\end{frame}
\begin{frame}[containsverbatim]
 \frametitle{分割代入}
 配列の要素の一部をまとめて別の変数に代入する。
\begin{Verbatim}
>\Verb+[a,,b,...c] = [0,1,2,3,4,5,6,7];
a ==> 0
b ==> 2
c ==> [3, 4, 5, 6, 7]
\end{Verbatim}
       \begin{itemize}
        \item  変数\Verb+a+には0番目の、\Verb+b+には3番目の、\Verb+c+に
               は4番目以降の配列の要素が代入される。
        \item \Verb+a+ と \Verb+b+ の間に\Verb+,,+がある
       ので2番目の要素は代入されない。
        \item 4番目以降の要素はまとめて変数\Verb+c+に代入
       \end{itemize}
 配列が関数の戻り値のとき、戻り値の必要なところだけ利用す
       るために便利な機能

 \Verb+...+は展開演算子と呼ばれ、配列に対して要素を並べたものを表す。
\end{frame}
\subsection{配列のメソッド}
\begin{frame}[containsverbatim]
 \frametitle{配列のメソッド}
\begin{center}
 \begin{tabular}{|c|m{20zw}|}\hline
 メンバー&\multicolumn{1}{c|}{説明} \\\hline
  \Verb+length+ &配列の要素の数\\ \hline
  \Verb+join(separator)+& 配列を文字列に変換する。\Verb+separator+はオプ
      ションの引数で、省略された場合はカンマ\Verb+,+である。\\ \hline
  \Verb+pop()+& 配列の最後の要素を削除し、その値を返す。配列をスタックと
      して利用できる。\\ \hline
  \Verb+push(i1,i2,...)+& 引数で渡された要素を配列の最後に付け加える。配
      列をスタックやキューとして利用できる。\\ \hline
  \Verb+shift()+&配列の最初の要素を削除し、その値を返す。配列をキューと
      して利用できる。\\ \hline
\end{tabular}
\end{center}
\end{frame}
\begin{frame}[containsverbatim]
 \frametitle{配列のメソッド}
\begin{center}
 \begin{tabular}{|c|m{12zw}|}\hline
   \Verb+slice(start,end)+&\Verb+start+から\Verb+end+の前の位置にある要素を取
      り出した配列を返す。元の配列は変化しない。\\ \hline
  \Verb+splice(start,end,i1,i2,...)+&\Verb+start+から\Verb+end+の位置に
      ある要素を取
      り出した配列を返す。元の配列からこれらのデータを取り除き、その位置
      に\Verb+i1,i1,...+以下の要素を付け加える。\\ \hline
\end{tabular}
\end{center}
\end{frame}
\section{演算子}
\subsection{代入、四則演算}
\begin{frame}[containsverbatim]
 \frametitle{代入、四則演算}
\begin{itemize}
 \item \Verb-+-演算子は文字列の連接にも使用できる。\Verb-+-演算子は左右のオペ
ランドがNumberのときだけ、数の和をとる。どちらかが数でもう一方が文字列の
場合は数を文字列に直して、文字列の連接を行う。
\begin{Verbatim}
1+2  => 3
1+"2" => 12
\end{Verbatim}
 \item そのほかの演算子(\Verb+-*/+)については文字列を数に変換してから数
       として計算する。
 \item 文字列全体が数にならない場合には変換の結果が\Verb+NaN+になる。

 \item 整数を整数で割った場合、割り切れなければ小数となる。
       小数部分を切り捨てたいときは\Verb+Math.floor()+を用いる。
\begin{Verbatim}
1/3 => 0.3333333333333333
Math.floor(1/3) =>0
\end{Verbatim}
\end{itemize}
\end{frame}
\subsection{論理演算子と比較演算子}
\begin{frame}[containsverbatim]
 \frametitle{論理演算子}
Boolean 型に対して使用できる演算子は次の3つ
\begin{itemize}
 \item \Verb+!+ 論理否定
 \item \Verb+&&+ 論理積
 \item \Verb+||+ 論理和
\end{itemize}
\end{frame}
\begin{frame}[containsverbatim]
 \frametitle{論理演算子に関する注意}
論理演算子をBoolean 型でない値を与えると元の値がBoolean型に変換されてか
ら実行される。次の値は \Verb+false+ に変換される。
\begin{itemize}
 \item 空文字列 \Verb+""+
 \item \Verb+null+
 \item \Verb+undefined+
 \item 数字の $0$
 \item 数値の \Verb+NaN+
 \item Boolean の\Verb+false+
\end{itemize}
\end{frame}
\begin{frame}[containsverbatim]
 \frametitle{論理演算子に関する注意}
論理和や論理積では左のオペランドの結果により、式の値が決まる場合は右のオ
ペランドの評価は行われない。たとえば、論理和の場合、左の値が
\Verb+true+であれば右のオペランドの評価が行われない。
\begin{Verbatim}
let a=1; true ||(a=3); 
\end{Verbatim}
では変数 \Verb+a+ の値は $1$ のままである。
\end{frame}
\begin{frame}[containsverbatim]
 \frametitle{比較演算子}
比較演算子は比較の結果、Boolean の値を返す演算である。C言語と同様の比較
演算子が使用できる。\texttt{>,>=,<,<=}など。等しいことを比較するためには
\Verb+==+(等価比較演算子)のほかに 型変換を伴わない等価比較演算子
\Verb+===+ がある。等価比較演算子\texttt{==}は必要に応じて型変換を行う。
\begin{Verbatim}
0 == "0" => true
0 === "0" => false
\end{Verbatim}
同様に非等価比較演算子\texttt{!=}にも型変換を伴わない非等価演算子
\texttt{!==}がある。

 また、\Verb+NaN == NaN+ の結果は \Verb+false+ である。値が \Verb+NaN+ で
あるかどうかを調べる関数がある。
\end{frame}
\section{制御構造}
\begin{frame}[containsverbatim]
 \frametitle{制御構造}
 \begin{itemize}
  \item C言語などと同様に\ElmJ{if}文、\ElmJ{for}文、\ElmJ{while}
文、\ElmJ{switch}文がある。
  \item 使い方は同様なのでここでは説明をしない。
  \item Java やC++のように
\ElmJ{for}文の初期化のところに現れる変数を\ElmJ{let}を用いて宣言すること
もできる。
  \item この宣言された変数に関するJavaScript特有の注意は次回解説
 \end{itemize}
\end{frame}
 \section{組み込みオブジェクト}
\subsection{組み込み関数}
\begin{frame}[containsverbatim]
 \frametitle{組み込み関数}
JavaScript に初めから組み込まれている関数としては次のものがある。
\begin{itemize}
 \item \Verb+parseInt(string,radix)+ \Verb+string+(文字列)と
       \Verb+radix+(基数、省略したときは$10$)をとり、先頭から見て正しい
       整数表現のところまで整数に変換する。
 \item \Verb+parseFloat(string)+ \Verb+string+(文字列)
       をとり、先頭から見て正しい
       浮動小数点表現のところまで浮動小数に変換する。
 \item \Verb+isNaN(N)+ \Verb+N+ が数であれば \Verb+true+、そうでなければ
       \Verb+false+ を返す関数
 \item \Verb+isFinite(N)+ \Verb+N+ が数値または数値に変換できる値でかつ
       \Verb+Infinity+ または \Verb+-Infinity+ でないときに \Verb+true+、
       そうでないとき、\Verb+false+ を返す。
\end{itemize}
このほかにも関数があるが、省略する。
\end{frame}

\begin{frame}[containsverbatim]
 \frametitle{組み込みオブジェクト}
\begin{itemize}
 \item \texttt{Date}オブジェクト

日付や時間に関するデータを扱うオブジェクトである。
基本的にはミリ秒単位の値が返ってくるので、実行時間の測定など
にも使える。
 \item {\texttt{Math}オブジェクト}

数学的な定数の定義(円周率など)や三角関数などの関数が定義されている。
\end{itemize}
\end{frame}

\end{document}

\subsection{第2回目復習課題}
\newcommand{\Rule}{\rule[-2ex]{0em}{5ex}}
次の式の評価結果を求めなさい。
\begin{center}
 \begin{tabular}{|>{\Rule}c|p{5zw}|p{23zw}|}\hline
  式&\multicolumn{1}{c|}{結果} &\multicolumn{1}{c|}{理由} \\\hline
  \Verb+5%3+& & \\ \hline
  \Verb-4+"5"-& & \\ \hline
  \Verb+4-"5"+& & \\ \hline
  \Verb-4+"ff"-& & \\ \hline
  \Verb-4+"0xff"-& & \\ \hline
  \Verb-4+parseInt("ff")-& & \\ \hline
  \Verb-4+parseInt("0xff")-& & \\ \hline
  \Verb-4+parseInt("ff",16)-& & \\ \hline
  \Verb-4+"1e1"-& & \\ \hline
  \Verb-4+parseInt("1e1")-& & \\ \hline
  \Verb-4+parseFloat("1e1")-& & \\ \hline
  \Verb+"4"*"5"+& & \\ \hline
  \Verb+"4"/"5"+& & \\ \hline
  \Verb+[].length+& & \\ \hline
  \Verb+[[]].length+& & \\ \hline
  \Verb+0 == "0"+& & \\ \hline
  \Verb+0 == []+& & \\ \hline
  \Verb+![]+& & \\ \hline
  \Verb+false == []+& & \\ \hline
  \Verb+"false" == []+& & \\ \hline
  \Verb+[] == []+& & \\ \hline
  \Verb+typeof []+& & \\ \hline
  \Verb+null == undefined+& & \\ \hline
 \end{tabular}
\end{center}
