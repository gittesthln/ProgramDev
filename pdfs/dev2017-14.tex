\input devHead.tex
\SetTheme{AnnArbor} %11
\title{ソフトウェア開発\\第14回目授業}
\author{平野 照比古}
\institute{}
\date{2017/1/13}
\newtheorem{Prob}{解説}
\newcommand{\Elm}[1]{\texttt{<#1>}}
\setbeamercovered{transparent}

\newcommand{\DOMM}{\texttt}
\newcommand{\Event}{\texttt}
\newcommand{\DOMP}{\texttt}
\newcommand{\DOM}{\texttt{DOM}}
\newcommand{\keyitem}{\relax}
\newcommand{\HTML}{HTML文書}
\begin{document}
\frame{\maketitle}
\section{コードの構造}
\subsection{良いコードとは}
 \begin{frame}[containsverbatim]
  \frametitle{良いコードとは}
多くの公開されているコードを眺めているといくつかの共通点が
  見つかる。
\begin{itemize}
 \item プログラムの構造がわかりやすくなるように字下げ(インデント)を付け
			 る。
 \item 変数名や関数名は実態を表すようにする。\\いくつかの単語をつなげて
			 変数名や関数名にすることが多くなる。これらが読みやすいように途中に出てく
			 る単語の先頭は大文字にすることが最近の傾向である(キャメル形式と呼
			 ばれる)。
 \item 不必要に長いプログラム単位を作らない。\\
			 ある程度まとまった作業をする部分は関数にすると見通しの良いプログ
			 ラムとなる。
 \item プログラム自体が説明書になっている。\\
			 コメントを多用して必要な説明をすべて記述する。
\end{itemize}
 \end{frame}
 \begin{frame}[containsverbatim]
  \frametitle{プログラムとデータの分離}
  \begin{itemize}
   \item ある処理をするときの分岐の条件をプログラム内に直接記述すると、分
			 岐の条件が変わったときにはプログラムをコンパイルしなおす必要が生
         ずる。
   \item これを避けるためには外部からデータを読み込んで、それに基づ
			 いた処理を行うようにする。

  \end{itemize}
 \end{frame}
\subsection{プログラムとデータの分離}
 \begin{frame}[containsverbatim]
  \frametitle{成績評価}
  期末試験の結果成績をつける。
  \begin{itemize}
   \item 90点以上ならば\texttt{S}
   \item 80点以上ならば\texttt{A}
   \item 70点以上ならば\texttt{B}
   \item 60点以上ならば\texttt{C}
   \item それ以外は\texttt{E}
  \end{itemize}
 期末試験の点
 を与えて、評価の\texttt{S}などを返す関数\texttt{seiseki}を作成する。
\end{frame}
 \begin{frame}[containsverbatim]
  \frametitle{簡単な方法(JavaScript)}
\begin{Verbatim}
function seiseki(val) {
  if(val >= 90) return "S";
  if(val >= 80) return "A";
  if(val >= 70) return "B";
  if(val >= 60) return "C";
  return "E";
}
\end{Verbatim}
\end{frame}
\begin{frame}[containsverbatim] 
\frametitle{配列の利用(1)}
\begin{Verbatim}
function seiseki(val) {
  var Score = [90,80,70,60,0];
  var Results= ["S","A","B","C","E"];
  var i;
  for(i=0;i<Grade.length;$i++) {
    if(val>=Grade[i]) return Hyouka[i];
  }
  return "Error";
}
\end{Verbatim}
\end{frame}
\begin{frame}[containsverbatim]
 \frametitle{配列の利用(2)}
 配列をまとめる。
\begin{Verbatim}
var Results =[["S",90],["A",80],["B",70],["C",60],["E",0]];
\end{Verbatim}
 連想配列の利用
\begin{Verbatim}
var Results ={S:90,A:80,B:70,C:60,E:0};
function seiseki(val) {
  for(v in Results) {
    if(val >=Results[v]) return v;
  }
}
\end{Verbatim}
JavaScriptでは\texttt{for(.. in ..)}で連想配列の要素をすべて渡ることがで
き、かつわたる順序が定義された順なのでこのコードが可能
\end{frame}
\begin{frame}[containsverbatim]
 \frametitle{JavaScriptにおける注意}
\begin{itemize}
 \item JavaScript内から外部ファイルを自動で読み込むことがで
きない
 \item 変数の形で値を用意
 \item この変数部
分を別のプログラムから作成する方法も可能
\end{itemize}
\end{frame}
\section{コードの管理}
 \begin{frame}[containsverbatim]
  \frametitle{複数でシステム開発をするときの問題点}
\begin{itemize}
 \item 複数の人間で開発している場合、開発者の間で最新のコードの共有
 \item ファイルを間違って削除したり、修正がうまくいかなかったときに過去
       のコードに戻す
 \item 安定しているコードに新規の機能を付け加えてテストをする場合、安定
       したコードに影響を与えないようにする
\end{itemize}
このようなコードの変更履歴の管理するソフトウェアをバージョン管理ソフ
トという。バージョン管理ソフトウェアの対象となるファイルはテキストベース
のものを主としている。
 \end{frame}
\subsection{バージョン管理の概念}
\begin{frame}[containsverbatim]
 \frametitle{バージョン管理ソフトとは}
バージョン管理ソフトは各時点におけるファイルの状態を管理するデータベース
 である。このデータベースは一般にリポジトリと呼ばれる。
\end{frame}
\begin{frame}[containsverbatim]
 \frametitle{バージョン管理ソフトを用いた開発手順}
\begin{itemize}
 \item リポジトリからファイルの最新版を入手
 \item ローカルな環境でファイルを変更。
 \item ファイルをリポジトリに登録
\end{itemize}
\end{frame}
\begin{frame}[containsverbatim]
 \frametitle{バージョン管理ソフトを用いた開発手順の問題点}
\begin{itemize}
 \item 同じファイルを複数の開発者が変更した場合、競合が
発生していないかをチェックする機能
 \item この機能がない場合には
同じファイルの変更は同時に一人しか変更ができないようにファイルをロック
\end{itemize}\end{frame}
\begin{frame}[containsverbatim]
 \frametitle{開発の流れ}
 \begin{itemize}
  \item 開発の流れをブランチと呼ぶ
  \item あるブランチに対して新規の
機能を付け加えたいときなどには元のブランチには手を付けずに別のブランチを
作成して、そこで開発
  \item 機能が安定すれば元のブランチに統合(merge)
 \end{itemize}
\end{frame}
\subsection{バージョン管理ソフトの概要}
\begin{frame}[containsverbatim]
 \frametitle{CVS(Concurrent Versions System)とSubversion}
 \begin{itemize}
  \item 管理するためにサーバーが必要
  \item 開発者はこのサーバーにアクセスしてファ
イルの更新などを行う
 \end{itemize}
\end{frame}
\begin{frame}[containsverbatim]
 \frametitle{gitの場合}
 \begin{itemize}
  \item 各開発者のローカルな環境にサーバー上のリポジトリの複製
を持つ
  \item ネットワーク環境がない状態でもバージョン管理が行える
  \item git ではネットワーク上に GitHub と呼ばれるサーバーを用意しており、ここに
リポジトリを作成することで開発者間のデータの共有が可能
  \item データを公開すれば無料で利用できる
  \item 有料のサービスやサーバー自体を個別に持つサービスも提供
  \item 有名なオープンソースのプロジェクトが GitHub を利用
 \end{itemize}
\end{frame}
\subsection{gitの使い方}
\begin{frame}[containsverbatim]
 \frametitle{git の特徴}
詳しい使い方は \Verb+https://git-scm.com/book/ja/v2/+を参照
\begin{itemize}
 \item 分散型のバージョンシステムなので、ローカルでブランチの作成が可能

       集中型のバージョン管理システムではブランチの作成が一部の人しかで
       きない。
 \item 今までのファイルの変更履歴などが見える。
 \item  GitHub 上ではファイルの更
       新などの通知を受けることができる。
 \item 開発者に対して変更などの要求が可能(pull request)
\end{itemize}
\end{frame}
\begin{frame}[containsverbatim]
 \frametitle{git のインストール}
 \begin{itemize}
  \item git はローカルにリポジトリを持つのでそれを処理する環境をインストー
        ルする
  \item git for windows が良い
  \item 「Git Bash」、「GitCMD」と「Git GUI」の3つがインストールされる。
  \item Unix 風のコマンドプロンプトが起動する Git Bash がおすすめ
 \end{itemize}
\end{frame}
\begin{frame}[containsverbatim]
 \frametitle{初期設定}
 \begin{itemize}
  \item GitHubで使用するリポジトリのユーザ名を登録

        \Verb+git --config --global user.name "Foo"+

ここでは\Verb+Foo+がユーザ名となる。
  \item GitHubからの通知を受けるメールアドレス

        \Verb+git --config --global user.email "Foo@example.com"+
 \end{itemize}
 \end{frame}
\begin{frame}[containsverbatim]
 \frametitle{SSH Keyの登録}
 \begin{itemize}
  \item GitHub との接続には SSH による通信で行う
  \item この通信はは公開鍵暗号方式で行う
  \item キーの生成は次のコマンドで行う
  \item \Verb+ssh-keygen -t rsa -C "Foo@example.com"+

\Verb+-C+のオプションはコメント 
  \item キーを保存するフォルダが聞かれるが、デフォルトでかまわない
  \item passphrase の入力が求められる
  \item 指定したフォルダの\Verb+.ssh+の下に\Verb+id_rsa+(秘密鍵)と
\Verb+id_rsa.pub+(公開鍵)の2つのファイルが作成される。 
 \end{itemize}
\end{frame}
\begin{frame}[containsverbatim]
 \frametitle{GitHub のアカウントの取得}
 \begin{itemize}
  \item ローカルだけで開発をするのであればアカウントは不要
  \item データを交換するのであればアカウントを取ることを勧める
  \item 作成した公開鍵の内容をアカウントで設定する
  \item アカウントを取ったら GitHub 上でテストのプロジェクトを中身が空で作成
 \end{itemize}
\end{frame}
\begin{frame}[containsverbatim]
 \frametitle{リポジトリの作成と運用}
リポジトリの初期のブランチ名は\Verb+master+

 \footnotesize
\begin{tabular}{|c|c|c|}\hline
コマンド&パラメータ&説明\\\hline
\Verb+git init+ & &プロジェクトの初期化\\\hline
\Verb+git add+ & ファイルリスト&プロジェクトへのファイルの登録 \\\hline
\Verb+git commit+& \Verb+-m+ コメント& リポジトリへの変更の登録\\ \hline
\Verb+git remote add+ &短縮名 リポジトリ & リポジトリの短縮名の登録\\
 \hline
\Verb+git push+ &\Verb+-u+ 短縮名 ブランチ名 &ブランチへの変更の登録 \\ \hline
\Verb+git clone+ &リポジトリ名 &リモートのリポジトリのコピーを作成\\ \hline
\Verb+git pull+ &リポジトリ名 ブランチ名 &リモートのリポジトリのコピーを作成\\ \hline
\end{tabular}
\end{frame}
\begin{frame}[containsverbatim]
 \frametitle{新しいプロジェクトの構成}
\begin{enumerate}
 \item \Verb+git init+ で新たにリポジトリを作成、または、
			 \Verb+git clone+でGitHubからリポジトリのコピーを取得する。
 \item ファイルを編集
 \item 編集したファイルを\Verb+git add+ でステージ
 \item \Verb+git commit+ でその時点での変更を登録
 \item \Verb+-m+オプションで簡単なコメントをつけることを忘れないこと。
 \item このオプションをつけないと\Verb+vim+というテキストエディタが立ち
       上がり、コメントの編集を行う
 \item \Verb+git push+ リモート名 ブランチ名 で変更をリモートのリポジト
			 リに反映される。
 \item これが拒否された場合には誰かが\Verb+push+している
			 のでその変更を\Verb+pull+してから調整して再度\Verb+push+する。
\end{enumerate}
\end{frame}
\begin{frame}[containsverbatim]
 \frametitle{ファイルの状態の注意}
\begin{itemize}
 \item リポジトリ内のファイルは追跡されている(tracked)ものと追跡されていない
(untracked)に分けられる。
 \item ファイルの状態として変更されていない
(unmodified)、変更されている(modified)とステージされている(staged)の3つ
の状態がある。
 \item modified から staged にするコマンドが \Verb+git add+ であ
る。このコマンドを新規にリポジトリに追加するという意味にとらえないことが
必要である。
\end{itemize}
\end{frame}
\end{document}
