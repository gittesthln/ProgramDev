\input devHead.tex
\SetTheme{AnnArbor} %10
\title{ソフトウェア開発\\第9回目授業}
\author{平野 照比古}
\institute{}
\date{2018/11/30}
\newtheorem{Prob}{解説}
\newcommand{\Elm}[1]{\texttt{<#1>}}
\setbeamercovered{transparent}

\newcommand{\DOMM}{\texttt}
\newcommand{\Event}{\texttt}
\newcommand{\DOMP}{\texttt}
\newcommand{\DOM}{\texttt{DOM}}
\newcommand{\keyitem}{\relax}
\newcommand{\HTML}{HTML文書}
\begin{document}
\frame{\maketitle}
\section{前回のレポートから}
\subsection{問題1}
 \begin{frame}[containsverbatim]
  \frametitle{オブジェクトリテラルでプルダウンメニューを作成(1)}
  連番のプルダウンメニュー作成関数を真似ればよい。
  \VerbatimInput[firstline=9,lastline=17,fontsize=\footnotesize]{../html/09-03pulldownObj.html}

  \texttt{for}文を\texttt{for(.. in ..)}で書き直せばよい。
 \end{frame}
 \begin{frame}[containsverbatim]
  \frametitle{オブジェクトリテラルでプルダウンメニューを作成(2)}
  連番のプルダウンメニュー作成関数を真似ればよい。
  \VerbatimInput[firstline=17,lastline=24,fontsize=\footnotesize]{../html/09-03pulldownObj.html}

  プルダウンメニューの表示文字列は引数内にあるので不要な形にした。
 \end{frame}
 \begin{frame}[containsverbatim]
  \frametitle{オブジェクトリテラルでラジオボタンを作成}
  連番のプルダウンメニュー作成関数を真似ればよい。
  \VerbatimInput[firstline=25,lastline=32,fontsize=\footnotesize]{../html/09-03pulldownObj.html}

\begin{itemize}
 \item 全体をまとめるために\texttt{div}要素内にラジオボタンを置く。
 \item ラジオボタンの説明文字列はラジオボタンと同じレベルの子要素にする。
 \item ラジオボタンの子要素では表示されない。
\end{itemize}
 \end{frame}
 \subsection{問題2}
  \begin{frame}[containsverbatim]
   \begin{itemize}
    \item \ElmJ{clientX}と\ElmJ{pageX}などの違いについての操作はほとんど
          合っていた。
		\item $\ElmJ{pageY}=\ElmJ{clientY}+\ElmJ{window.pageOffsetY}$の指摘
					がなかった。
    \item \ElmJ{target}と\ElmJ{currentTarget}の違いを確認するところで
          \ElmJ{target.value}のように要素の\ElmJ{value}を表示させていた
          誤答が目立った。
%    \item \ElmJ{alert}で表示させている誤答もあった。このように表示させる
%          と文字列に変換されるので要素の詳細がわからない。
   \end{itemize}
  \end{frame}
  \subsection{問題3}
  \begin{frame}[containsverbatim]
   \frametitle{階層を持ったプルダウンメニューの作成}
   \begin{itemize}
    \item 問題の意図はデータの持ち方を考えてもらうこと
    \item 重複がなく修正や追加がしやすい構造を考えることが重要
    \item ここでは地方毎にそれに属する地域をリストアップ
    \item 全国のデータはそれらを寄せ集めて作成
   \end{itemize}
  \end{frame}
  \begin{frame}[containsverbatim]
   \frametitle{作成例(1)}
   \VerbatimInput[firstline=6,lastline=18,fontsize=\footnotesize]{../html/jma.html}
   \begin{itemize}
    \item 気象庁のページのソースを尊重してプルダウンメニューの
          \ElmJ{value}をオブジェクトリテラルに記述
    \item 地域のリストはこのオブジェクトのキーから作成
    \item 全国のリストは値から作成
   \end{itemize}
  \end{frame}
  \begin{frame}[containsverbatim]
   \frametitle{作成例(2)}
   \VerbatimInput[firstline=161,lastline=183,fontsize=\scriptsize]{../html/jma.html}
  \end{frame}
\section{PHPの超入門}
\subsection{Web サーバーの基礎知識}
\begin{frame}[containsverbatim]
\frametitle{Webサーバーソフト}
\begin{itemize}
 \item hypertext transfer protocol(HTTP) を利用して、クライアン
トプログラム(Webブラウザなど)からの要求を処理
 \item このサービスを提供するプログラムとしては Apache が有名
 \item Apache によるサービスの設定ファイルは
       {\texttt{httpd.conf}}
\begin{itemize}
 \item ポート番号(HTTPのポート番号は80番)
 \item ドキュメントルート:Apache では Webサーバーに直接要求できるファイ
       ルはこのフォルダ(ディレクトリ)の下にあるものだけ
\end{itemize}
\end{itemize}
%IANA(Internet Assigned Numbers Authority)は DNS Root, IPアドレスやインター
%       ネットプロトコルなどに含まれる番号などを管理しているところである。
%\texttt{https://www.iana.org/assignments/}\\ \texttt{service-names-port-numbers/service-names-port-numbers.xhtml}
\end{frame}
\begin{frame}[containsverbatim]
\frametitle{ポート番号の種類}
\begin{itemize}
 \item TCP/IP におけるサービスを識別するための番号
 \item ポート番号は大別して次の3種類
\begin{center}
 \begin{tabular}{|c|r@{番$\sim$}r<{番}|%c|
}
\hline
種類 &\multicolumn{2}{c|}{範囲}% &内容
 \\\hline
  System Ports(Well Known Ports)& 1&1023 %& assigned by IANA
\\ \hline
  User Ports, (Registered Ports)& 1024&49151 %& assigned by IANA
\\  \hline
  Dynamic Ports(Private or Ephemeral Ports)& 49152& 65535%&never
  %assigned
 \\ \hline
 \end{tabular}
\end{center}
\end{itemize}
\end{frame}
\subsection{CGI}
\begin{frame}[containsverbatim]
\frametitle{Common Gateway Interface(CGI)}
\begin{itemize}
 \item Webコンテンツをダイナミックに生成する標準の方法
 \item Webサーバー上ではWebサーバーとWebコンテンツを生成するためのインター
       フェイス
 \item この授業ではCGIのプログラムはPHP(PHP:Hypertext Preprocessor)を使用
\end{itemize}
\end{frame}
\subsection{Webサーバーのインストール}
\begin{frame}[containsverbatim]
\frametitle{Webサーバーのインストール}
\begin{itemize}
 \item HTTPのサーバーを動かすためにはApacheやPHPを
インストールした後、いくつかの設定
 \item XAMPPは Apache、MySQL、PerlとPHPを一括してインストールパッケージ
\begin{itemize}
  \item インストール時に Apache をサービスとして実行するとパソコンを立ち
       上げた時にApacheを起動させる手間が省ける。
 \item 標準の文字コードは{UTF-8}である。
 \item PHPの初期設定でタイムゾーンがヨーロッパ大陸になっている。
 \item タイムスタンプを利用するプログラムの動作に注意すること。
\end{itemize}
\end{itemize}
\end{frame}
\input dev2016-09php1.tex
\input dev2016-09php2.tex
\end{document}

