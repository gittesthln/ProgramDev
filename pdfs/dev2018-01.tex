\input devHeadDarmstadt.tex
\title{ソフトウェア開発\\第1回目授業}
\author{平野 照比古}
\institute{}
\date{2018/9/28}
\begin{document}
\frame{\maketitle}
%\frame{\tableofcontents}
\section{受講に関する注意}
\subsection{授業について}
\begin{frame}
 \frametitle{授業内容 -- 概略}
\begin{itemize}
 \item JavaScript を通常の計算機言語として利用するための解説を10回の講義
       で行う。
 \item 進度が今までのプログラミングの授業より早いので復習をよくすること。
 \item パソコンを授業に持参す必要がある場合はその旨、前回の授業で指示
       する(多分ない)。
\end{itemize}
\end{frame}
\begin{frame}
 \frametitle{授業内容 -- 復習用課題}
\begin{itemize}
   \item 配布された復習用の課題は用紙に解答を直接記入するか、印刷したものを添付して次
        回の授業の前日の木曜日10時までに6階の平野のレポートボックスに提
        出
 \item 復習用課題と同時に配布されるルーブリック評価表も{\bfseries 自己評価を付けて}
       提出
 \item レポートの内容はルーブリック評価表の「標準的」の項目を参照するこ
       と。
 \item レポート課題に関して(必須)が付いているものは必ず解答すること。
 \item 復習課題の提出方法はメール提出に変更する場合もある。
\end{itemize}
\end{frame}
\begin{frame}[containsverbatim]
 \frametitle{授業内容 -- その他}
\begin{itemize}
 \item 復習課題とそのルーブリック評価表は次回の授業の開始時に添削、評価
       をしたものを返却する。
 \item 復習用課題は10回程度を予定している。成績評価の60\%を占める。
\item 最終回の授業に試験を行う。成績評価の40\%を占める。
 \item 配布資料等は\texttt{http://www.hilano.org/hilano-lab}で公開する予
       定である。
\end{itemize}
\end{frame}
\subsection{参考図書}
\begin{frame}
 \frametitle{参考図書}
\begin{thebibliography}{9}
 \bibitem{FirstJS} E. Brown, 初めてのJavaScript 第3版, --ES2015以降の最新ウェブ開発, オラ
       イリージャパン\label{ES2016}
% \bibitem{GoodParts}D. Crockford, JavaScript: The Good Parts 「良いパー
%	 ツ」によるベストプラ クティス, オライリージャパン
 \bibitem{JS6}David Flanagan, JavaScript 第6 版, オライリージャパン
 \bibitem{JSR6}David Flanagan, JavaScript クイックリファレンス第6 版, オ
	 ライリージャパン
 \bibitem{HPJS}Nicholas C. Zakas, ハイパフォーマンスJavaScript, オライリー
	 ジャパン
 \bibitem{MenJS}Nicholas C. Zakas, メンテナブルJavaScript  読みやすく保守しやすいJavaScript 
コードの作法, オライリージャパン
\end{thebibliography}
\end{frame}
\section{シラバス}\subsection{}
\begin{frame}
 \begin{center}
\begin{tabular}{|c|m{24zw}|}\hline
 授業回数&\multicolumn{1}{c|}{内容}\\\hline
 第1回&授業のガイダンスとブラウザの開発者モードについて \newline
     JavaScriptの実行環境の確認\\\hline
 第2回&JavaScript が取り扱うデータ\newline
     データの型と演算子に関する注意など\\\hline
 第3回& 関数の定義方法と変数のスコープ\\\hline
 第4回& オブジェクトの定義とクラス\\\hline
 第5回&オブジェクトの詳細\newline
     関数に要るオブジェクトの定義、エラー処理\\\hline
 第6回&正規表現と文字列の処理  \\\hline
 第7回&DOMの利用\newline
     HTML文書、CSSと DOM の基礎\\\hline
\end{tabular}
\end{center}
\end{frame}
\begin{frame}
 \begin{center}
\begin{tabular}{|c|m{23zw}|}\hline
 授業回数&\multicolumn{1}{c|}{内容}\\\hline
 第8回&イベント処理 \newline
     イベントモデルとイベント処理の例\\\hline
 第9回&PHP超入門\newline
       PHPに関する簡単なプログラム\\\hline
 第10回&サーバーとのデータの交換\newline
     PHP入門の続きとサーバーとのデータ交換の基礎\\\hline
 第11回&jQuery \newline
     DOMの処理を簡単にするライブラリーの紹介\\   \hline
 第12回&XMLファイルの処理\newline
     JavaScript と PHP によるXMLファイルの処理の例\\\hline
 第13回&クラスの例\\\hline
 第14回&システム開発のヒント\\\hline
 第15回&最終試験と解説\\ \hline
\end{tabular}
\end{center}
\end{frame}
\section{JavaScriptの実行方法}
\subsection{}
\begin{frame}
 \frametitle{JavaScriptの実行方法}
 最近のブラウザはJavaScriptの統合環境を提供している。
 \begin{itemize}
  \item Chromeのデベロッパーツール、FireFoxのWeb開発、Internet Explorer
        の開発者ツールなどはJavaScriptに
        おけるプログラミングにおいてデバッグなどの統合環境を提供
  \item これらのツールは「F12」
または「Control+Shift+I」というショートカットキーで表示、非表示ができる。
  \item このときに表示されるタブの内容は名前が
        異なっていても機能はほとんど同じ
 \end{itemize}
\end{frame}
\begin{frame}[containsverbatim]
\frametitle{Strict モードについて}
\begin{itemize}
 \item ECMAScript の最新版では\Strict と呼ばれる厳密な解釈をするモー
ドが導入
 \item このモードでは従来見つけにくい単純なバグがエラーとな
る
 \item プログラムを\Strict にするためにはプログラムの先頭に\Verb+"use strict";+
       を記述
 \item 関数の定義の直後に\Verb+"use strict";+を記述するとその関数内は
       \Strict になる。
\end{itemize}
\end{frame}
\begin{frame}[containsverbatim]
\frametitle{非\Strict と\Strict の主な違い}
  \begin{tabular}{|m{10zw}|c|c|}\hline
   &非\Strict & {\Strict}\\\hline
   変数の宣言&必要ではない&必要\\ \hline
   書き込み不可なプロパティへの代入&エラーが発生しない&エラーが発生\\
   \hline
   関数の\Verb+arguments+オブジェクトの値の変更&可能&不可能 \\ \hline
   関数の\newline\Verb+arguments.caller+&参照可能&エラーが発生 \\ \hline
   関数の\newline\Verb+arguments.callee+&参照可能&エラーが発生 \\ \hline
   8進リテラル(\Verb+0+で始まる数)&使用可能&エラーが発生 \\ \hline
 \end{tabular}
\end{frame}
\begin{frame}[containsverbatim]
\frametitle{JavaScriptの実行例}
\LISTN{01example.html}{1}{last}{\footnotesize}
\end{frame}
\begin{frame}[containsverbatim]
\frametitle{JavaScriptの実行例--解説(1)}
\begin{itemize}
 \item 1行目 HTML5におけるHTML文書の宣言
 \item 4行目 このファイルの文字コード(エンコーディング)を UTF-8 に指定。
 \item 6行目 スクリプトの開始の要素。プログラミング言語がECMAScriptであ
       ることを宣言している。
 \item 7行目 \texttt{//}は行末までの部分をコメントにするJavaScriptの記法。
残りの部分はこれ以降12行目までは通常の文字として解釈することを指定
       (\texttt{CDATA}セクションの開始)。

      7行目と13行目を消去したらどうなるのか確認すること
またその理由も考えること\footnote{最近のブラウザではエラーが起きないかも
       しれない。}。
\end{itemize}
 \end{frame}
\begin{frame}[containsverbatim]
\frametitle{JavaScriptの実行例--解説(2)}
\begin{itemize}
\item 8行目は関数\Verb+foo()+の宣言。13行目までがこの関数の定義範囲
 \item 9行目は変数 \texttt{i}の宣言
 \item 10行目 C言語などでおなじみの繰り返しの指定
 \item 11行目 引数内の式をコンソールに出力
 \item 14行目はメッセージボックスにコンソールを開くことを指示。
\end{itemize}
\end{frame}
\section{レポートの形式について}\subsection{}
\begin{frame}[containsverbatim]
 \frametitle{レポートの形式}
 \begin{itemize}
  \item 一番上に復習問題の用紙を置き、全体をステープラで止める。表紙は不
        要。
  \item 問題の結果をそのまま手書きでもよい。
  \item キャプチャや画像を印刷したときは、関係する説明文や考察も同じ用紙
        に記述すること。
  \item ブラウザを全画面表示してものをキャプチャしないこと。アクティブウィ
        ンドウだけのキャプチャすること。
  \item 必要な結果が入る範囲でウィンドウの表示のサイズを決めること。また、
        キャプチャ内の文字が読める程度にすること。
  \item ルーブリック評価は点数も含めて必ず自己評価をすること。採点結果と
        の差がなくなるようにするためである。
 \end{itemize}
\end{frame}
\end{document}

