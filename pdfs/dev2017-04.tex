\input devHeadMalmoe.tex
%\input devHeadGoettingen.tex
\title{ソフトウェア開発\\第4回目授業}
\author{平野 照比古}
\institute{}
\date{2017/10/13}
\begin{document}
\frame{\maketitle}
%\frame{\tableofcontents}
\section{オブジェクト}
\subsection{配列とオブジェクト}
\begin{frame}[containsverbatim]
\frametitle{配列とオブジェクトの特徴}
配列
\begin{itemize}
 \item 配列はいくつかのデータをまとめて一つの変数に格納
 \item 各データを利用
するためには \Verb+foo[1]+ のように数による添え字を使用
\end{itemize}
オブジェクト
\begin{itemize}
 \item オブジェクトでは添え字に任意の文字列を使うことができる
\end{itemize}
\end{frame}
\begin{frame}[containsverbatim]
\frametitle{オブジェクトの例}
\begin{Verbatim}
let person = {
  name : "foo",
  birthday :{
    year : 2001,
    month : 4,
    day : 1
  },
  "hometown" : "神奈川",
}
\end{Verbatim}
この形式で表したデータをオブジェクトリテラルという。
\end{frame}
\begin{frame}[containsverbatim]
\frametitle{コードの解説}
\begin{itemize}
 \item オブジェクトは全体を \Verb+{}+ で囲む。
 \item 各要素はキーと値の組で表される。両者の間は \Verb+:+ で区切る。
 \item キーは任意の文字列でよい。キー全体を \Verb+""+ で囲わなくてもよい。
 \item 値はJavaScriptで取り扱えるデータなあらば何でもよい。上の例ではキー
       \Verb+birthday+ の値がまたオブジェクトとなっている。
 \item 各要素の値を取り出す方法は2通りある。

一つは\Verb+.+演算子を用いてオブジェクトのキーをそのあとに書く。もう一つ
       は配列と同様に\Verb+[]+内にキーを文字列として指定する方法である。
\end{itemize}
\end{frame}
\begin{frame}[containsverbatim]
\frametitle{データへのアクセスの例}
\begin{Verbatim}
>person.name;
"foo"
>person["name"];
"foo"
\end{Verbatim}
\end{frame}
\begin{frame}[containsverbatim]
\frametitle{データへのアクセスの例--解説}
\begin{itemize}
 \item 
オブジェクトの中にあるキーをすべて網羅するようなループを書く場合や変数名
       として利用できないキーを参照する場合には後者
       の方法が利用される。
 \item キーの値が再びオブジェクトであれば、前と同様の方法で値を取り出せ
       る。
\begin{Verbatim}
>person.birthday;
Object {year: 2001, month: 4, day: 1}
>person.birthday.year;
2001
\end{Verbatim}
\end{itemize}
\end{frame}
\begin{frame}[containsverbatim]
\frametitle{データへのアクセスの例--解説}
\begin{itemize}
 \item 取り出し方は混在してもよい。
\begin{Verbatim}
>person.birthday["year"];
2001
\end{Verbatim}
 \item キーの値は代入して変更できる。
\begin{Verbatim}
>person.hometown;
"神奈川"
>person.hometown="北海道";
"北海道"
>person.hometown;
"北海道"
\end{Verbatim}
\end{itemize}
\end{frame}
\begin{frame}[containsverbatim]
\frametitle{データへのアクセスの例--解説}
\begin{itemize}
 \item 存在しないキーを指定すると値として\Verb+undefined+が
       返る。
\begin{Verbatim}
>person.mother;
undefined
\end{Verbatim}
 \item 存在しないキーに値を代入すると、キーが自動で生成される。
\begin{Verbatim}
>person.mother = "aaa";
"aaa"
>person.mother;
"aaa"
\end{Verbatim}
\end{itemize}
\end{frame}
\begin{frame}[containsverbatim]
\frametitle{オブジェクトのキーをすべて渡るループ}
\begin{itemize}
 \item オブジェクトのキーをすべて渡るループは \Verb+for-in+で実現できる。
\begin{itemize}
 \item \Verb+for( v in obj)+ の形で使用する。変数 \Verb+v+ はループ内で
       キーの値が代入される変数、\Verb+obj+ はキーが走査されるオブジェク
       トである。
 \item キーの値は \Verb+obj[v]+ で得られる。
\end{itemize}
\begin{Verbatim}
>for(i in person) { console.log(i+" "+person[i]);};
name foo
birthday [object Object]
hometown 北海道
mother aaa
undefined
\end{Verbatim}
最後の\Verb+undefined+は\Verb+for+ループの戻り値である。
\end{itemize}
\end{frame}
\section{オブジェクトリテラルとJSON}
\begin{frame}[containsverbatim]
 \frametitle{JSON}
 \begin{itemize}
  \item JSON(JavaScript Object Notation)はデータ交換のための軽量なフォー
        マット(文字列)
  \item 形式はJavaScriptのオブジェクトリテラルの記述法と全く同じ
 \item 正しく書かれたJSONフォーマットの文字列をブラウザとサーバーの間で
       データ交換の手段として利用
 \item JSONフォーマットの文字列をJavaScriptのオブジェク
       トに変換可能
 \item JavaScript内のオブジェクトをJSON形式の文字列に変換可能
 \end{itemize}
\end{frame}
\begin{frame}[containsverbatim]
 \frametitle{JSONオブジェクト}
JSONオブジェクトはJavaScriptのオブジェクトとJSONフォーマットの文字列の相
 互変換の手段を提供

次の例は2つの同じ形式からなるオブジェクトを通常の配列に入れたものを定義
\begin{Verbatim}
let persons = [{
  name : "foo",
  birthday :{ year : 2001, month : 4, day : 1},
  "hometown" : "神奈川",
},
{
  name : "Foo",
  birthday :{ year : 2010, month : 5, day : 5},
  "hometown" : "北海道",
}];
\end{Verbatim}
 \end{frame}
\begin{frame}[containsverbatim]
 \frametitle{JSONオブジェクトの利用例}
 次の例はこのオブジェクトを \texttt{JSON}に処理させたものである。
 {\scriptsize
\begin{Verbatim}
>s = JSON.stringify(persons);
"[{"name":"foo","birthday":{"year":2001,"month":4,"day":1},
"hometown":"神奈川"},
{"name":"Foo","birthday":{"year":2010,"month":5,"day":5},
"hometown":"北海道"}]"
>s2 = JSON.stringify(persons,["name","hometown"]);
"[{"name":"foo","hometown":"神奈川"},{"name":"Foo","hometown":"北海道"}]"
>o = JSON.parse(s2);
[Object, Object]
>o[0];
Object {name: "foo", hometown: "神奈川"}
\end{Verbatim}
 }
\end{frame}
\begin{frame}[containsverbatim]
 \frametitle{JSONオブジェクトの利用例(解説)}
\begin{itemize}
 \item JavaScriptのオブジェクトを文字列に変更する方法は
       \texttt{JSON.stringify()}を用いる。
 \item \texttt{JSON.stringigy()}の二つ目の引数として対象のオブジェクトの
       キーの配列を与えることができる。このときは、指定されたキーのみが
       文字列に変換
 \item ここでは、\verb+"name"+ と \verb+"hometown"+が指定されているので
\verb+"birthday"+のデータは変換されていない。
 \item JSONデータをJavaScriptのオブジェクトに変換するための方法は
       \texttt{JSON.parse()}を用いる。
 \item ここではオブジェクトの配列に変換されたことがわかる。
 \item 各配列の要素が正しく変換されていることがわかる。
\end{itemize}
\end{frame}
\section{クラス}
\subsection{クラスの宣言とインスタンスの生成}
\begin{frame}[containsverbatim]
 \frametitle{クラスの必要性}
 \begin{itemize}
  \item 同じキーを持つオブジェクトを複数作成したい場合に同じようなコードを
繰り返す必要がある。
  \item キーの追加をするときにはそれぞれのオブジェクトに対して修正が必要
        でプログラムのメンテナンスが面倒
 \end{itemize}
 オブジェクトのひな形を作成し、それをもとにオブジェクトを構成する。
 \begin{itemize}
  \item オブジェクトのひな形は通常クラスと呼ばれる。
  \item \ES ではクラスの宣言ができる。
  \item クラスから作成されたオブジェクトはこのクラスのインスタンスと呼ばれる。
 \end{itemize}
\end{frame}
\begin{frame}[containsverbatim]
 \frametitle{クラスの例}
 前の例をクラスを用いて書き直す。
 \begin{Verbatim}[numbers=left]
class Person{
  constructor(name, year, month, day, hometown="神奈川"){
    this.name = name;
    this.birthday = {
      year : year,
      month : month,
      day : day
    };
    this["hometown"] = hometown;
  }
\end{Verbatim}
\end{frame}
\begin{frame}[containsverbatim]
 \frametitle{クラスの例--解説}
\begin{itemize}
 \item 1行目がクラスの宣言
       \begin{itemize}
        \item キーワード\ElmJ{class}の後にクラス名
       \texttt{Person}を記述
        \item 通常、クラス名は大文字で始まる。
        \item 関数の時と同様にクラスの宣言を変数に代入することもできる。
       \end{itemize}
 \item その後の\Verb+{+と\Verb+}+内にクラスの記述を行う。
 \item クラスの定義には初期化を行う\ElmJ{constructor()}が必須
       \begin{itemize}
        \item こ
       こでは5つの引数を取るコンストラクタが定義
        \item 番目の引数
       はデフォルトの値が指定
       \end{itemize}
 \item キーワード\ElmJ{this}はクラスから作成されたインスタンスを指す。
 \item 3行目から9行目で、そのインスタンスのオブジェクトのキーはプロパティ
			 と呼ばれる。プロパティの名前は前のオブジェクトと同じである。
 \item 9行目の定義はコンストラクタの最後の引数はデフォルトの値が与えられ
			 ている。
 \item クラス内のコードは\Strict で実行
\end{itemize}
\end{frame}
\begin{frame}[containsverbatim]
 \frametitle{クラスの例--実行例}
 \begin{itemize}
  \item クラスの定義からインスタンスを作成するためにはキーワード
        \ElmJ{new}をつけてクラスを呼び出す。
        {\small
\begin{Verbatim}
>p = new Person("foo",2001,8,18);
Person {name: "foo", birthday: {…}, hometown: "神奈川"}
>p.birthday
{year: 2001, month: 8, day: 18}
>p.hometown;
"神奈川"
\end{Verbatim}
        }
  \item コンストラクタの最後の引数にはデフォルト値が指定されているので、この実行例の
ように値が指定されていない場合にはデフォルト値に設定されている
 \end{itemize}
\end{frame}
\subsection{クラスメソッド}
\begin{frame}[containsverbatim]
 \frametitle{クラスメソッド}
\begin{itemize}
 \item クラス内では\ElmJ{constructor}のほかにメソッドと呼ばれる関数で定義された
プロパティが定義可能
 \item メソッドにはアクセッサプロパティと呼ばれるオブジェ
クトに値を渡すセッター、値を得るゲッターの2種類を指定することも可能
 \item \ES では
ゲッターやセッターにはキーワード\ElmJ{get}と\ElmJ{set}を用いる。
\end{itemize} 
 \end{frame}
\begin{frame}[containsverbatim]
 \frametitle{メソッド定義の例}
\texttt{Person}に文字列に変換する\ElmJ{toString()}と、実行時に
 おける年令を返すゲッター\texttt{age}を追加したもの
 {\scriptsize
\begin{Verbatim}[numbers=left]
class Person{
  constructor(name, year, month, day, hometown = "神奈川"){
.... //以前と同じなので省略
  }
  toString() {
    return `${this.name}さんは`+
      `${this.birthday.year}年${this.birthday.month}月${this.birthday.day}日に` +
     `${this.hometown}で生まれました。`;
  }
  get age() {
    let today = new Date();
    let age = today.getFullYear() - this.birthday.year;
    if(today.getTime() <
         new Date(today.getFullYear(),
                  this.birthday.month-1,
                  this.birthday.day).getTime()) age--;
    return age;
  }
}
\end{Verbatim}
 }
\end{frame}
\begin{frame}[containsverbatim]
 \frametitle{メソッド定義の例--解説(1)}
 \begin{itemize}
  \item 11行目から15行目でメソッド\ElmJ{toString()}が定義
        \begin{itemize}
         \item \ElmJ{toString()}メソッドはすべてのオブジェクトに定義されていて、
 文字列が必要な時に呼び出される。
         \item ここでは「~さんは~年~月~日に~
               で生まれました。」という文字列を返す。
        \end{itemize}
 \end{itemize}
\begin{Verbatim}
p.toString();//toString()の明示的呼び出し
"fooさんは2001年4月1日に}神奈川で生まれました。"
`${p}`;      //暗黙のtoString()呼び出し
"fooさんは2001年4月1日に}神奈川で生まれました。"
\end{Verbatim}
 \end{frame}
\begin{frame}[containsverbatim]
 \frametitle{メソッド定義の例--解説(2)}
 16行目から24行目でゲッター\texttt{age}が定義
\begin{itemize}
 \item キーワード\ElmJ{get}をつけて、ゲッター\texttt{age()}を宣
       言
 \item 関数なので\texttt{()}をつける必要がある。
 \item ゲッターに仮引数をつけることはできない。
 \item 17行目でアクセス時の時間を変数\texttt{today}に保存
 \item 18行目でアクセス時の年から誕生日の年を引く。
 \item 19行目から22行目で、今年の誕生日が過ぎているかどうかをチェック
 \item アクセス時の年と誕生日の月を日をもとに日付を作成し、その
       時間(\ElmJ{getTime()}を利用)を比較
 \item 誕生日前ならば18行目で求めた値を$1$減少
\end{itemize}
\end{frame}
\begin{frame}[containsverbatim]
 \frametitle{}
\begin{Verbatim}
>p.age;
16
>p.age = 50;
50
>p.age;
16
>p.age(10);
VM87:1 Uncaught TypeError: p.age is not a function
    at <anonymous>:1:3
\end{Verbatim}
\begin{itemize}
 \item 定義の方はにはメソッドであることを示す\texttt{()}があるが、利用す
       るときはプロパティと同じようになる。\texttt{()}を付けるとエラーに
       なる。
 \item 代入の式はエラーがなく実行できるが、ゲッターの機能は
       変わらない。代入はセッターの呼び出しが行われる。
\end{itemize}
 \end{frame}
\subsection{継承}
\begin{frame}[containsverbatim]
 \frametitle{クラスの継承}
 \begin{itemize}
  \item 継承とはあるクラスをもとに機能の追加や変更を加えた新しいクラスを
        作ること
  \item 新しいクラスはもとになるクラスを継承しているという。
  \item 新しいクラスはもとになるクラスのサブクラス
  \item もとのクラスは新しいクラスのスーパークラス
  \item JavaScript では複数のクラスから同時に継承する多重継承はサポー
トされていない
 \end{itemize}
 \end{frame}
\begin{frame}[containsverbatim]
 \frametitle{継承の例}
  次の例はクラス\texttt{Person}を継承して
  新しいクラス\texttt{Student}を作成するものである。
 {\small
\begin{Verbatim}[numbers=left]
class Student extends Person {
  constructor(name, id, year, month, day, hometown) {
    super(name, year, month, day, hometown);
    this.id = id;
  }
}
\end{Verbatim}
 }
  \begin{itemize}
   \item クラスを継承するためにはクラス名の後に、キーワード
         \ElmJ{extends}を付けて継承するクラス名を書く(1行目)。
   \item \texttt{Student}クラスのコンストラクタには\texttt{Person}クラス
         で必要なパラメタのほかに\texttt{id}が付け加えている(2行目)。
   \item 親クラス(スーパークラス)のコンストラクタを呼び出すために
         \ElmJ{super()}を実行する(3行目)
   \item 4行目で追加のプロパティの設定を行っている。
  \end{itemize}
 \end{frame}
\begin{frame}[containsverbatim]
 \frametitle{継承の例--実行例}
 {\scriptsize
\begin{Verbatim}
>s = new Student("foo",1523999,2001,4,1);//Person のデフォルト引数値が設定
Student {name: "foo", birthday: {…}, hometown: "神奈川", id: 1523999}
> `${s}` //toString()もPersonで定義されたものを使用される
"fooさんは2001年4月1日に神奈川で生まれました。"
>s2 = new Student("foo",1523999,2001,4,1,"厚木");//デフォルト以外の設定もできる
Student {name: "foo", birthday: {…}, hometown: "厚木", id: 1523999}
>`${s2}`;
"fooさんは2001年4月1日に厚木で生まれました。"
\end{Verbatim}
 }
 \end{frame}
\subsection{\protect\texttt{instanceof}演算子}
\begin{frame}[containsverbatim]
 \frametitle{\protect\texttt{instanceof}演算子}
\texttt{instanceof}演算子はオブジェクトを生成したコンストラクタ関数が指
定されたものかを判定する。
\begin{Verbatim}
>p instanceof Person
true
>p instanceof Object;
true
>p instanceof Date;
false
\end{Verbatim}
\Verb+Person+ オブジェクトが \verb+Object+ を継承している
 ので\texttt{p instanceof Object} が\texttt{true}となっている。
 \end{frame}
\subsection{静的メソッド}
\begin{frame}[containsverbatim]
 \frametitle{}
 \begin{itemize}
  \item クラスに対して、インスタンスを作成しないで使用できる静的メソッドを定義で
きる。
  \item 静的メソッドはキーワード\ElmJ{static}を付ける。
  \item インスタンス化されたオブジェクトからは使用不可
 \end{itemize}
 \ElmJ{Math}オブジェクトにある各関数が\ElmJ{Math}オブジェクトの静的メソッ
ドの例
 \end{frame}
\begin{frame}[containsverbatim]
 \frametitle{クラスメソッドを用いて連続した\protect\texttt{id}を作成}
クラス\texttt{Student}のプロパティ\texttt{id}を重複がなくかつ連続な値に
自動で設定しようとするとクラスにおいて変数を管理する変数(クラス変数)が必
 要
\begin{Verbatim}[numbers=left]
class Student extends Person {
  static getNextId(){
    return Student.nextId++;
  }
  constructor(name, year, month, day, hometown) {
    super(name, year, month, day, hometown);
    this.id = Student.getNextId();
  }
}
Student.nextId = 10000;
\end{Verbatim}
 \end{frame}
\begin{frame}[containsverbatim]
 \frametitle{クラスメソッドを用いて連続した\protect\texttt{id}を作成--
 解説}
\begin{itemize}
 \item 2行目から4行目で次の\texttt{id}を求めるクラスメソッドを定義してい
       る。
 \item ここではクラス変数\texttt{nextId}を$1$増加させている(3行目)
 \item \texttt{nextId}の初期化は10行目で行っている。
 \item 初期化の方法からも推察されるように、この値は途中で変更が可能
\end{itemize}
 \end{frame}
\begin{frame}[containsverbatim]
 \frametitle{クラスメソッドを用いて連続した\protect\texttt{id}を作成--
 実行例}
 {\scriptsize
\begin{Verbatim}
>s1 = new Student("foo1",2001,4,1);
Student {name: "foo", birthday: {…}, hometown: "神奈川", id: 10000}
>s2 = new Student("foo2",2001,5,1);
Student {name: "foo", birthday: {…}, hometown: "神奈川", id: 10001}
Student.nextId = -100;
-100
>s3 = new Student("foo2",2001,6,1);
Student {name: "foo", birthday: {…}, hometown: "神奈川", id: -100}
\end{Verbatim}
 }
 \begin{itemize}
  \item 3つ目のインスタンスを作成する前にこの値を上書きしている
  \item 最後のインスタンスは前の2つと異なるものに設定
  \item この欠点を解消するにはクラスメソッドで定義する代わりにクラスをク
        ロージャの中で定義して、その中に変数を置く。
 \end{itemize}
 \end{frame}
\begin{frame}[containsverbatim]
 \frametitle{クラス式のクロージャ}
次の例は即時実行関数内に\texttt{id}を管理する変数を用意して、その関数が
 クラス式を返すようにした。
 {\scriptsize
\begin{Verbatim}[numbers=left]
const Student = (function(){
  let id = 10000;
  return class extends Person {
    constructor(name, year, month, day, hometown) {
      super(name, year, month, day, hometown);
      this.id = id++;
    }
  }
})();
\end{Verbatim}
 }
\end{frame}
\begin{frame}[containsverbatim]
 \frametitle{クラス式のクロージャ--解説}
 \begin{itemize}
  \item 即時実行関数の戻り値はクラス式であり、それを変数\texttt{Student}
        に代入している。同じ名前のクラスが再定義されないようにするため変
        数を\ElmJ{const}で宣言している。
  \item 2行目で次に設定する\texttt{id}を保存する変数を初期化している。
  \item 関数の戻り値としてクラス式を返す。クラスの定義は以前のものとほと
        んど同じである。
  \item 違いはインスタンスに設定した後で\texttt{id}を管理する変数を
        増加させているところ(6行目の右辺)。
 \end{itemize}
 \end{frame}
\begin{frame}[containsverbatim]
 \frametitle{クラス式のクロージャ--実行例}
 {\scriptsize
\begin{Verbatim}
>s = new Student("foo",2001,4,1);  //初めのインスタンスの id は 10000
{name: "foo", birthday: {…}, hometown: "神奈川", id: 10000}
>`${s}`;
"fooさんは2001年4月1日に}神奈川で生まれました。"
>s2 = new Student("foo",2001,4,1,"厚木");  //次のインスタンスの id は 10001
{name: "foo", birthday: {…}, hometown: "厚木", id: 10001}
>`${s2}`;
"fooさんは2001年4月1日に}厚木で生まれました。"
\end{Verbatim}
 }
 \end{frame}
\end{document}
\begin{frame}[containsverbatim]
 \frametitle{}
 \end{frame}
