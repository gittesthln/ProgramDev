%-*- coding: utf-8 -*-
\documentclass[dvipsk]{beamer}
\usepackage{pgfpages}
\usepackage{moreverb,array}
%\usetheme{Malmoe}
%\usetheme{Goettingen} %7 8 9
%\usetheme{Darmstadt}% 1, 2, 3
%\usetheme{Boadilla}
%\usetheme{CambridgeUS}%4, 5 6 
%
\usetheme{AnnArbor} %10
%\usetheme{Marburg}
\iffalse\else
\renewcommand{\familydefault}{\sfdefault}
\renewcommand{\kanjifamilydefault}{\gtdefault}
\setbeamerfont{title}{size=\Large,series=\bfseries}%,color=white}
\setbeamerfont{frametitle}{size=\large,series=\bfseries}
\setbeamerfont{beamergotobutton}{size=\Large}
%\setbeamercolor{frametitle}{fg=yellow}
\setbeamertemplate{frametitle}[default][center]
%\addtobeamertemplate{footline}{}{\insertframenumber/\inserttotalframenumber}
\usefonttheme{professionalfonts}
\fi
%\iftrue
\title{ソフトウェア開発\\第12回目授業}
\author{平野 照比古}
\institute{}
\date{2015/12/15}
\newtheorem{Prob}{解説}
\newcommand{\Elm}[1]{\texttt{<#1>}}
\setbeamercovered{transparent}

\newcommand{\DOMM}{\texttt}
\newcommand{\Event}{\texttt}
\newcommand{\DOMP}{\texttt}
\newcommand{\DOM}{\texttt{DOM}}
\newcommand{\keyitem}{\relax}
\newcommand{\HTML}{HTML文書}
\begin{document}
\frame{\maketitle}
\section{サーバーとのデータ-の交換}
\subsection{サーバーとのデータのやり取り}
\begin{frame}[containsverbatim]
\frametitle{サーバーとのデータ交換の基本}
Webページにおいてサーバーにデータを送る方法には\texttt{POST}と
\texttt{PUT}の2通りの方法がある。
\end{frame}
\begin{frame}[containsverbatim]
\frametitle{\texttt{POST}による送信}
\texttt{windows.onload =function()}内に次のコードを追加する。
\begin{verbatim}
    var Form = document.getElementsByTagName("form")[0];
    Form.setAttribute("method","POST");
    Form.setAttribute("action","09sendData.php");
\end{verbatim}
HTMLの要素に対しては次のことを行う。
\begin{itemize}
 \item \texttt{<select>}要素の属性に\verb+name="select"+ を追加する。
 \item \texttt{id}が\verb+"colorName"+であるテキストボックスに
       \verb+name="colorName"+ を追加する。
 \item 「設定」ボタンの要素の後に次の要素を追加する。
\begin{center}
\verb+<input type="submit" value="送信" id="Send"></input>+ 
\end{center}
\end{itemize}
一時期は\texttt{name}属性を指定しておくと、\texttt{id}属性を兼ねていた時
 期もあったが、最近では両者は厳密に区別されている。
\end{frame}
\begin{frame}[containsverbatim]
\frametitle{\texttt{POST}による送信(解説)}
\begin{itemize}
 \item このページでは「送信」ボタンを押すと\texttt{<form>}の
\texttt{action}属性で指定されたプログラムが呼び出される。
\item ここではWebペー ジと同じ場所にある\texttt{09sendData.php}が呼び出
      される。
\end{itemize}
\end{frame}
\begin{frame}[containsverbatim]
\frametitle{サーバープログラムのリスト}
{\scriptsize
\begin{listing}{1}
 <?php
print <<<_EOL_
<!DOCTYPE html>
<head>
<meta charset="UTF-8"/>
<title>サーバーに送られたデータ</title>
</head>
<body>
<table>
_EOL_;
foreach($_POST as $key=>$value) {
  print "<tr><td>$key</td><td>$value</td></tr>\n";
}
print <<<_EOL_
</table>
</body>
</html>
_EOL_;
?>
\end{listing}
}
\end{frame}
\begin{frame}[containsverbatim]
\frametitle{サーバープログラムのリスト(解説)}
\begin{itemize}
 \item 2行目から10行目の間はヒアドキュメント形式でHTML文書の初めの部分を
       出力させている。
 \item \verb+method="POST"+で呼び出されたときには\texttt{form}要素内の
       \texttt{name}属性が指定されたものの値がスーパーグローバル
       \verb+$_POST+内の連想配列としてアクセスができる。
 \item 11行目から13行目でそれらの値を\texttt{table}要素内の要素として出
       力している。
\end{itemize}
\end{frame}
\begin{frame}[containsverbatim]
\frametitle{サーバープログラムのリスト(ソース)}
\begin{verbatim}
<!DOCTYPE html>
<head>
<meta charset="UTF-8"/>
<title>サーバーに送られたデータ</title>
</head>
<body>
<table><tr><td>select</td><td>yellow</td></tr>
<tr><td>color</td><td>green</td></tr>
<tr><td>colorName</td><td>gray</td></tr>
</table>
</body>
</html>
\end{verbatim}
\end{frame}
\begin{frame}[containsverbatim]
\frametitle{\texttt{GET}による通信}
\begin{itemize}
 \item \verb+method="PUT"+で呼び出した場合にはスーパーグローバル
\verb+$_GET+を用いる。
 \item スーパーグローバル\verb+$_REQUEST+は
\verb+method="POST"+でも\verb+method="PUT"+で呼び出された場合の
\verb+$_POST+や\verb+$_GET+の代わりに使用できる。
\end{itemize}
\end{frame}
\begin{frame}[containsverbatim]
\frametitle{通信に関する注意}
\begin{itemize}
 \item \verb+type="submit"+の\texttt{input}要素は、ボタンが押されたときに直ちに、
\texttt{action}属性で指定された処理が呼び出される。
 \item サーバーにデータを送る前に最低限のエラーチェックを行い、エラーが
       ない場合にだけサーバーと通信するのが良い。
\end{itemize}
\end{frame}
\begin{frame}[containsverbatim]
\frametitle{スパーグローバルの補足}
\begin{itemize}
 \item {\texttt{\$\_SERVER}}\\
サーバーにアクセスしたときのクライ
アントの情報などを提供。具体的な内容はクライアントごとに異なる。
 \item \texttt{\$\_COOKIE}\\
COOKIE とはWebサーバー側からクライアント側に一時的にデータを保存させる仕
組み。すでに訪問したことがあるサイトに対して情報を開始
時に補填する機能などを実現できる。
 \item \texttt{\$\_SESSION}\\
セッションとはある作業の一連の流れを指す。たとえば会員制のサイトではログイ
ン後でなければページを見ることができない。情報のページに直接行くことがで
       きないような仕組みが必要

  セッションの間で受け渡すデータはこの変数の連想配列内に格納する。
\end{itemize}
\end{frame}
\begin{frame}[containsverbatim]
\frametitle{セッション}
\begin{itemize}
 \item HTTP通信はセッションレスな通信である(各ページが独立して存在し、ページ間
のデータを直接渡せない)
 \item セッションを確立するためには、クライアント側から情報を送り、それに基づい
てサーバー側が状況を判断するなどの操作を意識的にする必要
 \item PHPではセッションを開始するための関数\texttt{session\_start()}とセッショ
ンを終了させる\texttt{session\_destroy()}が用意されている。
 \item セッションを通じで保存させておきたい情報はこの連想配列に保存
 \item セッションの管理はサーバーが管理
 \item この機能はCOOKIEの機能を利用して実現
\end{itemize}
\end{frame}
 \iffalse
\begin{frame}[containsverbatim]
 \frametitle{Web Storage}
\begin{itemize}
 \item localStoarage と sessionStorage の2種類
 \item localStorage は文字列をキーに、文字列の値を持つStorageオブジェク
ト
 \item 同一の出身(プロトコルやポート番号も含む)のすべてのドキュメント
がおなじlocalStorageを共有
 \item このデータは意識的に消さない限り存在
 \item sessionStorageはウインドウやブラウザが閉じられると消滅
 \item セッション間の情報の移動を可能にしている。
\end{itemize}
\end{frame}
\begin{frame}[containsverbatim]
\frametitle{Web Storageの補足}
\begin{itemize}
 \item いくつかのサイトではこの機能を用いており、その開発者ツールで見る
       ことが可能
 \item データの形式は文字列である。構造化されたデータはJSON形式で保存するのがよい
\end{itemize}
\end{frame}
\fi
\subsection{Ajax}
\begin{frame}[containsverbatim]
\frametitle{Ajax\footnote{Ajaxの名称を提案したブログ
 \texttt{http://www.adaptivepath.com/publications/essays/archives/000385.php} 
は2014年12月8日現在アクセスできないようである。} とは}
\begin{itemize}
 \item Asynchronous Javascript+XML の略
 \item 非同期(Asynchronous)でWeb
ページとサーバーでデータの交換を行い、クライアント側で得られたデータをも
とにそのWebページを書き直す手法
\end{itemize} 

\begin{itemize}
 \item Google Maps がこの技術を利用したことで一気に認知度が高まった。
 \item 検索サイト
では検索する用語の一部を入力していると検索用語の候補が出てくる。これも
Ajaxを使用している(と考えられる)。
\end{itemize}
\end{frame}
\begin{frame}[containsverbatim]
\frametitle{\texttt{XMLHTTPRequest}}
Ajax の機能はこのオブジェクトを用いて実現

\begin{itemize}
 \item 日付が変わったときに、その日の記念日を
 メニューの下部に示す
 \item 記念日のデータは
 \texttt{http://ja.wikipedia.org/wiki/日本の記念日一覧}の表示画面からコ
 ピーして作成
\end{itemize}
\end{frame}
\begin{frame}[containsverbatim]
\frametitle{リスト(1)}
\begin{listing}{46}
    var changePulldown = function(){
      var d2 = Form.children[2].value
      d = new Date(Year.value, Month.value, 0).getDate();
      if( d != Form.children[2].children.length) {
        Form.replaceChild(Days[d],Form.children[2]);
        d2 = Math.min(Form.children[2].length, d2);
        Form.children[2].value = d2;
      }
\end{listing} 
\end{frame}
\begin{frame}[containsverbatim]
\frametitle{リスト---概要}
\begin{itemize}
 \item 以前のものとは46行目以降が異なっている。イベントハンドラーを関数とし
       て定義している。
 \item 47行目から52行目は以前と同じプルダウンメニューの処理である。
\end{itemize}
\end{frame}
\begin{frame}[containsverbatim]
\frametitle{リスト(2)--- XMLHttpRequestオブジェクトの生成}
\begin{listingcont}
  var xmlHttpObj = null;
  if(window.XMLHttpRequest) {
    xmlHttpObj = new XMLHttpRequest();
  } else if(window.ActiveXObject ) {
    xmlHttpObj = new ActiveXObject("Msxml2.XMLHTTP");//IE6
  } else {
    try {
      xmlHttpObj = new ActiveXObject("Microsoft.XMLHTTP");//IE5
    } catch(e) {
    }
  }
\end{listingcont}
\end{frame}
\begin{frame}[containsverbatim]
\frametitle{\texttt{XMLHttpRequest}の作成}
53行目から63行目は裏でサーバーと通信をするための
\texttt{XMLHttpRequest}オブジェクトを作成している。
\begin{itemize}
 \item 古いバージョンのIEは別のオブジェクトで通信をするので、ブラウザが
       \texttt{XMLHttpRequest}メソッドを持つか確認し(54行目)、持っている
       場合はそのオブジェクトを新規作成する(55行目)。
 \item 56行目から62行目は古いIEのためのコードである。
 \item このようにブラウザの機能の違いで処理を変えることをクロスブラウザ
       対策という。通常はブラウザがその機能を持つかどうかで判断する。
\end{itemize}
\end{frame}
\begin{frame}[containsverbatim]
\frametitle{リスト(3)--- コールバック関数の登録}
{\scriptsize
\begin{listingcont}
  if(xmlHttpObj) {
    xmlHttpObj.onreadystatechange = function(){
      if(xmlHttpObj.readyState == 4 && xmlHttpObj.status == 200) {
        document.getElementById("details").firstChild.nodeValue = 
            xmlHttpObj.responseText;
      } 
    }
    xmlHttpObj.open("GET",
      "./aniversary.php?month=" + Month.value+ "&day="+d2,true);
    xmlHttpObj.send(null);
  }
 }
 Form.addEventListener("change", changePulldown,false);
 changePulldown();
}
//]]>
</script>
</head>
\end{listingcont}
}
\end{frame}
\begin{frame}[containsverbatim]
\frametitle{イベントハンドラーの説明}
\texttt{XMLHttRequest}が生成できたら(65行目)、このオブジェクトが
       生成する\texttt{onreadystatechange}イベントのイベントハンドラーを
       登録する(66行目から71行目)。
 \begin{itemize}
  \item \texttt{XMLHttRequest}の\texttt{readyState}は通信の状態を表す。
	$4$ は通信終了を意味する。
  \item 通信が終了しても正しくデータが得られたかを調べる必要がある。
	$200$ は正しくデータが得られたことを意味する。
  \item 得られたデータは\texttt{responseText}で得られる。この場合、得ら
	れたデータは文字列となる。このほかに\texttt{responXML}でXMLデー
	タが得られる。
 \end{itemize}
%\end{frame}
%\begin{frame}[containsverbatim]
\frametitle{通信の開始}
\begin{itemize}
 \item 72行目から73行目が通信の開始する。ここでは、\texttt{GET}で行うので、
       URLの後に必要なデータを付ける。
 \item \texttt{GET}では送るデータ本体がないので、通信の終了をのため
       \texttt{null}を送信する。\texttt{POST}のときはここでデータ本体を
       送る。
 \item プルダウンメニューが変化したときのイベントハンドラーを登録し(77行
       目)、最後に現在の日付データをサーバーに要求する。、
\end{itemize}
\end{frame}
\begin{frame}[containsverbatim]
\frametitle{リスト(4)---HTML文書}
\begin{listingcont}
<body>
  <form id="menu"></form>
  <p id="details"> </p>
</body>
</html>
\end{listingcont}
\end{frame}
\begin{frame}[containsverbatim]
\frametitle{送られてきたデータの処理}
\begin{itemize}
 \item 得られたデータは85行目の\texttt{p}要素の中に入れる(68行目から69行
       目)。
 \item この要素の\texttt{firstChild}を指定しているので85行目には
       \texttt{<p>}と\texttt{</p>}の間に空白を設けて、テキストノードが存
       在するようにしている。
\end{itemize}\end{frame}
\begin{frame}[containsverbatim]
\frametitle{Ajaxで呼び出されるPHPのプログラム}
{\small
\begin{listing}{1}
<?php
mb_internal_encoding("UTF8");
//print mb_internal_encoding();
$m = isset($_GET["month"])?$_GET["month"]: $argv[1];
$d = isset($_GET["day"])?$_GET["day"]:$argv[2];
$data = file("aniversary.txt",FILE_IGNORE_NEW_LINES);
for($i=0;$i<count($data);$i++) {
  $data[$i] = mb_convert_encoding($data[$i],"UTF8");
  $mm = mb_split("\[",$data[$i]);
  if(count($mm) >1) {
    if(mb_convert_encoding($m."月","UTF8") === $mm[0]) break;
  }
}
for($i++;$i<count($data);$i++) {
  $data[$i] = mb_convert_encoding($data[$i],"UTF8");
  $dd = mb_split("\s-\s",$data[$i]);
  if(($d."日") === $dd[0]) break;
}
//  print mb_convert_encoding($dd[1],"SJIS");
print $dd[1]; 
?>
\end{listing}
}
\end{frame}
\begin{frame}[containsverbatim]
\frametitle{Ajaxで呼び出されるPHPのプログラム--解説}
\begin{itemize}
 \item 2行目で内部で処理をするエンコーディングを\texttt{UTF8}にしている。
       関数、\texttt{mb\_internal\_encoding}関数を引数なしで呼び出すと
       現在採用されているエンコーディングを得ることができる。
 \item 4行目と5行目では月(\verb+$m+)と日(\verb+$d+)の値をそれぞれの変数
       に設定している。
\begin{itemize}
 \item ここではコマンドプロンプトからもデバッグで
       きるように、スーパーグローバル\verb+$_GET+内に値があれば
       (\verb+isset()+)が\verb+true+になれば、その値を、そうでなければコ
       マンドからの引数を設定している。
 \item スーパーグローバル\verb+$argv+はの先頭は呼び出したファイル名であ
       り、その後に引数が順に入る\footnote{C言語の\texttt{main}関数は通
       常、\texttt{int main(int argc, char* argv[])}と宣言される。%
       \texttt{argc}は\texttt{argv}の配列の大きさを表し、渡された引数の
       リストが\texttt{argv[]}に入っている。このとき、\texttt{argv[0]}は実行
       したときのファイル名が入る。}。
\end{itemize}
\end{itemize}
\end{frame}
\begin{frame}[containsverbatim]
\frametitle{\texttt{file()}関数}
指定されたファイルを行末文字で区切って配列として返す関数\\
\begin{itemize}
\item 引数にはURLも指定できる。
\item この関数は2番目の引数をとることができる。
\begin{center}
 \begin{tabular}{|c|c|}\hline
 \verb+FILE_USE_INCLUDE_PATH+ & \verb+include_path+ のファイルを探す\\\hline
 \verb+FILE_IGNORE_NEW_LINES+ & 配列の各要素の最後に改行文字を追加しない
      \\ \hline
  \verb+FILE_SKIP_EMPTY_LINES+&空行を読み飛ばす \\ \hline
 \end{tabular}
\end{center}
 \item \verb+file_get_contents()+はファイルの内容を一つの文字列として
       読み込む。Webページの解析にはこちらの関数を使うとよい。
\end{itemize}
\end{frame}
\begin{frame}[containsverbatim]
\frametitle{読み込むファイルの一部}
\begin{verbatim}
1月[編集]
1日 - 鉄腕アトムの日
2日 - 月ロケットの日
[中略]
31日 - 生命保険の日、愛妻家の日
2月[編集]
1日 - テレビ放送記念日、ニオイの日
2日 - 頭痛の日
[以下略]
\end{verbatim}
\end{frame}
\begin{frame}[containsverbatim]
\frametitle{}
\begin{itemize}
 \item 月の部分の後には\texttt{[}がある。
 \item 日の情報は\verb*+ - +で区切られている(\verb*+ +は空白を表す)。
 \item すべての日の情報が入っている。
\end{itemize}
\end{frame}
\begin{frame}[containsverbatim]
\frametitle{データの処理---月を探す}
\begin{itemize}
 \item 8行目で念のためコードを\texttt{UTF8}に変更
 \item 関数\verb+mb_split()+関数は第1引数に指定された文字列パターンで第2
       引数で指定された文字列を分割して配列として返す関数
 \item 分割を指定する文字列には正規表現が使えるので、文字\verb+[+で分割
       するために、\verb+"\["+としている(9行目)。
 \item 指定された文字列があれば配列の大きさが1より大きくなる。その行に対
       して求める月と一致しているか判定し、等しければループを抜ける(11行
       目)。
\end{itemize}
\end{frame}
\begin{frame}[containsverbatim]
\frametitle{}
\begin{itemize}
 \item 14行目から18行目までは指定された月での指定された日の情報を探して
       いる。日を決定する方法も月と同じである。文字列の分割は
       \verb+"\s-\s"+となっている\footnote{これは\texttt{"\textbackslash
       s"}ではうまく行かなかっ
       たためである。}。
 \item 20行目で得られた情報をストリームに出力している。
\end{itemize}
\end{frame}
\end{document}

\begin{frame}[containsverbatim]
\frametitle{}
\end{frame}
%-*- coding: utf-8 -*-

\listinginput{1}{DEV/09-01whatday.html}
次のリストはである。
\listinginput{1}{DEV/aniversary.php}
 \item 読み込むファイルの一部を次に記す。
\end{itemize}
 \item 7行目から13行目までは指定された月の行を見つける。
\end{itemize}
\end{Exec}
\begin{Prob}
上のHTMLのリストの85行目の\texttt{<p>}と\texttt{</p>}の空白をとり除い
 たら正しく動かなくなることを確認せよ。
\end{Prob}
なお、構造化されたデータとしてはXML形式でもよいが、より軽量なJSONで与え
ることも可能である。このときは、\texttt{JSON.parse()}でJavaScriptのオブ
ジェクトに直せばよい。