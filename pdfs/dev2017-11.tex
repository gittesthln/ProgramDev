\input devHead.tex
\SetTheme{AnnArbor} %11
\title{ソフトウェア開発\\第11回目授業}
\author{平野 照比古}
\institute{}
\date{2017/12/8}
\newtheorem{Prob}{解説}
\newcommand{\Elm}[1]{\texttt{<#1>}}
\setbeamercovered{transparent}

\newcommand{\DOMM}{\texttt}
\newcommand{\Event}{\texttt}
\newcommand{\DOMP}{\texttt}
\newcommand{\DOM}{\texttt{DOM}}
\newcommand{\keyitem}{\relax}
\newcommand{\HTML}{HTML文書}
\begin{document}
\frame{\maketitle}
\section{前回のレポートから}
 \begin{frame}[containsverbatim]
  \frametitle{問題1}
  \begin{itemize}
   \item \texttt{PUT}ではURLにパラメータが記述される
   \item 直接記述してもサーバーのプログラムが起動される
   \item したがって、元のページでデータのチェックをしても、呼び出しの方
         でもデータのチェックをする必要がある。
   \item \texttt{POST}では直接見えないが、呼び出し側で直接記述が可能
         (前回のAjax通信を\texttt{POST}で行うところを参照)
  \end{itemize}
 \end{frame}
 \begin{frame}[containsverbatim]
  \frametitle{問題2}
  \begin{itemize}
   \item ブラウザのバージョンで内容が異なる。
   \item 最近のブラウザはそれほど多くの情報を与えないようである。
   \item セキュリティに関する考察が欲しい。
  \end{itemize}
 \end{frame}
 \section{前回補足}
\begin{frame}[containsverbatim]
\frametitle{Ajaxで呼び出されるPHPのプログラム}
\LISTNN{aniversary.php}{1}{last}{\scriptsize}
\end{frame}
\begin{frame}[containsverbatim]
\frametitle{Ajaxで呼び出されるPHPのプログラム--解説}
\begin{itemize}
 \item 2行目で内部で処理をするエンコーディングを\texttt{UTF8}にしている。
       関数、\texttt{mb\_internal\_encoding}関数を引数なしで呼び出すと
       現在採用されているエンコーディングを得ることができる。
 \item 4行目と5行目では月(\verb+$m+)と日(\verb+$d+)の値をそれぞれの変数
       に設定している。
\begin{itemize}
 \item ここではコマンドプロンプトからもデバッグで
       きるように、スーパーグローバル\verb+$_GET+内に値があれば
       (\Verb+isset()+)が\Verb+true+になれば、その値を、そうでなければコ
       マンドからの引数を設定している。
 \item スーパーグローバル\verb+$argv+はの先頭は呼び出したファイル名であ
       り、その後に引数が順に入る\footnote{C言語の\texttt{main}関数は通
       常、\texttt{int main(int argc, char* argv[])}と宣言される。%
       \texttt{argc}は\texttt{argv}の配列の大きさを表し、渡された引数の
       リストが\texttt{argv[]}に入っている。このとき、\texttt{argv[0]}は実行
       したときのファイル名が入る。}。
\end{itemize}
\end{itemize}
\end{frame}
\begin{frame}[containsverbatim]
\frametitle{\texttt{file()}関数}
指定されたファイルを行末文字で区切って配列として返す関数\\
\begin{itemize}
\item 引数にはURLも指定できる。
\item この関数は2番目の引数をとることができる。
\begin{center}
 \begin{tabular}{|c|m{15zw}|}\hline
 \Verb+FILE_USE_INCLUDE_PATH+ & \Verb+include_path+ のファイルを探す\\\hline
 \Verb+FILE_IGNORE_NEW_LINES+ & 配列の各要素の最後に改行文字を追加しない
      \\ \hline
  \Verb+FILE_SKIP_EMPTY_LINES+&空行を読み飛ばす \\ \hline
 \end{tabular}
\end{center}
 \item \verb+file_get_contents()+はファイルの内容を一つの文字列として
       読み込む。Webページの解析にはこちらの関数を使うとよい。
\end{itemize}
\end{frame}
\begin{frame}[containsverbatim]
\frametitle{読み込むファイルの一部}
\begin{Verbatim}
1月[編集]
1日 - 鉄腕アトムの日
2日 - 月ロケットの日
[中略]
31日 - 生命保険の日、愛妻家の日
2月[編集]
1日 - テレビ放送記念日、ニオイの日
2日 - 頭痛の日
[以下略]
\end{Verbatim}
\end{frame}
\begin{frame}[containsverbatim]
\frametitle{}
\begin{itemize}
 \item 月の部分の後には\texttt{[}がある。
 \item 日の情報は\verb*+ - +で区切られている(\verb*+ +は空白を表す)。
 \item すべての日の情報が入っている。
\end{itemize}
\end{frame}
\begin{frame}[containsverbatim]
\frametitle{データの処理---月を探す}
\begin{itemize}
 \item 8行目で念のためコードを\texttt{UTF8}に変更
 \item 関数\Verb+mb_split()+関数は第1引数に指定された文字列パターンで第2
       引数で指定された文字列を分割して配列として返す関数
 \item 分割を指定する文字列には正規表現が使えるので、文字\Verb+[+で分割
       するために、\Verb+"\["+としている(9行目)。
 \item 指定された文字列があれば配列の大きさが1より大きくなる。その行に対
       して求める月と一致しているか判定し、等しければループを抜ける(11行
       目)。
\end{itemize}
\end{frame}
\begin{frame}[containsverbatim]
\frametitle{}
\begin{itemize}
 \item 14行目から18行目までは指定された月での指定された日の情報を探して
       いる。日を決定する方法も月と同じである。文字列の分割は
       \verb+"\s-\s"+となっている\footnote{これは\texttt{"\textbackslash
       s"}ではうまく行かなかっ
       たためである。}。
 \item 20行目で得られた情報をストリームに出力している。
\end{itemize}
\end{frame}
\section{jQuery}
\subsection{jQueryとは}
\begin{frame}[containsverbatim]
\frametitle{jQueryとは}
jQuery の開発元\footnote{\texttt{jQuery.com}参照日:2017/11/29}には次のよ
 うに書かれている。
\begin{quotation}
jQuery is a fast, small, and feature-rich JavaScript library. It makes
 things like HTML document traversal and manipulation, event handling,
 animation, and Ajax much simpler with an easy-to-use API that works
 across a multitude of browsers. With a combination of versatility and
 extensibility, jQuery has changed the way that millions of people write
 JavaScript.
\end{quotation}
\end{frame}
\begin{frame}[containsverbatim]
\frametitle{jQueryの利用法}
\begin{enumerate}
 \item jQuery は\href{jquery.com}{jquery.com}のページの右上にある「Download
jQuery」をクリックすることでダウンロードのページに行く
 \item 次の2つの最新版がダウンロード可能

     Download the compressed, production jQuery 3.2.1

    Download the uncompressed, development jQuery 3.2.1
 \item ファイルの違いはコメントなどがそのまま入っているもの(uncompressed)と、
コメントを取り除き、変数名などを短いものに置き換えて、ファイルサイズを小
さくしたもの(compressed)
 \item ダウンロードしたファイルを取り込む\Elm{script}要素を記述すれば利
       用できる。
 \item または、CDN(Contents Deliverly Network)を利用
\begin{Verbatim}[fontsize=\small]
<script
 src="https://code.jquery.com/jquery-3.2.1.min.js"
 integrity="sha256-hwg4gsxgFZhOsEEamdOYGBf13FyQuiTwlAQgxVSNgt4="
 crossorigin="anonymous"></script>
\end{Verbatim}
\end{enumerate}

\end{frame}
\newcounter{Func}
\newcommand{\FuncRef}[1]{%
\refstepcounter{Func}\label{#1}({\bfseries 機能\arabic{Func}})}
\section{jQueryの基本}
\subsection{jQuery()関数とjQueryオブジェクト}
\begin{frame}[containsverbatim]
\frametitle{jQuery オブジェクト}
\begin{itemize}
 \item jQueryライブラリーは\texttt{jQuery()}というグローバル関数を一つだ
       け定義
 \item この関数の短縮形として、\texttt{\$}というグローバル関数も定義
 \item jQuery() 関数はコンストラクタではない。
 \item jQuery関数は引数の与え方により動作が異なる
 \item 戻り値としてjQueryオブジェクトとと呼ばれるDOMの要素群を返す
 \item このオブジェクトには多数のメソッドが定義されていて、これらの要素
       群を操作可能
\end{itemize}
\end{frame}
\begin{frame}[containsverbatim]
\frametitle{jQuery関数の引数の型(1)}
jQuery() 関数の呼び出し方は引数の型により返される要素群が異なる。
\begin{itemize}
 \item 引数が(拡張された)CSSセレクタ形式の場合\FuncRef{CSSSelector}

\begin{itemize}
 \item CSSセレクタにマッチした要素群を返す。
 \item 省略可能な2番目の引数として要素やjQueryオブ
       ジェクトを指定した場合には、その要素の子要素からマッチしたものを
       返す。
 \item この形はすでに解説したDOMのメソッド
       \texttt{querySelectorAll()}と似ている。
\end{itemize}  
 \item 引数が要素やDocumentオブジェクトなどの場合\FuncRef{ObjSelector}

       与えられた要素をjQueryオブジェクトに変換する。
\end{itemize}
\end{frame}
\begin{frame}[containsverbatim]
\frametitle{jQuery関数の引数の型(2)}
\begin{itemize}
 \item 引数としてHTMLテキストを渡す場合\FuncRef{HTMLText}

\begin{itemize}
 \item テキストで表される要素を作成し、この要素を表すjQueryオブジェクト
       を返す。\texttt{createElement()}メソッドに相当する。
 \item 省略可能な2番目の引数は属性を定義するものであり、オブジェクトリテ
       ラルの形式で与える。
\end{itemize}       

 \item 引数として関数を渡す場合\FuncRef{Function}

       引数の関数はドキュメントがロードされ、DOMが操作で可能になった時に実行される。
\end{itemize}
\end{frame}
\subsection{jQueryオブジェクトのメソッド}
\begin{frame}[containsverbatim]
\frametitle{jQueryオブジェクトのメソッドの基本}
jQueryオブジェクトに対する多くの処理はHTMLの属性やCSSスタイルの値を設定
したり、読み出したりすること
\begin{itemize}
 \item これらのメソッドに対してセッターと
ゲッターを同じメソッドを使う。メソッドに値を与えるとセッターになり、ない
とゲッターになる。
 \item セッターとして使った場合は戻り値がjQueryオブジェクトとなるので、
       メソッドチェインが使える。
 \item ゲッターとして使った場合は要素群の最初の要素だけ問い合わせる。
\end{itemize}
\end{frame}
\begin{frame}[containsverbatim]
\frametitle{HTML属性の取得と設定}
\texttt{attr()}メソッドはHTML属性用のjQueryのゲッター/セッターである。
\begin{itemize}
 \item 属性名だけを引数に与えるとゲッターとなる。
 \item 属性名と値の2つを与えるとセッターになる。
 \item 引数にオブジェクトリテラルを与えると複数の属性を一時に設定できる。
 \item 属性を取り除く\texttt{removeAttr()}もある。
\end{itemize}
\end{frame}
\begin{frame}[containsverbatim]
\frametitle{CSS属性の取得と設定}
\texttt{css()}メソッドはCSSのスタイルを設定する。
\begin{itemize}
 \item CSSスタイル名は元来のCSSスタイル名(ハイフン付)でもJavaScriptの
       キャメル形式でも問い合わせ、設定が可能である。
 \item 戻り値は単位を含めて文字列で返される。
\end{itemize}
\end{frame}
\begin{frame}[containsverbatim]
\frametitle{HTMLフォーム要素の値の取得と設定}
\begin{itemize}
 \item \texttt{val()}はHTMLフォーム要素の\texttt{value}属性の値の設定や
       取得が可能
 \item これにより、\texttt{<select>}要素の選択された値を得ることなどが可
       能
\end{itemize}
\end{frame}
\begin{frame}[containsverbatim]
\frametitle{要素のコンテンツの取得と設定}
\begin{itemize}
 \item \texttt{text()}と\texttt{html()}メソッドはそれぞれ要素のコンテンツを通常
のテキストまたはHTML形式で返す。
 \item 引数がある場合には、既存のコンテンツを置き換える。 
\end{itemize}
\end{frame}
\subsection{ドキュメントの構造の変更}
\begin{frame}[containsverbatim]
\frametitle{DOM構造を変える}
\begin{table}[ht]
 \caption{ドキュメントの構造の変更するメソッド}\label{Insertreplace}
\begin{tabular}{|c|c|m{10zw}|}
 \hline
 \texttt{\$(T).method(C)}&\texttt{\$(C).method(T)}&
    \multicolumn{1}{c|}{機能}\\\hline
 \texttt{append}&\texttt{appendTo}&
	 要素\texttt{T}の最後の子要素として\texttt{C}を付け加える\\ \hline
 \texttt{prepend}&\texttt{prependTo}&
	 要素\texttt{T}の初めの子要素として\texttt{C}を付け加える\\ \hline
 \texttt{before}&\texttt{insertBefore}&
	 要素\texttt{T}の直前の要素として\texttt{C}を付け加える\\ \hline
 \texttt{after}&\texttt{insertAfter}&
	 要素\texttt{T}の直後の要素として\texttt{C}を付け加える\\ \hline
 \texttt{replaceWith}&\texttt{replaceAll}&
	 要素\texttt{T}と\texttt{C}を置き換える\\ \hline
\end{tabular}\end{table}
\end{frame}
\begin{frame}[containsverbatim]
\frametitle{DOMの操作(補足)}
\begin{itemize}
 \item 要素をコピーする\texttt{clone()}
 \item 要素の子要素をすべて消す\texttt{empty()}
 \item 選択された要素(とその子要素すべて)を削除する\texttt{remove()}
\end{itemize}
\end{frame}
\subsection{イベントハンドラーの取り扱い}
\begin{frame}[containsverbatim]
\frametitle{イベントハンドラーの登録}
\begin{itemize}
 \item イベントの種類ごとにメソッドが定義されていてる。
 \item 指定されたjQueryオブジェクトが複数の場合にはそれぞれに対してイ
ベントハンドラーが登録される。
\end{itemize}
\end{frame}
\begin{frame}[containsverbatim]
\frametitle{イベントハンドラー登録のメソッド}
 \begin{tabular}{llllll}
\texttt{blur()}& \texttt{focusin()}&\texttt{mouseenter()} &\texttt{scroll()}\\
\texttt{change()}&\texttt{keydown()}&\texttt{mouseleave()} &\texttt{select()} \\
  \texttt{click()}& \texttt{keypress()} &\texttt{mousemove()} &\texttt{submit()} \\
\texttt{dbleclick()}& \texttt{keyup()} &\texttt{mouseover()}& \texttt{unload()} \\
\texttt{error()}&\texttt{load()} &\texttt{mouseup()} &\\
 \texttt{focus()}&\texttt{mousedown()}&\texttt{resize()}&\\
 \end{tabular}
\end{frame}
\begin{frame}[containsverbatim]
\frametitle{イベントハンドラーの登録}
\begin{itemize}
 \item \texttt{hover()}:\texttt{mouseenter}イベントと
       \texttt{mouseleave}イベントに対するハンドラ-を同時に登録できる。
 \item \texttt{toggle()}はクリックイベントに複数のイベントハンドラ-を登
       録し、イベントが発生するごとに順番に呼び出す。
\end{itemize}
\end{frame}
\begin{frame}[containsverbatim]
 \frametitle{イベントハンドラ-の登録解除}
 イベントハンドラ-の登録解除は\texttt{unbind()}メソッドで行う。
 \begin{itemize}
  \item \texttt{removeEvenListener()}
と同様な形式として1番目の引数にイベントタイプ(文字列で与える)、2番目の引
数に登録した関数を与えるものがある
  \item 登録したイベントハンドラ-には名前が必要
 \end{itemize}
\end{frame}
\section{Ajaxの処理}
\subsection{\texttt{jQuery.ajax()}関数}
\begin{frame}[containsverbatim]
 \frametitle{\texttt{jQuery.ajax()}関数}
 \begin{itemize}
  \item jQueryではAjaxの処理に関するいろいろな方法を提供
  \item \texttt{jQuery.ajax()}関数を紹介
  \item この関数は引数にオブジェクトリテラルをとる。
 \end{itemize}
\end{frame}
\begin{frame}[containsverbatim]
 \frametitle{\texttt{jQuery.ajax()}関数の引数}
 \begin{table}[ht]
 \caption{jQuery.ajax()関数で利用できる属性(一部))}\label{jQueryAjax}
\begin{center}
 \begin{tabular}{|c|m{22zw}|}\hline
属性名  &\multicolumn{1}{c|}{説明} \\\hline
  \texttt{type}&通信の種類。通常は\texttt{"POST"}または\texttt{"GET"}を指定\\
  \hline
  \texttt{url}&サーバーのアドレス\\ \hline
  \texttt{data}&\texttt{"GET"}のときはURLの後に続けるデータ。
      \texttt{"POST"}のときは\texttt{null}\\ \hline
  \texttt{dataType}&戻り値のデータの型を指定する。\texttt{"text"}、
      \texttt{"html"}、\texttt{"script"}、      
\texttt{"json"}、\texttt{"xml"}などがある\\ \hline
\texttt{success}&通信が正常終了したときに呼び出されるコールバック関数
      \\ \hline
\texttt{error}&通信が成功しなかったときに呼び出されるコールバック関数
      \\ \hline
  \texttt{timeout}&タイムアウト時間をミリ秒単位で指定\\ \hline
 \end{tabular}
\end{center}
\end{table}
\end{frame}
\subsection{jQuery のサンプル}
\begin{frame}[containsverbatim]
 \frametitle{今日は何の日---jQuery版(1)}
 \LISTN{09-01whatday-jquery.html}{1}{7}{\scriptsize}
 6行目はローカルに保存したjQueryライブラリーを読み込んでいる。
\end{frame}
\begin{frame}[containsverbatim]
\frametitle{今日は何の日---jQuery版(2)}
 \LISTN{09-01whatday-jquery.html}{8}{21}{\scriptsize}
\end{frame}
\begin{frame}[containsverbatim]
\frametitle{今日は何の日---jQuery版(2)解説}
 \begin{itemize}
  \item 10行目の\texttt{\$(window).redy()}はドキュメントが解釈されたとき
	に引数の関数が呼び出されるイベントである。ここでの要素の参照は機
	能\ref{ObjSelector}を用いている。
  \item 11行目から21行目は連続した番号を値に持つプルダウンメニューを作成
	する関数。仕様は以前と同じ
	\begin{itemize}
	 \item 13行目では\texttt{<select>}要素を作成し、同時に属性も設定
	       (機能\ref{HTMLText})。
	 \item 15行目から19行目で\texttt{<option>}要素を定義し、
	       \texttt{<select>}要素の子要素にしている。なお、15行目と16
	       行目は\texttt{appendTo()}メソッドを用いると18行目のコメン
	       トアウトしてあるように記述することができる。
	\end{itemize}
 \end{itemize}
\end{frame}
\begin{frame}[containsverbatim]
\frametitle{今日は何の日---jQuery版(3)}
 \LISTN{09-01whatday-jquery.html}{22}{36}{\scriptsize}
 \end{frame}
\begin{frame}[containsverbatim]
 \frametitle{今日は何の日---jQuery版(3)解説}
 \begin{itemize}
   \item 22行目から31行目も依然とほとんど同じである。25行目では機能
	\ref{HTMLText}を用いて要素を参照している。
  \item 32行目、33行目と36行目はプルダウンメニューの初期値を本日に設定し
	ている。
  \end{itemize}
 \end{frame}
\begin{frame}[containsverbatim]
 \frametitle{今日は何の日---jQuery版(4)}
 \LISTN{09-01whatday-jquery.html}{37}{42}{\scriptsize}
 \end{frame}
\begin{frame}[containsverbatim]
 \frametitle{今日は何の日---jQuery版(4)解説}
 \begin{itemize}
  \item jQueryのメソッドを用いていることをのぞけは38行目から43行
	目までは以前と同じである。
  \item 40行目のセレクタは\texttt{id}が
	\texttt{day}(\texttt{\#day})の下にある
	\texttt{<option>}要素を選択し、その数を\texttt{length}で
	調べている。
\end{itemize}
 \end{frame}
\begin{frame}[containsverbatim]
\frametitle{今日は何の日---jQuery版(5)}
 \LISTN{09-01whatday-jquery.html}{43}{59}{\scriptsize}
 \end{frame}
\begin{frame}[containsverbatim]
\frametitle{今日は何の日---jQuery版(5)解説}
\begin{itemize}
 \item  44行目から53行目はAjaxの処理を定義
 \item ブラウザによる違いに対する対処や、XMLHTTPRequestの処理部分が大幅に減っ
	       ていて見やすいコードになっている
\end{itemize}
 \end{frame}
\begin{frame}[containsverbatim]
\frametitle{今日は何の日---jQuery版(6)}
 \LISTN{09-01whatday-jquery.html}{60}{last}{\scriptsize}
\end{frame}
\end{document}

