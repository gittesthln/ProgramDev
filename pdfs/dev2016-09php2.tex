\subsection{関数}
\begin{frame}[containsverbatim]
\frametitle{関数の特徴}
\begin{itemize}
 \item 関数はキーワード\texttt{function} に引き続いて\texttt{()}内に、仮
       引数のリストを書く。そのあとに\texttt{\{\}}内にプログラム本体を書
       く。
 \item 引数は値渡しである。参照渡しをするときは仮引数の前に\texttt{\&}を
       付ける。
 \item 関数のオーバーロードはサポートされない。
 \item 関数の宣言を取り消せない。
 \item 仮引数の後に値を書くことができる。この値は引数がなかった場合のデ
       フォルトの値となる。デフォルトの値を与えた仮引数の後にデフォルト
       の値がない仮引数を置くことはできない。
 \item 関数は使用される前に定義する必要はない。
 \item 関数内で関数を定義できる。関数内で定義された関数はグローバルスコー
       プに存在する。ただし、外側の関数が実行されないと定義はされない。
 \item 関数の戻り値は \texttt{return}文の後の式の値。複数の値を関
       数の戻り値にしたいときは配列にして返す。
\end{itemize}
\end{frame}
\begin{frame}[containsverbatim]
\frametitle{ユーザー定義関数の使用例}
\begin{Verbatim}[fontsize=\small,numbers=left]
<?php
function example($a, $as, &$b, $f=false) {
  print "\$a = $a\n";
  print_r($as);
  print "\$b = $b\n";
  if(!isset($x)) $x = "defined in function";
  print "\$x = $x\n";
  if($f) {
    print "\$GLOBALS['x'] = ". $GLOBALS['x']."\n";
  }
  $a = $a*2;
  $as[0] += 10;
  $b = $b*2;
  return array($a,$as);
}
\end{Verbatim}
\end{frame}
\begin{frame}[containsverbatim]
\frametitle{ユーザー定義関数の使用例---解説}
\begin{itemize}
 \item 1行目での関数では3番目の引数が参照渡し、4番目の引数がデフォルトの値
       が設定されている。
 \item 6行目の関数\texttt{isset()}は与えられた変数に値がセットされている
       かどうかを確かめるものである。
 \item ここでは\verb+$x+は仮引数ではないのでローカルな変数となり、
       \verb+isset($x)+は\texttt{false}となる。\verb+!isset($x)+は
       \texttt{true}となるので変数\verb+$x+には
       \verb+"defined in function"+が代入される。
 \item 9行目ではデフォルトの引数のチェックのための部分である。
 \item 11行目から13行目までは変数に値を代入して、呼び出し元の変数が変わ
       るかどうかのチェックをする。
 \item 14行目は初めの二つの仮引数を配列にして戻り値としている。
\end{itemize}
\end{frame}
\begin{frame}[containsverbatim]
\frametitle{関数の動作チェック}
\begin{Verbatim}
$a = 10;
$as = array(1,2);
$b = 15;
$x = "\$x is defined at top level";
\end{Verbatim}
ここでは、関数に渡す引数の値を設定している。
\begin{Verbatim}
example($a, $as, $b, true);
print "\$a = $a\n";
print_r($as);
print "\$b = $b\n";
\end{Verbatim}
\end{frame}
\begin{frame}[containsverbatim]
 \frametitle{実行結果}
 20行目で呼び出した関数内での出力結果は次のようになる。
\begin{Verbatim}
$a = 10
Array
(
    [0] => 1
    [1] => 2
)
$b = 15
$x = defined in function
$GLOBALS['x'] = $x is defined at top level
\end{Verbatim}
\end{frame}
\begin{frame}[containsverbatim]
\frametitle{関数の動作チェック---解説}
\begin{itemize}
 \item 仮引数の値は正しく渡されている。
 \item 6行目は判定が\texttt{true}になるのでここで新たに値が設定され、そ
       れが7行目で出力される。
 \item デフォルトの仮引数に対して\texttt{true}が渡されているので、9行目
       が実行される。スーパーグローバル\verb+$GLOBAL+にはグローバルスコー
       プ内の変数が格納されている。ここでは19行目に現れる変数の値が表示
       される。
\end{itemize}
\end{frame}
\begin{frame}[containsverbatim]
\frametitle{関数の動作チェック---続き}
21行目から23行目の出力は次のようになる。
\begin{Verbatim}
$a = 10
Array
(
    [0] => 1
    [1] => 2
)
$b = 30
\end{Verbatim}
\end{frame}
\begin{frame}[containsverbatim]
\frametitle{関数の動作チェック---続き(解説)}
\begin{itemize}
 \item 初めの2つの引数は配列であっても書き直されていない。11行目と12行目
       の設定は戻り値にしか反映されない。
 \item 参照渡しの変数\verb+$b+は書き直されている。
\end{itemize}
\end{frame}
\begin{frame}[containsverbatim]
\frametitle{関数の動作チェック---続き}
\begin{Verbatim}
list($resa) = example($a, $as, $b);
print "\$resa = $resa\n";
?>
\end{Verbatim}
\begin{itemize}
 \item 24行目の関数呼び出しはデフォルトの引数がない。
 \item したがって、引数\verb+$f+の値が\texttt{false}に設定され、9行目は
       実行されない。
 \item \texttt{list()}は配列である右辺の値のうち先頭から順に指定された変
       数に代入
 \item ここでは関数内の11行目で計算された値を変数\verb+$resa+に代入
\end{itemize}
\end{frame}
\begin{frame}[containsverbatim]
\frametitle{関数の動作チェック---続き}
\begin{Verbatim}
$a = 10
Array
(
    [0] => 1
    [1] => 2
)
$b = 30
$x = defined in function
$resa = 20
\end{Verbatim}
\end{frame}
\iffalse\else
\section{可変変数と可変関数}
 \begin{frame}[containsverbatim]
  \frametitle{可変変数}
ある変数の値が文字列のとき、その前に\verb+$+をつけると、その文字列が変
  数名として使える。
\begin{Verbatim}[numbers=left, fontsize=\small]
<?php
$a = 1;
$b = 2;

$c = "a";
print "\$\$c = " . $$c ."\n";  // $$c = 1

$c = "b";
print "\$\$c = " . $$c . "\n"; // $$c = 2
?>
\end{Verbatim}
 \end{frame}
 \begin{frame}[containsverbatim]
  \frametitle{可変変数--解説}
  \begin{itemize}\upshape
   \item 変数\verb+$a+ と \verb+$b+ にそれぞれ$1$と$2$を代入(1行
         目と2行目)。
   \item 変数\verb+$c+に文字列\verb+"a"+を代入し、\verb+$$c+を出力させ
         ると\verb+$a+の値が表示(5行目と6行目)。
   \item 同様に、変数\verb+$c+に文字列\verb+"b"+を代入して、\verb+$$c+を出力させ
         ると\verb+$b+の値が表示(8行目と9行目)。
  \end{itemize}
 \end{frame}
\fi
 \begin{frame}[containsverbatim]
  \frametitle{可変関数}
 文字列が代入された変数の後に\texttt{()}をつけると、その文字列の関数
  を呼び出すことができる。
\begin{Verbatim}[fontsize=\small,numbers=left]
<?php
function add($a, $b) { return $a+$b;}
function sub($a, $b) { return $a-$b;}
$f = "add";
print "$f(5,2) = " . $f(5,2) ."\n";// add(5,2) = 7
$f = "sub";
print "$f(5,2) = " . $f(5,2) ."\n";// sub(5,2) = 3
?>
\end{Verbatim}
 \end{frame}
  \begin{frame}[containsverbatim]
  \frametitle{可変関数--解説}
\begin{itemize}
\item 2行目と3行目で2つの関数\texttt{add()}と
       \texttt{sub()}を定義
 \item 4行目で変数\Verb+$f+に文字列\Verb+"add"+を代入して、5行目で可変
       関数として呼び出すと、関数\texttt{add}が呼び出されている。
 \item 同様に、変数\Verb+$f+に文字列\Verb+"sub"+を代入して、可変
       関数として呼び出すと、関数\texttt{sub}が呼び出されている。
\end{itemize}
\end{frame}
