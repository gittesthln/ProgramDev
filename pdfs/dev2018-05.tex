%\input devHeadMalmoe.tex
\input devHeadGoettingen.tex
\title{ソフトウェア開発\\第5回目授業}
\author{平野 照比古}
\institute{}
\date{2018/10/25}
\begin{document}
\frame{\maketitle}
%\frame{\tableofcontents}
\section{レポート課題について}
\begin{frame}[containsverbatim]
 \frametitle{レポート課題4.1につ
%\frame{\tableofcontents}
\section{レポート課題について}
\begin{frame}[containsverbatim]いて}
\begin{Verbatim}
for( i in window) console.log(`${i} ${window[i]}`);
\end{Verbatim}
 として実行する。
 \begin{itemize}
  \item 特定のものをチェックするのではなくすべてを見て、どのようなプロパ
        ティがあるかを見てほしい。
  \item 見慣れた関数名(\Verb+alert()+)がいくつか見つかるはず
  \item 新しいオブジェクトが出てきて、文書化されていないときなどに利用で
        きる。
  \item コンソールでオブジェクトを出力すれば同様のことは可能
 \end{itemize}
\end{frame}
\begin{frame}[containsverbatim]
 \frametitle{レポート課題4.2について}
 \begin{itemize}
  \item      出力結果は\Verb+[{},{}]+
\begin{Verbatim}
>JSON.stringify(persons,["year"]);
"[{},{}]"
 \end{Verbatim}
  \item \texttt{year}の値を出すためには親のキーも含める。
        {\footnotesize
\begin{Verbatim}
>JSON.stringify(persons,["year","birthday","day","month","name"]);
"[{"birthday":{"year":2001,"day":1,"month":4},"name":"foo"},
{"birthday":{"year":2010,"day":5,"month":5},"name":"Foo"}]"
\end{Verbatim}
        }
 \end{itemize}
\end{frame}
\begin{frame}[containsverbatim]
 \frametitle{レポート課題4.3について}
 同様のことができることの確認が少ないレポートが多かった。
\end{frame}
\begin{frame}[containsverbatim]
 \frametitle{レポート課題4.4について}
\begin{itemize}
 \item ルーブリックにヒントを書いておいた。月と日の値はデフォルトの値を
       書いておくとプログラムが簡単になる。
 \item 仮引数に値が与えられないときは\texttt{undefined}が渡されることを
       利用する。
\end{itemize}
 {\small
\begin{Verbatim}
  age(year, month=1, day=1) {
    let today;
    if(year === undefined) today = new Date();
    else today = new Date(year, month-1, day);
    let Age = today.getFullYear() - this.birthday.year;
    if(today.getTime() <
      new Date(today.getFullYear(),
               this.birthday.month-1,
               this.birthday.day).getTime()) Age--;
    return Age;
  }
\end{Verbatim}
 }
\end{frame}
\begin{frame}[containsverbatim]
 \frametitle{レポート課題4.5について}
 \begin{itemize}
  \item 実行方法は\texttt{()}を付ける(\texttt{get}がないとき)かどうか
  \item メソッドの定義(\texttt{get}がない場合)
        \begin{itemize}
         \item 仮引数を与えることができる。
         \item メソッドの書き換えが可能(できないようにすることは可能)
        \end{itemize}
 \end{itemize}
\end{frame}
\begin{frame}[containsverbatim]
 \frametitle{レポート課題4.6について}
\begin{Verbatim}
  set age(x){
    window.alert("値をセットできません");
  }
\end{Verbatim}
\begin{itemize}
 \item セッターには仮引数が一つ必要
 \item 通常はこの値をチェックして、条件を満たせば該当する値をセットし、
       しかるべき追加の処理を行う
\end{itemize}
\end{frame}
\section{\protect\texttt{class}の実体}
\begin{frame}[containsverbatim]
\frametitle{\ElmJ{class}の実体}
\ES で導入された\ElmJ{class}の実体はこれまでの
 ECMAScript に新しい機能を追加したわけではなく、
 従来のオブジェクトや継承に関して記述を簡単にしたものに過ぎない
 \footnote{このことを syntactic sugar(糖衣構文)と呼ぶ。}。
 
 前回のクラス\texttt{Person}に対し、
 \texttt{typeof Person}の結果は次のとおり
\begin{Verbatim}
>typeof Person;
"function"
\end{Verbatim}
従来の ECMAScript では関数を用いてクラスを定義する。\ElmJ{class}でもその手法が使
われていることを示す。
\end{frame}
\begin{frame}[containsverbatim]
 \frametitle{従来の方法によるオブジェクト指向}
 前回の授業の\texttt{Person}を従来の方法で定義する。
 {\small
\begin{Verbatim}
function PersonF(name, year, month, day, hometown = "神奈川"){
    this.name = name;
    this.birthday = {
      year : year,
      month : month,
      day : day
    };
    this["hometown"] = hometown;
}
\end{Verbatim}
 }
 \begin{itemize}
  \item この定義では\ElmJ{constructor}の内容をそのまま関数にしている。
  \item メソッドなどが定義されていないが、それについては後で解説
 \end{itemize}
\end{frame}
\begin{frame}[containsverbatim]
\frametitle{実行結果}
\begin{Verbatim}
>p = new PersonF("foo",2001,4,1);
PersonF {name: "foo", birthday: {…}, hometown: "神奈川"}
\end{Verbatim}
\end{frame}
\section{オブジェクト属性}
\subsection{\protect\ElmJ{class}属性}
\begin{frame}[containsverbatim]
 \frametitle{関数オブジェクトの属性}
 \begin{itemize}
  \item JavaScriptの関数オブジェクトの属性は\ElmJ{prototype}、\ElmJ{class}と
\ElmJ{extensible}の3つ
  \item \ElmJ{prototype}属性はオブジェ
クトの継承に関係するので、最後に説明
 \end{itemize}
\end{frame}
\begin{frame}[containsverbatim]
 \frametitle{\protect\ElmJ{class}属性}
 \begin{itemize}
  \item \ElmJ{class}属性はオブジェクトの型情報を表す文字列
  \item この属性を設定する方法はない。
  \item クラス属性は間接的にしか得られない。
  \item 組み込みのコンストラクタで生成されたオブジェクトではそのクラス名が間接的
        に得られる.
  \item 独自のコンストラクタ関数では、\texttt{"Object"}しか得られ
ない。
 \end{itemize}
\end{frame}
\begin{frame}[containsverbatim]
 \frametitle{実行例}
 前の例\texttt{PersonF}では次のようになる。
 \begin{Verbatim}
>p = new PersonF("foo",2001,4,1);
PersonF {name: "foo", birthday: {…}, hometown: "神奈川"}
>`${p}`;
"[object Object]"
\end{Verbatim}
\begin{itemize}
 \item \texttt{Person`F()}コンストラクタには\ElmJ{toString()}メソッドが定
       義されていない。
 \item もともとの\ElmJ{Object}で定義されている
 \texttt{toString()}が呼び出される。
\end{itemize}
\end{frame}
\subsection{\protect\ElmJ{extensible}属性}
\begin{frame}[containsverbatim]
 \frametitle{\protect\ElmJ{extensible}属性}
\begin{itemize}
 \item オブジェクトに対してプロパティの追加ができるかどうかを指定
 \item この属性の取得や設定ができる関数がある。
 \item 属性の取得は\ElmJ{Object.isExtensible()}に調べたいオブジェクトを引数
にして渡す。
 \item オブジェクトのプロパティを拡張できなくするためには
\ElmJ{Object.preventExtension()}に引数として設定したいオブジェクトを渡す。
\end{itemize} 
\end{frame}
\begin{frame}[containsverbatim]
 \frametitle{\protect\ElmJ{extensible}属性の実行例}
 {\scriptsize
 \begin{Verbatim}
>p = new PersonF("foo",2001,4,1);
PersonF {name: "foo", birthday: {…}, hometown: "神奈川"}
PersonF
>p.mother = "Alice";
Alice
>p.grandmother = "Old Alice";
Old Alice
>Object.preventExtensions(p);
PersonF
>p.father = "Bob";
Bob
>p.father;
undefined
>delete p.mother;
true
>p.mother;
undefined
\end{Verbatim}
}
\end{frame}
\begin{frame}[containsverbatim]
\frametitle{\protect\ElmJ{extensible}属性の実行例--解説}
\begin{itemize}
 \item \texttt{Object.preventExtensions(p)}を実行する前では存在しな
       い属性の追加が可能(\texttt{p.mother})。
 \item \texttt{Object.preventExtensions(p)}を実行後は、新しい属性が定義
			 できない(\texttt{p.father})。
 \item \texttt{Object.preventExtensions(p)}では属性の削除は禁止でき
       ない(\texttt{delet p.mother}が成功し、
			 値を参照すると\ElmJ{undefined}が返る)。
\end{itemize}
\end{frame}
\begin{frame}[containsverbatim]
 \frametitle{属性の削除の禁止--\protect\ElmJ{Object.seal()}}
 この関数を実行されたオブジェクトは削除の禁止を解除できない。
%\end{frame}
%\begin{frame}[containsverbatim]
% \frametitle{属性の削除の禁止--実行例}
 {\scriptsize
\begin{Verbatim}
>Object.seal(p);
PersonF
>Object.isSealed(p);
true
>delete p.grandmother;
false
>p.grandmother;
Old Alice
>p.grandmother = "Very Old Alice";
Very Old Alice
>p.grandmother;;
Very Old Alice
\end{Verbatim}
}
\end{frame}
\begin{frame}[containsverbatim]
 \frametitle{属性の削除の禁止--実行例(解説)}
\begin{itemize}
	 \item すでに\texttt{seal}された状態になっているかどうかは、
				 \ElmJ{Object.isSealed()}で調べる。
 \item \texttt{seal}された状態ではプロパティを削除できない
			 (\texttt{delete p.grandmother}の結果が\ElmJ{false})。
 \item \texttt{seal}された状態ではプロパティの値を変えることができる
			 (\texttt{p.grandmother}の値の書き換えができている)。
\end{itemize}
\end{frame}
\begin{frame}[containsverbatim]
 \frametitle{オブジェクトの固定化--\protect\ElmJ{Object.freeze()}}
\begin{itemize}
 \item オブジェクトを最も強く固定するためには\ElmJ{Object.freeze()}を用
			 いる。
 \item この状態を確認するためには\ElmJ{Object.isFrozen()}を用いる。
\end{itemize}
%\end{frame}
%\begin{frame}[containsverbatim]
% \frametitle{オブジェクトの固定化--実行例}
\begin{Verbatim}
>Object.freeze(p);
PersonF
>p.grandmother = "Very Very Old Alice";
Very Very Old Alice
>p.grandmother;
Very Old Alice
\end{Verbatim}
\end{frame}
\begin{frame}[containsverbatim]
 \frametitle{オブジェクトの固定化--実行例(解説)}
 \begin{itemize}
  \item \texttt{p.grandmother}の値が設定できていない。
  \item このような状態で属性値を変えたい場合には、属性にたいするセッターメソッド
  を定義する
 \end{itemize}
\end{frame}
\begin{frame}[containsverbatim]
 \frametitle{オブジェクトの固定化--注意}
 このオブジェクトのプロパティ\texttt{birthday}の値はオブジェクトで、
 \ElmJ{seal}されていない。
 {\scriptsize
\begin{Verbatim}
>p.birthday;
{year: 2001, month: 4, day: 1}
>delete p.birthday;
false
>p.birthday = {};
{}
>p.birthday;
{year: 2001, month: 4, day: 1}
>delete p.birthday.year;
true
>p.birthday;
{month: 4, day: 1}
\end{Verbatim}
 }
 \texttt{p.birthday}は削除できないし、値の変更もできない。
 一方で、\texttt{p.birthday.year}は削除できる。

 理由はレポート問題(この後にも同じような状況が起こる)
\end{frame}
\begin{frame}[containsverbatim]
 \frametitle{オブジェクトの操作に対する安全性の確保}
 \begin{itemize}
  \item JavaScript におけるクラスとそれから生成されるインスタンスは
 通常のオブジェクト指向言語と異なり、構造を変えることができる。
  \item  固定化はインスタンスごとに行う。
	\item 固定化などはコンストラクタ内で\ElmJ{this}の\ElmJ{extensible}属性
				を設定することで可能
  \item 別の方法について後で解説する。
 \end{itemize}
\end{frame}
 \subsection{\protect\ElmJ{prototye}属性}
 \begin{frame}[containsverbatim]
  \frametitle{\ElmJ{prototye}属性}
  \begin{itemize}
   \item オブジェクトの\ElmJ{prototype}属性の値は、同じコンストラクタ関数で生成
 された間で共通
   \item オブジェクトリテラルで生成されたオブジェクトは
         \ElmJ{Object.prototype}で参照可能
   \item \ElmJ{new}を用いて生成されたオブジェクトはそのコンストラクタ関
         数の\ElmJ{prototype}を参照
   \item このコンストラクタ関数の\ElmJ{prototype}もオブジェクトであるから、それ
 の\ElmJ{prototype}も存在
   \item この一連の\ElmJ{prototype}オブジェクトをプロトタイプチェーンとよぶ。
  \end{itemize}
 \end{frame}
 \subsection{\protect\texttt{prototype}の使用例}
\begin{frame}[containsverbatim]
 \frametitle{前の例でメソッドを\ElmJ{prototype}に追加}
{\scriptsize
\begin{Verbatim}
PersonF.prototype = {
  toString:function() {
    return `${this.name}さんは`+
      `${this.birthday.year}年${this.birthday.month}月${this.birthday.day}日に` +
   `${this.hometown}で生まれました。`;
  },
  get age(){
    let today = new Date();console.log(x);
    let age = today.getFullYear() - this.birthday.year;
    if(today.getTime() <
         new Date(today.getFullYear(),
                  this.birthday.month-1,
                  this.birthday.day).getTime()) age--;
    return age;
  },
  constructor: PersonF
}
\end{Verbatim}
}
\end{frame}
\begin{frame}[containsverbatim]
 \frametitle{メソッドを\ElmJ{prototype}に追加--解説}
 \begin{itemize}
  \item これらの定義は前と同じ
  \item オブジェクトリテラルの形式で\ElmJ{prototype}に代入
  \item 異なるところは\ElmJ{constructor}プロパティに自分自身を定義している
 \end{itemize}
\end{frame}

\section{関数によるオブジェクトの継承}
\begin{frame}[containsverbatim]
 \frametitle{関数によるメソッドの継承}
 JavaScript では\ElmJ{prototype}を用
 いることでメソッドの継承が可能

 \texttt{Person}を継承して、学籍番号
 を追加のプロパティとする\texttt{Student}オブジェクトのコンストラクタ関
 数である。
 {\scriptsize
\begin{Verbatim}
const StudentF = (function(){
  let id = 10000;
  return function(name, year, month, day, hometown = "神奈川") {
    this.name = name
    this.birthday = {
      year : year,
      month : month,
      day : day
    };
    this["hometown"] = hometown;
    this.id = id++;
    }
  })();
StudentF.prototype = new PersonF();
StudentF.prototype.constructor = StudentF;
\end{Verbatim}
 }
\end{frame}
\begin{frame}[containsverbatim]
 \frametitle{}
 \begin{itemize}
 \item 関数を用いた継承では\ElmJ{super}が使えないので、継承元のクラスで
       定義されたプロパティ(\texttt{name}など)は継承先で定義する
 \item メソッドはインスタンスで共有されるので継承元で定義をする必
       要はないが、参照できるようにする必要がある。
  \item 継承元のクラスを作成して継承するクラスの\ElmJ{prototype}を書き直し
       ている
 \item このままでは\ElmJ{constructor}プロパティが\texttt{PersonF}になるの
       で、\texttt{StudentF.prototype.constructor}を
       継承するクラスに置き換える
\end{itemize}
\end{frame}
\section{\protect\ElmJ{WeakMap}によるインスタンスの安全性の確保}
\begin{frame}[containsverbatim]
 \frametitle{\protect\ElmJ{WeakMap}}
 \begin{itemize}
  \item あるクラスのインスタンスごとにプロパティを保存するのに
        \texttt{this.name}のようなキーを用いると、データの安全性が保障さ
        れない。
  \item クラス式のクロージャに入れるとクラスで共通のものになってしまい、
        目的に合わない。
  \item インスタンスごとに値をしまうためにはオブジェクトと値の集まりを
        関連付ける必要がある。
  \item \ElmJ{WeakMap}はオブジェクトと同じようにキーと値の集まり
  \item \ElmJ{WeakMap}のキーはオブジェクトしか使用できない。
 \end{itemize}
\end{frame}
\begin{frame}[containsverbatim]
 \frametitle{\protect\ElmJ{Person}クラスの書き換え(1)}
 {\scriptsize
\begin{Verbatim}
const Person = (function() {
  const properties = new WeakMap();
  return class {
    constructor(name, year, month, day, hometown="神奈川"){
      properties.set(this,
        {
          "name" : name,
          "birthday" :{
            "year" : year,
            "month": month,
            "day" : day
          },
          "hometown" : hometown
        });
    }
    get name(){
      return properties.get(this).name;
    }
    get birthday(){
      return properties.get(this).birthday;
    }
    get hometown(){
      return properties.get(this).hometown;
    }
\end{Verbatim}
 }
 \end{frame}
\begin{frame}[containsverbatim]
 \frametitle{\protect\ElmJ{Person}クラスの書き換え(2)}
 {\scriptsize
\begin{Verbatim}
    toString() {
      return `${this.name}さんは`+
        `${this.birthday.year}年${this.birthday.month}月${this.birthday.day}日に` +
        `${this.hometown}で生まれました。`;
    }
    get age(){
      let today = new Date();
      let age = today.getFullYear() - this.birthday.year;
      if(today.getTime() <
         new Date(today.getFullYear(),
                  this.birthday.month-1,
                  this.birthday.day).getTime()) age--;
      return age;
    }
  }
})()
\end{Verbatim}
 }
\end{frame}
\begin{frame}[containsverbatim]
 \frametitle{\protect\ElmJ{Person}クラスの書き換え--解説}
 \begin{itemize}
  \item \texttt{Person}クラスのインスタンスの(隠された)プロパティ
        値を格納するための\ElmJ{WeakMap}のインスタンスを作成
  \item \ElmJ{constructor}を定義
        \begin{itemize}
         \item 定義されているオブジェクトは前と同じ
         \item そのオブジェクト(\ElmJ{this})をキーにして、
               \ElmJ{WeakMap}のインスタンスに\ElmJ{set}メ
               ソッドを用いて登録
        \end{itemize}
  \item 各インスタンスのプロパティを読み出すために、\texttt{name}、
        \texttt{birthday}、\texttt{hometown}の3つのメソッド定義
  \item インスタンスの(隠し)プロパティを保存している\ElmJ{WeakMap}から
        値を取り出すために、\ElmJ{get}メソッドを使用
  \item \texttt{toString()}メソッドとプロパティ\ElmJ{age}は以前の定義と同じ
 \end{itemize}
\end{frame}
\begin{frame}[containsverbatim]
 \frametitle{実行例(1)}
\begin{Verbatim}
>p = new Person("foo",2001,4,1);
{}
\end{Verbatim}
インスタンスのプロパティは設定していな
 いため空である。結果を展開するとメソッドが見える。

インスタンスのそれぞれのプロパティは読み出すことができる。
\begin{Verbatim}
>p.name;
"foo"
>p.birthday;
{year: 2001, month: 4, day: 1}
>p.hometown;
"神奈川"
\end{Verbatim}
\end{frame}
\begin{frame}[containsverbatim]
 \frametitle{実行例(2)}
 メソッドも正しく動作する。
\begin{Verbatim}
>p.age;
16
>`${p}`;
"fooさんは2001年4月1日に神奈川で生まれました。"
\end{Verbatim}
プロパティの値は変更できない。
\begin{Verbatim}
>p.name = "Alice"; //name の値を "Alice" にする試み
>p.name;
"foo"              //name の値は変更されていない。
\end{Verbatim}
\end{frame}
\begin{frame}[containsverbatim]
 \frametitle{実行例(3)}
プロパティは消去できない。
\begin{Verbatim}
>delete p.name;  //プロパティ name の消去の試み
true             //成功?
>p.name
"foo"            //プロパティ name は消去されていない。
\end{Verbatim}
 \texttt{p.birthday}のプロパティの値は変更できる。
\begin{Verbatim}
>p.birthday.year = 2010;       // 戻り値 2010
>p.birthday;
{year: 2010, month: 4, day: 1} //year の値が変更されている。
\end{Verbatim}
\end{frame}
 \section{エラーオブジェクトについて}
\begin{frame}[containsverbatim]
 \frametitle{エラーオブジェクト}
 \begin{itemize}
	\item エラーオブジェクトとはエラーが発生したことを知らせるオブジェクト
	\item 通常は計算の継続ができなくなったときにエラーオブジェクトをシステムに送る
操作が必要
	\item これを通常、エラーを投げる(\ElmJ{throw}する)という。
 \end{itemize}
下表のようなプロパティがある。
\begin{table}[ht]
 \caption{エラーオブジェクトのプロパティ}
 \begin{center}
	 \begin{tabular}{|c|m{20zw}|}\hline
		プロパティ&\multicolumn{1}{c|}{説明}\\ \hline
		\texttt{message}&エラーに関する詳細なメッセージ。コンストラクタで渡
				された文字列か、デフォルトの文字列\\ \hline
		\texttt{name}&エラーの名前。エラーを作成したコンストラクタ名になる\\ \hline

	\end{tabular}
 \end{center}
\end{table}
\end{frame}
\subsection{エラー処理の例}
\begin{frame}[containsverbatim]
 \frametitle{エラー処理の例}
以前に定義した\texttt{Person}クラスのコンストラクタに与えられ
	た引数をチェックして不正な値の場合にはエラーを投げるように書き直したも
 の
 {\scriptsize
\begin{Verbatim}
class Person{
  static checkName(name) {
    if(name === "") throw new Error("名前がありません");
    return name;
  }
  static checkDate(y, m, d) {
    if(m<1 || m>12) throw new Error("月が不正です");
    let date = new Date(y,m,0);
    if(d<1 || d>date.getDate()) throw new Error("日が不正です");
    return {year: y, month: m, day: d};
  }
  constructor(name, year, month, day, hometown="神奈川"){
    this.name = Person.checkName(name);
    this.birthday = Person.checkDate(year, month, day);
    this["hometown"] = hometown;
  }
  toString() {
  ...
  }
  get age(){
    ...
  }
}
const Student = (function(){
  ...
})();
\end{Verbatim}
 }
\end{frame}
\begin{frame}[containsverbatim]
 \frametitle{エラー処理の例--解説}
\begin{itemize}
 \item \texttt{name} が空文字であればエラーを発生させ、
       正しければ値をそのまま返すクラスメソッドを定義
 \item 正しい日付でないとエラーを発生させ、正しいときは
       日付のオブジェクトを返すクラスメソッドを定義
       \begin{itemize}
        \item 月の値の範囲をチェックしている。
 \item 与えられた年と月からその月の最終の日を求めている。
			 \ElmJ{Date.getMonth()}の戻り値が0(1月)から11(12月)になっているの
			 で、\texttt{new Date(y,m,0)}により翌月の1日の1日前、つまり、問題
			 としている月の最終日が設定できる。
 \item 与えられた範囲に日が含まれていなければエラーを発生
        \item 日付のオブジェクトを作成し、戻り値として返す。
       \end{itemize}
 \item 名前と日付を指定したプロパティにクラスメソッド
       からの戻り値で設定
% \item 残りの部分は前のリストと同じである。
\end{itemize}
\end{frame}
\begin{frame}[containsverbatim]
 \frametitle{エラー処理の例--実行例}
\begin{itemize}
 \item 通常の日時ならば問題なく、オブジェクトが構成される。
\begin{Verbatim}
>p = new Person("foo",1995,4,1);
Person {name: "foo", year: 1995, month: 4, day: 1}
\end{Verbatim}
 \item うるう年の1996年には2月29日あるので、エラーは起こ
			 らずオブジェクトが作成できる。
\begin{Verbatim}
>p = new Person("foo",1996,2,29);
Person {name: "foo", year: 1996, month: 2, day: 29}
\end{Verbatim}
 \item うるう年ではない1995年には2月29日がないので、エラーが起
			 きる。
\begin{Verbatim}
>p = new Person("foo",1995,2,29);
 Uncaught Error: 日が不正です(…)
\end{Verbatim}
 \item 不正な月や日では当然、エラーが起こる。
\begin{Verbatim}
>p = new Person("foo",1995,13,29);
 Uncaught Error: 月が不正です(…)
>p = new Person("foo",1995,12,0);
 Uncaught Error: 日が不正です(…)
\end{Verbatim}
\end{itemize}
\end{frame}
\subsection{エラーからの復帰}
\begin{frame}[containsverbatim]
 \frametitle{エラーからの復帰--\ElmJ{try\{...\}catch\{...\}}構文}
\begin{itemize}
 \item 前節の例ではエラーが発生するとそこでプログラムの実行が停止
 \item エラーが発生したときに、投げられた(\ElmJ{throw}された)エラーを捕まえる
(\ElmJ{catch}する)ことが必要
 \item このためには\ElmJ{try\{...\}catch\{...\}}構文を使用
\end{itemize}
\end{frame}
\begin{frame}[containsverbatim]
 \frametitle{\ElmJ{try\{...\}catch\{...\}}構文の解説}
 \begin{itemize}
 \item \ElmJ{try}のブロック内にエラーが発生する可能性があるコードを記述
 \item エラーが発生したときは\ElmJ{throw}を用いてエラーを発生
 \item エラーが発生すると\ElmJ{catch(e)}ブロックが実行
	\item \ElmJ{catch}の後にある\texttt{e}は投げられたエラーオブジェクトが
				渡される。
	\item この変数のスコープは\ElmJ{catch}内
 \item \ElmJ{finally\{\}}を付けることもできる。
	\item \texttt{try}や\texttt{catch}の処理がが実行後、必ず呼び出される。
       これは\ElmJ{try}や
       \ElmJ{catch}の部分が\ElmJ{return}文、\ElmJ{break}文、
       \ElmJ{continue}文、\ElmJ{return}文や新しい例外を投げたとしても呼
       び出される。
 \item \ElmJ{try\{...\}catch\{...\}}構文は入れ子にできる。投げられたエラー
			 に一番近い\texttt{catch}にエラーが捕まえられる。
 \end{itemize}
\end{frame}
\begin{frame}[containsverbatim]
 \frametitle{実行例}
\begin{itemize}
 \item   前の例で\ElmJ{try\{...\}catch\{...\}}構文を用いてオブジェクトが正しくでき
	るまで繰り返すようにしたもの
 \item ブラウザで実行することを想定
\end{itemize}
\end{frame}
\begin{frame}[containsverbatim]
 \frametitle{プログラムリスト}
 {\scriptsize
\begin{Verbatim}
function test() {
  for(;;) {
    try {
      let y = Number(prompt("生まれた年を西暦で入力してください"));
      console.log(`年:${y}`)
      let m = Number(prompt("生まれた月を入力してください"));
      console.log(`月:${m}`)
      let d = Number(prompt("生まれた日を入力してください"));
      console.log(`日:${d}`)
      return new Person("foo", y, m, d);
    } catch(e) {
      console.log(e.name+":"+e.message);
    } finally{
      console.log("finnally");
    }
  }
}
\end{Verbatim}
}
\end{frame}
\begin{frame}[containsverbatim]
 \frametitle{解説}
\begin{itemize}
 \item テストを繰り返す関数\texttt{test()}が定義
 \item 無限ループのなかで正しいパラメータが与えられた
			 ときに作成されたオブジェクトを戻り値にして関数の実行が終了
 \item \ElmJ{try\{\}}内にはエラーが発生するかもしれないコードを中に含め
			 る。
	\begin{itemize}
	 \item 年、月、日の入力を\ElmJ{prompt}によるダイアログボックス
				 から入力
	 \item 戻り値は文字列なので、\ElmJ{Number}で数に変換
   \item 代入された値が正当であればその値をコンソールに出力
	\end{itemize}
 \item 与えられた入力が正しくなければエラーが投げられ、
			 \ElmJ{catch}\texttt{(e)}の中に制御が移動
 \item \ElmJ{catch}\texttt{(e)}における\texttt{e}には発生したエラーオブ
			 ジェクトが渡されるので、コンソールにその情報を出力
\end{itemize}
\end{frame}
\begin{frame}[containsverbatim]
 \frametitle{実行例}
 {\scriptsize
\begin{Verbatim}
>p = test();
 年:2010
 月:4
 日:31
 Error:日が不正です
 finnally
 年:2010
 月:4
 日:30
finnally
Person {name: "foo", birthday: {…}, hometown: "神奈川"}
\end{Verbatim}
 }
\end{frame}
\begin{frame}[containsverbatim]
 \frametitle{実行例--解説}
\begin{itemize}
 \item 200年4月31日は不正な日付なので、\ElmJ{console.log()}の出力
       はない。
 \item その代り、エラーの内容が出力されている。
 \item エラー出力後は\ElmJ{finally}ブロックが実行されている。
 \item 入力データが正しい場合でも\ElmJ{finally}ブロックが実行されている。
\end{itemize}\end{frame}
\end{document}
