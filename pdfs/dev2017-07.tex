\input devHead.tex
\SetTheme{Goettingen}
\title{ソフトウェア開発\\第7回目授業}
\author{平野 照比古}
\institute{}
\date{2017/11/10}
\newtheorem{Prob}{解説}
\setbeamercovered{transparent}
\begin{document}
\frame{\maketitle}
%\frame{\tableofcontents}
\newcommand{\Elm}[1]{\texttt{<#1>}}
\setbeamercovered{transparent}
\section{前回の演習}
\subsection{正規表現を作る}
\begin{frame}[containsverbatim]
 \frametitle{C言語の変数名にマッチする}
\begin{itemize}
%\onslide<1->
 \item C言語の変数名(正確には識別子)は英字で始まり、そのあとに英数字が並んだも
の(正確にはもう少し使える文字がある)
 \item 先頭の文字は文字クラスを使うと\texttt{[A-Za-z]}
 \item そのあとは英数字
 \item その文字クラスは\texttt{\textbackslash w}
 \item $0$ 個でもよいので、\texttt{\textbackslash w*}
 \item 全体がこれだけであることを保証するためには位置指定子をつける
\end{itemize}
%\onslide<2->
{\LARGE
\Verb+^[A-Za-z]\w*$+}
\end{frame}
\begin{frame}[containsverbatim]
 \frametitle{浮動小数リテラルをにマッチする正規表現(1)}
浮動小数リテラルは次の部分から成り立っている。

[符号][整数部][小数点][小数部][指数部]

このうち、[符号]や小数点以下の部分はなくてもよい。

\end{frame}
\begin{frame}[containsverbatim]
 \frametitle{浮動小数リテラルをにマッチする正規表現(2)}
\begin{itemize}
 \item 符号部は\texttt{+}または\texttt{-}からなる一文字からなる。一度だ
       けまで現れてよいので、この部分は \texttt{[+-]?}または\texttt{(+|-)?}で表される。
 \item 整数部は10進数の並びであり最低1文字は必要であるので反復の指定は
       \texttt{+}となる。したがって、この部分は \texttt{\textbackslash d+}で表される。
 \item 小数点\texttt{.}は正規表現では任意の文字にマッチするのでエスケー
       プする必要がある。したがってこの部分は \texttt[\textbackslash .]
       となる。
 \item 小数部は数字が並べられる。全くなくてもよいので反復の指定は
       \texttt{*}となる。
\end{itemize}
\end{frame}
\begin{frame}[containsverbatim]
  \frametitle{浮動小数リテラルをにマッチする正規表現(3)}
\begin{itemize}
\item 指数部は指数の開始を表す文字\texttt{E}または\texttt{e}で始まる10
       進数である。数字は最低一つ必要であるのでこの部分は
       \texttt{(E|e)\textbackslash d+}となる。
 \item これらを合わせると求める正規表現が得られる。小数部などがなくても
       よいのでそれらの部分には反復指定\texttt{?}を付ければよい。
 {\Large
\begin{Verbatim}
^[+-]?\d+(\.\d*)?((E|e)[+-]?\d+)?$
\end{Verbatim}
}
\end{itemize}
\end{frame}
\begin{frame}[containsverbatim]
  \frametitle{浮動小数リテラルをにマッチする正規表現(4)}
正式な数値リテラルでは小数点の前に整数部がない \texttt{.1}など
       も許しているが、ここではマッチしない。
%"-.1e10".match(/^[+-]?((\d+(\.\d*)?)|(\.\d+))((E|e)[+-]?\d+)?$/);
\end{frame}
\begin{frame}[containsverbatim]
 \frametitle{24時間制の時刻の表し方}
時、分、秒はすべて2桁とし、それらの区切りは\texttt{:}
%\end{frame}
%\begin{frame}[containsverbatim]
\begin{itemize}
 \item 時間は\texttt{00}から\texttt{23}までであるので時間の初めの文字が
       \texttt{0}と\texttt{1}のときと、\texttt{2}のときで分ける必要があ
       る。
 \item 時間の先頭が\texttt{0}と\texttt{1}のときはそのあとの文字は
       \texttt{0}から\texttt{9}まで取れるので、
       \texttt{[01]\textbackslash d} となる。
 \item \texttt{2}ではじまるときは\texttt{0}から\texttt{3}まで取れるので、
\texttt{2[0-3]}となる。
 \item 同様に、分と秒は先頭の文字が\texttt{0}から\texttt{5}までであるの
       で \texttt{[0-5]\textbackslash d}となる。
 \item \texttt{|}の範囲を限定す
			 るため時間のところの\texttt{()}を忘れないこと。
\end{itemize}
求めるものは次のとおりである。
{\Large
\begin{Verbatim}
^([01]\d|2[0-3]):[0-5]\d:[0-5]\d$
^([01]\d|2[0-3])(:[0-5]\d){2}$
\end{Verbatim}
}
 \end{frame}
 \subsection{日付を得る}
 \begin{frame}[containsverbatim]
	\frametitle{日付を分ける--ほしい部分を指定(\texttt{match}を利用)}
	\begin{itemize}
	 \item \Verb-/(\d+)年(\d+)月(\d+)日/-
				 \begin{itemize}
					\item 戻り値の1,2, 3番目に求める値が入る
					\item したがって分割代入\texttt{[,y,m,d]}で
								変数に代入ができる。
				 \end{itemize}
	 \item この形ならば\verb-/\d+/g-で十分
	\end{itemize}
 \end{frame}
 \begin{frame}[containsverbatim]
	\frametitle{日付を分ける--区切る部分を指定(\texttt{split}を利用)}
	\begin{itemize}
	 \item \texttt{/年|月|日/}\\
					戻り値の 0, 1,2番目に求める値が入る
	 \item 文字クラスでもよいが「平成」が付いた場合には対応できない
	 \item \Verb-/\D+/-(数字以外で分ける)
	\end{itemize}
 \end{frame}
 \iffalse\else
\subsection{正規表現のマッチを確認する}
\newcommand{\ShowCh}[1]{{\Large \texttt{#1}}}
\newcommand{\Stringi}{%
\begin{center}
\begin{tabular}{*{10}{c}}
 1&2&3&4&5&6&7&8&9&10\\
 \ShowCh{a}&
 \ShowCh{a}&
 \ShowCh{a}&
 \ShowCh{a}&
 \ShowCh{b}&
 \ShowCh{a}&
 \ShowCh{a}&
 \ShowCh{a}&
 \ShowCh{b}&
 \ShowCh{b}\\
\end{tabular}
\end{center}
}
\newcommand{\Stringii}{%
\begin{center}
\setlength{\tabcolsep}{0.2em}
\begin{tabular}{*{20}{c}}
 1&2&3&4&5&6&7&8&9&10&11&12&13&14&15&16&17&18&19%&20
\\
 \ShowCh{a}&
 \ShowCh{b}&
 \ShowCh{c}&
 \ShowCh{c}&
 \ShowCh{b}&
 \ShowCh{c}&
 \ShowCh{c}&
 \ShowCh{k}&
 \ShowCh{k}&
 \ShowCh{c}&
 \ShowCh{c}&
 \ShowCh{a}&
 \ShowCh{a}&
 \ShowCh{M}&
 \ShowCh{M}&
 \ShowCh{a}&
 \ShowCh{a}&
 \ShowCh{c}&
 \ShowCh{c}\\
\end{tabular}
\end{center}
}
\newcommand{\Stringiii}{%
\begin{center}
\setlength{\tabcolsep}{0.2em}
\begin{tabular}{*{20}{c}}
 1&2&3&4&5&6&7&8&9&10&11&12&13&14&15&16&17&%18&19%&20
\\
 \ShowCh{a}&
 \ShowCh{b}&
 \ShowCh{c}&
 \ShowCh{c}&
 \ShowCh{b}&
 \ShowCh{c}&
 \ShowCh{c}&
 \ShowCh{k}&
 \ShowCh{k}&
 \ShowCh{c}&
 \ShowCh{c}&
 \ShowCh{a}&
 \ShowCh{a}&
 \ShowCh{M}&
 \ShowCh{M}&
 \ShowCh{a}&
 \ShowCh{a}\\
\end{tabular}
\end{center}
}
\newcommand{\Stringiv}{%
\begin{center}
\setlength{\tabcolsep}{0.2em}
\begin{tabular}{*{20}{c}}
 1&2&3&4&5&6&7&8&9&10&11&12&13&14&15&16&17&18&19%&20
\\
 \ShowCh{a}&
 \ShowCh{b}&
 \ShowCh{c}&
 \ShowCh{c}&
 \ShowCh{b}&
 \ShowCh{c}&
 \ShowCh{c}&
 \ShowCh{k}&
 \ShowCh{k}&
 \ShowCh{c}&
 \ShowCh{c}&
 \ShowCh{a}&
 \ShowCh{a}&
 \ShowCh{M}&
 \ShowCh{M}&
 \ShowCh{c}&
 \ShowCh{c}&
 \ShowCh{a}&
 \ShowCh{a}
\\
\end{tabular}
\end{center}
}


\begin{frame}[containsverbatim]
 \frametitle{\protect\texttt{"aaaabaaabb".match(/.*b/);}}
\Stringi
\begin{Verbatim}
["aaaabaaabb"]
\end{Verbatim}
\begin{itemize}
 \item 最後に\texttt{b}が来る文字列がマッチ
 \item \texttt{.*}は貪欲であるのでで
       きるだけ長い任意の文字列とマッチする
 \item 最後の\texttt{b}は10番目のものが対応
\end{itemize}
\end{frame}
\begin{frame}[containsverbatim]
 \frametitle{\protect\texttt{"aaaabaaabb".match(/.*b/g);}}
\begin{Verbatim}
["aaaabaaabb"]
\end{Verbatim}
前と同じ理由により、結果は同じ
\end{frame}
\begin{frame}[containsverbatim]
  \frametitle{\protect\texttt{"aaaabaaabb".match(/.*?b/);}}
\Stringi
\begin{Verbatim}
["aaaab"]
\end{Verbatim}
\begin{itemize}
 \item \texttt{b}で終わる文字列を表している
 \item その前の\texttt{.*?}は非貪欲なマッ
       チをする
 \item 最後の\texttt{b}は5番目のもの
\end{itemize}
\end{frame}
\begin{frame}[containsverbatim]
  \frametitle{\protect\texttt{"aaaabaaabb".match(/.*?b/g);}}
\Stringi
 \begin{Verbatim} 
["aaaab", "aaab", "b"]
\end{Verbatim}
\begin{itemize}
 \item 前と同様に非貪欲なマッチング
 \item \texttt{g}オプションがついているの
       で一つ目以降見つかった位置から再度マッチするものを探す
 \item 全体で3つ答えを返す
\end{itemize}
\end{frame}
\iffalse
\begin{frame}[containsverbatim]
   \frametitle{\protect\texttt{"aaaabaaabb".match(/.*?b\textbackslash B/);}}
\Stringi
\begin{Verbatim}
["aaaab"]
\end{Verbatim}
\begin{itemize}
 \item 非貪欲なマッチング
 \item \texttt{b}の位置が単語境界以外のところ
       (\texttt{\textbackslash B})を探す
 \item 5番目の\texttt{b}は単語境界に
       いないのでここまでがマッチ
\end{itemize}
\end{frame}
\begin{frame}[containsverbatim]
   \frametitle{\protect\texttt{"aaaabaaabb".match(/.*?b\textbackslash B/g);}}
\Stringi
\begin{Verbatim}
["aaaab", "aaab"]
\end{Verbatim}
\begin{itemize}
 \item 繰り返しの探索をする
 \item \texttt{b}が単語境界に
       いないものを探す
 \item 10番目の\texttt{b}は候補にならない。
\end{itemize}
\end{frame}
\begin{frame}[containsverbatim]
   \frametitle{\protect\texttt{"aaaabaaabb".match(/.*(?=b)/);}}
\Stringi
\begin{Verbatim}
["aaaabaaab"]
\end{Verbatim}
\begin{itemize}
 \item 任意の文字列(\texttt{.*})でそのあとが\texttt{b}であるもの
       (\texttt{(?=b)})を探す
 \item 貪欲な探索なので10番目の\texttt{b}が
       \texttt{(?=b)}で指定されたもの
 \item マッチした部分にはこの部分
       が含まれない
\end{itemize}
\end{frame}
\begin{frame}[containsverbatim]
   \frametitle{\protect\texttt{"aaaabaaabb".match(/.*?(?=b)/);}}
\Stringi
\begin{Verbatim}
["aaaa"]
\end{Verbatim}
\begin{itemize}
 \item 非貪欲なマッチ
 \item \texttt{(?=b)}で指定された部分が5番目の\texttt{b}
\end{itemize}
\end{frame}
\begin{frame}[containsverbatim]
   \frametitle{\protect\texttt{"aaaabaaabb".match(/.*?(?=b)/g);}}
\Stringi
\begin{Verbatim}
["aaaa", "", "aaa", "", ""]
\end{Verbatim}
\begin{itemize}
 \item 非貪欲なマッチで繰り返しを行うもの
 \item 戻り値に空文字列があ
       るのは、一度\texttt{(?=b)}でマッチした処理を行った後、もう一度
       \texttt{b}のところから探索を始めているためではないかと考えられる
\end{itemize}
\end{frame}
 \fi
\begin{frame}[containsverbatim]
   \frametitle{\protect\texttt{"abccbcckkccaaMMaacc".match(/((.)\textbackslash2).*\textbackslash1/);}}
\Stringii
\begin{Verbatim}
["ccbcckkccaaMMaacc", "cc", "c"]
\end{Verbatim}
\begin{itemize}
 \item 正規表現の\texttt{(.)}は左から数えて2番目の括弧になる
 \item この部分にマッ
      チした文字は\texttt{\textbackslash2}で参照できる
 \item \texttt{(.)\textbackslash2}は同じ文字が2つ並ぶものにマッチ
 \item この部分
      全体が再び括弧でくくられているので、この部分が
      \texttt{\textbackslash1}で参照できる
 \item この正規表現は同じ2つの文字
      で挟まれた文字列にマッチする
 \item 中央部の正規表現は任意の文
      字列を表す
 \item 貪欲なマッチなのではじめに現れる同じ文字が2つ続く3,4番
      目の\texttt{cc}が一番最後に現れる18番目と19番目の\texttt{cc}と組み
      合わされる
 \item グローバルな検索ではないときには戻り値に\texttt{\textbackslash1}と
\texttt{\textbackslash2}が含まれる。
\end{itemize}
\end{frame}
\begin{frame}[containsverbatim]
   \frametitle{\protect\texttt{"abccbcckkccaaMMaacc".match(/((.)\textbackslash2).*\textbackslash1/g);}}
\Stringii
\begin{Verbatim}
["ccbcckkccaaMMaacc"]
\end{Verbatim}
\begin{itemize}
 \item \texttt{g}フラグが付いているが、条件に合うものは一つしかない
 \item 戻り値に\texttt{\textbackslash1}と
\texttt{\textbackslash2}が含まれない
\end{itemize}
\end{frame}
\begin{frame}[containsverbatim]
   \frametitle{\protect\texttt{"abccbcckkccaaMMaacc".match(/((.)\textbackslash2).*?\textbackslash1/);}}
\Stringii
\begin{Verbatim}
["ccbcc", "cc", "c"]
\end{Verbatim}
\begin{itemize}
 \item 前問と異なり、中央部の任意の文字列が非貪欲になっている
 \item 3,4番目と6,7
       番目の\texttt{cc}が対応
 \item 戻り値の配列の2番目と3番目は
\texttt{\textbackslash1}と
\texttt{\textbackslash2}
\end{itemize}
\end{frame}
\begin{frame}[containsverbatim]
   \frametitle{\protect\texttt{"abccbcckkccaaMMaacc".match(/((.)\textbackslash2).*?\textbackslash1/g);}}
\Stringii
\begin{Verbatim}
["ccbcc", "ccaaMMaacc"]
\end{Verbatim}
\texttt{g}が付いているのでさらに
10番目と11番目、18番目と19番目の\texttt{cc}が対応
\end{frame}
\begin{frame}[containsverbatim]
 \frametitle{\protect\texttt{"abccbcckkccaaMMaa".match(/((.)\textbackslash2).*\textbackslash1/);}}
\Stringiii
\begin{Verbatim}
["ccbcckkcc", "cc", "c"]
\end{Verbatim}
\begin{itemize}
 \item これまでの文字列から最後の2文字を取り除いた文字列で同じことを行っ
       ている
 \item 3,4番目の\texttt{cc}に対応するのは11,12番目のもの
\end{itemize}。
\end{frame}
\begin{frame}[containsverbatim]
 \frametitle{\protect\texttt{"abccbcckkccaaMMaa".match(/((.)\textbackslash2).*\textbackslash1/g);}}
\Stringiii
 \begin{Verbatim}
["ccbcckkcc", "aaMMaa"]
\end{Verbatim}
\begin{itemize}
 \item \texttt{g}フラグが付いているのでマッチした部分列が返される
 \item 3,4番目の
      と10,11番目の \texttt{cc}が対応
 \item そのあとの部分列で12,13番目
       と16,17番目の\texttt{aa}が対応する
 \item マッチした部分列は2つ
\end{itemize}
\end{frame}
\begin{frame}[containsverbatim]
 \frametitle{\protect\texttt{"abccbcckkccaaMMccaa".match(/((.)\textbackslash2).*\textbackslash1/g);}}
\Stringiv
\begin{Verbatim}
["ccbcckkccaaMMcc"]
\end{Verbatim}
\begin{itemize}
 \item これまでの文字列と最後の4つが入れ替わっている
 \item 3,4番目と16,17番目の
       \texttt{cc}が対応
\end{itemize}
\end{frame}
\begin{frame}[containsverbatim]
 \frametitle{\protect\texttt{"abccbcckkccaaMMccaa".match(/((.)\textbackslash2).*?\textbackslash1/g);}}
\Stringiv
\begin{Verbatim}
["ccbcc", "ccaaMMcc"]
\end{Verbatim}
\begin{itemize}
 \item 前問と異なり、非貪欲な任意の部分文字列を途中に取る
 \item 3,4番目と6,7番
       目の\texttt{cc}、10,11番目と16,17番目の\texttt{cc}が対応する
\end{itemize}
\end{frame}
\fi
\subsection{正規表現の利用}
 \begin{frame}[containsverbatim]
 \frametitle{エラー処理の例(改良版)}
	2回前の授業で次のエラー処理を紹介した。
{\small
\begin{Verbatim}
function Person(name, y, m, d){
  if(name === "") throw new Error("名前がありません");
  this.name = name;
  this.year = y;
  if(m<1 || m>12) throw new Error("月が不正です");
  var date = new Date(y,m,0);
  if(d<1 || d>date.getDate()) throw new Error("日が不正です");
  this.month = m,
  this.day = d
}
Person.prototype = {
...
}
\end{Verbatim}
 }
\end{frame}
 \begin{frame}[containsverbatim]
	\frametitle{エラー処理が不十分}
 このリストでは十分なエラー処理がなされていない。
 \begin{itemize}
	\item 数に変換されない文字列の入力に対しては値が\texttt{NaN}になる。
	\item その結果、たとえば月の値の評価\texttt{m<1}は\texttt{false}となり、
				エラーチェックを通り抜けてしまう。
	\item 対処法としては\Verb+!(m>=1 && m<=12)+とすることも考えられるが、
				これでも小数点付きの数が排除できない。
	\item Web アプリケーションではテキストボックスの入力は文字列になるので、
				文字列の段階でチェックするほうが楽
 \end{itemize}
 \end{frame}
\begin{frame}[containsverbatim]
 \frametitle{\protect\texttt{Person2}の修正}
\texttt{Person}オブジェクトの範囲を外部から入力させる
ときに文字列を数に変換する前に数値リテラルになっているかを判定することでプログラ
 ムが不正な値を受け付けないようにできる。

 整数値だけにするのであれば正規表現は\Verb-/^\d+$/-である。
\end{frame}
\begin{frame}[containsverbatim]
 \frametitle{\protect\texttt{Person2}の修正}\small
\begin{Verbatim}
<title>エラーオブジェクト(改良)</title>
function Person(name, y, m, d){
  if(name === "") throw new Error("名前がありません");
  this.name = name;
  this.year = checkNum(y,1900,2020,"年");
  this.month = checkNum(m,1,12,"月");
  var date = new Date(y,this.month,0);
  this.day = checkNum(d,1,date.getDate(),"日");
}
function checkNum(s, low, high, mes) {
  if(s.match(/^\d+$/)) {
    if(s>=low &&s<=high) return s-0;//文字列を数字に変換
  }
  throw new Error(mes+"が不正です");
}
	
...
\end{Verbatim}
\end{frame}
\begin{frame}[containsverbatim]
 \frametitle{解説}
\texttt{prompt()}で戻ってきた文字列が数字だけからなっているかをチェック
 したうえで、与えられた範囲内にあるかを調べている。
\end{frame}
\begin{frame}[containsverbatim]
 \frametitle{外部入力のチェックを!}
\begin{itemize}
 \item 外部からのデータの入力に対しては、データを吟味してから利用する
 \item 特にWebページのテキストボックスからのデータ入力を利用し
て不正行為を行う手法が知られている
\end{itemize}
\end{frame}
\section{HTML文書の構成}
\begin{frame}[containsverbatim]
\frametitle{\texttt{Google Maps} の利用(1)}
次のリストは\texttt{Google Maps} を利用して地図を表示するものである。
\LISTN{07firstMap.html}{1}{7}{\footnotesize}
\end{frame}
\begin{frame}[containsverbatim]
\frametitle{\texttt{Google Maps} の利用(2)}
\LISTN{07firstMap.html}{8}{last}{\footnotesize}
\end{frame}
\begin{frame}[containsverbatim]
 \frametitle{リストの解説(1)}
\begin{itemize}
 \item 1行目はHTML文書の\texttt{DOCTYPE}宣言である。この形はHTML5にお
       けるもの
 \item 2行目はこのHTML文書のルート要素と呼ばれるものである。最後の26行目
       の \Elm{/html}までが有効となる。
 \item すべての要素はこの範囲になければならない。
\end{itemize}
\end{frame}
\begin{frame}[containsverbatim]
 \frametitle{リストの解説(2)--\texttt{<head>}}
3行目から始まる\Elm{head}はブラウザに表示されない、いろいろ
       なHTML文書の情報を表す。
 \begin{itemize}
  \item 4,5行目はこの文書の形式や文字集合を記述している。ここでは
	内容は\texttt{text/html}の形式、つまり、テキストで書かれ
	た\texttt{html}の形式で書かれていることを表す。
	\footnote{このような方法でファイルのデータ形式を表すこと
	をMIME(Multipurpose
	Internet Mail Extension)タイプと呼ぶ。元来、テキストデー
	タしか扱えない電子メールに様々なフォーマットのデータを扱
	えるようにする規格である。}
  \item 6行目の\Elm{title}はブラウザのタブに表示される文字
	列を指定している。
  \item 6,7行目はGoogle Maps のライブラリーを読み込むため
	のものである。このようにJavaScriptのプログラムは外部ファ
	イルとすることができる。
  \item 9行目から19行目はHTML文書内に書かれたJavaScriptである。
	詳しい解説は後の授業で行う。
  \item 20行目はHTML文書の見栄えなどを規定する\texttt{CSS}ファイ
	ルを外部から読み込むことをしている。
 \end{itemize}
\end{frame}
\begin{frame}[containsverbatim]
 \frametitle{リストの解説(2)--\texttt{<body>}}
HTML文書で実際にブラウザ内で表示される情報は\Elm{body}要素
       内に現れる。
\begin{itemize}
 \item このリストではGoogle Mapsを表示するための\Elm{div}要素が一つ
       あるだけである。このとき、\Elm{div}は\Elm{body}の子要素
       であるといい、\Elm{body}は\Elm{div}の親要素という。
 \item 各要素名または要素の終了を示すタグ(\texttt{<...>})の間に文字列が
       ある場合、その部分はテキストノードと呼ばれるノードが作成されてい
       る。
\end{itemize}
\end{frame}
\begin{frame}[containsverbatim]
 \frametitle{要素の構成について}
各要素は\texttt{<}と\texttt{}の中に現れる。初めに現れる文字列が要素名で
 あり、そのあとに属性と属性値の組がいくつか並ぶ。
\begin{itemize}
 \item 属性とその属性値は\texttt{=}で結ばれる。
 \item 属性値は\texttt{”}ではさまれた文字列として記述
 \item \Elm{script}要素では属性\texttt{type}と\texttt{src}が
       設定
 \item 24行目の\Elm{div}要素では属性\texttt{id}に属性値
       \texttt{map\textunderscore canvas}を設定している。なお、この要素
       はCSSによっても属性が定義されている。
\end{itemize}
\end{frame}
\begin{frame}[containsverbatim]
 \frametitle{20行目で参照しているCSSファイル}
 \LISTN{map.css}{1}{last}{\normalsize}
 \begin{itemize}
 \item CSSの各構成要素はHTML文書の要素を選択するセレクタ(ここでは
       \Verb+#map_canvas+)とそれに対する属性値の並び([属性]:[属性値];)か
       らなる。
 \item \Verb+#+で始まるセレクタはそのあとの文字列を\Elm{id}の属性値に
       持つ要素に適用される。
 \item したがって、ここの規則は24行目の\Elm{div}要素に適用
 \item その内容はGoogle Maps が表示される画面の大きさ(\texttt{width}と
       \texttt{height})、配置の位置(\texttt{float})と要素の外に配置され
       る空白(\texttt{margin})を指定
 \end{itemize}
\end{frame}
\section{CSSの利用}
\begin{frame}[containsverbatim]
 \frametitle{CSSとは}
\begin{itemize}
 \item カスケーディングスタイルシート(CSS)はHTML文書の要素の表示方法を指定する
もの
 \item CSSはJavaScriptからも制御可能
 \item 文書のある要素に適用されるスタイルルールは、複数の異なるルールを結合(カ
スケード)したもの
 \item スタイルを適用するためには要素を選択するセレクタで選ぶ。
\end{itemize}
\end{frame}
\begin{frame}[containsverbatim]
 \frametitle{セレクタの種類}
セレクタの種類は配布資料を参照のこと
いくつか注意する点を挙げる。
\begin{itemize}
 \item 属性\texttt{id}の属性値の前に\texttt{\#}をつけることでその要素が
       選ばれる。
 \item 属性\texttt{class}の属性値の前に\texttt{.}をつけることでその要素が
       選ばれる。
 \item \texttt{nth-child(n)}には単純な式を書くことができる。
このセレクタは複数書いても
       よい。
 \item \Verb+E F+ と \Verb+E > F+ の違いを理解しておくこと。たとえば
       \texttt{div div}というセレクタは途中に別の要素が挟まれていてもよ
       い。また、\texttt{<div>}要素が3つある場合にはどのような2つの組み
       合わせも対象となる。
\end{itemize}
\end{frame}
\section{DOMの利用}
\subsection{DOMとは}
\begin{frame}[containsverbatim]
 \frametitle{DOMとは}
\begin{itemize}
 \item Document Object Model(DOM)はHTML文書などの要素をノードとしたツリー構造で
管理する方法
 \item DOMのメソッドやプロパティを使うことで各要素にアクセ
スしたり、属性値やツリーの構造を変化させることが可能
 \item DOMの構造は開発者ツールなどで見ることができる。
\end{itemize}
\end{frame}
\begin{frame}[containsverbatim]
 \frametitle{代表的なブラウザの開発者ツール}
\begin{itemize}
 \item  Google Chromeでは開発者ツールからElementsタブで確
       認する。ここで要素上で右クリックしてEdit as HTML を選択するとテキ
       ストとして編集できる。
 \item FireFoxでは開発ツールから「開発ツールを表示」を
       選択し、インスペクタタブでDOMツリーが確認できる。要素上で右クリッ
       クから「HTMLとして編集」とするとテキストとして編集できる。
 \item IEでは開発者ツールを開き左にあるタブの一番上にあるDOM Explorerで同様の
       ことができる。
\end{itemize}
\end{frame}
\subsection{DOM のメソッド}
\begin{frame}[containsverbatim]
 \frametitle{DOM のメソッド}
DOM では DOM ツリーを操作するためにメソッドやプロパティが規定されている。
 メソッドとはそのオブジェクトに対する操作である。次のような手段を
       提供している。
\begin{itemize}
 \item 条件に合う要素または要素のリストを得る。
 \item 要素の属性を参照、変更ができる。
 \item 要素を新規に作成する。
 \item ある要素に子要素を追加したり、取り除いたりする。
\end{itemize}
\end{frame}
\newcommand{\DOMM}{\texttt}
\begin{frame}[containsverbatim]
 \frametitle{条件に合う要素または要素のリストを得る(1)}
\begin{itemize}
 \item \DOMM{getElementById}{(id)}\\
      属性\texttt{id}の値が引数\texttt{id}である要素を得る。
 \item \DOMM{getElementsByTagName}{(Name)}\\
     要素名が\texttt{Name}である要素のリストを得る。リストの各要素は
	  配列と同様に\texttt{[ ]}で参照できる。
 \item \DOMM{getElementsByClassName}{(Name)}\\
     属性\texttt{class}の値が\texttt{Name}である要素のリストを得る。リス
	  トの各要素は配列と同様に\texttt{[ ]}で参照できる。
 \item \DOMM{getElementsByName}{(Name)}\\
     属性\texttt{name}が\texttt{Name}である要素のリストを得る。得られた各要素は
	  配列と同様に\texttt{[ ]}で参照できる。
\end{itemize}
\end{frame}
\begin{frame}[containsverbatim]
 \frametitle{条件に合う要素または要素のリストを得る(2)}
\begin{itemize}
 \item \DOMM{querySelector}{(selectors)}\\
     \texttt{selectors}で指定されたCSSのセレクタに該当する一番初めの要素
	  を得る
 \item \DOMM{querySelectorAll}{(selectors)}\\
     \texttt{selectors}で指定されたCSSのセレクタに該当する要素のリストを得る。
	  配列と同様に\texttt{[ ]}で参照できる
\end{itemize}
\end{frame}
\begin{frame}[containsverbatim]
 \frametitle{要素の属性を参照、変更ができる}
\begin{itemize}
 \item {\DOMM{getAttribute}{(Attrib)}} \\
     対象要素の属性\texttt{Attrib}の値を得る。戻り値の型は文字列である。
 \item {\DOMM{setAttribute}{(Attrib,Val)}} \\
     対象要素の属性\texttt{Attrib}の値を\texttt{Val}にする。数を渡しても
	  文字列に変換される。
 \item {\DOMM{hasAttribute}{(Attrib)}} \\
     対象要素に属性\texttt{Attrib}がある場合は\texttt{true}を、ない場合
 は\texttt{false}を返す。
 \item {\DOMM{removeAttribute}{(Attrib)}}\\
     対象要素の属性\texttt{Attrib}を削除
\end{itemize}
\end{frame}
\begin{frame}[containsverbatim]
 \frametitle{要素を新規に作成する}
\begin{itemize}
 \item \DOMM{createElement}{(Name)} \\
     \texttt{Name}で指定した要素を作成
 \item \DOMM{createElementNS}{(NS,Name)} \\
     {名前空間}\texttt{NS}で定義されている要素\texttt{Name}を作成
 \item \DOMM{createTextNode}{(text)}\\
     \texttt{text}を持つテキストノードを作成
 \item {\DOMM{cloneNode}{(bool)}} \\
\texttt{bool}が  \texttt{true}のときは対象要素の子要素すべてを、%複製する。
  \texttt{false}のときは対象要素だけの複製を作る。
\end{itemize}
\end{frame}
\begin{frame}[containsverbatim]
 \frametitle{名前空間}
\begin{itemize}
 \item 指定した要素が定義されている規格を指定するもの
 \item 一つの文書内で複数の規格を使用する場合、作成する要素がどこで定義
されているのかを指定
 \item 異なる規格で同じ要素名が定義されていてもそれらを区別することが可能
 \item 通常のHTML文書では\texttt{ http://www.w3.org/1999/xhtml}を指定
\end{itemize}
\end{frame}
\begin{frame}[containsverbatim]
 \frametitle{子要素の追加と削除}
\begin{itemize}
 \item {\DOMM{appendChild}{(Elm)}} \\
  \texttt{Elm}を対象要素の最後の子要素として付け加える。\texttt{Elm}がすでに
	  対称要素の子要素のときは元の位置から最後の位置に移動
 \item {\DOMM{insertBefore}{(newElm, PElm)}} \\
   対象要素の子要素\texttt{PElm}の前に\texttt{newElm}を子要素として付け
  加える。\texttt{Elm}がすでに対称要素の子要素のときは元の位置から指定さ
	  れた位置に移動
 \item \DOMM{removeChild}{(Elm)} \\
対象要素の子要素  \texttt{Elm}を取り除く。
 \item \DOMM{replaceChild}{(NewElm, OldElm)}
   対象要素に含まれる子要素  \texttt{OldElm}を\texttt{NewElm}で置き換え
       る。
 \item \DOMM{setValue}{(value)} \\
 {対象のテキストノードの値を\texttt{value}にする。}
\end{itemize}
\end{frame}
\begin{frame}[containsverbatim]
 \frametitle{プルダウンメニューの例}
1月から12月までを選択できるプルダウンメニューを作成
\begin{Verbatim}[fontsize=\small]
<body>
  <form id="menu">
    <select>
      <option value="1">1月</option>
      <option value="2">2月</option>
      <option value="3">3月</option>
      <option value="4">4月</option>
      <option value="5">5月</option>
      <option value="6">6月</option>
      <option value="7">7月</option>
      <option value="8">8月</option>
      <option value="9">9月</option>
      <option value="10">10月</option>
      <option value="11">11月</option>
      <option value="12">12月</option>
    </select>
  </form>
</body>
\end{Verbatim}
\end{frame}
\begin{frame}[containsverbatim]
 \frametitle{プルダウンメニュ――解説}
\begin{itemize}
 \item ユーザからの入力を受け付ける要素は通常、\Elm{form}要素内に記
       述
 \item プルダウンメニュ-の要素名は\Elm{select}
 \item 選択する内容は\Elm{option}要素
 \item \Elm{option}要素の属性\texttt{value}の値が選択した値として
       利用できる。
 \item \Elm{option}要素内の文字列(テキストノード)がプルダウンメニューに
       表示される
 \item \Elm{select}は\Elm{form}の子要素であり、各\Elm{option}は
       \Elm{select}の子要素
\end{itemize}
\end{frame}
\begin{frame}[containsverbatim]
 \frametitle{プルダウンメニュ――プログラムで作成(1)}
\LISTN{07-02pulldown.html}{7}{11}{\normalsize}
\end{frame}
\begin{frame}[containsverbatim]
 \frametitle{プルダウンメニュ――プログラムで作成(2)}
\LISTN{07-02pulldown.html}{12}{last}{\normalsize}
\end{frame}
\begin{frame}[containsverbatim]
 \frametitle{プルダウンメニュ――プログラムで作成--解説}
\begin{itemize}
 \item 8行目の\texttt{window.onload}はファイルのロードが終わった後に発生
       するイベントを表す。\texttt{function()}が設定されているのでこの関
       数がロード後、実行
 \item 10行目では26行目から27行目にある\Elm{form}要素を得ている。
 \item 11行目では\Elm{select}要素を作成している。
 \item 13行目で11行目で作成した\Elm{select}要素を\Elm{form}要素の子要素
       に設定している。
 \item 14行目から始まる\texttt{for}ループで12個の\Elm{option}要素を作成
       し、\Elm{select}要素の子要素としている。
       \begin{itemize}
	\item 15行目で\Elm{option}要素を新規に作成している。
	\item 16行目で、その要素の属性\texttt{value}に値を設定している。
	\item 17行目ではその\Elm{option}要素を\Elm{select}要素の子要素と
	      している。
	\item さらに、18行目では表示する文字列をもつテキストノードを作成
	      し、19行目でそれを\Elm{option}要素の子要素としている。
       \end{itemize}
\end{itemize}
\end{frame}
\newcommand{\DOMP}{\texttt}
\subsection{DOMのプロパティ}
\begin{frame}[containsverbatim]
 \frametitle{DOMのプロパティ}
\begin{table}[ht]
\caption{DOM要素に対するプロパティ(1)}\label{PropertyDOM}
\begin{center}
 \begin{tabular}{|c|m{16zw}|}
  \hline
プロパティ名  &
 \hspace*{\fill}説{\hfill}明\hspace*{\fill}\rule{0em}{0em}\\ \hline
\DOMP{firstChild} &指定された要素の先頭にある子要素 \\ \hline
\DOMP{lastChild} & 指定された要素の最後にある子要素\\ \hline
\DOMP{nextSibling} & 指定された子要素の次の要素\\ \hline
\DOMP{previousSibling} & 現在の子要素の前にある要素\\ \hline
\DOMP{parentNode} & 現在の要素の親要素\\ \hline
\DOMP{hasChildNodes} &その要素が子要素を持つかどうか \\ \hline
\DOMP{nodeName}& その要素の要素名前\\ \hline
\DOMP{nodeType}& 要素の種類($1$は普通の要素、$3$はテキストノード)\\ \hline
\DOMP{nodeValue}&(テキスト)ノードの値 \\ \hline
\DOMP{childNodes}& 子要素の配列\\ \hline
 \end{tabular}
\end{center}
\end{table}
\end{frame}
\begin{frame}
 \begin{table}[ht]
\caption{DOM要素に対するプロパティ(2)---DOM4}
\begin{center}
 \begin{tabular}{|c|m{13zw}|}
  \hline
プロパティ名  &
 \hspace*{\fill}説{\hfill}明\hspace*{\fill}\rule{0em}{0em}\\ \hline
\DOMP{children}& 子要素のうち通常の要素だけからなる要素の配列\\ \hline
\DOMP{firstElementChild} &指定された要素の先頭にある通常の要素である子要素\\ \hline
\DOMP{lastElementChild} & 指定された要素の最後にある通常の要素である子要素\\ \hline
\DOMP{nextElementSibling} & 指定された子要素の次の通常の要素\\ \hline
\DOMP{previousElementSibling} & 現在の子要素の前にある通常の要素\\ \hline
 \end{tabular}
\end{center}
\end{table}
\end{frame}
\iffalse
\section{レポート問題}
\begin{frame}[containsverbatim]
 \frametitle{レポート問題}
課題7.1から7.4までレポートにして提出のこと。
\end{frame}
\fi
\end{document}
