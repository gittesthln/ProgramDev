%-*- coding: utf-8 -*-
\documentclass[a4j]{jbook}
\usepackage{fancyvrb,array,longtable,xcolor,amsmath,graphicx%
,hyperref
}
\title{ソフトウェア開発\ifText\else\\ {\large -- 授業配布版 --}\fi}
\author{平野 照比古}
\date{}
\newif\ifText
\newif\ifAnswer
\newtheorem{Prob}{課題}[chapter]
\newtheorem{Exec}{実行例}[chapter]
\newcommand{\ElmJ}[1]{\texttt{#1}}
\newcommand{\ElmJF}[2]{\texttt{#1(#2)}}
\newcommand{\ElmJP}[1]{\texttt{#1}}
\newcommand{\Event}[1]{\texttt{#1}}
\newcommand{\KeyItem}[1]{#1}

\newcommand{\ListAttribsF}[4]{%
  \begin{table}[ht]
     \caption{#2}\label{#1}%
   \hfill%\begin{center}
    \begin{tabular}[t]{#3}
       \hline
       \ShowExpF#4\relax\relax\relax
    \end{tabular}
    \hspace*{\fill}%\end{center}
  \end{table}
}
%
\newcommand{\ListAttribs}[3]{%
  \begin{table}[ht]
     \label{#1}\caption{#2}%
    \begin{center}
      \begin{tabular}[t]{|c|c|c|}
        \hline
        属性名& 意味& 属性値\\\hline
       \ShowExp#3\relax\relax\relax
      \end{tabular}
    \end{center}
  \end{table}
}
%
\newcommand{\ShowExpF}[3]{%
%  \def\TTT{#1}
%  \ifx\TTT\RELAX\else
  \ifx#1\relax\else
    {#1}&#2&#3\\\hline
    \expandafter\ShowExpF%
  \fi}
%
\newcommand{\ShowExp}[3]{%
%   \def\TTT{#1}
%   \ifx\TTT\RELAX\else
   \ifx#1\relax\else
     \Attrib{#1}&#2&#3\\\hline
     \expandafter\ShowExp%
   \fi}
%
\newcommand{\Attrib}[1]{\texttt{#1}}
\newcommand{\JS}{JavaScript}

\newcommand{\LIST}[3]{\LISTN{#1}{#2}{#3}{\footnotesize}}
\newcommand{\LISTN}[4]{%
\renewcommand{\theFancyVerbLine}{#4\arabic{FancyVerbLine}}
\VerbatimInput[numbers=left,fontsize=#4,
numbersep=4pt,firstline=#2,lastline=#3]{html/#1} 
}
\newcommand{\Strict}{\texttt{strict}モード}
\newcommand{\Correct}[1]{{\color{red}#1}}