\ProblemNN{
\input 17prob08-01.tex
\begin{Prob}\upshape\Must
 実行例8.1において次のことを行いなさい。
 \begin{enumerate}
	\item 10行目にある3つの\Elm{div}の大きさを大きくする。大きさは、ブラ
				ウザの表示画面を小さくしたときに、画面のスクロールバーが出るよう
				にすること。テキストボックスを作成して大きさを変えるようにしても
				よい。
	\item スクロールバーが表示された状態で、10行目にある3つの\Elm{div}の
				場所をクリックしたときの\ElmJ{clientX}などの値を確認する。確認す
				る際には毎回同じような場所をクリックすること。また、スクロールさ
				せて表示される値に違いがあるか確認すること。
	\item 18行目から22行目にあるラジオボタンをクリックしたときのイベントの
				\ElmJ{target}と\ElmJ{currentTarget}の値を確認する。
 \end{enumerate}
\end{Prob}
\begin{Prob}\upshape
 気象庁の天気予報のページ\texttt{http://www.jma.go.jp/jp/yoho/000.html}
 では天気予報の地域を選択する2つのプルダウンメニュー「地方」と「府県」が
 並んでいる。「地方」のプルダウンメニューで選択を変えると、「府県」のプ
 ラウダウンメニューが変化する。このとき、URLが移動している。

 このようなプルダウンメニューをページを移動しないで実現しなさい。次の点
 に注意すること。
 \begin{itemize}
  \item 「地方」のデータとそれに属する「府県」のデータをオブジェクトとし
        て持つ。
  \item そのデータを元にして2つのプルダウンメニューをできればプログラム
        で作成する。
  \item プルダウンメニュー以外は作成する必要はない。
 \end{itemize}
\end{Prob}
}
\RubricC{
復習の目的は次の項目を理解することである。
\begin{itemize}
 \item クリックイベントで得られる情報の理解と利用法
 \item DOMツリーをプログラムから制御する技法が使える
 \item イベントの発生でDOMの構造を変化させる技法を理解する
\end{itemize}
}
{
{\RProbNoM{1\ }{0.2}}{8}
 {
 {オブジェクトリテラルのデータからプルダウンメニューの作成している。}
 {プラウダウンメニューの作成を汎用性がある関数で行っている。}
 {オブジェクトリテラルのデータからラジオボタンのメニューの作成している。}
 {ラジオボタンのメニューの作成を汎用性がある関数で行っている。}
 {結果を説明するために十分な数のキャプチャ画面がある}
 {キャプチャ画面の大きさが結果を説明するために十分である}
 }
 {
 {オブジェクトリテラルのデータからプルダウンメニューの作成しているが改良
 の余地がある。}
 {プラウダウンメニューの作成を汎用性が足りない関数で行っている。}
 {オブジェクトリテラルのデータからラジオボタンのメニューの作成している。}
 {ラジオボタンのメニューの作成を汎用性が足りない関数で行っている。}
 {結果を説明するためにページのソースのキャプチャ画面がある}
 {結果を説明するためにDOMツリーのキャプチャ画面がある}
 {キャプチャ画面の大きさが結果を説明するために少し小さすぎて内容が読めな
 い。}
 {キャプチャ画面の範囲が結果を説明するために一部欠けている。}
 }
 {
 {オブジェクトリテラルのデータからプルダウンメニューの作成していない。}
 {プラウダウンメニューの作成を関数で行っていない。}
 {オブジェクトリテラルのデータからラジオボタンのメニューの作成していない。}
 {ラジオボタンのメニューの作成を関数で行っていない。}
 {結果を説明するためにページのソースのキャプチャ画面がない。}
 {結果を説明するためにDOMツリーのキャプチャ画面がないか、結果を示すため
 に重要な部分が欠けている。}
 {キャプチャ画面の大きさが結果を説明するために小さすぎる。}
 {キャプチャ画面の範囲が結果を説明するためには不十分である。}
 }
 {\RProbNoM{2.1}{0.2}}{4}
 {
 {\Elm{div}の大きさをCSSの値を直して十分大きくしている。}
 {画面を調整して、スクロールバーが表示されている。}
 {テキストボックスから大きさの変更ができるようになっている。}
 {リストやその解説、考察が十分にある}
 }
 {
 {\Elm{div}の大きさをCSSの値を直して大きくしているが、大きさが不十分であ
 る。}
 {画面を調整しないで、スクロールバーが表示されている。}
 {テキストボックスから大きさの変更ができるようになっていない。}
 {リストやその解説、考察がある}
 }
 {
 {\Elm{div}の大きさをCSSの値を直していない。}
 {画面を調整しないので、スクロールバーが表示されていない。}
 {テキストボックスから大きさの変更ができるようになっていない。}
 {リストやその解説、考察が不十分であるかない}
 }
 {\RProbNoM{2.2}{0.2}}{4}
 {
 {スクロールバーが表示された状態でかつスクロールバーを少し移動した状態で
 \ElmJ{clientX}などの値の確認を行っている。}
 {確認している値の種類が十分である。}
 {リストやその解説、考察が十分にある}
 }
 {
 {スクロールバーが表示された状態であるが、スクロールバーの位置が上端また
 は左端の位置にある状態で\ElmJ{clientX}などの値の確認を行っている。}
 {確認する値が4,5種類しかない。}
 {リストやその解説、考察がある。}
 }
 {
 {スクロールバーが表示された状態でない。}
 {スクロールバーの位置が上端または左端の位置にある状態で\ElmJ{clientX}な
 どの値の確認を行っている。}
 {核にしてある値の種類が2,3しゅらういしかない。}
 {リストやその解説、考察がないか不十分である。}
 }
 {\RProbNoM{2.3}{0.2}}{4}
 {
 {ラジオボタンやその文字の部分をクリックしたときの\ElmJ{target}と
 \ElmJ{currentTarget}の値の確認が十分にある。}
 {確認している位置の種類が十分ある。}
 {リストやその解説、考察が十分ある。}
 }
 {
 {ラジオボタンやその文字の部分をクリックしたときの\ElmJ{target}と
 \ElmJ{currentTarget}の値の確認が少しある。}
 {確認している位置の種類が少しある。}
 {リストやその解説、考察がある。}
 }
 {
 {ラジオボタンやその文字の部分をクリックした位置の種類が不十分である。}
 {\ElmJ{target}と\ElmJ{currentTarget}の値の確認がないか不十分である。}
 {確認している位置の種類が足りない。}
 {リストやその解説、考察が不十分であるか全くない。}
 }
 {問題3}{15}
 {
 {「地方」と「府県」の関連付けがなされたデータ構造が作成されている。}
 {データに基づいてプルダウンメニューが構成されている。}
 {\ElmJ{weakMap}またはオブジェクトを用いて「府県」のデータ管理を行っている。}
 {正しく動作する。}
 {リストとその解説が十分にある。}
 }
 {
 {「地方」と「府県」の関連付けが不十分なデータ構造が作成されている。}
 {「府県」のプルダウンメニューが変化のたびごとに新規に作成されている。}
 {データに基づいてプルダウンメニューが構成されているが、汎用性が少し低い。}
 {一部動作におかしいところがある。}
 {リストとその解説がある。}
 }
 {
 {「地方」と「府県」の関連付けがないデータが作成されている。}
 {データに基づいてプルダウンメニューが構成されているない。}
 {ほとんど期待したように動作しない。}
 {リストとその解説が不十分であるか全くない。}
 }
 }