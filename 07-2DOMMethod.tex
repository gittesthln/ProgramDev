表\ref{MethodDOM}は DOM のメソッドのリストである。\texttt{document}だけ
に適用できるものと、すべての要素に適応できるものとがある。一番右の項目
「PHP」はPHPの\texttt{DOMDocument}クラスにおけるサポート状態を示す。
\Yes はサポートされていて\No はサポートされていないこと
を示す。
なお、結果が要素のリストであるものについては通常の配列と同様に\texttt{[ ]}によ
りそれぞれの要素を指定できる。

なお、表中の名前空間(Namespace)とは、指定した要素が定義されている規格を指定するもの
である。一つの文書内で複数の規格を使用する場合、作成する要素がどこで定義
されているのかを指定する。これにより、異なる規格で同じ要素名が定義されて
いてもそれらを区別することが可能となる。通常はルート要素(\HTML では
\ElmJ{html})の名前空間が指定される。
HTML文書では\texttt{ http://www.w3.org/1999/xhtml}を指定する
\footnote{\texttt{http://dev.w3.org/html5/spec-LC/namespaces.html}}。
\newpage
{\setlength{\tabcolsep}{0.2em}
\begin{longtable}{|c|c|m{20em}|c|}
\caption{DOMのメソッド}\label{MethodDOM}
\\  \hline
メソッド名  & {対象要素}&
\hspace*{\fill}説{\hfill}明\hspace*{\fill}\rule{0em}{0em}&{\scriptsize
 PHP}\\ \hline
\endfirsthead
\caption{DOMのメソッド(続き)}
\\  \hline
メソッド名  & {対象要素}&
\hspace*{\fill}説{\hfill}明\hspace*{\fill}\rule{0em}{0em}&{\scriptsize
 PHP}\\ \hline
\endhead
\hline\multicolumn{3}{r}{次ページへ続く}
\endfoot
\hline
\endlastfoot
\DOMM{getElementById}{(id)}&\texttt{document}&
      属性\texttt{id}の値が\texttt{id}の要素を得る。&\Yes \\\hline
\DOMM{getElementsByTagName}{(Name)}&対象要素&
     要素名が\texttt{Name}の要素のリストを得る。&\Yes\\\hline
\DOMM{getElementsByClassName}{(Name)}&対象要素&
     属性\texttt{class}の値が\texttt{Name}の要素のリストを得る。&\No\\\hline
\DOMM{getElementsByName}{(Name)}&\texttt{document}&
 属性\texttt{name}が\texttt{Name}である要素のリストを得る。&\No\\\hline
\DOMM{querySelector}{(selectors)}&対象要素&
     \texttt{selectors}で指定されたCSSのセレクタに該当する一番初めの要素
	  を得る。&\No \\\hline
\DOMM{querySelectorAll}{(selectors)}&対象要素&
     \texttt{selectors}で指定されたCSSのセレクタに該当する要素のリストを得る。
 &\No\\\hline
\DOMM{getAttribute}{(Attrib)}&対象要素&
     対象要素の属性\texttt{Attrib}の値を読み出す。得られる値はすべて文字
	  列である。&\Yes\\ \hline
{\DOMM{setAttribute}{(Attrib,Val)}}  &対象要素&
     対象要素の属性\texttt{Attrib}の値を\texttt{Val}にする。数を渡しても
	  文字列に変換される。&\Yes\\ \hline
{\DOMM{hasAttribute}{(Attrib)}}  &対象要素&
     対象要素に属性\texttt{Attrib}がある場合は\texttt{true}を、ない場合
 は\texttt{false}を返す。&\Yes\\ \hline
{\DOMM{removeAttribute}{(Attrib)}}  &対象要素&
     対象要素の属性\texttt{Attrib}を削除する。&\Yes\\ \hline
%\DOMM{getNodeName}{()}&対象要素&対象要素の要素名を得る。\\\hline
\DOMM{createElement}{(Name)} &\texttt{document}&
     \texttt{Name}で指定した要素を作成する。&\Yes \\ \hline
\DOMM{createElementNS}{(NS,Name)} &\texttt{document}&
     \keyitem{名前空間}\texttt{NS}にある要素\texttt{Name}を作成
	  する。&\Yes \\ \hline
\DOMM{createTextNode}{(text)} &\texttt{document}&
     \texttt{text}を持つテキストノードを作成する。&\Yes\\ \hline
{\DOMM{cloneNode}{(bool)}} &対象要素&
%   対象要素の複製を作る。
\texttt{bool}が
  \texttt{true}のときは対象要素の子要素すべてを、%複製する。
  \texttt{false}のときは対象要素だけの複製を作る。&\Yes\\ \hline
{\DOMM{appendChild}{(Elm)}} &対象要素&
  \texttt{Elm}を対象要素の最後の子要素として付け加える。\texttt{Elm}がすでに
	  対称要素の子要素のときは最後の位置に移動する。&\Yes \\ \hline
{\DOMM{insertBefore}{(newElm, PElm)}} &対象要素&
   対象要素の子要素\texttt{PElm}の前に\texttt{newElm}を子要素として付け
  加える。\texttt{Elm}がすでに対称要素の子要素のときは指定さ
	  れた位置に移動する。 &\Yes\\ \hline
\DOMM{removeChild}{(Elm)} &対象要素& 対象要素の子要素
      \texttt{Elm}を取り除く。&\Yes\\ \hline
\DOMM{replaceChild}{(NewElm, OldElm)} &対象要素& 対象要素の子要素
      \texttt{OldElm}を\texttt{NewElm}で置き換える。&\Yes\\ \hline
%\DOMM{setValue}{(value)} &\small テキストノード& {対象のテキストノードの値を
%	  \texttt{value}にする。}\\ \hline
\end{longtable}
}