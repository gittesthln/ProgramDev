%
% enshuuhead.tex 2010 版
% 
%
%
%\usepackage[noreplace]{otf}
\usepackage{array%,times
}
%\usepackage{timesnew}
%\usepackage{mathabx}
\usepackage{amsmath,amssymb,graphicx,fancybox,
%epic}
%\usepackage{eepic}
multibox,fancyvrb}
%
% the following macros are taken from Eijkhoout'book
%	TeX by Topic Chap. 13.8.6 (p.137)
%   some modification made by T. Hilano
%
\def\ifEq#1#2{\ifnum#1=#2\relax\taketrue \else \takefalse \fi}
\def\ifGt#1#2{\ifnum#1>#2\relax\taketrue \else \takefalse \fi}
\def\ifLt#1#2{\ifnum#1<#2\relax\taketrue \else \takefalse \fi}
\def\takefalse\fi#1#2{\fi#2}
\def\taketrue\else\takefalse\fi#1#2{\fi#1}

\def\ifEqString#1#2{\def\stringone{#1}\def\stringtwo{#2}\ifx\stringone\stringtwo}
%
% End of Citation
%
\newcommand{\cosec}{\mathop{\rm\text{cosec}}\nolimits}
\newcommand{\goup}[1]{\raise1.0ex\hbox{#1}}

\newcommand{\FRAC}[2]{\displaystyle\frac{{#1}}{#2}}
\newcommand{\Expo}[1]{\raisebox{0.7ex}{{$\scriptstyle{#1}$}}}
\newcommand{\SUM}{\displaystyle\sum}
\newcommand{\PROD}{\displaystyle\prod}
\newcommand{\INT}{\displaystyle\int}

\newcommand{\DispInt}[4]{\makebox[#1]{%
	$\displaystyle#2_{%
		\mbox{\scriptsize$\begin{array}{c}#4\end{array}$}}%
	$}\hspace{#3}}
\newcommand{\DispInts}[4]{\makebox[#1]{%
	$\smash{\displaystyle#2_{%
		\mbox{\scriptsize$\begin{array}{c}#4\end{array}$}}}\vphantom{\INT}%
	$}\hspace{#3}}

\newcommand{\dint}{\mathop{\int\!\!\int}}
\newcommand{\DINT}[1]{\DispInt{2em}{\dint}{0.5em}{#1}}
\newcommand{\DINTs}[1]{\DispInts{2em}{\dint}{0.5em}{#1}}

\newcommand{\triint}{\mathop{\int\!\!\int\!\!\int}}
\newcommand{\TRIINT}[1]{\DispInt{2.2em}{\triint}{0.5em}{#1}}
\newcommand{\TRIINTs}[1]{\DispInts{2.2em}{\triint}{0.5em}{#1}}

\newcommand{\nint}{\mathop{\int\!\!\int\cdots\int}}
\newcommand{\nINT}[1]{\DispInt{2.6em}{\nint}{0.9em}{#1}}
\newcommand{\nINTs}[1]{\DispInts{2.6em}{\nint}{0.9em}{#1}}

\newcommand{\LIMto}[2]{\displaystyle{\lim_{#1\rightarrow#2}}}

\newcommand{\Pard}[1]{\partial#1}
\newcommand{\ParDD}[4]{\FRAC{\partial^{#3}#1}{\partial{#2}^{#4}}}
\newcommand{\ParDn}[3]{\ParDD{#1}{#2}{#3}{#3}}
\newcommand{\ParD}[2]{\ParDD{#1}{#2}{}{}}

\newcommand{\Del}{{\mit \Delta}}
\newcommand{\GAMMA}{{\mit \Gamma}}

\newcommand{\Arc}[1]{\stackrel{\frown}{\text{#1}}}
\newcommand{\VEC}[1]{\overrightarrow{\mathrm{#1}}}
\renewcommand{\Vec}[1]{\mbox{\boldmath $#1$}}

\newcommand{\Matrix}[2]{\left(\SetComponent{#1}{#2}{c}\right) }
\newcommand{\MatrixR}[2]{\left(\SetComponent{#1}{#2}{r}\right) }
\newcommand{\Det}[2]{\left| \SetComponent{#1}{#2}{c}\right| }
\newcommand{\DetR}[2]{\left| \SetComponent{#1}{#2}{r}\right| }
\newcommand{\Vector}[1]{\Matrix{1}{#1}}
\newcommand{\VectorR}[1]{\MatrixR{1}{#1}}
\newcommand{\Vect}[1]{\mathbf #1}
\newcounter{subNo}
%\def\tmpc{&}
\newcommand{\SetComponent}[3]{%
	\def\CoNo{#1}\setcounter{subNo}{-1}%
	\begin{array}{*{#1}{#3}}\MakeRaw#2\relax\end{array}}
\newcommand{\MakeRaw}[1]{
	\ifx#1\relax\relax
	\else%
		\stepcounter{subNo}%
		\ifEq{\value{subNo}}0{\stepcounter{subNo} #1}%
			{\ifEq{\value{subNo}}1{\cr #1}{& #1}}%
		\ifnum\value{subNo}=\number\CoNo {\setcounter{subNo}{0}} \fi%
		\expandafter\MakeRaw%
	\fi%
}

\newcommand{\qed}{\hspace{\fill}\rule{3pt}{1ex}\par}
\newcommand{\Qed}{\hfill\rule{3pt}{1ex}}
\newcommand{\qedss}{\intline\qed}
\newcommand{\EquationwithQed}[1]{%
	\par\noindent%
%	\hspace*{\fill}%
	\hspace{0pt plus 1fill}%
	#1\qed}

\newcommand{\DispMath}[2]{{\setlength{\arraycolsep}{0.15em}\begin{#1}%
		#2\end{#1}}\ignorespaces}
%\setlength{\oddsidemargin}{0pt}
%\setlength{\evensidemargin}{0pt}
%\setlength{\marginparwidth}{1in}
%\setlength{\marginparsep}{0pt}

\setlength{\topmargin}{-0.8in}
\setlength{\headheight}{0.01\paperheight}
\setlength{\headsep}{0.01\paperheight}
\setlength{\topskip}{0pt}
\setlength{\textheight}{\paperheight}
\addtolength{\textheight}{-0.8in}
\setlength{\oddsidemargin}{-0.5in}
\setlength{\textwidth}{\paperwidth}
\addtolength{\textwidth}{-1in}
\setlength{\parindent}{0mm}
%
%
%\thispagestyle{enshuu}
%\def\endpiece{?}
%
%  均等割付マクロ
%   再帰呼び出しをして可変引数が取れるようにしてあるので
%  \newcommand では定義できない。
%
\def\WellSpace#1{#1\WellSpaceDo}
\def\WellSpaceDo#1{%
%  \def\temp{#1}%
  \ifx#1\relax\relax%
  \else
       \hfill#1\expandafter\WellSpaceDo%
  \fi%
 }
%
%  文字列均等割付マクロ
%   文字列の前後に空白なし
%
\newcommand{\wellspace}[1]{\WellSpace#1\relax}
%
%  文字列の前後に空白あり
%
\newcommand{\wellspaceside}[1]{%
         \hfill{\WellSpace#1\relax}\hfill\rule{0em}{0em}}
%\newcommand{\VSpace}[1]{%
% \rule[-#1\textheight]{0em}{#1\textheight}
% }
%
%     番号なし問題文環境
%
\newtheorem{ProbN}{}
%     番号を出力しない設定
\renewcommand{\theProbN}{ }
%
%   番号つき問題文環境
%
\newtheorem{Prob}{}
\renewcommand{\theProb}{{\bfseries\arabic{Prob}. }}
%    小問用番号設定カウンター  問題ごとにリセット
\newcounter{Probsub}[Prob]
\renewcommand{\theProbsub}{\stepcounter{Probsub}\arabic{Probsub}. \ }
%
%  置換積分などの表による演習問題で式を中央に書くためのマクロ
%   \TableSol
%   #1  問題
%   #2  解答
%   #3  解答文記述位置上下移動量(単位をつける。)
%   #4  解答記述幅
%
\newcommand{\TableSol}[4]{%
#1\ifHMSol{\raisebox{#3}{\hspace{#4}#2}}\fi}
%
%   演習問題のヘッダー記述マクロ
%   \ExcerciseHeadN
%   #1 科目名
%   #2 科目名記述前に挿入されるコマンド  通常は空欄でよい。
%   #3 科目名記述位置調整
%   #4  ヘッダー記述前に挿入されるコマンド  通常は空欄でよい。
%   #5 表題の右上に使用されるタイトル
%
\newcommand{\ExcerciseHeadN}[5]{%
\global\def\Pre{#5}%
%\RunningHead%
{\hfill
\setlength{\arrayrulewidth}{0.002\paperwidth}\setlength{\tabcolsep}{0mm}
  #4
  \begin{tabular}[t]{|*{3}{p{0.175\paperwidth}|}%
                      p{0.22\paperwidth}|p{0.11\paperwidth}|}
  \hline
  \wellspaceside{科目名}& \wellspaceside{{学科$\bullet$組}}&
  \wellspaceside{学籍番号} &
  \wellspaceside{氏名} & \wellspaceside{採点}\\ \hline
  \multicolumn{1}{|c|}{#2%
  \rule{0mm}{0.033\paperheight}\raisebox{#3mm}{\hspace*{-1zw}#1}} &
           & & & \\ \hline
  \end{tabular}
\hfill}
\setcounter{Prob}{0}
\setcounter{Probsub}{0}
\\[-0.02\paperheight]
}
\makeatletter
\newcommand{\LenL}{0.28}
\newcommand{\ps@enshuu}{%
\renewcommand{\@oddhead}{%
\hfill{\raisebox{-0.004\paperheight}%
         {\rule{\LenL\paperwidth}{0.002\paperwidth}}%
  \hspace*{-\LenL\paperwidth}\hspace*{-1zw}
  \setcounter{TPage}{\value{page}}\addtocounter{TPage}{-\fpage}%
  \stepcounter{TPage}%
  \makebox[\LenL\paperwidth][l]{\LecName\upshape 課題\theLecNo\hfill\Pre-p.\arabic{TPage}}}
}%
\renewcommand{\@evenhead}{}%
\renewcommand{\@oddfoot}{}%
\renewcommand{\@evenfoot}{}%
}
\makeatother
\newcommand{\AnsBox}{%
{\setlength{\fboxrule}{0.002\paperheight}%
 \fbox{\rule{0em}{1.em}\rule{2.5em}{0em}}
}}
%
\newcommand{\MKSpace}{\rule{2.5em}{0em}}
%
% 問題文記述用マクロ
%
\newcounter{LecNo}
\newcounter{TPage}[page]
\newcommand{\MakeTitleHead}[3]{%
\expandafter\newcommand\csname Problem#1\endcsname[1]{%
\setcounter{TPage}{0}
\edef\fpage{\arabic{page}}%
   \ExcerciseHeadN{\LecName}{}{2}{}{#2}%
   ##1%
\immediate\write\htmldata{S::#3::\fpage::\arabic{page}}%
   \clearpage}%
}
%
%  演習用紙ごとに通し番号をつけるためのカウンターとコマンドの設定
%
\MakeTitleHead{R}{資料}{Data}%    \ProblemR(授業中の資料用)
% A counter ProblemRNo and a command ProblemR are defined.
\MakeTitleHead{P}{予習}{Prelim}%    \ProblemR(授業中の予習用)
\MakeTitleHead{Pi}{予習1}{Prelimi}%    \ProblemR(授業中の予習用)
\MakeTitleHead{Pii}{予習2}{Prelimii}%    \ProblemR(授業中の予習用)
\MakeTitleHead{Piii}{予習3}{Prelimiii}%    \ProblemR(授業中の予習用)
\MakeTitleHead{Piv}{予習4}{Prelimiv}%    \ProblemR(授業中の予習用)
\MakeTitleHead{Pv}{予習5}{Prelimv}%    \ProblemR(授業中の予習用)
\MakeTitleHead{Pvi}{予習6}{Prelimvi}%    \ProblemR(授業中の予習用)
\MakeTitleHead{Pvii}{予習7}{Prelimvii}%    \ProblemR(授業中の予習用)
\MakeTitleHead{HM}{復習}{Rev}%    \ProblemHM(自宅学習用)
\MakeTitleHead{HMii}{復習2}{Revii}%    \ProblemHM(自宅学習用)
\MakeTitleHead{HMiii}{復習3}{Reviii}%    \ProblemHM(自宅学習用)
\MakeTitleHead{HMiv}{復習4}{Reviv}%    \ProblemHM(自宅学習用)
\MakeTitleHead{HMv}{復習5}{Revv}%    \ProblemHM(自宅学習用)
\MakeTitleHead{HMvi}{復習6}{Revvi}%    \ProblemHM(自宅学習用)
\MakeTitleHead{AD}{発展}{Adv}%    \ProblemAD(発展用)
\MakeTitleHead{}{演習}{Ex}%    \Problem(授業中の演習用)

%
% 各授業ごとに用意されたファイルを読み込むためのマクロの定義
% ホームページでは各回ごとに解答を記述した pdf ファイルを用意するため
% カウンターのデータを別ファイルに記述する機能を持たせてある。
%
\newcounter{Month}
\newcounter{Day}
\newcommand{\SetStart}[2]{%
  \setcounter{Month}{#1}%
  \setcounter{Day}{#2}%
}
\newcommand{\setNext}[1]{%
  \addtocounter{Day}{#1}
  \ifcase\value{Month}\or
     \setMonthDay{31}\or%1
     \setMonthDay{28}\or%2
     \setMonthDay{31}\or%3
     \setMonthDay{30}\or%4
     \setMonthDay{31}\or%5
     \setMonthDay{30}\or%6
     \setMonthDay{31}\or%7
     \setMonthDay{31}\or%8
     \setMonthDay{30}\or%9
     \setMonthDay{31}\or%10
     \setMonthDay{30}\or%11
     \setMonthDay{31}%12
  \fi
}
\newcommand{\setMonthDay}[1]{%
  \ifnum\value{Day}>#1\relax%
    \addtocounter{Day}{-#1}%
    \addtocounter{Month}{1}%
    \ifnum\value{Month}>12\setcounter{Month}{1}\fi
  \fi
}
\newcommand{\inputfile}[4]{%
\setcounter{LecNo}{#3}%
%
%  情報記述ファイルは解答なしのものを全体でTeXにかけたときに作る。
%   \write15 で使われるファイルは大元で開く。
%
\setNext{#4}
\immediate\write\htmldata{N::#1::#2::#3::\arabic{Month}::\arabic{Day}}%
 \input #1.tex
}
\newcommand{\KeepSpaceShow}[1]{\rule[-15mm]{0mm}{23mm}%
\raisebox{-5mm}{{#1}}}
\newcommand{\KeepSpaceShowR}[1]{\rule[-10mm]{0mm}{13mm}
\raisebox{-5mm}{\makebox[3em][c]{#1}}}
\newcommand{\HMSol}[2]{#1\ifHMSol #2\fi}
%
%   問題文の中に設けられる空欄記述用マクロ
%
%  \AnsBoxS
%   #1  解答欄の縦幅
%   #2  解答欄の横幅
%   #3  解答欄を下に広げるための調整量(- はマクロ内でつけられる。)
%   #4  解答
%
\newcommand{\AnsBoxS}[4]{\BoxS{#1}{#2}{#3}{#4}{1}}
\newcommand{\MBoxS}[4]{\BoxS{#1}{#2}{#3}{#4}{0}}
\newcommand{\BoxS}[5]{%
{\setlength{\fboxrule}{#5pt}%
    \hspace*{0.5em}\framebox[#2][c]{\ifHMSol\makebox[0em][c]{#4}\fi\rule[-#3]{0mm}{#1}}}%
\hspace*{0.5em}\ignorespaces 
}
\newcommand{\TBox}[1]{\BoxS{0em}{2em}{0em}{#1}{0}}
\newcommand{\TBoxS}[2]{\BoxS{0em}{#1}{0em}{#2}{0}}
%
% 解答欄横幅自動設定マクロ \setwd
%
%    \tmpwd に設定
%
\newsavebox{\tmpbox}
\newdimen\tmpdp 
\newdimen\tmpht 
\newdimen\tmpwd
\newdimen\mindim
\mindim=2.5mm
\newcommand{\setwd}{%
     \tmpwd=\wd\tmpbox
        \ifdim\tmpwd<10mm \tmpwd=15mm\else\tmpwd =1.5\tmpwd\fi
}
%
% 解答欄縦幅自動設定マクロ \setheight
%
%    \tmpht に設定
%
\newcommand{\setheight}{%
        \tmpdp=\dp\tmpbox%
        \ifdim\tmpdp<\mindim \tmpdp=\mindim\else\advance\tmpdp by \mindim\fi%
        \tmpht=\ht\tmpbox%
        \advance\tmpht by \mindim%\tmpht += \mindim
        \advance\tmpht by \tmpdp%
}
%
% 解答欄縦幅および横幅自動設定 \AnsBoxA
%
\newcommand{\AnsBoxA}[1]{%
     \savebox{\tmpbox}{\hbox{#1}}%
     \setwd\setheight%
    \BoxS{\tmpht}{\tmpwd}{\tmpdp}{#1}{1}%
}
\newcommand{\NAnsBoxA}[2]{%
     \savebox{\tmpbox}{\hbox{#1}}%
     \setwd\setheight%
    \BoxS{\tmpht}{\tmpwd}{\tmpdp}{#1}{#2}%
}
\newcommand{\MBoxA}[1]{%
     \savebox{\tmpbox}{\hbox{#1}}%
     \setwd\setheight%
    \BoxS{\tmpht}{\tmpwd}{\tmpdp}{#1}{0}%
}
\newcommand{\MBoxL}[1]{%
     \savebox{\tmpbox}{\hbox{#1}}%
     \setheight%
\rule[-\tmpdp]{0mm}{\tmpht}\ifHMSol\hbox{#1}\fi%
}
%
% 解答欄縦幅および横幅自動設定(添え字用) \AnsBoxScA
%
\newcommand{\AnsBoxScA}[1]{%
{\mindim=0.5mm
     \savebox{\tmpbox}{\hbox{#1}}%
     \setwd\setheight%
    \BoxS{\tmpht}{0.8\tmpwd}{\tmpdp}{#1}{1}%
}
}
%
% 解答欄縦幅および横幅半自動設定 \AnsBoxSA
%
\newcommand{\AnsBoxSA}[2]{%
    \savebox{\tmpbox}{\hbox{#2}}%
    \setheight\setwd%
    \BoxS{\tmpht}{\tmpwd}{\tmpdp}{#1}{1}%
}
\newcommand{\NAnsBoxSA}[2]{%
    \savebox{\tmpbox}{\hbox{#2}}%
    \setheight\setwd%
    \BoxS{\tmpht}{\tmpwd}{\tmpdp}{#1}{0.4}%
}
\newcommand{\MBoxSA}[2]{%
    \savebox{\tmpbox}{\hbox{#2}}%
    \setheight\setwd%
    \BoxS{\tmpht}{\tmpwd}{\tmpdp}{#1}{0}%
}
\newcommand{\AnsBoxN}[3]{\AnsBoxS{#1}{#2}{3mm}{#3}}
%
%  問題文と解答記述のためのマクロ(最新版)
%     今後はこのマクロに統一
%
%   高さが解答記述に不足するときは警告 private Warning が出る。
%
%    \ProbandAnsN
%    番号なし用
%    #1 全体の高さ(\textheight単位)
%    #2 問題文
%    #3 解答
%
\newcommand{\ProbandAnsN}[3]{%
\ProbandAnsBody{#1}{#2}{#3}{\relax}}
%
%  \ProbandAns
%
%    #1 全体の高さ(\textheight単位)
%    #2 問題文
%    #3 解答
%
\newcommand{\ProbandAns}[3]{%
\ProbandAnsBody{#1}{#2}{#3}{\theProbsub}}
%
%  \ProbandAnsBody
%
%    #1 全体の高さ(\textheight単位)
%    #2 問題文
%    #3 解答
%    #4 問題文の前に置かれるコマンド(ここに番号を出力するためのマクロを
%            記述すればよい。
%
\newcommand{\ProbandAnsBody}[4]{%
\setbox0=\hbox{\begin{minipage}[t]{0.99\textwidth}
#4% 
#2%
{#3}
\end{minipage}}
\ifdim\dp0>#1\textheight%\else%
\typeout{Private Warning:vspace #1(\textheight) of answer area 
\arabic{Prob}.\arabic{Probsub} is not sufficient.}
\fi
\addtocounter{Probsub}{-1}
 \rule[-#1\textheight]{0em}{#1\textheight}
\begin{minipage}[t]{0.99\textwidth}
#4% 
#2%
\ifHMSol%
{#3}
\fi
\end{minipage}}
%
%  end of \ProbandAnsBody
%
%  選択肢つき解答記述マクロ
%
%      #1 正解の選択肢
%      #2 選択肢のリスト  選択肢それぞれを {} でくくり全体をひとつのパラ
%              メータとして渡す。
%
\newcounter{SelectNo}
\renewcommand{\theSelectNo}{\Ding{\arabic{SelectNo}}\ }
\newcommand{\SelectAnsBox}[2]{%
\def\ansno{#1}
\setcounter{SelectNo}{0}
$\left(\right.$\makeAnsListBody#2\relax$\left.\right)$}
\def\makeAnsListBody#1{%
   \ifx#1\relax\relax%
   \else%
       \stepcounter{SelectNo}\let\currentbox=\mbox%
         \ifnum\value{SelectNo}=\number\ansno%
         \ifHMSol%
         \let\currentbox=\Ovalbox\fi\fi%
       \setbox0=\hbox{\Ovalbox{\theSelectNo #1}}%
       \vphantom{\box0}%
        \makebox[\wd0][c]{\currentbox{\theSelectNo #1}}%
	 \expandafter\makeAnsListBody%
   \fi%
}
\newcommand{\SelectAnsRem}{%
\\\ifHMSol {\upshape(答を選択するところは正解のほうが囲まれています。)}\fi\par}

%  丸付き数字記述マクロ(出典は忘れてしまいました。多分 LaTex Companion?)
%   Ding という ps フォントを使うこともできますが 0 の丸付きがないので
%   変更しました。
%
\newcommand{\Ding}[1]{{%
\setbox0=\hbox{\bfseries\large{$\bigcirc$}}%
\makebox[\wd0][c]{\textup{#1}}\hspace{-\wd0}\box0
}}
\newcommand{\Newpage}{\vfill\hfill{\bfseries 裏にもあります。}\newpage%
}
\pagestyle{enshuu}
\newwrite\htmldata
\immediate\openout\htmldata=\jobname.
