\begin{Prob}\upshape
 次の実行結果を確かめなさい\footnote{\ref{omitlet}以降をコンソールで連続
 して行う場合には\ref{omitlet2}以降にある\texttt{let}はつけないこと。}。
\begin{enumerate}\upshape
 \item \texttt{[1,2,[],3].length;}\Ans{6}
 \item \texttt{let a=[1,2,3]; console.log(a.pop());
       console.log(a.length);a;}\label{omitlet}\Ans{6}
 \item \texttt{let a=[1,2,3]; a.push(4,5); console.log(a.length);a;}\label{omitlet2}\Ans{6}
 \item \texttt{let a=[1,2,3]; a.shift(4,5); console.log(a.length);a;}\Ans{6}
 \item \texttt{let a=[1,2,3]; a.join(" ");}\Ans{6}
 \item \texttt{let a=[1,2,3,4,5]; console.log(a.slice(1,2)); console.log(a.length);a;}\Ans{6}
 \item \texttt{let a=[1,2,3,4,5]; console.log(a.splice(1,2)); console.log(a.length);a;}\Ans{6}
 \item \texttt{let a=[1,2,3,4,5]; console.log(a.indexOf(3)); console.log(a.indexOf(3,3));}\Ans{6}
 \item \texttt{let a=[3,1,2,3,4,5]; console.log(a.lastIndexOf(3));
       console.log(a.lastIndexOf(3,2));}%\\\rule{0em}{1em}
       \Ans{6}
\end{enumerate}
\end{Prob}
