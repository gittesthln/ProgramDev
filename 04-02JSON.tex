\section{オブジェクトリテラルとJSON}
JSON(JavaScript Object Notation)はデータ交換のための軽量なフォーマットで
ある。形式はJavaScriptのオブジェクトリテラルの記述法と全く同じである。
\begin{itemize}
 \item 正しく書かれたJSONフォーマットの文字列をブラウザとサーバーの間で
       データ交換の手段として利用できる。
 \item JavaScript内で、JSONフォーマットの文字列をJavaScriptのオブジェク
       トに変換できる。
 \item JavaScript内のオブジェクトをJSON形式の文字列に変換できる。
\end{itemize}

JavaScriptのオブジェクトとJSONフォーマットの文字列の相互変換の手段を提供
するのが\texttt{JSON}オブジェクトである。

\begin{Exec}\label{JSONExec}\upshape
次の例は2つの同じ形式からなるオブジェクトを通常の配列に入れたものを定義
 している。
\begin{Verbatim}
let persons = [{
  name : "foo",
  birthday :{ year : 2001, month : 4, day : 1},
  "hometown" : "神奈川",
},
{
  name : "Foo",
  birthday :{ year : 2010, month : 5, day : 5},
  "hometown" : "北海道",
}];
\end{Verbatim}
次の例はこのオブジェクトを \texttt{JSON}に処理させたものである。
\begin{Verbatim}
>s = JSON.stringify(persons);
"[{"name":"foo","birthday":{"year":2001,"month":4,"day":1},
"hometown":"神奈川"},
{"name":"Foo","birthday":{"year":2010,"month":5,"day":5},
"hometown":"北海道"}]"
>s2 = JSON.stringify(persons,["name","hometown"]);
"[{"name":"foo","hometown":"神奈川"},{"name":"Foo","hometown":"北海道"}]"
>o = JSON.parse(s2);
[Object, Object]
>o[0];
Object {name: "foo", hometown: "神奈川"}
\end{Verbatim}
\begin{itemize}
 \item JavaScriptのオブジェクトを文字列に変更する方法は
       \texttt{JSON.stringify()}を用いる。このまま見ると\verb+"+がおかし
       いように見えるが表示の関係でそうなっているだけである。%"
なお、結果は途中で改行を入れているが実際は一つの文字列となっている。
 \item \texttt{JSON.stringigy()}の二つ目の引数として対象のオブジェクトの
       キーの配列を与えることができる。このときは、指定されたキーのみが
       文字列に変換される。
 \item ここでは、\verb+"name"+ と \verb+"hometown"+が指定されているので
\verb+"birthday"+のデータは変換されていない。
 \item JSONデータをJavaScriptのオブジェクトに変換するための方法は
       \texttt{JSON.parse()}を用いる。
 \item ここではオブジェクトの配列に変換されたことがわかる。
 \item 各配列の要素が正しく変換されていることがわかる。
\end{itemize}
\end{Exec}
\input 04-02JSONProb.tex
