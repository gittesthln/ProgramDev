%-*- coding: utf-8 -*-
%\subsection{}
\section{サーバーとのデータのやり取り}
\input 0923EXchangeDataFund.tex
\subsection{サーバーとのデータ交換の基本}
Webページにおいてサーバーにデータを送る方法には\texttt{POST}と
\texttt{PUT}の2通りの方法がある。
\begin{Exec}\upshape
 次のリストは実行例\ref{EventInput}のリストに\texttt{form}のデータを
 \texttt{POST}で送るようにしたものである。JavaScriptの部分だけ異なるので
 そこだけのリストになっている。

\texttt{windows.onload =function()}内に次のコードを追加する。
\begin{verbatim}
    var Form = document.getElementsByTagName("form")[0];
    Form.setAttribute("method","POST");
    Form.setAttribute("action","09sendData.php");
\end{verbatim}
HTMLの要素に対しては次のことを行う。
\begin{itemize}
 \item \texttt{<select>}要素の属性に\verb+name="select"+ を追加する。
 \item \texttt{id}が\verb+"colorName"+であるテキストボックスに
       \verb+name="colorName"+ を追加する。
 \item 「設定」ボタンの要素の後に次の要素を追加する。
\begin{center}
\verb+<input type="submit" value="送信" id="Send"></input>+ 
\end{center}
\end{itemize}
このページでは「送信」ボタンを押すと\texttt{<form>}の
\texttt{action}属性で指定されたプログラムが呼び出される。ここではWebペー
 ジと同じ場所にある\texttt{09sendData.php}が呼び出される。このファイル
 のリストは次の通りである。
\listinginput{1}{09sendData.php}
\verbatiminput{09sendData.res}
\end{Exec}