\section{サーバーとのデータ交換の基本}
Webページにおいてサーバーにデータを送る方法には\texttt{POST}と
\texttt{PUT}の2通りの方法がある。
\begin{Exec}\upshape\label{dataexchange}
 次のリストは実行例\ref{EventInput}のリストに\texttt{form}のデータを
 \texttt{POST}で送るようにしたものである。JavaScriptの部分だけ異なるので
 そこだけのリストになっている。

\texttt{windows.onload =function()}内に次のコードを追加する。
\begin{Verbatim}
    var Form = document.getElementsByTagName("form")[0];
    Form.setAttribute("method","POST");
    Form.setAttribute("action","09sendData.php");
\end{Verbatim}
HTMLの要素に対しては次のことを行う。
\begin{itemize}
 \item \texttt{<select>}要素の属性に\Verb+name="select"+ を追加する。
 \item \texttt{id}が\Verb+"colorName"+であるテキストボックスに
       \Verb+name="colorName"+ を追加する。
 \item 「設定」ボタンの要素の後に次の要素を追加する。
\begin{center}
\Verb+<input type="submit" value="送信" id="Send"></input>+ 
\end{center}
\end{itemize}
このページでは「送信」ボタンを押すと\texttt{<form>}の
\texttt{action}属性で指定されたプログラムが呼び出される。ここではWebペー
 ジと同じ場所にある\texttt{09sendData.php}が呼び出される。このファイル
 のリストは次の通りである。
\LISTN{09sendData.php}{1}{last}{\normalsize}
\begin{itemize}
 \item 2行目から10行目の間はヒアドキュメント形式でHTML文書の初めの部分を
       出力させている。
 \item \Verb+method="POST"+で呼び出されたときには\texttt{form}要素内の
       \texttt{name}属性が指定されたものの値がスーパーグローバル
       \Verb+$_POST+内の連想配列としてアクセスができる。
 \item 11行目から13行目でそれらの値を\texttt{table}要素内の要素として出
       力している。
\end{itemize}
出力された結果は次のようになる。
\VerbatimInput{09sendData.res}
\end{Exec}
なお、\Verb+method="PUT"+で呼び出した場合にはスーパーグローバル
\Verb+$_GET+を用いる。また、スーパーグローバル\Verb+$_REQUEST+は
\Verb+method="POST"+でも\Verb+method="PUT"+で呼び出された場合の
\Verb+$_POST+や\Verb+$_GET+の代わりに使用できる。

\Verb+type="submit"+の\texttt{input}要素は、ボタンが押されたときに直ちに、
\texttt{action}属性で指定された処理が呼び出される。通常は、サーバーにデー
タを送る前に最低限のエラーチェックを行い、エラーがない場合にだけサーバー
と通信するのが良い。
\begin{Prob}\upshape
 \texttt{PUT}と\texttt{GET}を用いたHTMLファイルで実行後のURLがどうなって
 いるか確認しなさい。
 さらに、実行例\ref{dataexchange}のPHPのプログラムを直接URLに記述して呼び出した
 らどうなるか。また、
   \Verb+Form.setAttribute("method","POST");+
 の部分を\Verb+Form.setAttribute("method","PUT");+に変更して\texttt{PUT}
 による通信の場合について調べよ。
\end{Prob}
\section{スパーグローバルの補足}
スパーグローバルのうち、これまでに説明してこなかったものについて解説をす
る。
\paragraph{\texttt{\$\_SERVER}}
この変数はサーバーにアクセスしたときのクライ
アントの情報などを提供する。具体的な内容はクライアントごとに異なる。
\begin{Prob}\upshape
 \texttt{foreach}構文を用いて\Verb+$_SERVER+の内容を表示し、どのようなも
 のが送られている確認せよ。
\end{Prob}
\paragraph{\texttt{\$\_COOKIE}}
COOKIE とはWebサーバー側からクライアント側に一時的にデータを保存させる仕
組みである。サーバーと通信するたびに、これらのデータがクライアント側から
サーバー側に送られる。これにより、すでに訪問したことがあるサイトに対して
情報を開始時に補填する機能などを実現できる。
\paragraph{\texttt{\$\_SESSION}}
セッションとはある作業の一連の流れを指す。たとえば会員制のサイトではログイ
ン後でなければ会員専用のページを見ることができない。情報のページに直接行くことがで
きないような仕組みが必要である。

HTTP通信はセッションレスな通信である(各ページが独立して存在し、ページ間
のデータを直接渡せない)。
セッションを確立するために、クライアント側から情報を送り、それに基づい
てサーバー側が状況を判断するなどの操作を意識的にする必要がある。

初期のころはページ間のデータを受け渡すために表
示しない\texttt{<input>}要素の中に受け渡すデータを入れていた。これはセキュ
リティ上問題が生ずるので、COOKIEによる認証が行われるようになった。

PHPではセッションを開始するための関数\texttt{session\_start()}とセッショ
ンを終了させるための関数\texttt{session\_destroy()}が用意されている。セッションを
通じで保存させておきたい情報はこの連想配列に保存する。セッションの管理は
サーバーが管理することとなる。
なお、この機能はCOOKIEの機能を利用して実現されている。
\section{Web Storage}
HTML5から独立した規格である Web Storage には localStoarage と sessionStorage の2種類
のオブジェクトがある。これらのオブジェクトは文字列をキーに、文字列の値を持つ Storage オブジェクト
である。同一の出身(プロトコルやポート番号も含む)のすべてのドキュメント
が同じ localStorage と sessionStorage を共有する。sessionStorage のデー
タは意識的に消さない限り存在する。
%
これに対し、sessionStorageはウインドウやブラウザが閉じられると消滅する。

これにより、セッション間の情報の移動や以前の訪れたことのあるページの情報
の保存を可能にしている。

\begin{Exec}\upshape
次の例は \texttt{localStorage} を用いて、ページのアクセス時間を保存し、他のページ
 に移動してもそのデータが利用できることを示すものである。

ページのアクセス時間を記録するページのリストは次のとおりである。
 \LISTN{10storage.html}{1}{last}{\normalsize}
 \begin{itemize}
	\item 8行目で記述を簡単にするためと後で\texttt{localStorage}を
				\texttt{sessionStorage}に簡単に修正できるように変数
				\texttt{Storage}を導入している。
	\item 11行目で結果を表示するための\texttt{<div>}要素の得ている。
	\item 12行目ではアクセスされた時間を求めている。
	\item WebStorageに\texttt{access}の要素が存在するかを調べ、存在する場
				合には変数\texttt{accessList}に配列のデータとして復元する
				(13行目から14行目)。これは、今までのアクセス時間を新しい順に並べ
				た配列である。
	\item 存在しない場合には、変数\texttt{accessList}を初期化し(16行目)、
				初回のアクセスである旨の表示を行う(17行目)。テキストを表示すある
				関数は26行目から30行目に定義してある。
	\item 新しい順に並べてあるアクセス時間の先頭に今回のアクセス時間
				(1970年1月1日0時(GMT)からのミリ秒単位の時間)を挿入
				し(19行目の\texttt{unshift()}は配列の先頭に引数のデータを挿入す
				る配列のメソッドである)、WebStorageにJSON形式で保存する(20行目))。
	\item 以下のデータがアクセス時間のリストであるメッセージを表示した(21
				行目)後、アクセス時間をすべて表示している(22行目から24行目)。
\begin{itemize}
 \item 配列の\texttt{forEach}メソッドは、配列の個々の要素に対して引数で
				与えられた関数を実行する。
 \item このメソッドの戻り値はない。
 \item 引数で渡される関数の引数
				は3つあり、処理する配列の現在の値(\texttt{D})、配列の位置
			 (\texttt{i})と処理する配列(\texttt{A})である。したがって、
			 \texttt{A[i]}と\texttt{D}は同じ値である。
 \item 処理結果を配列に戻したければ\texttt{A[i]}に値を代入すればよい。
\end{itemize}
	\item 26行目から30行目がWebページ上にデータを表示する関数を定義してい
				る部分である。
				\begin{itemize}
				 \item 引数は、表示するための親要素(\texttt{P})と表示する文字列
							 (\texttt{Mess})である。
				 \item \texttt{<div>}要素を作成し(27行目)、与えられた親要素の子要素にす
							 る(28行目)。
				 \item さらに作成した子要素に与えられた文字列を表示するテキスト
							 要素を子要素に付け加える(29行目)
				\end{itemize}
				
 \end{itemize}
前のページで「次のページ」ボタンを押したとき、移動先のページのリストは次
 のとおりである。
 \LISTN{10next.html}{1}{last}{\normalsize}
\end{Exec}
このリストは呼び出す前のページのものとほとんど変わらないが、アクセス時間
の表示が今回のものを含めて4つだけ表示されるようになっている
(15行目から18行目)。
\begin{itemize}
 \item 前のリストの22行目で\Verb+AccessList.forEach+となっている部分が
			 \Verb+AccessList.every+となっている。
 \item この配列のメソッドは引き渡された関数をすべての要素を順に適用し、
			 戻り値の論理積を返す。
 \item どこかのところで\Verb+false+となれば全体の論理積は\Verb+false+と
			 なるので実行はそこで打ち切られる。
 \item 論理和を求める\Verb+some+という配列のメソッドもある。
\end{itemize}
 \begin{Prob}\upshape
	このリストで、\Verb+AccessList.every+となっている部分を
	\Verb+AccessList.some+でも正しく実行できるようにしなさい。
\end{Prob}

すでに、いくつかのサイトではこ
の機能を用いており、その開発者ツールで見ることが可能である。これらのデー
タの形式は文字列である。構造化されたデータはJSON形式で保存するのがよいで
あろう。
\begin{Prob}\upshape
いくつかのサイトにおいてStorageが利用されているか、どのようなデータ
 が保存されているか調べよ。
\end{Prob}
