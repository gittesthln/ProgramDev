%-*- coding: utf-8 -*-
\newcommand{\Elm}[1]{\texttt{<#1>}}
\newcommand{\Yes}{$\bigcirc$}
\newcommand{\No}{$\times$}
\chapter{DOMの利用}
この節からはHTML文書の取り扱いを始める。この授業では2014年10月28日にW3C
のRecommendationとなったHTML5
を用いる\footnote{http://www.w3.org/TR/2014/REC-html5-20141028/}。
\section{HTML文書の構成}
\begin{Exec}\upshape\label{ExGoogleMaps}
 次のリストは\texttt{Google Maps} を利用して地図を表示するものである
 \footnote{Google Maps API3 の公開当初は API キーは存在しなかったが、そ
 の後 API キーが必要となった。利用当初の状況から API キーがなくて
 も動作するが、API キーを取得することを勧める。取得の方法は付録に示す。}。
 \LISTN{07firstMap.html}{1}{last}{\normalsize}
ここではこのHTML文書の構成について解説をする。
\begin{itemize}
 \item 1行目はHTML文書の\texttt{DOCTYPE}宣言である。この形はHTML5にお
       けるもので、以前のものと比べて記述が簡単になっている。
 \item 2行目はこのHTML文書のルート要素と呼ばれるものである。最後の25行目
       の \Elm{/html}までが有効となる。すべての要素はこの範囲になけ
       ればならない。
 \item 3行目から始まる\Elm{head}はブラウザに表示されない、いろいろ
       なHTML文書の情報を表す。
         \begin{itemize}
	  \item 4行目はこの文書の形式や文字集合を記述している。ここでは
		内容は\texttt{text/html}の形式、つまり、テキストで書かれ
		た\texttt{html}の形式で書かれていることを表す。
		\footnote{このような方法でファイルのデータ形式を表すこと
		をMIME(Multipurpose
		Internet Mail Extension)タイプと呼ぶ。元来、テキストデー
		タしか扱えない電子メールに様々なフォーマットのデータを扱
		えるようにする規格である。}
	  \item 5行目の\Elm{title}はブラウザのタブに表示される文字
		列を指定している。
	  \item 6行目から7行目はGoogle Maps のライブラリーを読み込むため
		のものである。このようにJavaScriptのプログラムは外部ファ
		イルとすることができる。
	  \item 8行目から19行目はHTML文書内に書かれたJavaScriptである。
		詳しい解説は後の授業で行う。
	  \item 20行目はHTML文書の見栄えなどを規定する\texttt{CSS}ファイ
		ルを外部から読み込むことをしている。
	 \end{itemize}
 \item HTML文書で実際にブラウザ内で表示される情報は\Elm{body}要素
       内に現れる。
 \item このリストではGoogle Mapsを表示するための\Elm{div}要素が一つ
       あるだけである。このとき、\Elm{div}は\Elm{body}の子要素
       であるといい、\Elm{body}は\Elm{div}の親要素という。
 \item 各要素名または要素の終了を示すタグ(\texttt{<...>})の間に文字列が
       ある場合、その部分はテキストノードと呼ばれるノードが作成されてい
       る。
\end{itemize}
各要素は\texttt{<}と\texttt{}の中に現れる。初めに現れる文字列が要素名で
 あり、そのあとに属性と属性値がいくつか並ぶ。
\begin{itemize}
 \item 属性とその属性値は\texttt{=}で結ばれる。
 \item 属性値は\texttt{”}ではさまれた文字列として記述する。
 \item 6行目の\Elm{script}要素では属性\texttt{type}と\texttt{src}が
       設定されている。
 \item 23行目の\Elm{div}要素では属性\texttt{id}に属性値
       \texttt{map\textunderscore canvas}を設定している。なお、この要素
       はCSSによっても属性が定義されている。
\end{itemize}
次のリストは20行目で参照しているCSSファイルの内容である。
\LISTN{map.css}{1}{last}{\normalsize}
\begin{itemize}
 \item CSSの各構成要素はHTML文書の要素を選択するセレクタ(ここでは
       \Verb+#map_canvas+)とそれに対する属性値の並び([属性]:[属性値];)か
       らなる。
 \item \Verb+#+で始まるセレクタはそのあとの文字列を\Elm{id}の属性値に
       持つ要素に適用される。
 \item したがって、ここの規則は23行目の\Elm{div}要素に適用される。
 \item その内容はGoogle Maps が表示される画面の大きさ(\texttt{width}と
       \texttt{height})、配置の位置(\texttt{float})と要素の外に配置され
       る空白(\texttt{margin})を指定している。
\end{itemize}
\end{Exec}
\section{CSSの利用}
\input 07-1selectors.tex
\section{DOMとは}
Document Object Model(DOM)はHTML文書などの要素をノードとしたツリー構造で
管理する方法である。DOMのメソッドやプロパティを使うことで各要素にアクセ
スしたり、属性値やツリーの構造を変化させることができる。DOMの構造は開発
者ツールなどで見ることができる。
\begin{itemize}
 \item Google Chrome では開発者ツールの Elementsタブで確認でき
       る。ここで要素上で右クリックしてEdit as HTML
       を選択するとテキストとして編集できる。
 \item FireFoxでは開発ツールの「インスペクタ」タブでDOMツリーが確認でき
       る。要素上で右クリックから「HTMLとして編集」とするとテキストとし
       て編集できる。 
 \item IEでは開発者ツールを開き、左にある「HTML」タブで同様の
       ことができる。
\end{itemize}
\newcommand{\DOMM}{\texttt}
\newcommand{\DOMP}{\texttt}
\newcommand{\DOM}{\texttt{DOM}}
\newcommand{\keyitem}{\relax}
\newcommand{\HTML}{HTML文書}
\section{DOM のメソッドとプロパティ}
%\DOM の構造や属性値の操作をプログラムから可能にするために
DOM では DOM ツリーを操作するために%表\ref{MethodDOM}や\ref{PropertyDOM}にある
\keyitem{メソッド}や
\keyitem{プロパティ}が規定されている。
これらの手段を用いてDOM をサポートする文書にアクセスができる。

 メソッドとはそのオブジェクトに対する操作である。次のような手段を
       提供している。
\begin{itemize}
 \item 条件に合う要素または要素のリストを得る。
 \item 要素の属性を参照、変更ができる。
 \item 要素を新規に作成する。
 \item ある要素に子要素を追加したり、取り除いたりする。
\end{itemize}
プロパティはそのオブジェクトが持つ性質であり、それらの値を参照できる。ほ
とんどのプロパティは書き直しできない。

なお、ここでのメソッドやプロパティは DOM文書で使用可能なものである。したがっ
て、\HTML も \DOM{} をサポートするブラウザであれば同様の方法で
部分的に書き直すことが可能である。
\subsection{DOMのメソッド}
\input 07-2DOMMethod.tex

%実行例\ref{ExGoogleMaps}では16行目でGoogle Mapsを表示する\Elm{div}
%要素を得て、17行目でその要素内に地図を表示するように指定している。
\begin{Exec}\upshape\label{PullDown}
 次の例は1月から12月までを選択できるプルダウンメニューを作成するものであ
 る。この様なプルダウンメニューをテキストエディタで入力すると次のように
 なる。\label{pulldown1}
\begin{Verbatim}
<!DOCTYPE html>
<head>
<meta charset="UTF-8"/>
<title>プルダウンメニュー</title>
</head>
<body>
  <form id="menu">
    <select>
      <option value="1">1月</option>
      <option value="2">2月</option>
      <option value="3">3月</option>
      <option value="4">4月</option>
      <option value="5">5月</option>
      <option value="6">6月</option>
      <option value="7">7月</option>
      <option value="8">8月</option>
      <option value="9">9月</option>
      <option value="10">10月</option>
      <option value="11">11月</option>
      <option value="12">12月</option>
    </select>
  </form>
</body>
</html>
\end{Verbatim}
\begin{itemize}
 \item ユーザからの入力を受け付ける要素は通常、\Elm{form}要素内に記
       述する。
 \item プルダウンメニュ-の要素名は\Elm{select}で、選択する内容は
       \Elm{option}要素である。
 \item \Elm{option}要素の属性\texttt{value}の値が選択した値として
       利用できる。
 \item \Elm{option}要素内の文字列(テキストノード)がプルダウンメニューに
       表示されるものになる。
 \item \Elm{select}は\Elm{form}の子要素であり、各\Elm{option}は
       \Elm{select}の子要素となっている。
\end{itemize}
\label{pulldown2}
\LISTN{07-02pulldown.html}{1}{last}{\normalsize}
\begin{itemize}
 \item 8行目の\texttt{window.onload}はファイルのロードが終わった後に発生
       するイベントの処理をする関数を表す。
       イベントの詳細については後で解説する。
% \item 10行目では26行目から27行目にある\Elm{form}要素を得ている。
 \item 10行目では\Elm{select}要素を作成している。
 \item 11行目では10行目で作成した\Elm{select}要素を
       \texttt{getElementById("menu")}で得た\Elm{form}要素の子要素に設定
       している。 
 \item 12行目から始まる\texttt{for}ループで12個の\Elm{option}要素を作成
       し、\Elm{select}要素の子要素としている。
       \begin{itemize}
	\item 13行目で\Elm{option}要素を新規に作成し、14行目で、その要素の属性
        \texttt{value}に値を設定している。
	\item 15行目ではその\Elm{option}要素を\Elm{select}要素の子要素と
	      している。
	\item さらに、15行目では表示する文字列をもつテキストノードを作成
	      し、16行目でそれを\Elm{option}要素の子要素としている。
       \end{itemize}
\end{itemize}
\end{Exec}
\begin{Prob}\upshape
実行例\ref{PullDown}に対して次のことを行いなさい。
\begin{enumerate}
 \item ブラウザの「ページのソースを表示」を選択して、ソースコードがどの
       ようになっているか確認する。
 \item DOMツリーが実行例\ref{PullDown}のリストと同じようにできていることを確認する。
% \item 8行目と21行目を取り除いた場合に起こることを確認する。また、その理
%       由を考える。
 \item \Elm{select}要素を構成する部分を関数化して、月だけでなく日(1日か
       ら31日)が選べるプルダウンメニューも作成できるようにしなさい。関数
       の引数を工夫すること。
\end{enumerate}
\end{Prob}
\subsection{DOMの要素のプロパティ}
表\ref{PropertyDOM}は DOM の要素に対するプロパティである。
%該当するものがない場合には \texttt{null} の値になる。
\begin{table}[ht]
\caption{DOM要素に対するプロパティ}\label{PropertyDOM}
\begin{center}
 \begin{tabular}{|c|m{25zw}|c|}
  \hline
プロパティ名  &
 \hspace*{\fill}説{\hfill}明\hspace*{\fill}\rule{0em}{0em}&{\scriptsize PHP}\\ \hline
\DOMP{firstChild} &指定された要素の先頭にある子要素&\Yes \\ \hline
\DOMP{lastChild} & 指定された要素の最後にある子要素&\Yes\\ \hline
\DOMP{nextSibling} & 指定された子要素の次の要素&\Yes\\ \hline
\DOMP{previousSibling} & 現在の子要素の前にある要素&\Yes\\ \hline
\DOMP{parentNode} & 現在の要素の親要素&\Yes\\ \hline
\DOMP{hasChildNodes} &その要素が子要素を持つ場合は\texttt{true}持たない
      場合は\texttt{false}である。&\No\\ \hline
\DOMP{nodeName}& その要素の要素名前&\Yes\\ \hline
\DOMP{nodeType}& 要素の種類($1$は普通の要素、$3$はテキストノード)&\Yes\\ \hline
\DOMP{nodeValue}&(テキスト)ノードの値 &\Yes\\ \hline
\DOMP{childNodes}& 子要素の配列&\Yes\\ \hline
\DOMP{children}& 子要素のうち通常の要素だけからなる要素の配列
      (DOM4で定義)&\No\\ \hline
\DOMP{firstElementChild} &指定された要素の先頭にある通常の要素である子要
			素(DOM4で定義)&\No\\ \hline 
\DOMP{lastElementChild} & 指定された要素の最後にある通常の要素である子要
			素(DOM4で定義)&\No\\ \hline
\DOMP{nextElementSibling} & 指定された子要素の次の通常の要素(DOM4で定義)&\No\\ \hline
\DOMP{previousElementSibling} & 現在の子要素の前にある通常の要素(DOM4で定義)&\No\\ \hline
 \end{tabular}
\end{center}
\end{table}

なお、DOM4\footnote{\texttt{http://www.w3.org/TR/domcore/}}とは2015年11月現在
W3Cが定める次の DOM の規格の Proposed Recommendation(勧告案) である。
DOM の規格は今までに Level 1 から Level 3 までが Recommendation(勧告) と
なっている。
\begin{itemize}
 \item これらのプロパティのうち、\texttt{nodeValue}を除いてはすべて、読み取り専用である。
 \item ある要素に子要素がない場合にはその要素の\texttt{firstChild}や
\texttt{lastChild}は\texttt{null}となる。
 \item ある要素に子要素がある
場合、その\texttt{firstChild.previousSibling}や
\texttt{lastChild.nextSibling}も\texttt{null}となる。
       \texttt{firstElementChild}などでも同様である。
\end{itemize}

プロパティに関する例はイベント処理のところで示す。
\begin{Prob}\upshape
 \pageref{pulldown1}ページから始まるリストと\pageref{pulldown2}ページから始
 まるリストにおける\texttt{<select>}要素の子要素の数を調べよ。異なる場合にはその理由
 を述べよ。

 \texttt{document.getElementsByTagName("select")[0].childNodes.length}を使
 うとよい。
%
また、
 \texttt{document.getElementsByTagName("select")[0].childNodes[0].nodeType}
 も調べるとよいかもしれない。また、\texttt{childNodes}の代わりに
 \texttt{children}を用いたらどうなるかも調べるとよい。
\end{Prob}

