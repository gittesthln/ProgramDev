%-*- coding: utf-8 -*-
\section{Ajax}
Ajax とは Asynchronous Javascript+XML の略で、非同期(Asynchronous)でWeb
ページとサーバーでデータの交換を行い、クライアント側で得られたデータをも
とにそのWebページを書き直す手法である\footnote{%
\texttt{http://adaptivepath.org/ideas/ajax-new-approach-web-applications/}}。
Google Maps がこの技術を利用したことで一気に認知度が高まった。検索サイト
では検索する用語の一部を入力していると検索用語の候補が出てくる。これも
Ajaxを使用している(と考えられる)。

Ajax の機能は \texttt{XMLHTTPRequest}というオブジェクトの機能を用いて実現されている。
\begin{Exec}\upshape\label{WhatDay}
次の例は実行例\ref{pulldownDate}で日付が変わったときに、その日の記念日を
 メニューの下部に示すものである。記念日のデータは
 \texttt{http://ja.wikipedia.org/wiki/日本の記念日一覧}の表示画面からコ
 ピーして作成した。
\LISTN{09-01whatday.html}{1}{last}{\normalsize}
\begin{itemize}
 \item 実行例\ref{pulldownDate}とは41行目以降が異なっている。
 \item イベントハンドラーを関数として定義している(41行目から60行目)。
 \item 42行目から48行目は以前と同じプルダウンメニューの処理である。
 \item 49行目では裏でサーバーと通信をするための
       \texttt{XMLHttpRequest}オブジェクトを作成している。
 \item \texttt{XMLHttpRequest}が生成できたら(50行目)、このオブジェクトの
       \texttt{onreadystatechange}イベントのイベントハンドラーを
       登録する(51行目から55行目)。
 \begin{itemize}
  \item \texttt{XMLHttpRequest}の\texttt{readyState}は通信の状態を表す。
	$4$ は通信終了を意味する。これらの値については表\ref{XMLHttpRequestRes}を参照
				のこと。
  \item 通信が終了しても正しくデータが得られたかを調べる必要がある。
	$200$ は正しくデータが得られたことを意味する\footnote{Http通信の終了
				コードについては
				\texttt{http://www.w3.org/Protocols/rfc2616/rfc2616-sec10.html}
				を参照のこと}。
  \item 得られたデータは\texttt{responseText}で得られる。この場合、得ら
	れたデータは文字列となる。このほかに\texttt{responXML}でXMLデー
	タが得られる。
 \end{itemize}
 \item 56行目から58行目が通信の開始部分である。ここでは、\texttt{GET}で行うので、
       URLの後に必要なデータを付ける。
 \item \texttt{GET}では送るデータ本体がないので、通信終了のため
       \texttt{null}を送信する。\texttt{POST}のときはここでデータ本体を
       送る。
 \item プルダウンメニューが変化したときのイベントハンドラーを登録し(62行
       目)、最後に現在の日付データをサーバーに要求する(63行目)。
 \item 得られたデータは69行目の\texttt{p}要素のテキスト
       (\texttt{innerText})として代入する(53行目)。
       %この要素の\texttt{firstChild}を指定しているので
       %\texttt{<p>}と\texttt{</p>}の間に空白を設けて、テキストノードが存
      % 在するようにしている。
\end{itemize}
 表\ref{XMLHttpRequestRes}の状態は\texttt{XMLHttpRequest}オブジェクトの
 プロパティである。たとえば、\texttt{XMLHttpRequest.DONE}で利用できる。
 \begin{table}[ht]
	\caption{XMLHttpRequestの通信の状態
\protect\footnote{\protect\texttt{https://developer.mozilla.org/ja/docs/Web/API/XMLHttpRequest}
	より引用}} \label{XMLHttpRequestRes}
		\begin{tabular}{|c|c|m{30zw}|}\hline
		 値&状態&\multicolumn{1}{c|}{詳細}\\\hline
		 0&\Verb+UNSENT+&\Verb+open()+ がまだ呼び出されていない。\\\hline
1&\Verb+OPENED+&\Verb+send()+ がまだ呼び出されていない。\\\hline
2&\Verb+HEADERS_RECEIVED+&\Verb+send()+ が呼び出され、ヘッダーとステータ
						 スが通った。\\\hline 
3&\Verb+LOADING+&ダウンロード中;\Verb+responseText+ は断片的なデータ
						 を保持している。\\\hline 
4&\Verb+DONE+&一連の動作が完了した。\\\hline
		\end{tabular}
 \end{table}
\input 17dev10-05.tex
 次のリストはAjaxで呼び出されるPHPのプログラムである。
\LISTN{aniversary.php}{1}{last}{\normalsize}
\begin{itemize}
 \item 2行目で内部で処理をするエンコーディングを\texttt{UTF8}にしている。
       関数、\texttt{mb\_internal\_encoding}関数を引数なしで呼び出すと
       現在採用されているエンコーディングを得ることができる。
 \item 4行目と5行目では月(\Verb+$m+)と日(\Verb+$d+)の値をそれぞれの変数
       に設定している。
\begin{itemize}
 \item ここではコマンドプロンプトからもデバッグで
       きるように、スーパーグローバル\Verb+$_GET+内に値があれば
       (\Verb+isset()+)が\Verb+true+になれば、その値を、そうでなければコ
       マンドからの引数を設定している。
 \item コマンドから実行したときの引数はスーパーグローバル\Verb+$argv+に
       格納される。一番目は呼び出したファイル名であ
       り、その後に引数が順に入る\footnote{C言語の\texttt{main}関数は通
       常、\texttt{int main(int argc, char* argv[])}と宣言される。%
       \texttt{argc}は\texttt{argv}の配列の大きさを表し、渡された引数の
       リストが\texttt{argv[]}に入っている。このとき、\texttt{argv[0]}は実行
       したときのファイル名が入る。}。
\end{itemize}
 \item 6行目の\texttt{file}関数は指定されたファイルを行末文字で区切って配
       列として返す関数である。この引数にはURLも指定できる。
\begin{itemize}
 \item この関数は2番目の引数をとることができる。次の定数を組み合わせて使
       う。
\begin{center}
 \begin{tabular}{|c|c|}\hline
 \Verb+FILE_USE_INCLUDE_PATH+ & \Verb+include_path+ のファイルを探す\\\hline
 \Verb+FILE_IGNORE_NEW_LINES+ & 配列の各要素の最後に改行文字を追加しない
      \\ \hline
  \Verb+FILE_SKIP_EMPTY_LINES+&空行を読み飛ばす \\ \hline
 \end{tabular}
\end{center}
 \item \Verb+file_get_contents()+はファイルの内容を一つの文字列として
       読み込む。Webページの解析にはこちらの関数を使うとよい。
\end{itemize}
 \item 読み込むファイルの一部を次に記す。
\begin{Verbatim}
1月[編集]
1日 - 鉄腕アトムの日
2日 - 月ロケットの日
[中略]
31日 - 生命保険の日、愛妻家の日
2月[編集]
1日 - テレビ放送記念日、ニオイの日
2日 - 頭痛の日
[以下略]
\end{Verbatim}
\begin{itemize}
 \item 月の部分の後には\texttt{[}がある。
 \item 日の情報は\Verb*+ - +で区切られている(\Verb*+ +は空白を表す)。
 \item すべての日の情報が入っている。
\end{itemize}
 \item 7行目から13行目までは指定された月の行を見つける。
\begin{itemize}
 \item 8行目で念のためコードを\texttt{UTF8}に変更している。
 \item 関数\Verb+mb_split()+関数は第1引数に指定された文字列パターンで第2
       引数で指定された文字列を分割して配列として返す関数である。
 \item 分割を指定する文字列には正規表現が使えるので、文字\Verb+[+で分割
       するために、\Verb+"\["+としている(9行目)。
 \item 指定された文字列があれば配列の大きさが1より大きくなる。その行に対
       して求める月と一致しているか判定し、等しければループを抜ける(11行
       目)。
\end{itemize}
 \item 14行目から18行目までは指定された月での指定された日の情報を探して
       いる。日を決定する方法も月と同じである。文字列の分割は
       \Verb+"\s-\s"+となっている\footnote{これは\texttt{"\textbackslash
       s"}ではうまく行かなかっ
       たためである。}。
 \item 20行目で得られた情報をストリームに出力している。
\end{itemize}
\end{Exec}
なお、構造化されたデータとしてはXML形式でもよいが、より軽量なJSONで与え
ることも可能である。このときは、\texttt{JSON.parse()}でJavaScriptのオブ
ジェクトに直せばよい。
\input 17dev10-06.tex

Ajax の通信で大きなデータを渡すためには\texttt{POST}で行う行う。この例
においては57行目から59行目の部分を次のように変更すれば\texttt{POST}で通
信できる。
\LISTN{09-01whatdayPOST.html}{57}{60}{\normalsize}
\begin{itemize}
 \item \ElmJ{open}メソッドの一番目の引数を\Verb+"POST"+に変更し、2番目の
			 引数を呼び出すURLにする。URLには\Verb+"GET"+のときのようにデータ
			 を記述しない。
 \item その後、\ElmJ{setRequestHeader}メソッドで\Verb+Content-type+を指定
			 する。
 \item その後、\Verb+send+メソッドで\Verb+GET+のときと同じ形式でデータを
			 送信する。
 \item 受け取る側のPHPのプログラムでは\Verb+$_GET+のところを
			 \Verb+$_POST+に変更する。
\end{itemize}

\begin{Exec}\upshape\label{contPrimes}
 次のリストは$10,000,000$以下の素数を$1,000,000$づつの区間に分けて求める
 ものである。
 \LISTN{../primes/countPrimes.html}{1}{last}{\normalsize}
 \LISTN{../primes/countPrimes-Ajax.js}{1}{last}{\normalsize}
 \LISTN{../primes/countPrimes.php}{1}{last}{\normalsize}

\end{Exec}