\documentclass[a4j]{jarticle}
\newif\ifHMSol
\newif\ifText
\newif\ifAnswer
\newcommand{\LecName}{\hspace{1zw}ソフトウェア開発}
\input enshuuhead.tex
\Texttrue
\renewcommand{\label}[1]{\relax}
\newcommand{\Ans}[1]{\hfill\underline{\makebox[#1zw][l]{\theenumi.}}}
%\renewcommand{\postchaptername}{回}
\newcommand{\ProblemN}{%
\stepcounter{LecNo}
\Problem{
\ \\[\baselineskip]
足りない場合は裏面を使用してよい。}
}
\newcommand{\ProblemNN}[1]{\stepcounter{LecNo}\Problem{#1}}
\newcommand{\ProblemNR}[1]{\stepcounter{LecNo}\ProblemRev{#1}}
\input 00dev-tmp.aux
\begin{document}
\pagestyle{enshuu}
\renewcommand{\theLecNo}{第\arabic{LecNo}回}
%\chapter{ガイダンス}
\ProblemNN{
 \begin{Prob}\upshape
  次の問いに答えよ。
  \begin{enumerate}
   \item 使用するブラウザの名称とバージョン\\[0.05\textheight]
   \item そのブラウザでの開発者ツールほ開き方とタブの種類
         \\[0.1\textheight]
   \item 簡単な式を実行させたときのキャプチャ画面を印刷してこの提出用紙
				 に張り付ける。貼り付ける用紙の裏面には学籍番号と名前、問題の番
				 号を書いておくこと(はがれたとき、誰のかわかるようにするため)\\[0.2\textheight]
  \end{enumerate}
 \end{Prob}
 \Newpage
 \begin{Prob}\upshape
  課題1.1の実行結果のキャプチャ画面を張り付けなさい。貼り付ける用紙の裏
  面には学籍番号と名前、問題の番号を書いておくこと。出力の形式を変える
  なり、自分で内容を変更してもかまわない。内容を変更した場合にはプログラ
  ムリストもつけること。
 \end{Prob}
}
%\chapter{JavaScriptが取り扱うデータ}
\ProblemNN{
\input prob02-01string.tex
\input prob02-02asign.tex
\input prob02-03.tex
\input prob02-04Date.tex
\Texttrue
\begin{Prob}\upshape
\input prob02-05Check.tex
\end{Prob}
}
%\chapter{関数}
\ProblemNN{
 \begin{Prob}\upshape
  次のプログラムを実行したときのコンソールの出力を記せ。また、その理由も
  述べよ。
  \input prob03-01.tex
 \end{Prob}
\underline{\makebox[10zw][l]{(1)}}\hfill
\underline{\makebox[10zw][l]{(2)}}\hfill
\underline{\makebox[10zw][l]{(3)}}\hfill
\underline{\makebox[10zw][l]{(4)}}\\[\baselineskip]
\underline{\makebox[\textwidth][l]{\bfseries 理由:}}
%\input 03-01OrderOfDefFunction.tex
\input 03-02FunctionWithArbituraryArgs.tex
\input prob03-03declairVAriablesByVar.tex
\Newpage
\input 03-04setTimeout.tex
\begin{Prob}\upshape
 実行例3.7における\texttt{func2()}、\texttt{func3()}と\texttt{func4()}の
 動作を確認しなさい。コンソール画面のキャプチャを貼り付けること。
\end{Prob}
}
%\chapter{オブジェクト}
%\chapter{オブジェクトのプロトタイプと継承}
%\chapter{正規表現}
%\chapter{DOMの利用}
%\chapter{イベント}
%\chapter{PHPの超入門}
%\chapter{サーバーとのデータのやり取り}
%\chapter{jQuery}
%\chapter{JavaScriptライブラリーの配布}
%\chapter{XMLファイルの処理}
%\chapter{システム開発のためのヒント}
\end{document}