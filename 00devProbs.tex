\documentclass[a4j]{jarticle}
\input rubricHead.tex
\input rubricProgDevHead.tex
\newcommand{\Cent}[1]{\hspace*{\fill}#1\hspace*{\fill}\rule{0em}{1ex}}
\newcounter{NN}
\newcounter{STEP}
\newcommand{\ShowNo}[2]{%
  \setcounter{NN}{#1}\setcounter{STEP}{-#2}\Cent{#1}\ShowNoDo}
\newcommand{\ShowNoDo}{%
  \addtocounter{NN}{\value{STEP}}\ifnum\value{NN}>-1\newline\Cent{\arabic{NN}}%
  \expandafter\ShowNoDo\fi
}
\newif\ifHMSol
\newif\ifText
\newif\ifAnswer
\newcommand{\ElmJ}[1]{\texttt{#1}}
\newcommand{\LName}{ソフトウェア開発}
\newcommand{\LecName}{\hspace{1zw}\LName}
\input enshuuhead.tex
\Texttrue
\renewcommand{\label}[1]{\relax}
\newcommand{\Ans}[1]{\hfill\underline{\makebox[#1zw][l]{\theenumi.}}}
%\renewcommand{\postchaptername}{回}
\newcommand{\ProblemN}{%
\stepcounter{LecNo}
\Problem{
\ \\[\baselineskip]
足りない場合は裏面を使用してよい。}
}
\newcommand{\ProblemNN}[1]{\stepcounter{LecNo}\Problem{#1}}
\newcommand{\ProblemNR}[1]{\stepcounter{LecNo}\ProblemRev{#1}}
%\input 00dev-tmp.aux
\begin{document}
\pagestyle{enshuu}
\renewcommand{\theLecNo}{第\arabic{LecNo}回}
%\chapter{ガイダンス}
\input Ex01.tex
%\chapter{JavaScriptが取り扱うデータ}
\input Ex02.tex
%\chapter{関数}
\input Ex03.tex
%\chapter{オブジェクト}
\ProblemNN{
\input 04-01windowProperty.tex
\input 04-02JSONProb.tex
\ \vspace{0.15\textheight}
\input 04-03ChangeProperties.tex
\input 04-04ageAtToday.tex
\Newpage
\input 04-05ageRemoveGet.tex
\input 04-06ageSetter.tex
}
%\chapter{オブジェクトの詳細}
\ProblemNN{
\input 05-01extensibleClass.tex
\input 05-02checkPrototype.tex
\input 05-03errorCheck.tex
\Newpage
\input 05-04errorCheckMoreStrict.tex
}
%\chapter{正規表現}
\ProblemNN{
\input 06-01makeRegEx.tex
\begin{Prob}\upshape
日付を表す文字列 \Verb+"2017年10月27日"+から年(\Verb+2017+)、月(\Verb+10+)、
 日(\Verb+27+)をそれぞれ変数\Verb+y+、\Verb+m+、\Verb+d+に代入するプログ
 ラムを書け。\\[0.02\textheight]
\end{Prob}
\Newpage
\input 06-02checkRegEx.tex
\input 06-03checkPromptValue.tex
}
%\chapter{DOMの利用}
\ProblemNN{
}
%\chapter{イベント}
\ProblemNN{
}
%\chapter{PHPの超入門}
\ProblemNN{
}
%\chapter{サーバーとのデータのやり取り}
\ProblemNN{
}
%\chapter{jQuery}
\ProblemNN{
}
%\chapter{JavaScriptライブラリーの配布}
\ProblemNN{
}
%\chapter{XMLファイルの処理}
\ProblemNN{
}
%\chapter{システム開発のためのヒント}
\ProblemNN{
}
\end{document}