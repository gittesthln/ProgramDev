\chapter{システム開発のためのヒント}
\section{コードの構造}
\subsection{良いコードとは}
良いコードとはどのようなものであろうか。それは個人によっても違うであろう。
しかしながら、多くの公開されているコードを眺めているといくつかの共通点が
見つかる。
\begin{itemize}
 \item プログラムの構造がわかりやすくなるように字下げ(インデント)を付け
			 る。
 \item 変数名や関数名は実態を表すようにする。\\いくつかの単語をつなげて
			 変数名や関数名にすることが多くなる。これらが読みやすいように途中に出てく
			 る単語の先頭は大文字にすることが最近の傾向である(キャメル形式と呼
			 ばれる)。
 \item 不必要に長いプログラム単位を作らない。\\
			 ある程度まとまった作業をする部分は関数にすると見通しの良いプログ
			 ラムとなる。
 \item プログラム自体が説明書になっている。\\
			 コメントを多用して必要な説明をすべて記述する。
 \item プログラムとデータの分離\\
			 ある処理をするときの分岐の条件をプログラム内に直接記述すると、分
			 岐の条件が変わったときにはプログラムをコンパイルしなおす必要が生
			 ずる。これを避けるためには外部からデータを読み込んで、それに基づ
			 いた処理を行うようにすることができるか考えてみる。
\end{itemize}
\subsection{プログラムとデータの分離}
\begin{Exec}\upshape
 期末試験の結果、90点以上ならば\texttt{S}、
 80点以上ならば\texttt{A}、70点以上ならば\texttt{B}、60点以上ならば
 \texttt{C}、それ以外は\texttt{E}の成績をつけることにした。期末試験の点
 を与えて、評価の\texttt{S}などを返す関数\texttt{seiseki}を作成しなさい。
\end{Exec}
一番思いつくコードは次のようなものであろう(JavaScriptで記述)。
\begin{Verbatim}
function seiseki(val) {
  if(val >= 90) return "S";
  if(val >= 80) return "A";
  if(val >= 70) return "B";
  if(val >= 60) return "C";
  return "E";
}
\end{Verbatim}
これに対し、数値と戻り値をそれぞれ配列の値として用意するとつぎのようなコー
ドとなるであろう。
\begin{Verbatim}
function seiseki(val) {
  var Score = [90,80,70,60,0];
  var Results= ["S","A","B","C","E"];
  var i;
  for(i=0;i<Grade.length;$i++) {
    if(val>=Grade[i]) return Hyouka[i];
  }
  return "Error";
}
\end{Verbatim}
二つの配列をまとめて一つの配列にするほうがコードの管理上わかりやすくなる
であろう。
\begin{Verbatim}
var Results =[["S",90],["A",80],["B",70],["C",60],["E",0]];
\end{Verbatim}
または、次のような連想配列にしてもよい。
\begin{Verbatim}
var Results ={S:90,A:80,B:70,C:60,E:0};
function seiseki(val) {
  for(v in Results) {
    if(val >=Results[v]) return v;
  }
}
\end{Verbatim}
JavaScriptでは\texttt{for(.. in ..)}で連想配列の要素をすべて渡ることがで
き、かつわたる順序が定義された順なのでこのコードが可能となる。

このように配列をうまく利用するとロジックの部分とパラメータの部分の分離が
可能となる。

なお、JavaScriptではJavaScript内から外部ファイルを自動で読み込むことがで
きないので上のような変数\texttt{Results}を宣言する必要がある。この変数部
分を別のプログラムから作成すればもう少し管理が楽になるであろう。


\section{コードの管理}
\subsection{複数でシステム開発をするときの問題点}
システムを開発していると次のような問題が発生する。
\begin{itemize}
 \item 複数の人間で開発している場合、開発者の間で最新のコードの共有
 \item ファイルを間違って削除したり、修正がうまくいかなかったときに過去
       のコードに戻す
 \item 安定しているコードに新規の機能を付け加えてテストをする場合、安定
       したコードに影響を与えないようにする
\end{itemize}
このようなコードの変更履歴の管理するソフトウェアをバージョン管理ソフ
トという。バージョン管理ソフトウェアの対象となるファイルはテキストベース
のものを主としている。
\subsection{バージョン管理の概念}
バージョン管理ソフトは各時点におけるファイルの状態を管理するデータベース
である。このデータベースは一般にリポジトリと呼ばれる。

開発者は通常次の手順でシステムの開発を行う。
\begin{itemize}
 \item リポジトリからファイルの最新版を入手
 \item ローカルな環境でファイルを変更。
 \item ファイルをリポジトリに登録
\end{itemize}
バージョン管理ソフトでは同じファイルを複数の開発者が変更した場合、競合が
発生していないかをチェックする機能が備わっている。この機能がない場合には
同じファイルの変更は同時に一人しか変更ができないようにファイルをロックす
る。
バージョン管理システムでは一連の開発の流れがある。この開発の流れをブラン
チと呼ぶ。あるブランチに対して新規の
機能を付け加えたいときなどには元のブランチには手を付けずに別のブランチを
作成して、そこで開発をする。機能が安定すれば元のブランチに統合(merge)す
る。
\subsection{バージョン管理ソフトの概要}
CVS(Concurrent Versions System)やこの改良版であるSubversionはバージョン
管理するためにサーバーが必要となる。開発者はこのサーバーにアクセスしてファ
イルの更新などを行う。

これに対し、git は各開発者のローカルな環境にサーバー上のリポジトリの複製
を持つ。これにより、ネットワーク環境がない状態でもバージョン管理が行える。
git ではネットワーク上に GitHub と呼ばれるサーバーを用意しており、ここに
リポジトリを作成することで開発者間のデータの共有が可能となっている。デー
タを公開すれば無料で利用できるほか、有料のサービスやサーバー自体を個別に
持つサービスも提供している。

現在では有名なオープンソースのプロジェクトが`GitHub を利用して開発を行っ
ている。
\subsection{gitの使い方}
git の詳しい使い方については次のサイトを参照すること。

\Verb+https://git-scm.com/book/ja/v2/+

ここでは簡単な使い方を解説する。

\subsubsection{git の特徴}
git のバージョン管理システムとして次のような点が挙げられている。
\begin{itemize}
 \item 分散型のバージョンシステムなので、ローカルでブランチの作成が可能

       集中型のバージョン管理システムではブランチの作成が一部の人しかで
       きない。
 \item 今までのファイルの変更履歴などが見える。
 \item  GitHub 上ではファイルの更
       新などの通知を受けることができる。
 \item 開発者に対して変更などの要求が可能(pull request)
\end{itemize}

\subsubsection{git のインストール}
git はローカルにリポジトリを持つのでそれを処理する環境をインストールする
必要がある。ここでは git for windows を紹介する。このソフトは次のとこか
らダウンロードできる。

\Verb+https://git-for-windows.github.io/+

デフォルトの設定で十分である。

「Git Bash」、「GitCMD」と「Git GUI」の3つがインストールされる。
Unix 風のコマンドプロンプトが起動する Git Bash がおすすめである。

\subsubsection{初期設定}
ローカルでいくつかの準備が必要となる。

\paragraph{ユーザーネームの登録}
GitHubで使用するリポジトリのユーザ名を登録は次のコマンドで行う。

\Verb+git --config --global user.name "Foo"+

ここでは\Verb+Foo+がユーザ名となる。
\paragraph{メールアドレスの登録}
GitHubからの通知を受けるメールアドレスは次のコマンドで行う。

\Verb+git --config --global user.email "Foo@example.com"+

%git config --global core.editor "'C:/Program Files/Notepad++/notepad++.exe' -multiInst -nosession"
%git config --global core.editor "'C:/Program
%Files/Notepad++/notepad++.exe' -multiInst -nosession"

%設定の確認
%git config --list
\paragraph{SSH Keyの登録}
GitHub との接続には SSH による通信で行う。この通信はは公開鍵暗号方式で行
うので、キーを生成する必要がある。キーの生成は次のコマンドで行う。

\Verb+ssh-keygen -t rsa -C "Foo@example.com"+

\Verb+-C+のオプションはコメントである。 

この後でキーを保存するフォルダが聞かれるが、デフォルトでかまわない。その
後、passphrase の入力が求められるので、何かの文字列を入力する。
マニュアルによれば passphrase はパスワードのようなもので、10から20文字の
長さで大文字、小文字、数字と記号が混在することが推奨されている。再度、同
じものの入力が求められ、一致すれば通信のためのキーが生成される。

指定したフォルダの\Verb+.ssh+の下に\Verb+id_rsa+(秘密鍵)と
\Verb+id_rsa.pub+(公開鍵)の2つのファイルが作成される。 

\subsubsection{GitHub のアカウントの取得}
ローカルだけで開発をするのであればアカウントは必要ないが、データを交換す
るのであればアカウントを取ることを勧める。有料で使用するのでなければこの
ためのアカウントは新たにとることを勧める。

作成した公開鍵の内容をアカウントで設定する必要がある。

アカウントを取ったら GitHub 上でテストのプロジェクトを中身が空で作成する。


\subsubsection{リポジトリの作成と運用}
リポジトリの初期のブランチ名は\Verb+master+となっている。

\begin{tabular}{|c|c|c|}\hline
コマンド&パラメータ&説明\\\hline
\Verb+git init+ & &プロジェクトの初期化\\\hline
\Verb+git add+ & ファイルリスト&プロジェクトへのファイルの登録 \\\hline
\Verb+git commit+& \Verb+-m+ コメント& リポジトリへの変更の登録\\ \hline
\Verb+git remote add+ &短縮名 リポジトリ & リポジトリの短縮名の登録\\
 \hline
\Verb+git push+ &\Verb+-u+ 短縮名 ブランチ名 &ブランチへの変更の登録 \\ \hline
\Verb+git clone+ &リポジトリ名 &リモートのリポジトリのコピーを作成\\ \hline
\Verb+git pull+ &リポジトリ名 ブランチ名 &リモートのリポジトリのコピーを作成\\ \hline
\end{tabular}

通常は次の手順で新しいプロジェクトを構成する。
\begin{enumerate}
 \item \Verb+git init+ で新たにリポジトリを作成する。または、
			 \Verb+git clone+でGitHubからリポジトリのコピーを取得する。
 \item ファイルを編集する。
 \item 編集したファイルを\Verb+git add+ でステージする。
 \item \Verb+git commit+ でその時点での変更を登録する。\Verb+-m+オプショ
			 ンで簡単なコメントをつけることを忘れないこと。このオプションをつ
			 けないと\Verb+vim+というテキストエディタが立ち上がり、コメントの
			 編集を行うこととなる。
 \item \Verb+git push+ リモート名 ブランチ名 で変更をリモートのリポジト
			 リに反映される。これが拒否された場合には誰かが\Verb+push+している
			 のでその変更を\Verb+pull+してから調整して再度\Verb+push+する。
\end{enumerate}
ファイルの状態については次のことに注意する必要がある。
\begin{itemize}
 \item リポジトリ内のファイルは追跡されている(tracked)ものと追跡されていない
(untracked)に分けられる。
 \item ファイルの状態として変更されていない
(unmodified)、変更されている(modified)とステージされている(staged)の3つ
の状態がある。
 \item modified から staged にするコマンドが \Verb+git add+ であ
る。このコマンドを新規にリポジトリに追加するという意味にとらえないことが
必要である。
\end{itemize}