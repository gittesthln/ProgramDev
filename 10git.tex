\chapter{バージョン管理}
\subsection{バージョン管理とは}
システムを開発していると次のような問題が発生する。
\begin{itemize}
 \item 複数の人間で開発している場合、開発者の間で最新のコードの共有
 \item ファイルを間違って削除したり、修正がうまくいかなかったときに過去
       のコードに戻す
 \item 安定しているコードに新規の機能を付け加えてテストをする場合、安定
       したコードに影響を与えないようにする
\end{itemize}
このようなコードの変更履歴の管理するソフトウェアをバージョン管理ソフ
トという。バージョン管理ソフトウェアの対象となるファイルはテキストベース
のものを主としている。
\subsection{バージョン管理の概念}
バージョン管理ソフトは各時点におけるファイルの状態を管理するデータベース
である。このデータベースは一般にリポジトリと呼ばれる。

開発者は通常次の手順でシステムの開発を行う。
\begin{itemize}
 \item リポジトリからファイルの最新版を入手
 \item ローカルな環境でファイルを変更。
 \item ファイルをリポジトリに登録
\end{itemize}
バージョン管理ソフトでは同じファイルを複数の開発者が変更した場合、競合が
発生していないかをチェックする機能が備わっている。この機能がない場合には
同じファイルの変更は同時に一人しか変更ができないようにファイルをロックす
る。
\subsection{バージョン管理ソフトの概要}
CVS(Concurrent Versions System)やこの改良版であるSubversionはバージョン
管理するためにサーバーが必要となる。開発者はこのサーバーにアクセスしてファ
イルの更新などを行う。

これに対し、git は各開発者のローカルな環境にサーバー上のリポジトリの複製
を持つ。これにより、ネットワーク環境がない状態でもバージョン管理が行える。
git ではネットワーク上に github と呼ばれるサーバーを用意しており、ここに
リポジトリを作成することで開発者間のデータの共有が可能となっている。デー
タを公開すれば無料で利用できるほか、有料のサービスやサーバー自体を個別に
持つサービスも提供している。

  