次のリストは\texttt{jQuery-3.1.1.js}の冒頭の部分である\footnote{元来の
ソースどのインデントはタブでつけているがここでは空白2文字に変更してある。
また、一部の空行も省略した。}。
\LISTN{jQuery-3.1.1.js}{1}{41}{\small}

このリストでは詳細なコメントが付いていて比較的読みやすい。コードから次の
ことがわかる。
\begin{itemize}
 \item このバージョンでは\Strict で動作する(15行目と41行目)
 \item クロスブラウザ対策が行われている(37行目から40行目のコメント)。
 \item 36行目で定義されている関数の仮引数に\Verb+window+が使われている。
\end{itemize}
次のリストはこの部分の最小化をした部分のリストである。元来は改行が入って
いないが、対応をわかりやすくするために改行を入れてある。
\LISTN{jQuery-3.1.1.min.js}{1}{8}{\small}


短縮化では次のことを行っていることがわかる。
\begin{itemize}
 \item 各関数内で変数名は1文字から始めている。
 \item 関数の仮引数も\texttt{a}から付け直している。
 \item JavaScriptの固有の関数は当然のことながら変換されていない。
 \item このライブラリーは一つの関数を定義して、その場で実行している。14
       行目の\texttt{function()}の前に\texttt{()}がついている。
 \item 短縮化されたコードではこの部分が\texttt{!function()}となっている。
       関数オブジェクトを演算の対象とすることでその場で実行する。
 \item そのほかにもキーワード\texttt{true}の代わりに\texttt{!0}としてい
       る。
 \item \texttt{if()\{\}else\{\}}構文は\texttt{?}に置き換えている。
\end{itemize}
