%-*- coding: utf-8 -*-
%\subsection{ソースコードの最小化}
\iffalse
jQueryのソースファイルには\texttt{jQuery-<Version No>.js}と
\texttt{jQuery-<Version No>.min.js}の2種類が存在する。
\texttt{jQuery-<Version No>.min.js}は\texttt{jQuery-<Version No>.js}から
コメントや、空白を取り除き、変数名を短いものに置き換えるという操作を行っ
て、ファイルサイズを小さくしたものである。
\fi
次のリストは\texttt{jQuery-1.11.3.js}の冒頭の部分である\footnote{元来の
ソースどのインデントはタブでつけているがここでは空白2文字に変更してある。
また、一部の空行も省略した。}。
\begin{Verbatim}[numbers=left,fontsize=\small]
/*!
 * jQuery JavaScript Library v1.11.3
 * http://jquery.com/
 *
 * Includes Sizzle.js
 * http://sizzlejs.com/
 *
 * Copyright 2005, 2014 jQuery Foundation, Inc. and other contributors
 * Released under the MIT license
 * http://jquery.org/license
 *
 * Date: 2015-04-28T16:19Z
 */
(function( global, factory ) {
  if ( typeof module === "object" && typeof module.exports === "object" ) {
    // For CommonJS and CommonJS-like environments where a proper window is present,
    // execute the factory and get jQuery
    // For environments that do not inherently posses a window with a document
    // (such as Node.js), expose a jQuery-making factory as module.exports
    // This accentuates the need for the creation of a real window
    // e.g. var jQuery = require("jquery")(window);
    // See ticket #14549 for more info
    module.exports = global.document ?
      factory( global, true ) :
      function( w ) {
        if ( !w.document ) {
          throw new Error( "jQuery requires a window with a document" );
        }
        return factory( w );
      };
  } else {
    factory( global );
  }
// Pass this if window is not defined yet
}(typeof window !== "undefined" ? window : this, function( window, noGlobal ) {
// Can't do this because several apps including ASP.NET trace
// the stack via arguments.caller.callee and Firefox dies if
// you try to trace through "use strict" call chains. (#13335)
// Support: Firefox 18+
//
var deletedIds = [];
var slice = deletedIds.slice;
var concat = deletedIds.concat;
var push = deletedIds.push;
var indexOf = deletedIds.indexOf;
var class2type = {};
var toString = class2type.toString;
var hasOwn = class2type.hasOwnProperty;
var support = {};
var
  version = "1.11.3",
  // Define a local copy of jQuery
  jQuery = function( selector, context ) {
    // The jQuery object is actually just the init constructor 'enhanced'
    // Need init if jQuery is called (just allow error to be thrown if not included)
    return new jQuery.fn.init( selector, context );
  },
  // Support: Android<4.1, IE<9
  // Make sure we trim BOM and NBSP
  rtrim = /^[\s\uFEFF\xA0]+|[\s\uFEFF\xA0]+$/g,
  // Matches dashed string for camelizing
  rmsPrefix = /^-ms-/,
  rdashAlpha = /-([\da-z])/gi,
  // Used by jQuery.camelCase as callback to replace()
  fcamelCase = function( all, letter ) {
    return letter.toUpperCase();
  };
\end{Verbatim}

このリストでは詳細なコメントが付いていて比較的読みやすい。またコメントを
見るとクロスブラウザ対策が行われていることも見て取れる。

ソースコードの最小化のテクニックとしてグローバルなオブジェクトを短い変数
名で置き換えることを行っている。たとえば、35行目では冒頭の定義された関数
の引数に、関数を渡している。その関数の仮引数は\texttt{window}となってい
る。これにより、開発時はわかりやすい変数名が使えるメリットがある。

次のリストはこの部分の最小化をした部分のリストである。元来は改行が入って
いないが、対応をわかりやすくするために改行を入れてある。

\begin{Verbatim}[numbers=left, fontsize=\small]
/*! jQuery v1.11.3 | (c) 2005, 2015 jQuery Foundation, Inc. | jquery.org/license */
!function(a,b){"object"==typeof module&&"object"==typeof
    module.exports?module.exports=a.document?b(a,!0):function(a){if(!a.document)throw
    new Error("jQuery requires a window with a document");
  return b(a)}:b(a)}
    ("undefined"!=typeof window?window:this,function(a,b){
      var c=[],d=c.slice,e=c.concat,f=c.push,g=c.indexOf,h={},i=h.toString,
  j=h.hasOwnProperty,k={},l="1.11.3",
  m=function(a,b){return new  m.fn.init(a,b)},
  n=/^[\s\uFEFF\xA0]+|[\s\uFEFF\xA0]+$/g,o=/^-ms-/,
  p=/-([\da-z])/gi,q=function(a,b){return b.toUpperCase()};
\end{Verbatim}

短縮化のために次のことを行っていることがわかる。
\begin{itemize}
 \item 各関数内で変数名は1文字から始めている。
 \item 関数の仮引数も\texttt{a}から付け直している。
 \item JavaScriptの固有の関数は当然のことながら変換されていない。
 \item このライブラリーは一つの関数を定義して、その場で実行している。14
       行目の\texttt{function()}の前に\texttt{(}がついている。
 \item 短縮化されたコードではこの部分が\texttt{!function()}となっている。
       関数オブジェクトを演算の対象とすることで実行する。
 \item そのほかにもキーワード\texttt{true}の代わりに\texttt{!0}としてい
       る。
 \item \texttt{if()\{\}else\{\}}構文は\texttt{?}に置き換えている。
\end{itemize}
