%-*- coding: utf-8 -*-
\chapter{PHPの超入門}
\section{Web サーバーの基礎知識}
\subsection{サーバーの重要な設定項目}
Web サーバーは hypertext transfer protocol(HTTP) を利用して、クライアン
トプログラム(Webブラウザなど)からの要求を処理する。Apache はこのサービスを提供す
るものとして有名である。Apache の設定ファ
イル \texttt{httpd.conf} の中で設定される重要な項目は
次のとおりである。
\begin{itemize}
 \item ポート番号:TCP/IP のサービスを識別するための
       番号で、次の3種類に分けられる
\footnote{\texttt{https://tools.ietf.org/html/rfc6335\#section-6}}。
\begin{center}\vspace{-\baselineskip}
 \begin{tabular}{|c|r@{番$\sim$}r<{番}|c|}
\hline
種類 &\multicolumn{2}{c|}{範囲} &内容 \\\hline
  System Ports(Well Known Ports)& 1&1023 & assigned by IANA\\ \hline
  User Ports, (Registered Ports)& 1024&49151 & assigned by IANA\\
  \hline
  Dynamic Ports(Private or Ephemeral Ports)& 49152& 65535&never assigned \\ \hline
 \end{tabular}
\end{center}
IANA(Internet Assigned Numbers Authority)は DNS Root, IPアドレスやインター
       ネットプロトコルのポート番号などを管理している団体である。


 \item ドキュメントルート:Apache では Webサーバーに直接要求できるファイルはこのフォルダ
       (ディレクトリ)の下にあるものだけである。
\end{itemize}
 なお、HTTPのSystem ポート番号は80番となっている\footnote{ポート番号の一覧は次のところにある。\\
%
 \texttt{https://www.iana.org/assignments/}\texttt{service-names-port-numbers/service-names-port-numbers.xhtml}}。

\subsection{CGI}
Common Gateway Interface(CGI)とはWebコンテンツを動的に生成する標
準の方法で、Webサーバー上ではWebサーバーとWebコンテンツを生成するた
めのインターフェイスを与える。
この授業ではCGIのプログラムはPHP(PHP:
%
Hypertext Preprocessor)を使用する。

\subsection{Webサーバーのインストール}
Webサーバー(Apache)やPHPをインストールし、HTTPのサーバー上でPHPと連携さ
せるためにはいくつかの設定をする必要がある。これらのインストール
を一括で行うことができるパッケージも存在する。
\paragraph{XAMPP}XAMPPは Apache、MySQL、PerlとPHPを一括してインストール
するパッケージである。インストール時にはいくつかのパッケージをインストー
ルしない選択も可能である。%なお、先頭の X はクロスプラットフォームを意味
													%する。
XAMPPをインストールしたときに注意する点は次のとおりである。
\begin{itemize}
 \item インストール時に Apache をサービスとして実行するとパソコンを立ち
       上げた時にApacheを起動させる手間が省ける。
 \item 標準の文字コードはISO-8859-1である。
 \item PHPの設定でタイムゾーンがヨーロッパ大陸になっている。タイムスタン
       プを利用するプログラムがうまく動かない場合あがあるので注意するこ
       と。
\end{itemize}

\input 0910PHP.tex

\input 0920ExchangeData.tex

