%-*- coding: utf-8 -*-
\chapter{PHPの超入門}
\section{Web サーバーの基礎知識}
\subsection{HTTPプロトコルの基礎}
 Web ページとサーバー間の通信には \texttt{http}(hypertext transfer
 protocol) または \texttt{https} が
 用いられる。これらのプロトコルには次のようなメソッドが定義されている。
	\begin{table}[htb]
	 \caption{httpのよく使われるメソッド}
 \begin{center}
	\begin{tabular}{|c|l|}\hline
	 \ElmJ{GET} & 指定されたリソースを要求する。\\\hline
	 \ElmJ{POST}& 指定されたURIのリソースを作成する。\\ \hline
	 \ElmJ{PUT}& 指定されたリソースを置き換えたり、集めたりする。\\ \hline
	 \ElmJ{DELETE}& 指定したリソースを消去する。\\ \hline
	 \ElmJ{HEAD}& 指定したリソースのHTTPヘッダーの部分だけ要求する。\\ \hline
	\end{tabular}
 \end{center}
	\end{table}

	後で見るように、これらのメソッドの定義を無視してサーバーのデータを操作
	したり、クライアント側に返すことも可能である。サーバーと
	クライアント側のやり取りを提供するWebサービス(Web API)ではメソッドに即
	した設計が求められている。このような設計思想で作られた API は RESTfull
	なAPIと呼ばれているようである\footnote{
	\texttt{https://ja.wikipedia.org/wiki/Representational\textunderscore
	State\textunderscore Transfer}}。
\subsection{サーバーの重要な設定項目}
Web サーバーは HTTPプロトコルを利用して、クライアン
トプログラム(Webブラウザなど)からの要求を処理する。
%
Apache はこのサービスを提供す
るものとして有名である。Apache の設定ファ
イル \texttt{httpd.conf} の中で設定される重要な項目は
次のとおりである。
\begin{itemize}
 \item ポート番号:TCP/IP のサービスを識別するための
       番号で、次の3種類に分けられる
\footnote{\texttt{https://tools.ietf.org/html/rfc6335\#section-6}}。
\begin{center}\vspace{-\baselineskip}
 \begin{tabular}{|c|r@{番$\sim$}r<{番}|c|}
\hline
種類 &\multicolumn{2}{c|}{範囲} &内容 \\\hline
  System Ports(Well Known Ports)& 1&1023 & assigned by IANA\\ \hline
  User Ports, (Registered Ports)& 1024&49151 & assigned by IANA\\
  \hline
  Dynamic Ports(Private or Ephemeral Ports)& 49152& 65535&never assigned \\ \hline
 \end{tabular}
\end{center}
IANA(Internet Assigned Numbers Authority)は DNS Root, IPアドレスやインター
       ネットプロトコルのポート番号などを管理している団体である。


 \item ドキュメントルート:Apache では Webサーバーに直接要求できるファイルはこのフォルダ
       (ディレクトリ)の下にあるものだけである。
\end{itemize}
 なお、HTTPのポート番号は80番、HTTPSのポート番号は443番となっている\footnote{ポート番号の一覧は次のところにある。\\
%
 \texttt{https://www.iana.org/assignments/}\texttt{service-names-port-numbers/service-names-port-numbers.xhtml}}。


\subsection{CGI}
Common Gateway Interface(CGI)とはWebコンテンツを動的に生成する標
準の方法で、Webサーバー上ではWebサーバーとWebコンテンツを生成するた
めのインターフェイスを与える。
この授業ではCGIのプログラムはPHP(PHP:
%
Hypertext Preprocessor)を使用する。

\subsection{Webサーバーのインストール}
Webサーバー(Apache)やPHPをインストールし、HTTPのサーバー上でPHPと連携さ
せるためにはいくつかの設定をする必要がある。これらのインストール
を一括で行うことができるパッケージも存在する。
\paragraph{XAMPP}XAMPPは Apache、MySQL、PerlとPHPを一括してインストール
するパッケージである。インストール時にはいくつかのパッケージをインストー
ルしない選択も可能である。%なお、先頭の X はクロスプラットフォームを意味
													%する。
XAMPPをインストールしたときに注意する点は次のとおりである。
\begin{itemize}
 \item インストール時に Apache をサービスとして実行するとパソコンを立ち
       上げた時にApacheを起動させる手間が省ける。
 \item 標準の文字コードは{UTF-8}である。
 \item PHPの設定でタイムゾーンがヨーロッパ大陸になっている。タイムスタン
       プを利用するプログラムがうまく動かない場合あがあるので注意するこ
       と\footnote{\Verb+date\_default\_timezone\_get()+関数で調べることが
       できる。}。
\end{itemize}

\input 0910PHP.tex

