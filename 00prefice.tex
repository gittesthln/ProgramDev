%-*- coding: utf-8 -*-
JavaScript はHTML文書内で使われる場合が多いが、通常の計算機言語として取
り扱うこともできる。
この授業ではJavaScriptの計算機言語としての特徴について述べた後、
Webブラウザ内での処理について解説を行う。しかし、JavaScriptを単独
で実行する環境はほとんどなく\footnote{コマンドラインからJavaScriptを実行
する環境としては Node.js があるが、新規にインストールする必要がある。}、
このテキストではブラウザ内で実行する例だけを取り上げる。

最近のブラウザはJavaScriptの実行環境を提供するだけではなく、開発環境も提
供している。Google Chromeの「デベロッパーツー
ル」、FireFoxの「Web開発」、%Opera の開発者用ツール、
Internet Explorer の「開発者ツール」などがそうである。

これらのツールは「F12」または「Control+Shift+I」のショートカットキーで表
示、非表示ができる。表示されるタブには次のようなもの
がある\footnote{この例は Chrome のものの一部である。}。
%この機能の具体的な使い方についてはこのテキスト内で解説をする。
\begin{itemize}
 \item {\bfseries Elements }
       HTMLの構造(DOMの構造)を表示する。対応する要素の部分が反転
       する。
 \item {\bfseries Console }エラーや警告が表示される。また、
       JavaScriptのプログラムが対話形式で実行できる。
 \item {\bfseries Sources }HTML文書のソースが表示される。JavaScriptのソー
       スの場合にはプログラムの実行を中断するブレイクポイントの設定が可
       能である。
 \item {\bfseries Network }ファイルの取得などの順序やかかった時間などが
       表示される。
 \item {\bfseries Timeline }ブラウザの処理に関する情報が表示される。
 \item {\bfseries Profiles }JavaSCriptの関数で使われた実行時間などが記録
       できる。
\end{itemize}