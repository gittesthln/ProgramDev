\newcommand{\Elm}[1]{\texttt{<#1>}}
  \ProblemNN{
\input 17prob07-1.tex
\input 17prob07-2.tex
\Newpage
\input 17prob07-3.tex
}
\RubricC{
復習の目的は次の項目を理解することである。
\begin{itemize}
 \item CSSセレクタの基礎を理解する。
 \item ブラウザでDOMツリーの構造を調べる方法を取得する。
 \item 実行時にDOMを作成する方法を理解する。
 \item \texttt{childNodes}と\texttt{children}の違いを理解する。
\end{itemize}
}
{
{\RProbNoM{1.1}{0.2}}{8}
 {
 {すべての結果が正しく説明されている。}
 {リストの数をを増減して問題以外の場合も確認して、考察が正しいかを検証している}
 {結果を説明するために十分な数のキャプチャ画面がある}
 {キャプチャ画面の大きさが結果を説明するために十分である}
 }
 {
 {(a)の結果が正しい。}
 {(b)の結果が正しい。}
 {(c)の結果が正しい。}
 {(c)の説明がリストの数が与えられた場合(6)のときだけ正しい。}
 {(d)の結果が正しい。}
 {(d)の説明がリストの数が与えられた場合(6)のときだけ正しい。}
 {キャプチャ画面の大きさが結果を説明するためは大きすぎる。}
 }
 {
 {多くの結果が間違っている。}
 {(a)の考察ががないか間違っている。}
 {(b)の考察ががないか間違っている。}
 {(c)の考察ががないか間違っている。}
 {(d)の考察ががないか間違っている。}
 {結果を説明するキャプチャ画面がないか少なすぎる。}
 }
{\RProbNoM{1.2}{0.2}}{8}
 {
 {すべての結果が正しく説明されている。}
 {リストの数をを増減して問題以外の場合も確認して、考察が正しいかを検証している}
 {結果を説明するために十分な数のキャプチャ画面がある}
 {キャプチャ画面の大きさが結果を説明するために十分である}
 }
 {
 {(a)の結果が正しい。}
 {(b)の結果が正しい。}
 {(c)の結果が正しい。}
 {(c)の説明がリストの数が与えられた場合(6)のときだけ正しい。}
 {(d)の結果が正しい。}
 {(c)の説明がリストの数が与えられた場合(6)のときだけ正しい。}
 {キャプチャ画面の大きさが結果を説明するためは大きすぎる。}
 }
 {
 {多くの結果が間違っている。}
 {(a)の考察ががないか間違っている。}
 {(b)の考察ががないか間違っている。}
 {(c)の考察ががないか間違っている。}
 {(d)の考察ががないか間違っている。}
 {結果を説明するキャプチャ画面がないか少なすぎる。}
 }
{\RProbNoMM{2.1~2.2}{0.95}{0.1}}{6}
 {
 {ブラウザの「ソースの表示」の適切な大きさのキャプチャがある。}
 {DOMツリーを展開して、\Elm{option}要素がすべて見えるようにして、内容が
 読める程度の大きさのキャプチャがある。}
 {十分な量の考察がある。}
 }
 {
 {ブラウザの「ソースの表示」のキャプチャが大きすぎる。または小さくて内容
 が読みにくい。}
 {DOMツリーを展開して、\Elm{option}要素がすべて見えるキャプチャがあるが、内容が
 読める程度の大きさになっていない。または図が大きすぎる。}
 {考察があが、内容の一部に間違い、不十分な点がある。}
 }
 {
 {ブラウザの「ソースの表示」のキャプチャがない。または間違ったものをキャ
 プチャしている。}
 {DOMツリーを展開していない。または、\Elm{option}要素がすべて見えるキャプチャがない。
 大きさが小さくて内容が全く読めない。}
 {考察がない。または、内容が間違っている。}
 }
{問題2.3}{8}
 {
 {要求した仕様に合う関数が定義されている。}
 {定義した関数のリストがあり、十分な解説がある。}
 {実行結果のキャプチャの大きさが適切である。}
 }
 {
 {\Elm{option}要素の初期値が指定できる。}
 {\Elm{option}要素の最後の値が指定できる。}
 {表示する数字の前後の文字の指定ができる。}
 {リストの解説が少し不十分である。}
 {作成したプルダウンメニューを開いたキャプチャがある。}
 {実行結果のキャプチャの大きさが大きすぎるが、小さすぎる。}
 }
 {
 {\Elm{option}要素の初期値が1に固定されている。}
 {\Elm{option}要素の最後の値が固定されている。}
 {表示する数字の前後の文字のいずれかまたは両方が指定ができない。}
 {リストの解説がないか、不十分である。}
 {作成したプルダウンメニューを開いたキャプチャがない。}
 {実行結果のキャプチャの内容が読めない。}
 }
{\RProbNoM{3}{0.5}}{7}
 {
 {\Elm{select}要素の子要素の数を正しく調べている。}
 {2つのリストの子要素の違いについて理由が正しい。}
 }
 {
 {\texttt{childNodes.length}を使
 うかコンソールに出力して2つのリストの子要素の数を調べていて、結果のキャ
 プチャ画面がある。}
 {\texttt{childNodes[0].nodeType}を調べている結果のキャ
 プチャ画面がある。}
 {\texttt{childNodes}の代わりに\texttt{children}で調べている結果のキャ
 プチャ画面がある。}
 {考察が少し不十分である。}
 }
 {
 {\pageref{pulldown1}ページから始まるリストに関する子要素の数の調べ方が間違って
 いる。}
 {\pageref{pulldown1}ページから始まるリストに関する子要素の数が間違って
 いる。}
 {\pageref{pulldown2}ページから始まるリストに関する子要素の数の調べ方が間違って
 いる。}
 {\pageref{pulldown2}ページから始まるリストに関する子要素の数が間違って
 いる。}
 {\texttt{childNodes[0].nodeType}で調べていない。また、結果のキャ
 プチャ画面がない。}
 {\texttt{childNodes}の代わりに\texttt{children}で調べていない。}
 {考察がないか、間違っている。}
 }
}