% -*- coding: utf-8 -*-
\newcommand{\ElmJC}[2]{\ElmJ{Math.#1}%\IndexSetN{#2}{}{}{}{JS}
}
\ListAttribsF{JSMath}{\JS における\ElmJ{Math}オブジェクトの種類}{|c|c|c|}%
{{名称}&{種類}&{説明}}
{{\ElmJC{E}{e@$e$}}{定数}{自然対数の底($2.71828182\dots$)}
{\ElmJC{LN10}{log10@$\protect\log10$}}{定数}{$\log_{e}10$}
{\ElmJC{LN2}{log2@$\protect\log2$}}{定数}{$\log_{e}2$}
{\ElmJC{LOG2E}{log2e@$\protect\log_2 e$}}{定数}{$\log_{2}e$}
{\ElmJC{LOG10E}{\log10e$\protect\log_{10}e$}}{定数}{$\log_{10}e$}
{\ElmJC{PI}{pi@$\protect\pi$}}{定数}{円周率($\pi=3.141592\dots$)}
{\ElmJC{SQRT1\textunderscore2}{sqrt1/2@$\protect\sqrt{\protect\dfrac{1}{2}}$}}{%
定数}{$\dfrac{1}{2}$の平方根 $\sqrt{\dfrac{1}{2}}$}
{\ElmJC{SQRT2}{sqrt2@$\sqrt{2}$}}{定数}{$2$の平方根 $\sqrt{2}$}
{\ElmJC{abs(x)}{|x|@$|x|$}}{関数}{$x$ の絶対値}
{\ElmJC{acos(x)}{arccos@$\protect\arccos x$}}{関数}{逆余弦関数 $\arccos x$}
{\ElmJC{asin(x)}{arcsin@$\protect\arcsin x$}}{関数}{逆正弦関数 $\arcsin x$}
{\ElmJC{atan(x)}{arctan@$\protect\arctan x$}}{関数}{逆正接関数 $\arctan x$}
{\ElmJC{atan2(y, x)}{arctan2@$\protect\arctan \dfrac{y}{x}$}}{関数}%
   {$\arctan\dfrac{y}{x}$を計算。$x$が$0$のときでも正しく動く}
{\ElmJC{ceil(x)}{}}{関数}{$x$以上の整数で最小な値を返す。}
{\ElmJC{cos(x)}{}}{関数}{余弦関数 $\cos x$}
{\ElmJC{exp(x)}{}}{関数}{指数関数 $e^x$}
{\ElmJC{floor(x)}{}}{関数}{$x$の値を超えない整数}
{\ElmJC{log(x)}{}}{関数}{自然対数$\log x$}
{\ElmJC{max([x1, x2, ... , xN])}{}}{関数}{与えられた引数のうち最大値を返す}
{\ElmJC{min([x1, x2, ... , xN])}{}}{関数}{与えられた引数のうち最小値を返す}
{\ElmJC{pow(x, y)}{xy@$x^y$}}{関数}{指数関数 $x^y$}
{\ElmJC{random()}{}}{関数}{$0$と$1$の間の擬似乱数を返す}
{\ElmJC{round(x)}{}}{関数}{$x$の値を四捨五入する}
{\ElmJC{sin(x)}{}}{関数}{正弦関数 $\sin x$}
{\ElmJC{sqrt(x)}{}}{関数}{平方根を求める。 $\sqrt{x}$}
{\ElmJC{tan(x)}{}}{関数}{正接関数 $\tan x$}
}
いくつか注意をしておく。
\begin{itemize}
 \item 配列の要素の最大値や最小値を求めるために\Verb+Math.max+ や
       \Verb+Math.min+を直接使用できない。展開演算子\Verb+...+を用いると
       計算ができる。次の実行例を参照のこと。
\begin{Verbatim}
>a = [3,5,23,1,4];
(5) [3, 5, 23, 1, 4]
>Math.max(a);
NaN
>Math.max(...a);
23
\end{Verbatim}
 \item EcmaScript 2016 ではべき乗の演算子\texttt{**}が導入されているので
\texttt{Math.pow()}は使わなくてもよい。
\end{itemize}


