%-*- coding: utf-8 -*-
\documentclass[dvipsk]{beamer}
\usepackage{pgfpages}
\usepackage{moreverb,array}
%\usetheme{Malmoe}
%\usetheme{Goettingen} %7 8 9
%\usetheme{Darmstadt}% 1, 2, 3
%\usetheme{Boadilla}
%\usetheme{CambridgeUS}%4, 5 6 
%
\usetheme{AnnArbor} %10
%\usetheme{Marburg}
\iffalse\else
\renewcommand{\familydefault}{\sfdefault}
\renewcommand{\kanjifamilydefault}{\gtdefault}
\setbeamerfont{title}{size=\Large,series=\bfseries}%,color=white}
\setbeamerfont{frametitle}{size=\large,series=\bfseries}
\setbeamerfont{beamergotobutton}{size=\Large}
%\setbeamercolor{frametitle}{fg=yellow}
\setbeamertemplate{frametitle}[default][center]
%\addtobeamertemplate{footline}{}{\insertframenumber/\inserttotalframenumber}
\usefonttheme{professionalfonts}
\fi
%\iftrue
\title{ソフトウェア開発\\第14回目授業}
\author{平野 照比古}
\institute{}
\date{2015/11/16}
\newtheorem{Prob}{解説}
\newcommand{\Elm}[1]{\texttt{<#1>}}
\setbeamercovered{transparent}

\newcommand{\DOMM}{\texttt}
\newcommand{\Event}{\texttt}
\newcommand{\DOMP}{\texttt}
\newcommand{\DOM}{\texttt{DOM}}
\newcommand{\keyitem}{\relax}
\newcommand{\HTML}{HTML文書}
\begin{document}
\frame{\maketitle}
\section{Google Maps入門}
\subsection{Google Mapsとは}
\begin{frame}[containsverbatim]
\frametitle{Google Mapsとは}
Google Maps は Google が提供する地図サービスである。Google Maps
JaaScript APIのページ\footnote{
\texttt{https://developers.google.com/maps/documentation/javascript/?hl=ja}}
には次のような記述がある。
\begin{quotation}
 Google Maps JavaScript API を使用すると、Google マップをウェブページに埋
 め込むことができます。この API のバージョン 3 は、従来のパソコン用ブラ
 ウザ アプリケーションとしてだけでなく、携帯端末でも快適に動作するように
 設計されています。 

この API では http://maps.google.co.jp サイトで使用できるような地図を操
 作し、さまざまなサービスを介してコンテンツを地図に追加するための多数の
 ユーティリティを提供しています。これを利用して、ウェブサイトにパワフル
 な地図アプリケーションを作成できます。 

JavaScript Maps API V3 は、誰でも自由にアクセスできるウェブサイトであれ
 ば、無料で利用できるサービスです。詳細については、利用規約をご覧くださ
 い。
\end{quotation}
\end{frame}
\begin{frame}[containsverbatim]
\frametitle{}
\begin{itemize}
 \item このAPIを用いて地図を表示する例を与えている。
 \item APIの仕様はデヴェロッパーガイドやAPIリファレンス(英語です)コー
       ドサンプルが参考になる。
\end{itemize}
\end{frame}
\subsection{Google Maps のオブジェクト}
\begin{frame}[containsverbatim]
\frametitle{Google Maps のオブジェクト}
Google Mapsで提供されるものの代表的なもの
\begin{itemize}
 \item Map

       地図を表示するためのオブジェクトである。
 \item Controls

       地図をコントロールするオブジェクトである。ズームや、地図の移動を
       可能にするコントロールが含まれる。
 \item Overlays

       地図上に表示されるオブジェクトである。折れ線、多角形、マーカーな
       どが含まれる。
 \item サービス

       地図を利用する際に必要となるサービスである。緯度経度からその位置
       の情報(建物や住所)を得るGeocoder、ルート検索をするDirectionサー
       ビスなどを含む。
\end{itemize}
\end{frame}
\subsection{GPSログを表示する}

\begin{frame}[containsverbatim]
\frametitle{GPS機器で記録された移動記録をGoogle Maps 上に表示するサンプル}
\begin{itemize}
 \item GPSのログデータのリストを記述したファイルを読み込む
 \item その内容に基づいてルートを表示
 \item GPSのログデータのファイルの内容は各行に一つファ
イル名が書かれている。
\begin{verbatim}
20140412.gpx
20140922.gpx
20141128.gpx
20141128-plan.gpx
\end{verbatim}
\end{itemize}
\end{frame}

\begin{frame}[containsverbatim]
\frametitle{地図を表示するページのリスト(1)}
URLで指定したオプションパラメーターの処理を行う関数を二つ定義
{\small
\begin{listing}{1}
<?php
function setParam($name, $default, $min) {
  $val = $default;
  if(array_key_exists($name,$_GET)) {
    $val = $_GET[$name];
    if($val <$min) $val = $min;
  }
  return $val;
}
function setParamS($name, $default) {
  $val =$default;
  if(array_key_exists($name,$_GET)) {
    $val = $_GET[$name];
  }
  return $val;
 }
\end{listing}
}
\end{frame}
\begin{frame}[containsverbatim]
\frametitle{地図を表示するページのリスト(1)--解説}
\begin{itemize}
 \item 関数\verb+function setParam($name, $default, $min)+はパラメータ
       が数値の場合を取り扱う。この関数は次の3つの引数をとる。
\begin{itemize}
 \item 一番目の引数\verb+$name+はパラメータの名称
 \item 2番目の引数\verb+$default+はパラメータの値がなかった場合のデフォル
       ト値
 \item 3番目の引数\verb+$min+は設定するパラメータの最小値
\end{itemize}
値をデフォルト値に設定したのち(3行目)、URLに指定した引数があるかチェック
       し(4行目)、あった場合にはその値が指定された最小値より大きいか調べ、
       小さい場合には指定された最小値に設定する(5行目から6行目)。
 \item 関数\verb+setParamS($name, $default)+は文字列のパラメータを設定す
       るための関数である。この関数には設定される値のチェックはしない。
\end{itemize}
\end{frame}
\begin{frame}[containsverbatim]
\frametitle{地図を表示するページのリスト(2)}
\begin{listingcont}
$w = setParam("w",1000,200);
$h = setParam("h",800,200);
$hl =setParamS("hl","ja");
\end{listingcont}
ここでは地図の横幅(\verb+$w+)、高さ(\verb+$h+)、
%ズームレベルの補正(\verb+$zoom+)、
表示に使用する言語(\verb+$hl+)などの値を設定している。
\end{frame}

\begin{frame}[containsverbatim]
\frametitle{地図を表示するページのリスト(3)}
{\footnotesize
\begin{listingcont}
$latMaxG=$latminG=35.486210;
$lonMaxG=$lonminG=139.341443;

$filenames = file("history.dat",FILE_IGNORE_NEW_LINES|FILE_SKIP_EMPTY_LINES);
Print <<<_EOL_
<!DOCTYPE html>
<html>
<head>
<title>My Trace</title>
<meta http-equiv="Content-Type" content="text/html; charset=UTF-8"/>
<link rel="stylesheet" type="text/css" href="map.css"/>
<script src="http://maps.google.com/maps/api/js?sensor=false" 
        type="text/javascript" charset="UTF-8"></script>
<script src="jquery-1.11.2.min.js" type="text/javascript" 
        charset="UTF-8"></script>
<script src="map4.js" type="text/ecmascript"></script>
\end{listingcont}
}
\end{frame}

\begin{frame}[containsverbatim]
\frametitle{}
\begin{itemize}
 \item 21行目と22行目ではデフォルト地点の緯度(\verb+$latminG+)、経度
       (\verb+latminG+)を定義(この値は神奈川工科大学のもの)
 \item 24行目では読み込むGPSファイルのリストが入ったファイルを行単位で配
       列に格納
 \item 関数\texttt{file()}の2番目の引数の定数は、行末
       の改行文字を取り除き(\verb+FILE_IGNORE_NEW_LINES+)、かつ空行を
       読み飛ばす(\verb+FILE_IGNORE_NEW_LINES+)を論理和(\texttt{|})で
       つないでいる。
 \item 25行目から46行目では出力するHTML文書の先頭部分をヒアドキュメント
       の形式で記述し、出力
\item 32行目から33行目ではGoogle Maps のライブラリーを、34行目か
      ら35行目ではjQuery のライブラリーを読み込んでいる。
\item 36行目では39行目で呼び出される関数\texttt{initialize()}な
      どが定義されていいるJavaScriptのファイルを読み込んでいる。
\end{itemize}
\end{frame}

\begin{frame}[containsverbatim]
\frametitle{地図を表示するページのリスト(4)}
\begin{listingcont}
<script type="text/ecmascript">
window.onload = function() {
  initialize($latMaxG,$lonMaxG,$latminG,$lonminG);
}
</script>
  </head>
 <body>
<div id="map_canvas" style="width:{$w}px; height:{$h}px"></div>
  <form id="filenames">
_EOL_;
\end{listingcont}
\begin{itemize}
 \item 38行目から40行目ではこのHTML文書での処理が終わった後の関数
	      を定義。単純に\texttt{initialize()}を呼び出し。
 \item 44行目では地図を表示するための\texttt{<div>}要素を定義して
	      いる。この中ではPHPの変数\verb+\w+と\verb+$h+の値が展開さ
	      れて埋め込まれるる。
\end{itemize}
\end{frame}
\begin{frame}[containsverbatim]
\frametitle{地図を表示するページのリスト(5)}
{\footnotesize
\begin{listingcont}
for($i=0;$i<count($filenames); $i++) {
  print '<div class="fn"><div><input type="checkbox" id="file'.$i.
      '"/></div><div>'. $filenames[$i]. '</div><div> </div></div>';
}
print '</form></div></body> </html>'
?>
\end{listingcont}
}
47行目から50行目では表示するGPXデータを一覧できる、チェックボック
       ス、ファイル名とこのパスの全体距離の3つが一つになった\texttt{<div>}要素を、
       ファイルの数だけ表示するようにしている。
\end{frame}
\begin{frame}[containsverbatim]
\frametitle{地図を表示するページのリスト(6)---CSSファイル}
 {\tiny
\listinginput{1}{DEV/cycling-club/map.css}
 }
\end{frame}
\begin{frame}[containsverbatim]
\frametitle{地図を表示するページのリスト(6)---CSSファイル--解説}
\begin{itemize}
 \item 6行目のセレクタは、\texttt{id}が\texttt{filename}である要素の直下
       の\texttt{<div>}要素を意味する。この中にはGPXデータの情報などを示
       す3つの\texttt{<div>}要素が含まれる。
 \item 10行目のセレクタはチェックボックスを選択する。
 \item 14行目のセレクタはファイル名を示す\texttt{<div>}要素を選択する。
 \item 19行目のセレクタはGPXデータの全体の距離を示す\texttt{<div>}要素を選択する。
\end{itemize}
\end{frame}
\begin{frame}[containsverbatim]
\frametitle{地図を表示するページのリスト(7)}
 地図を表示するページで呼ばれるJavaScriptファイル

 Ajaxを用いてルートのデータをサーバーに要求し、順にルートを表示するようになっ
ている。
\end{frame}
\begin{frame}[containsverbatim]
\frametitle{地図を表示するページのリスト(8)}
 地図を表示するための準備の部分
 {\footnotesize
\begin{listing}{1}
function initialize(latMax,lonMax,latmin,lonmin) {
  var Clrs = [[255,0,0],[63,63,63],
              [0,0,255],[127,127,0],[127,0,127],[0,127,127]];
  var GMap = google.maps;
  var Routes =[];
  var map = new GMap.Map($("#map_canvas")[0],
        {center: new GMap.LatLng((latMax+latmin)/2,(lonMax+lonmin)/2),
         mapTypeId: google.maps.MapTypeId.ROADMAP,
         scaleControl:true,
         scaleControlOptions:google.maps.ControlPosition.BOTTOM_LEFT
        });
  var latLngBounds = new GMap.LatLngBounds(
        new GMap.LatLng(latmin,lonmin),
        new GMap.LatLng(latMax,lonMax));
  map.fitBounds(latLngBounds);
  }
\end{listing}
 }
\end{frame}
\begin{frame}[containsverbatim]
\frametitle{地図を表示するページのリスト(8)---解説}
\begin{itemize}
 \item 関数\texttt{initialize()}の引数は表示すべき地図の範囲の南西部の緯
       度と経度、北東部の緯度と経度
 \item 2行目から3行目では、GPSログを示すルートの色の種類を与えている。数
       が増えたときは、この要素の数で割った余りの色となる。
 \item 6行目から11行目で地図を表示している。なお、ここではjQuery関数で
       HTML文書内の要素を得ている。jQueryオブジェクトに対してGoogle Maps の地図
       を割り当てる要素として使えないので引数を与えて、HTML要素を得てい
       る。
 \item 12行目から15行目で13行目と14行目で与えられた緯度経度の地点を含む
       範囲(\texttt{LatLngBounds})を構成
 \item 15行目でこの範囲を含むような最大のズームレベルにするメソッド
       \texttt{fitBounds()}を呼び出している。
\end{itemize}
\end{frame}
\begin{frame}[containsverbatim]
 \frametitle{地図を表示するページのリスト(9)}
 {\small
\begin{listingcont}
  var InfoWindow = InfoWindow = new google.maps.InfoWindow({});
    var Colors =[],C2 =[], i, j;
  for(i=0;i<Clrs.length;i++) {
    Colors[i] = "rgb("+Clrs[i].join(",")+")";
    C2[i] = "rgb("+Math.floor((Clrs[i][0]+255*2)/3,1)+","+
                   Math.floor((Clrs[i][1]+255*2)/3,1)+","+
                   Math.floor((Clrs[i][2]+255*2)/3,1)+")";
\end{listingcont}
 }
\begin{itemize}
 \item 16行目では地図上に情報を表示するための\texttt{InfoWindow()}を作成
 \item 18行目から23行目ではGPSログのルートの一覧部分の背景色の設定をしてい
       る。パスの表示は重なった場合に備えて不透明度を$0.5$にしている(52
       行目)ので、定義の色より明るくしている。
\end{itemize} \end{frame}
\begin{frame}[containsverbatim]
\frametitle{地図を表示するページのリスト(9)--GPSログを地図上に表示する}
 ページがすぐに開かないことを避けるためにAjaxを用いてデータを得ている。

\begin{listingcont}
  var Files = $(".fn");
  for(j = 0;j<Files.length;j++){
    (function(j) {
      jQuery.ajax({
        type:"GET",
        url: "./getData.php",
        data:"file="+$($("div",Files[j])[1]).text(),
        dataType:"json",
        error: function(){alert("error");},
\end{listingcont}
\end{frame}
\begin{frame}[containsverbatim]
\frametitle{地図を表示するページのリスト(9)--解説}
\begin{itemize}
 \item 24行目でHTML内に表示されているルートを表示している要素を得る。
 \item 25行目からそのファイル名をサーバーに渡して、GPSログのデータを得
       て、地図上に表示する。
 \item 26行目では関数を定義してその場で実行することで、Ajaxの処理を行っ
       ている。これは、ループの制御変数\texttt{j}(30行目)がAjaxの通信が終了した
       段階(47行目など)で変化しているかもしれないことに対処するためである。
 \item 27行目から72行目でjQueryによるAjax関数の呼び出しを行っている。
\end{itemize}
\end{frame}
\begin{frame}[containsverbatim]
\frametitle{地図を表示するページのリスト(10)}
 通信終了後に呼び出される関数
 \begin{listingcont}
        success: function(Data){
          var ps, length, PP, P, LL, point, i,k;
          ps = [];
          length =0;
          for(k=0;k<Data.route.length;k++){
            P = Data.route[k];
            LL = P.length;
            ps.push(new GMap.LatLng(P[0][0], P[0][1]));
            for(i=1;i<LL;i+=1) {
              point = new GMap.LatLng(P[i][0], P[i][1]);
              length += getDistance(P[i-1],P[i]);
              ps.push(point);
            }
          }
 \end{listingcont}
\end{frame}
\begin{frame}[containsverbatim]
\frametitle{地図を表示するページのリスト(9)--解説}
\begin{itemize}
 \item サーバーから得たデータはすでにJavaScriptのオブジェクトとなってい
       るので、その\texttt{position}プロパティに緯度と経度の配列を要素と
       する配列が設定されている。
 \item 37行目から46行目でルートを示す\texttt{PolyLine}オブジェクトに渡す
       配列に変換している。さらに、ルートの長さも計算している。
\end{itemize}
\end{frame}
\begin{frame}[containsverbatim]
\frametitle{地図を表示するページのリスト(10)}
 表示すべき\texttt{PolyLine} を配列に格納
 {\footnotesize
\begin{listingcont}
          Routes[j] = [
            new GMap.Polyline({
              path:ps,
              strokeColor:Colors[j % Colors.length],
              strokeWeight:4,
              strokeOpacity:0.5,
              map:map
            }),true];
\end{listingcont}
}
\end{frame}
\begin{frame}[containsverbatim]
\frametitle{地図を表示するページのリスト(11)}
{\small
\begin{listingcont}
          $($("div",Files[j])[2]).text((length/1000).toFixed(2)+ "km");
          $(Files[j]).css("backgroundColor",C2[j % Colors.length]);
          $("input",Files[j]).attr("checked",true);
          var showId = (function(No){return function(E){
              InfoWindow.setOptions({
                 content: "<div>"+ No+"</div>",
                 position: E.latLng});
               InfoWindow.open(map);
             };
           })($($("div",Files[j])[1]).text());
           latLngBounds.extend(
             new GMap.LatLng(Data.position[0],Data.position[2]));
           latLngBounds.extend(
               new GMap.LatLng(Data.position[1], Data.position[3]));
           map.fitBounds(latLngBounds);
           GMap.event.addListener(Routes[j][0],"click", showId) ;
        }
      });
    })(j);
  }
\end{listingcont}
 }
\end{frame}
\begin{frame}
\frametitle{地図を表示するページのリスト(10)--解説}
\begin{itemize}
 \item 55行目では計算された距離をリスト内に表示し、背景色を定義
 \item 57行目では対応するチェックボックスをにチェックを入れる。
 \item 58行目から64行目では、ルート上をクリックしたときにそのファイル名
       を\texttt{InfoWindow}に表示する関数を定義。70行目でイベン
       トの登録を\texttt{addListener()}メソッドを用いてしている。
 \item サーバーからはルートが含まれる範囲の緯度経度が渡されているの
       でそれをもとに今までの範囲を\texttt{extend()}メソッドで拡大
\end{itemize}
\end{frame}
\begin{frame}[containsverbatim]
 \frametitle{地図を表示するページのリスト(11)}
ファイルのリストのチェックボックスの値が変化したとき
 に、ルートを表示のオンオフをする処理
 \begin{listingcont}
  document.getElementById("filenames").addEventListener("change",
    function (E) {
      var i =  $(E.target).attr("id").substring(4)-0;
      if(E.target.checked && !Routes[i][1]) {
        Routes[i][0].setMap(map);
        Routes[i][1] = true;
      }
      if(!E.target.checked && Routes[i][1]) {
        Routes[i][0].setMap(null);
        Routes[i][1] = false;
      }
    },false);
}
 \end{listingcont}
\end{frame}
\begin{frame}[containsverbatim]
\frametitle{地図を表示するページのリスト(12)}
 緯度と経度で与えられた2点間の距離を求める関数
 {\footnotesize
 \begin{listingcont}
function getDistance(P1, P2) {
    var R = 6378137;
    var P1LatRad = P1[0]*Math.PI/180;
    var P1LngRad = P1[1]*Math.PI/180;
    var P2LatRad = P2[0]*Math.PI/180;
    var P2LngRad = P2[1]*Math.PI/180;
    var P1LatCos = Math.cos(P1LatRad);
    var P2LatCos = Math.cos(P2LatRad);
    var Xdiff = R*(P1LatCos*Math.cos(P1LngRad) - P2LatCos*Math.cos(P2LngRad));
    var Ydiff = R*(P1LatCos*Math.sin(P1LngRad) - P2LatCos*Math.sin(P2LngRad));
    var Zdiff = R*(Math.sin(P1LatRad) - Math.sin(P2LatRad));
    return Math.sqrt(Xdiff*Xdiff+Ydiff*Ydiff+Zdiff*Zdiff);
}
 \end{listingcont}
 }
\end{frame}
\begin{frame}[containsverbatim]
\frametitle{地図を表示するページのリスト(13)}
GPSデータの各地点の緯度と経 度のリストをJSON形式で返す
 {\footnotesize
\begin{listing}{1}
<?php
function SetDatafromFile($fin,$folder) {
  $fn = "$folder$fin";
  if(!file_exists($fn)) return;

  $data = new DOMDocument();
  $data->load($fn);
  $trks = $data->getElementsByTagName("trk");
\end{listing}
 }
\end{frame}
\begin{frame}[containsverbatim]
\frametitle{地図を表示するページのリスト(13)--解説}
 \begin{itemize}
  \item GPXファイルはXML形式のファイルなので、構造を保持したままデータを
	扱うためにはPHPの\texttt{DOMDocument}オブジェクトに読み込むのが
	簡単である。
  \item 6行目で\texttt{DOMDocument}オブジェクトを作成し、7行目でファイル
	を読み込んでいる。
  \item 残りの捜査はDOMの扱いと同様に\texttt{getElementsByTagName()}など
	を用いてデータを処理すればよい。
  \item GPXには複数の\texttt{<trk>}要素があり、その中に位置情報を示す
	\texttt{trkpt}がある。
 \end{itemize}
\end{frame}
\begin{frame}[containsverbatim]
 \frametitle{地図を表示するページのリスト(14)}
 {\small
\begin{listingcont}
  $len = $trks->length;
  $trackdata = array();
  $lonMax = -181;
  $lonmin =  181;
  $latMax =  -91;
  $latmin =   91;
\end{listingcont}
 }
\end{frame}
\begin{frame}[containsverbatim]
 \frametitle{地図を表示するページのリスト(15)}
 {\tiny
\begin{listingcont}
  $cnt=0;
  for($i=0;$i<$len;$i++){
    $trk = $trks->item($i);
    $trksegs = $trk->getElementsByTagName("trkpt");
    $len2 = $trksegs->length;
    if($len2 <10) continue;
    $newtrk = array();
    for($j = 0;$j<$len2;$j++) {
      $trkseg = $trksegs->item($j);
      $lat = $trkseg->getAttribute("lat");
      $lon = $trkseg->getAttribute("lon");
      if($lat > $latMax) $latMax = $lat;
      if($lat < $latmin) $latmin = $lat;
      if($lon > $lonMax) $lonMax = $lon;
      if($lon < $lonmin) $lonmin = $lon;
      array_push($newtrk,"[$lat,$lon]");
    }
    if($lonMax - $lonmin >0.001 ||$latMax - $latmin > 0.001) {
      $cnt++;
      array_push($trackdata,"[".implode(",",$newtrk)."]");
    }
  }
  print "{\"position\":[$latMax,$latmin,$lonMax,$lonmin],\"route\":[".
      implode(",",$trackdata)."]}";
}
SetDatafromFile($_GET["file"],"");
?>
}
\end{listingcont}
 }
\end{frame}
\begin{frame}[containsverbatim]
\frametitle{地図を表示するページのリスト(14)--解説}
\begin{itemize}
  \item \texttt{trkpt}要素の属性に緯度(\texttt{lat})と経度(\texttt{lng})
	があるので、それを組み合わせたJSON形式の配列(の文字列)を配列に
	追加する(30行目)。
  \item 途中でルート全体を含む緯度経度の範囲を求めている(26行目から29行
	目)。
  \item これらのデータをまとめて出力する(37目から38行目)
 \end{itemize}
\end{frame}
\section{来週の試験}
\begin{frame}[containsverbatim]
 \frametitle{試験の内容}
 \begin{itemize}
  \item 授業で行った簡単なプログラムやCSSの設定に関するもの
  \item プログラムを組む問題はなし
  \item いくつかの項目について論述する。
\begin{itemize}
 \item JavaScriptの言語の特性に関する考察
 \item jQueryのライブラリーを使うメリットとデメリット
 \item Google Maps のライブラリの設計思想に関する意見
 \item そのほか、この授業を通じて理解したことについての論述(各自が設定し
       てよい)
\end{itemize}
 \end{itemize}
 \end{frame}

\end{document}
