\documentclass[a4j]{jarticle}
\usepackage{array}
\title{ソフトウェア開発レポートレポート課題}
\author{平野 照比古}
\date{2017/1/6}
\newtheorem{Prob}{}
\begin{document}
\maketitle
次のレポート問題を1月18日(水)17までに6階のレポート室の平野のポストまで提
出すること。別紙のルーブリック評価表を参考にレポートを作成すること。

問1は必須、問2と問3は選択でどちらか一方を解答すること。

ルーブリック評価表は最終授業の最後に返却する。
\begin{Prob}[必須]\upshape
 \input prob13.tex
\end{Prob}\newpage
\begin{Prob}\upshape
 適当なGPXファイルを利用しGPXファイルの道のりをブラウザ上に表示すること
 に関して次のことを報告しなさい。
 \begin{enumerate}
	\item 今回のコードでGPXの道のりが表示できることを確認しなさい。
	\item URLの後ろに\texttt{?w=800\&h=800}を付けて実行すると地図の大きさ
				が変化することを確認しなさい。
	\item このページを再読み込みしないでさらにGPXファイルを表示させようと
				すると、前の道のりの情報が消えないことを確認しなさい。
	\item 前項の不具合を修正しなさい。
	\item jQueryを用いてこのJavaScriptを書き直しなさい。
 \end{enumerate}
\end{Prob}
\begin{Prob}\upshape
 XMLファイルの処理をするときの方法としてブラウザでJavaScriptを用いるとき
 とPHPにおける\texttt{DOMDocument}を用いる方法で使用できるメソッドの比較
 を行いなさい。\texttt{http://www.php.net/manual/ja/}から情報を探すとよ
 い。
\end{Prob}
\end{document}