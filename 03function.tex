%-*- coding: utf-8 -*-
\chapter{関数}
\ifText
JavaScript における関数はいろいろな目的のために利用される。
ここでは次の事柄について説明する。
\else
\section{今回の内容}
JavaScript では関数はいろいろな目的のために利用される。
今回は次のことを取り扱う。
\fi
\begin{itemize}
 \item 関数の定義方法と使用法
 \item 関数への引数の渡し方
 \item 関数の戻り値(数や配列のほかに関数も戻り値にできる)
 \item JavaScriptにおける変数のスコープ
% \item 関数も単なるオブジェクト
\end{itemize}
さらに次のような JavaScript 特有の事柄についても説明する。
\begin{itemize}
 \item 無名関数とコールバック関数
 \item 自己実行関数
 \item 関数内で定義された関数
% \item 関数の戻り値が関数となる関数
 \item クロージャ
\end{itemize}
%関数はオブジェクトを作成するときのも用いられるが、その解説は次回で行う。
\section{関数の定義方法と呼び出し}
\subsection{簡単な関数の例}
次の例は関数 \Verb+sum()+を定義している。
\begin{Verbatim}
function sum(a,b) {
  let c = a + b;
  return c;  // return a + b; でもよい。
}
\end{Verbatim}
関数の定義は次の部分から成り立っている。
\begin{itemize}
 \item {\bfseries \ElmJ{function}キーワード} %\\
戻り値の型を記す必要はない。
 \item {\bfseries 関数の名前} %\\
\ElmJ{function} の後にある識別名が関数の名前になる。ここでは \texttt{sum}
       が関数の名前である。
 \item {\bfseries 引数のリスト} %\\
関数名の後に\Verb+()+内にカンマで区切られた引数を記述する。この場合は変数
       \Verb+a+と\Verb+b+が与えられている。引数はなくてもよい。
 \item {\bfseries 関数の本体であるコードブロック} %\\
\Verb+{}+で囲まれた部分に関数の内容を記述する。
\item {\bfseries \Verb+return+ キーワード} \\
関数の戻り値をこの後に記述する。戻り値を明示的に示さない場合は
       \Verb+undefined+が返される。
\end{itemize}
%\paragraph{実行例}
次にこの関数の実行例を掲げる。
\begin{Verbatim}
>sum(1,2)  // 戻り値は 3
>sum(1)    // 戻り値は NaN
>sum(1,2,3)// 戻り値は 3
\end{Verbatim} 
\begin{itemize}
 \item 引数に \Verb+1+ と \Verb+2+ を与えられているので期待通りの結果が得られる。
 \item 引数に \Verb+1+ だけを与えた場合、エラーが起こらず、\Verb+NaN+ と
       なる。これは、不足している引数 \Verb+b+ には
       \Verb+undefined+ が渡され、戻り値は \Verb-1+undefined- の計算
       結果(\Verb+NaN+)となる。
 \item 引数を多く渡してもエラーが発生しない。無視されるだけである。
\end{itemize}
これらのことから JavaScript の関数はオブジェクト指向で使われるポリモーフィ
ズムをサポートしていない。さらに、次の問題で見るように同じ関数を定義しても
エラーにならない。後の関数の定義が優先される。

\input 03-01OrderOfDefFunction.tex
\subsection{仮引数への代入}
\renewcommand{\theFancyVerbLine}{\normalsize\arabic{FancyVerbLine}}
仮引数に値を代入してもエラーとはならない。仮引数で渡された値がプリミティブなとき
とそうでないときとでは呼び出し元における変数の値が異なる。
\begin{Exec}\label{Params}\upshape
 次の例は呼び出した関数の中で仮引数の値を変化させたときの例である。
\end{Exec}
\begin{Verbatim}
function func1(a){
  a = a*2;
  return 0;
}
\end{Verbatim}
\Verb+func1()+では仮引数\Verb+a+の値を2倍している。次のように実行する。
\begin{Verbatim}
>a = 4;
>func1(a); // 戻り値は 0
>a;
4          //a の値は変化しない
\end{Verbatim}
       この結果から呼び出し元の変数の値には変化がないことがわかる。つ
       まり、プリミティブデータ型の引数は値そのものが渡され
       る(値渡し)。

次の例は配列を引数に渡す関数の例である。
\begin{Verbatim}
function func2(a){
  a[0] *=2;
  return 0;
}
\end{Verbatim}
\Verb+func2()+の仮引数は配列が想定してあり、その配列の先頭の値だ
       け2倍する関数である。
\begin{Verbatim}
>a = [1,2,3];
>func2(a);// 戻り値は 0
>a;
[2, 2, 3] //配列の先頭の値が2倍されている
\end{Verbatim}
この関数の実行後、配列の先頭の値が変化している。
この結果、プライミティブ型以外の引数の渡し方が参照渡しであることがわかる。
\iffalse
\subsection{\protect\texttt{aruguments}について}
JavaScriptでは引数リストで引数の値などが渡されるほかに\texttt{arguments}
という配列のようなオブジェクトでもアクセスできる。
このオブジェクトは非\Strict と\Strict では扱いが異なる部分がある。
\begin{itemize}
 \item 引き渡された変数の数は\Verb+length+で知ることができる。
\begin{Verbatim}
function sumN(){
  var i, s = 0;
  for(i = 0; i <arguments.length;i++) {
    s += arguments[i];
  }
  return s;
}
\end{Verbatim}
実行例は次のとおりである。
\begin{Verbatim}
>sumN(1,2,3,4);
10
>sumN(1,2,3,4,5);
15
\end{Verbatim}
 \item 引数があっても無視できる
\begin{Verbatim}
function sumN2(a,b,c){
  var i, s = 0;
  for(i = 0; i <arguments.length;i++) {
    s += arguments[i];
  }
  return s;
}
\end{Verbatim}
この例では引数が3個より少なくても正しく動く。実行例は次のとおりである。
\begin{Verbatim}
>sumN2(1,2,3,4,5);
15

\end{Verbatim}
 \item 非\Strict では仮引数と\Verb+arguments+は対応していて、片方を変更しても他の方も
       変更される。
実行例は次のとおりである。
\begin{Verbatim}
function sum2(a, b){
  var c;
  a *= 3;
  console.log(arguments[0]);
  return a + b;
}
\end{Verbatim}
\begin{Verbatim}
>sum2(1,2,3,4,5);
3 
5
\end{Verbatim}
一方、\Strict では\Verb+argument+は仮引数の静的なコピーが保持されるので、
       \Verb+a *=3+により\Verb+arguments[0]+の値は変更されない。
\begin{Verbatim}
>sum2(1,2,3,4,5);
1
5
\end{Verbatim}
\end{itemize}
\else
\subsection{可変引数をとる関数}
前節の引数の和を計算する関数で、引数の数を固定しないものを定義
するには仮引数に展開演算子をつける\footnote{これまでのJavaScript で
は、関数の引数を表す配列のような性質を持つ\texttt{arguments}オブジェクト
が用意されていたが、\ES 以降非推奨となった。}。
\begin{Exec}\upshape\label{DefOfFunctionwithIndefiniteArges}
次で定義された関数は任意個数の引数を取る関数の例である。
\begin{Verbatim}[numbers=left]
function sumN(...args) {
    let S = 0;
    for(let i=0;i<args.length;i++) {
        S += args[i];
    }
    return S;
}
\end{Verbatim}
\begin{itemize}
 \item 1行目で関数の仮引数\texttt{args}に展開演算子\texttt{...}をつけて
       いる。
 \item \texttt{args}は配列となるのでその大きさは\texttt{args.length}でわ
       かる。
 \item これを用いて総和のプログラムが2行目から5行目に記述されている。
\end{itemize}
実行例は次のとおりである。
\begin{Verbatim}
>sumN(1,2,3,4);  //戻り値は 10
>sumN(1,2,3,4,5);//戻り値は 15
\end{Verbatim}
\end{Exec}
\input 03-02FunctionWithArbituraryArgs.tex
\fi
\subsection{変数のスコープ}
変数のスコープとはある場所で使われている変数がどこで定義されているものかという
概念である。JavaScriptでは次の特徴がある。
\begin{enumerate}
 \item 非\Strict では変数は宣言しなくても使用できる。\label{3-4NoDeclare}
 \item \label{3-4DeclareInFunc}
       関数内で \ElmJ{let} により明示的に宣言された変数はブロック内で有
       効となる。ブロックとは\Verb+{+と\Verb+}+で囲まれた部分である。ただし、宣言
       よりも前では使用できない。同じブロック
       内で \ElmJ{let} による同一変数名の宣言はエラーになる。
 \item 変数を\ElmJ{var}で宣言すると関数の途中で宣言しても、関数の先頭で
       宣言したと同じ効果を持つ。これを変数の巻き上げという。関数の外で
       \ElmJ{var} で宣言された変数は、プログラムの先頭で宣言したものとみ
       なされる。ただし、初期化は宣言された位置で行われる。同じスコープ
       内で \ElmJ{var} による同一変数名の宣言はエラーにならない。
 \item 関数の外で宣言された変数や宣言されないで使用された変数はすべてど
       こからからでも参照できるグローバルとなる。\label{3-4Declare2}
\end{enumerate}
%\subsection{スコープチェイン}
関数やブロックの中で関数を定義したりブロックを作成すると、その
内側の関数やブロック内で\Verb+let+で宣言された変数のほかに、一つ上の関数
やブロックで利用できる(スコープにある)変数が利用できる。これがスコー
プチェインである。
例を挙げる。
\begin{Verbatim}
let G1, G2;
function func1(a) {
  let b, c;
  function func2() [
    let G2, c;
   ...
  }
}
\end{Verbatim}
\begin{itemize}
 \item 関数\Verb+func1()+内ではグローバル変数\Verb+G1+と\Verb+G2+、
       仮引数\Verb+a+ とローカル変数\Verb+b+と\Verb+c+が利用できる。
 \item 関数\Verb+func2()+内ではグローバル変数\Verb+G1+、
       \Verb+func1()+の仮引数\Verb+a+ と\Verb+func1()+のローカル変数
       \Verb+b+、\Verb+func2()+のローカル変数\Verb+G2+と\Verb+c+が利用で
       きる。
 \item \Verb+func2()+では同名の\texttt{G2}と
       \Verb+c+が宣言されているため、グローバル変数\texttt{G2}と
       \Verb+func1()+で定義された\Verb+c+は参照できない。
\end{itemize}
このように内側で定義された関数は自分自身の中で定義されたローカル変数があ
るかを探し、見つからない場合には一つ上のレベルでの変数を探す。これがスコー
プチェインである。JavaScriptの関数のスコープは関数が定義されたときのスコー
プチェインが適用される(次の実行例\ref{ExecScope}の14行目参照)。
これをレキシカルスコープと呼ぶ。レキシカルスコー
プは静的スコープとも呼ばれる。これに対して実行時にスコープが決まるものは
動的スコープと呼ばれる。
\begin{Exec}\label{ExecScope}\upshape
変数のスコープを次の例で確かめる。
\begin{Verbatim}[numbers=left]
let S = "global";
function func1(){
    console.log(S);
    return 0;
}
function func2(){
    console.log(S);
    let S = "local";
    console.log(S);
    return 0;
}
function func3(){
    let S = "local";
    func1();
    return 0;
}
function func4(){
    let S = "local in func4";
    func5 = function(){
        console.log(S);
        return 0;
    };
    console.log(S);
    return 0;
}
\end{Verbatim}
\end{Exec}
このリストは次のようになっている。
\begin{itemize}
 \item \Verb+func1()+から\Verb+func4()+まで4つの関数が定義されている。
 \item 1行目ではグローバル変数 \Verb+S+ が宣言されて、文字列\Verb+"global"+
       で初期化されている。
 \item 2行目から5行目で\Verb+func1()+が定義されている。
\begin{itemize}
 \item 3行目で\Verb+S+の値をコンソールに出力している。
 \item この関数内で変数\Verb+S+
       の宣言がないので1行目のグローバル変数\texttt{S}が参照される。
\end{itemize}       
\begin{Verbatim}
>func1();
global
0
\end{Verbatim}
 \item 6行目から11行目で\Verb+func2()+が定義されている。
\begin{itemize}
 \item 7行目と9行目で\Verb+S+の値をコンソールに2回出力している。
 \item この関数内で変数\Verb+S+ は8行目で
       \Verb+let+ を用いてローカル変数として宣言されている。
  \item \Verb+let+で宣言された変数はブロックで宣言される前に使用
       できないので実行時にエラーが発生する。
\begin{Verbatim}
>func2();
VM362:2 Uncaught ReferenceError: S is not defined
    at func2 (<anonymous>:2:17)
    at <anonymous>:1:1
\end{Verbatim}
\end{itemize}
 \item 12行目から16行目で\Verb+func3()+が定義されている。
\begin{itemize}
 \item 13行目でローカル変数\Verb+S+を定義して、初期値を
       \Verb+"local"+としている。
 \item 14行目で初めに定義した関数\Verb+func1()+を呼び出している。
 \item \Verb+func1()+の実行時は、この関数が定義された時の変数
       \Verb+S+(1行目)が参照される。
\end{itemize}
\begin{Verbatim}
>func3();
global
0
\end{Verbatim}
 \item 17行目から25行目で\Verb+func4()+が定義されている。
\begin{itemize}
 \item 18行目でローカル変数\Verb+S+の値を設定している。
 \item 23行目の出力は18行目での値となる。
\begin{Verbatim}
>func4();
local in func4
0
\end{Verbatim}
 \item 19行目では関数オブジェクトを変数\Verb+func5+に代入している。これ
       により\Verb+func5()+という関数が定義される。
 \item \Verb+let+による変数の宣言がないので変数\Verb+func5+はグローバル変数となる
       (\pageref{3-4Declare2}ページの変数のスコープの\ref{3-4Declare2}参照)。
 \item \Verb+func5()+内では変数\Verb+S+の内容を出力が定義されている。
 \item \Verb+func4()+を実行した後では\Verb+func5()+が実行できる。
 \item 関数\Verb+func5()+が定義された段階では変数\Verb+S+が、この関数を
       定義している\Verb+func4()+の中にあるので、18行目の変数\Verb+S+が参
       照される。%この状況についてはスコープチェーンの項で説明する。
\begin{Verbatim}
func5();
local in func4
0
\end{Verbatim}
\end{itemize}
\end{itemize}
\input prob03-03declairVAriablesByVar.tex
\section{JavaScriptにおける関数の特徴}
%JavaScript関数ではほかの言語では見られない関数の取り扱い方法がある。
\subsection{関数もデータ}
JavaScriptでは関数もデータ型のひとつなので次のようなことが可能となってい
る。
\begin{itemize}
 \item 関数の定義を変数に代入したり(実行例\ref{ExecScope}の19行目以降参
       照)関数の引数として渡す。
 \item 代入はいつでもできるので、実行時に関数の定義を変える。
 \item 関数の戻り値として関数自体を返す。
\end{itemize}

%\subsection{関数を返す関数}
\iffalse
この使用法の例としてはある処理がブラウザごとに異なる作業を必要とするとき
に、それに適応した判定を1回だけしてあとは戻した関数を実行するだけにする
ことができる。%他の利用法については後で例をあげる。
\fi
\subsection{コールバック関数}
実行例\ref{ExecScope}の19行目以降の関数オブジェクトでは
\Verb+function+の後には関数名がない。このよ
うな関数は無名関数と呼ばれる。HTML文書などでは、イベント(マウ
スがクリックされた、一定の時間が経過した)が発生したときに、その処理を行
う関数を登録する必要がある。関数に引数として渡される関数
をコールバック関数という。
\begin{Exec}\upshape
 次の例は、一定の経過時間後にある関数を呼び出す\Verb+window+オブジェクト
 の\Verb+setTimeout()+メソッドの使用例である。
\begin{Verbatim}[numbers=left]
let T = new Date();
window.setTimeout(
  function callMe(){
    let NT = new Date();
    if(NT.getTime()-T.getTime()<10000) {
      console.log(Math.floor((NT.getTime()-T.getTime())/1000));
      window.setTimeout(callMe,1000);
    }
   },1000);
\end{Verbatim}
\begin{itemize}
 \item 1行目では実行開始時の時間を変数\Verb+T+に格納している。
 \item このメソッドは一定時間経過後に呼び出される関数と、実行される経過
       時間(単位はミリ秒)を引数に取る。
 \item 実行する関数は3行目から9行目で定義されている。
 \item この関数内で一定時間経過後にこの関数を呼び出す。この関数
       に名前は\Verb+callMe+である(3行目)\footnote{\Verb+function+オブ
       ジェクトの引数を表す配列のようなオブジェクト\Verb+arguments+の
       プロパティ\texttt{callee}を使うのが従来の方法である。
       \ES では\Verb+arguments+が非推奨となったため、コールバック関数に
       名前を付けている。}。
 \item 4行目で呼び出されたときの時間を求め、経過時間が$10000$ミリ秒以下
       であれば(5行目)、経過時間を秒単位で表示する(6行目)。
 \item さらに、自分自身を1秒後に呼び出す(7行目)。

\iffalse			 非\Strict では
       \Verb+arguments+を仮引数としてもつ関数を\Verb+arguments.callee+で
       呼び出すことができる。つまり自分自身を呼び出せることとなる。
       \Strict では\Verb+arguments.callee+は使用できないので3行目の関数
       に名前を付けてその関数名を\Verb+arguments.callee+の代わりに用いる
       必要がある。
			 \fi
\end{itemize}
\end{Exec}
%なお、関数の宣言を \Verb+function foo(){...}+ でする代わりに、
%{\Verb+let foo = function(){...}+} と無名関数で定義してもよい。

\Verb+setTimeout()+メソッドのコールバック関数の引数に値を渡したいときは、
\Verb+setTimeout()+メソッドの3番目以降の引数に記述する。
\input 03-04setTimeout.tex
%\input prob03-03factorial.tex
  \subsection{自己実行関数}
関数を定義してその場で直ちに実行することができる。次のコードは課題
\ref{FisatJS}の関数の中身である。
\begin{Verbatim}
  let i;
  for(i=1;i<10;i++) {
    console.log(`${i} ${i*i}`);
  }
\end{Verbatim}
このプログラムを実行すると$1$から$9$までの値とそれの2乗の値がコンソール
に出力される。実行後に、コンソールに\Verb+i+と入力すれば\Verb+10+が出
力される。

一方、課題\ref{FisatJS}では関数が定義されただけで何も出力されないので、
\Verb+foo()+と入力して関数を実行する必要がある。この場合、変数
\Verb+i+は関数内のローカル変数なので関数実行後、変数\Verb+i+の値は参照で
きない(\Verb+undefined+が出力される)。

前者の場合は変数\texttt{i}、後者の場合は関数名\texttt{foo}がグローバル変
数が残っている。JavaScriptでは定義した関数をその場で実行する方法があり、
これを利用すると不要なグローバル変数を残さないことができる。

次の例はこれを実現している。
\begin{Verbatim}
(function(){
  let i;
  for(i=1;i<10;i++) {
    console.log(`${i} ${i*i}`);
  })();
\end{Verbatim}
関数の定義を全体で\Verb+()+で囲み、そのあとに関数の呼び出しを示
すための\Verb+()+を付ける。

この技法は、初期化の段階で1回しか実行しない事柄を記述し、かつグローバル
な空間に余計な変数などを残さない手段としてJavaScriptのライブラリーで用いられる。

\iffalse
なお、\ElmJ{let}で変数\Verb+i+を\ElmJ{for}内で宣言すると、
その変数は\ElmJ{for}文内でしか存在しない。また、スコープの規則が少し異な
る。実行例\ref{declareVariableInFor}を参照のこと。
\fi
\iffalse
\begin{Verbatim}
  for(let i=1;i<10;i++) {
    console.log(`${i} ${i*i}`);
  }
\end{Verbatim}
\fi
\section{クロージャ}
関数内部で宣言された変数は、その外側から参照することができない。
つまり、その関数は関数内のローカル変数を閉じ込めている。しかし、関数内部
で定義された関数を外部に持ち出す(グローバルな関数にする)と、持ち出された
関数のスコープチェイン内に定義された親の関数のスコープを引き継ぐので、親
の関数のローカル変数の参照が可能となる。

関数とその
依存する環境(変数や呼び出せる関数などのリスト)を合わせたものをその関数の
クロージャと呼ぶ。

\subsection{クロージャの例 -- 変数を隠す}
ここで上げる例は実用に乏しいと思われるかもしれないが、この後に出てくる
オブジェクトの項の例でより実用的なものと理解できるであろう。
%\newpage
\begin{Exec}\label{Closure1}\upshape
 次の関数を考える。
\begin{Verbatim}[numbers=left]
function f1() {
  let n=0;
  return function() {
    return n++;
  };
}
\end{Verbatim}
 \begin{itemize}\upshape
	\item ローカル変数\texttt{n}を定義し、初期値を\texttt{0}としている(2行
				目)。
	\item 戻り値はローカル変数\texttt{n}の値を返す関数である(3行目から5行
				目)。
	\item この戻り値の関数はローカル変数の値を\texttt{1}増加させている(5行
				目)。
 \end{itemize}
 これを次のように実行する。
 \begin{itemize}\upshape
	\item  変数\texttt{ff}に関数\texttt{f1()}で返される関数オブジェクトを代入する。
	これにより変数\Verb+ff+は関数オブジェクトになる。
\begin{Verbatim}
>ff= f1();  //関数の定義が表示される
\end{Verbatim}
	\item \texttt{f1()}内のローカル変数は参照できない。
\begin{Verbatim}
>n;
VM351:2 Uncaught ReferenceError: n is not defined(…)
\end{Verbatim}
% 関数内で定義された変数\texttt{n}は参照できないことがわかる。
	\item 戻り値の関数は変数\texttt{ff}を用いて参照できる。
\begin{Verbatim}
>ff(); // 戻り値は 0
>ff(); // 戻り値は 1
>ff(); // 戻り値は 2
\end{Verbatim}
何回か実行すると戻り値が順に増加している。つまり、ローカルに
は変数\texttt{n}が存在している。
	\item もう一度関数\texttt{f1()}を実行すると新しい関数が得られる。
\begin{Verbatim}
>ff2=f1();  // 関数の定義が表示
>ff();  // 戻り値は 3
>ff2(); // 戻り値は 0
\end{Verbatim}
 \end{itemize}
\end{Exec}
  \subsection{クロージャの例 -- 無名関数をその場で実行する}
 実行例\ref{Closure1}で定義された関数\texttt{f1}を一度だけ実行して、それがこ
				れ以上実行されないようにするためにはこの関数を無名関数としてそ
				の場で実行すればよい。
\begin{Exec}\upshape
  次のリストは、使い捨ての関数の例である。このような方法を用いるとオブジェ
    クトに共通の変数を持つことができる。
\begin{Verbatim}
let foo = (function () {
  let n=0;
  return function() {
    return n++;
  };
})();
\end{Verbatim}
\begin{itemize}
 \item 無名関数を定義した部分を\texttt{()}でくくり、引数リストをその後の
			 \texttt{()}に記述する。ここでは、引数がないので中は空である。
 	\item 戻された関数オブジェクトを変数\texttt{foo}に代入する。
	\item 前と同じように実行できる。
\begin{Verbatim}
>foo();  // 戻り値は 0
>foo();  // 戻り値は 1
>foo();  // 戻り値は 2
\end{Verbatim}
\end{itemize}
\end{Exec}
%\newpage
\begin{Exec}\upshape\label{functionsReturnLocalValue}
次の例は関数内のローカル変数の値を単純に返すと、不都合が起こる例である\footnote{こ
				の例は Stoyan Stefanov(水野貴明、渋川よしき訳)オブジェクト指向
				JavaScript、アスキーメディアワークス、2012年の103ページ以降から
 取りました。}。
\begin{Verbatim}[numbers=left]
function f2() {
  let a = [];
  let i;
  for(i=0; i<3; i++) {
    a[i] = function() {
      return i;
    };
  }
  return a;
}
\end{Verbatim}
 \begin{itemize}\upshape
	\item 変数\texttt{a}を空の配列で初期化する(2行目)。
	\item この配列に配列の添え字の値を返す関数を定義する
				(4行目から8行目)
	\item 配列全体を戻り値として返す(9行目)。
 \end{itemize}
 これを実行すると次のようになる\footnote{ここでの出力例はChromeによるもの
 である。ブラウザによっては出力が異なる。}。
\begin{Verbatim}
>funcs=f2()
[function a.(anonymous function)(), function a.(anonymous function)(),
		function a.(anonymous function)()]
>funcs[1]() // 戻り値は 3
\end{Verbatim}
 これは、\texttt{f2()}を実行した後では、変数\texttt{i}の値が\texttt{3}と
 なっており、それぞれの関数が実行されるときにはこの値が参照されるためで
 ある。
\end{Exec}
\begin{Exec}\upshape\label{declareVariableInFor}
この不具合は、\ElmJ{for}の制御変数\Verb+i+を\Verb+for+文の初
 期化のところで宣言すれば発生しない\footnote{\ES では\texttt{for}文の初
 期化内で\texttt{let}で宣言された変数のスコープ規則が別に定められている。}。
\begin{Verbatim}[numbers=left]
 function f2() {
  let a = [];
  for(let i=0; i<3; i++) {
    a[i] = function() {
      return i;
    };
  }
  return a;
}
\end{Verbatim}
これを実行すると、期待した結果が得られる。
\begin{Verbatim}
>funcs2 = f3();
[function anonymous(), function anonymous(), function anonymous()]
>funcs2[0]();  // 戻り値は 0
>funcs2[1]();  // 戻り値は 1
\end{Verbatim}

この形で\Verb+let+による変数の宣言を\Verb+var+に変えると変数の巻き上げに
 より正しく動作しない。
%
正しく動作させるためには値を関数の引数に(値渡しで)渡すことで、スコー
 プチェーンを切ればよい。
 \begin{Verbatim}[numbers=left]
function f3() {
  let a = [], i;
  for(i=0; i<3; i++) {
    a[i] = (function(x){
      return function() {
        return x;
      }
    })(i);
  };
  return a;
}
 \end{Verbatim}
\begin{itemize}
 \item 5行目から9行目で、その場で実行される、引数を取る無名関数を用意している。
 \item この戻り値は、5行目の無名関数の引数の値を返す無名
       関数(6行目から8行目)である。
 \item 仮引数には、実行されたときの\Verb+i+のコピーが渡されるので、その
       後、もとの変数の値が変わっても呼び出されたときの値が仮引数の
       \texttt{x}に保持される。
\end{itemize}
 \ElmJ{let}を利用することで記述が分かりやすくなっていることが分かる。
\end{Exec}
\begin{Prob}\upshape
 実行例\ref{functionsReturnLocalValue}において、\ElmJ{for}文の代わりに
 \ElmJ{while}文に書き換えたときに正しく動作するようにせよ。ただし、直前
 のリストのように即時実行関数を用いないで実現すること。
\end{Prob}
