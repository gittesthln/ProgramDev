%-*- coding: utf-8 -*-
%
% enshuuhead.tex 2004 版
% 
%
\usepackage{array}
%\usepackage{times}
%\usepackage{timesnewp}
%\usepackage{mathabx}
\usepackage{amsmath}
\usepackage{amssymb}
\usepackage{graphicx}
\usepackage{fancybox}
\usepackage{epic}
\usepackage{eepic}
\usepackage{multibox}
\usepackage{moreverb}
%\usepackage[all,2cell]{xy}
\usepackage{rotating}
\rotdriver{dvips}
%
% the following macros are taken from Eijkhoout'book
%	TeX by Topic Chap. 13.8.6 (p.137)
%   some modification made by T. Hilano
%
\def\ifEq#1#2{\ifnum#1=#2\relax\taketrue \else \takefalse \fi}
\def\ifGt#1#2{\ifnum#1>#2\relax\taketrue \else \takefalse \fi}
\def\ifLt#1#2{\ifnum#1<#2\relax\taketrue \else \takefalse \fi}
\def\takefalse\fi#1#2{\fi#2}
\def\taketrue\else\takefalse\fi#1#2{\fi#1}

\def\ifEqString#1#2{\def\stringone{#1}\def\stringtwo{#2}\ifx\stringone\stringtwo}
%
% End of Citation
%
\newcommand{\cosec}{\mathop{\nsh\text{cosec}}\nolimits}
\newcommand{\goup}[1]{\raise1.0ex\hbox{#1}}

\newcommand{\FRAC}[2]{\displaystyle\frac{{#1}}{#2}}
\newcommand{\Expo}[1]{\raisebox{0.7ex}{{$\scriptstyle{#1}$}}}
\newcommand{\SUM}{\displaystyle\sum}
\newcommand{\PROD}{\displaystyle\prod}
\newcommand{\INT}{\displaystyle\int}

\newcommand{\DispInt}[4]{\makebox[#1]{%
	$\displaystyle#2_{%
		\mbox{\scriptsize$\begin{array}{c}#4\end{array}$}}%
	$}\hspace{#3}}
\newcommand{\DispInts}[4]{\makebox[#1]{%
	$\smash{\displaystyle#2_{%
		\mbox{\scriptsize$\begin{array}{c}#4\end{array}$}}}\vphantom{\INT}%
	$}\hspace{#3}}

\newcommand{\dint}{\mathop{\int\!\!\int}}
\newcommand{\DINT}[1]{\DispInt{2em}{\dint}{0.5em}{#1}}
\newcommand{\DINTs}[1]{\DispInts{2em}{\dint}{0.5em}{#1}}

\newcommand{\triint}{\mathop{\int\!\!\int\!\!\int}}
\newcommand{\TRIINT}[1]{\DispInt{2.2em}{\triint}{0.5em}{#1}}
\newcommand{\TRIINTs}[1]{\DispInts{2.2em}{\triint}{0.5em}{#1}}

\newcommand{\nint}{\mathop{\int\!\!\int\cdots\int}}
\newcommand{\nINT}[1]{\DispInt{2.6em}{\nint}{0.9em}{#1}}
\newcommand{\nINTs}[1]{\DispInts{2.6em}{\nint}{0.9em}{#1}}

\newcommand{\LIMto}[2]{\displaystyle{\lim_{#1\rightarrow#2}}}

\newcommand{\Pard}[1]{\partial#1}
\newcommand{\ParDD}[4]{\FRAC{\partial^{#3}#1}{\partial{#2}^{#4}}}
\newcommand{\ParDn}[3]{\ParDD{#1}{#2}{#3}{#3}}
\newcommand{\ParD}[2]{\ParDD{#1}{#2}{}{}}

\newcommand{\Del}{{\mit \Delta}}
\newcommand{\GAMMA}{{\mit \Gamma}}

\newcommand{\Arc}[1]{\stackrel{\frown}{\text{#1}}}
\newcommand{\VEC}[1]{\overrightarrow{\mathrm{#1}}}
\renewcommand{\Vec}[1]{\mbox{\boldmath $#1$}}

\newcommand{\Matrix}[2]{\left(\SetComponent{#1}{#2}{c}\right) }
\newcommand{\MatrixR}[2]{\left(\SetComponent{#1}{#2}{r}\right) }
\newcommand{\Det}[2]{\left| \SetComponent{#1}{#2}{c}\right| }
\newcommand{\DetR}[2]{\left| \SetComponent{#1}{#2}{r}\right| }
\newcommand{\Vector}[1]{\Matrix{1}{#1}}
\newcommand{\VectorR}[1]{\Matrix{1}{#1}}
\newcommand{\Vect}[1]{\mathbf #1}
\newcommand{\SetComponent}[3]{%
	\def\CoNo{#1}\setcounter{subNo}{-1}%
	\begin{array}{*{#1}{#3}}\MakeRaw #2?\end{array}}
\newcommand{\MakeRaw}[1]{
	\def\temp{#1}
	\ifx\temp\endpiece\relax
	\else%
		\stepcounter{subNo}%
		\ifEq{\value{subNo}}0{\stepcounter{subNo} #1}%
			{\ifEq{\value{subNo}}1{\cr #1}{& #1}}%
		\ifnum\value{subNo}=\number\CoNo {\setcounter{subNo}{0}} \fi%
		\expandafter\MakeRaw%
	\fi%
}

\newcommand{\qed}{\hspace{\fill}\rule{3pt}{1ex}\par}
\newcommand{\Qed}{\hfill\rule{3pt}{1ex}}
\newcommand{\qedss}{\intline\qed}
\newcommand{\EquationwithQed}[1]{%
	\par\noindent%
%	\hspace*{\fill}%
	\hspace{0pt plus 1fill}%
	#1\qed}

\newcommand{\DispMath}[2]{{\setlength{\arraycolsep}{0.15em}\begin{#1}%
		#2\end{#1}}\ignorespaces}
%\setlength{\oddsidemargin}{0pt}
%\setlength{\evensidemargin}{0pt}
%\setlength{\marginparwidth}{1in}
%\setlength{\marginparsep}{0pt}

\setlength{\topmargin}{-0.8in}
\setlength{\headheight}{0.01\paperheight}
\setlength{\headsep}{0.01\paperheight}
\setlength{\topskip}{0pt}
\setlength{\textheight}{\paperheight}
\addtolength{\textheight}{-0.8in}
\setlength{\oddsidemargin}{-0.5in}
\setlength{\textwidth}{\paperwidth}
\addtolength{\textwidth}{-1in}
\setlength{\parindent}{0mm}
%
%
%\thispagestyle{enshuu}
\def\endpiece{?}
%
%  均等割付マクロ
%   再帰呼び出しをして可変引数が取れるようにしてあるので
%
\def\WellSpace#1{#1\WellSpaceDo}
\def\WellSpaceDo#1{%
  \def\temp{#1}%
  \ifx\temp\endpiece\relax%
  \else
       \hfill#1\expandafter\WellSpaceDo%
  \fi%
 }
%
%  文字列均等割付マクロ
%   文字列の前後に空白なし
%
\newcommand{\wellspace}[1]{\WellSpace#1?}
%
%  文字列の前後に空白あり
%
\newcommand{\wellspaceside}[1]{%
         \hfill{\WellSpace#1?}\hfill\rule{0em}{0em}}
%\newcommand{\VSpace}[1]{%
% \rule[-#1\textheight]{0em}{#1\textheight}
% }
%
%     番号なし問題文環境
%
\newtheorem{ProbN}{}
%     番号を出力しない設定
\renewcommand{\theProbN}{ }
%
%   番号つき問題文環境
%
\newtheorem{Prob}{問題}
%\renewcommand{\theProb}{{\bfseries\arabic{Prob}. }}
%    小問用番号設定カウンター  問題ごとにリセット
\newcounter{Probsub}[Prob]
\renewcommand{\theProbsub}{\stepcounter{Probsub}\arabic{Probsub}. \ }
%
%  置換積分などの表による演習問題で式を中央に書くためのマクロ
%   \TableSol
%   #1  問題
%   #2  解答
%   #3  解答文記述位置上下移動量(単位をつける。)
%   #4  解答記述幅
%
\newcommand{\TableSol}[4]{%
#1\ifHMSol{\raisebox{#3}{\hspace{#4}#2}}\fi}
%
%   演習問題のヘッダー記述マクロ
%   \ExcerciseHeadN
%   #1 科目名
%   #2 科目名記述前に挿入されるコマンド  通常は空欄でよい。
%   #3 科目名記述位置調整
%   #4  ヘッダー記述前に挿入されるコマンド  通常は空欄でよい。
%   #5 表題の右上に使用されるタイトル
%
\newcommand{\ExcerciseHeadN}[5]{%
\global\def\Pre{#5}%
%\RunningHead%
{\hfill
\setlength{\arrayrulewidth}{0.002\paperwidth}\setlength{\tabcolsep}{0mm}
  #4
  \begin{tabular}[t]{|*{3}{p{0.175\paperwidth}|}%
                      p{0.22\paperwidth}|p{0.11\paperwidth}|}
  \hline
  \wellspaceside{科目名}& \wellspaceside{{学科$\bullet$組}}&
  \wellspaceside{学籍番号} &
  \wellspaceside{氏名} & \wellspaceside{採点}\\ \hline
  {#2 \rule{0mm}{0.033\paperheight}%
    \wellspaceside{{\raisebox{#3mm}{\parbox{0.17\paperwidth}{#1}}}}} &
   & & & \\ \hline
  \end{tabular}
\hfill}
\setcounter{Prob}{0}
\setcounter{Probsub}{0}
\\[-0.02\paperheight]
}
\makeatletter
\newcommand{\ps@enshuu}{%
\renewcommand{\@oddhead}{%
\hfill{\raisebox{-0.004\paperheight}%
         {\rule{0.1\paperwidth}{0.002\paperwidth}}
  \hspace*{-0.1\paperwidth}%
  \makebox[0.09\paperwidth][l]{\upshape\Pre\hfill(\theLecNo 回)}}
}%
\renewcommand{\@evenhead}{}%
\renewcommand{\@oddfoot}{}%
\renewcommand{\@evenfoot}{}%
}
\makeatother
\newcommand{\RunningHead}{%
%\vspace*{-0.12\paperheight}%
\hfill%
  \smash{\raisebox{-0.004\paperheight}%
         {\rule{0.1\paperwidth}{0.002\paperwidth}}
  \hspace*{-0.1\paperwidth}%
  \makebox[0.09\paperwidth][l]{\upshape\Pre\hfill(\theLecNo 回)}}
%\\[-0.5\baselineskip]
%  \vspace*{-0.5\baselineskip}\newline%
}
\newcommand{\AnsBox}{%
{\setlength{\fboxrule}{0.002\paperheight}%
 \fbox{\rule{0em}{1.em}\rule{2.5em}{0em}}
}}
%
\newcommand{\MKSpace}{\rule{2.5em}{0em}}
%
% 問題文記述用マクロ
%
\newcounter{LecNo}
\newcommand{\MakeTitleHead}[2]{%
\expandafter\newcommand\csname Problem#1\endcsname[1]{%
   \ExcerciseHeadN{\LecName}{}{2}{}{#2}
   ##1
   \clearpage}%
}
%
%  演習用紙ごとに通し番号をつけるためのカウンターとコマンドの設定
%
\MakeTitleHead{R}{資料}%    \ProblemR(授業中の資料用)
% A counter ProblemRNo and a command ProblemR are defined.
\MakeTitleHead{P}{予習}%    \ProblemR(授業中の予習用)
\MakeTitleHead{HM}{復習}%    \ProblemHM(自宅学習用)
\MakeTitleHead{AD}{発展}%    \ProblemAD(発展用)
\MakeTitleHead{}{演習}%    \Problem(授業中の演習用)

%
% 各授業ごとに用意されたファイルを読み込むためのマクロの定義
% ホームページでは各回ごとに解答を記述した pdf ファイルを用意するため
% カウンターのデータを別ファイルに記述する機能を持たせてある。
%
\newcommand{\inputfile}[2]{%
\stepcounter{LecNo}
\ifHMSol\else%
%
%  情報記述ファイルは解答なしのものを全体でTeXにかけたときに作る。
%   \write15 で使われるファイルは大元で開く。
%
\immediate\write15{#1::#2::\arabic{LecNo}}\fi
 \input #1.tex
}
\newcommand{\KeepSpaceShow}[1]{\rule[-15mm]{0mm}{23mm}%
\raisebox{-5mm}{{#1}}}
\newcommand{\KeepSpaceShowR}[1]{\rule[-8mm]{0mm}{13mm}\raisebox{-3mm}%
{\makebox[3em][c]{#1}}}
\newcommand{\HMSol}[2]{#1\ifHMSol #2\fi}
%
%   問題文の中に設けられる空欄記述用マクロ
%
%  \AnsBoxS
%   #1  解答欄の縦幅
%   #2  解答欄の横幅
%   #3  解答欄を下に広げるための調整量(- はマクロ内でつけられる。)
%   #4  解答
%
\newcommand{\AnsBoxS}[4]{\BoxS{#1}{#2}{#3}{#4}{1}}
\newcommand{\MBoxS}[4]{\BoxS{#1}{#2}{#3}{#4}{0}}
\newcommand{\BoxS}[5]{%
{\setlength{\fboxrule}{#5pt}%
    \hspace*{0.5em}\framebox[#2][c]{\ifHMSol\makebox[0em][c]{#4}\fi\rule[-#3]{0mm}{#1}}}%
\hspace*{0.5em}\ignorespaces 
}
\newcommand{\TBox}[1]{\BoxS{0em}{2em}{0em}{#1}{0}}
\newcommand{\TBoxS}[2]{\BoxS{0em}{#1}{0em}{#2}{0}}
%
% 解答欄横幅自動設定マクロ \setwd
%
%    \tmpwd に設定
%
\newsavebox{\tmpbox}
\newdimen\tmpdp 
\newdimen\tmpht 
\newdimen\tmpwd
\newdimen\mindim
\mindim=2.5mm
\newcommand{\setwd}{%
     \tmpwd=\wd\tmpbox
        \ifdim\tmpwd<10mm \tmpwd=15mm\else\advance\tmpwd by 6mm\fi
}
%
% 解答欄縦幅自動設定マクロ \setheight
%
%    \tmpht に設定
%
\newcommand{\setheight}{%
        \tmpdp=\dp\tmpbox%
        \ifdim\tmpdp<\mindim \tmpdp=\mindim\else\advance\tmpdp by \mindim\fi%
        \tmpht=\ht\tmpbox%
        \advance\tmpht by \mindim%\tmpht += \mindim
        \advance\tmpht by \tmpdp%
}
%
% 解答欄縦幅および横幅自動設定 \AnsBoxA
%
\newcommand{\AnsBoxA}[1]{%
     \savebox{\tmpbox}{\hbox{#1}}%
     \setwd\setheight%
    \BoxS{\tmpht}{\tmpwd}{\tmpdp}{#1}{1}%
}
\newcommand{\NAnsBoxA}[2]{%
     \savebox{\tmpbox}{\hbox{#1}}%
     \setwd\setheight%
    \BoxS{\tmpht}{\tmpwd}{\tmpdp}{#1}{#2}%
}
\newcommand{\MBoxA}[1]{%
     \savebox{\tmpbox}{\hbox{#1}}%
     \setwd\setheight%
    \BoxS{\tmpht}{\tmpwd}{\tmpdp}{#1}{0}%
}
%
% 解答欄縦幅および横幅自動設定(添え字用) \AnsBoxScA
%
\newcommand{\AnsBoxScA}[1]{%
{\mindim=0.5mm
     \savebox{\tmpbox}{\hbox{#1}}%
     \setwd\setheight%
    \BoxS{\tmpht}{\tmpwd}{\tmpdp}{#1}{1}%
}
}
%
% 解答欄縦幅および横幅半自動設定 \AnsBoxSA
%
\newcommand{\AnsBoxSA}[2]{%
    \savebox{\tmpbox}{\hbox{#2}}%
    \setheight\setwd%
    \BoxS{\tmpht}{\tmpwd}{\tmpdp}{#1}{1}%
}
\newcommand{\NAnsBoxSA}[2]{%
    \savebox{\tmpbox}{\hbox{#2}}%
    \setheight\setwd%
    \BoxS{\tmpht}{\tmpwd}{\tmpdp}{#1}{0.4}%
}
\newcommand{\MBoxSA}[2]{%
    \savebox{\tmpbox}{\hbox{#2}}%
    \setheight\setwd%
    \BoxS{\tmpht}{\tmpwd}{\tmpdp}{#1}{0}%
}
\newcommand{\AnsBoxN}[3]{\AnsBoxS{#1}{#2}{3mm}{#3}}
%
%  問題文と解答記述のためのマクロ(最新版)
%     今後はこのマクロに統一
%
%   高さが解答記述に不足するときは警告 private Warning が出る。
%
%    \ProbandAnsN
%    番号なし用
%    #1 全体の高さ(\textheight単位)
%    #2 問題文
%    #3 解答
%
\newcommand{\ProbandAnsN}[3]{%
\ProbandAnsBody{#1}{#2}{#3}{\relax}}
%
%  \ProbandAns
%
%    #1 全体の高さ(\textheight単位)
%    #2 問題文
%    #3 解答
%
\newcommand{\ProbandAns}[3]{%
\ProbandAnsBody{#1}{#2}{#3}{\theProbsub}}
%
%  \ProbandAnsBody
%
%    #1 全体の高さ(\textheight単位)
%    #2 問題文
%    #3 解答
%    #4 問題文の前に置かれるコマンド(ここに番号を出力するためのマクロを
%            記述すればよい。
%
\newcommand{\ProbandAnsBody}[4]{%
\setbox0=\hbox{\begin{minipage}[t]{0.99\textwidth}
#4% 
#2%
{#3}
\end{minipage}}
\ifdim\dp0>#1\textheight%\else%
\typeout{Private Warning:vspace #1(\textheight) of answer area 
\arabic{Prob}.\arabic{Probsub} is not sufficient.}
\fi
\addtocounter{Probsub}{-1}
 \rule[-#1\textheight]{0em}{#1\textheight}
\begin{minipage}[t]{0.99\textwidth}
#4% 
#2%
\ifHMSol%
{#3}
\fi
\end{minipage}}
%
%  end of \ProbandAnsBody
%
%  選択肢つき解答記述マクロ
%
%      #1 正解の選択肢
%      #2 選択肢のリスト  選択肢それぞれを {} でくくり全体をひとつのパラ
%              メータとして渡す。
%
\newcounter{SelectNo}
\renewcommand{\theSelectNo}{\Ding{\arabic{SelectNo}}\ }
\newcommand{\SelectAnsBox}[2]{%
\def\ansno{#1}
\setcounter{SelectNo}{0}
$\left(\right.$\makeAnsListBody#2\relax$\left.\right)$}
\def\makeAnsListBody#1{%
   \ifx#1\relax\relax%
   \else%
       \stepcounter{SelectNo}\let\currentbox=\mbox%
         \ifnum\value{SelectNo}=\number\ansno%
         \ifHMSol%
         \let\currentbox=\Ovalbox\fi\fi%
       \setbox0=\hbox{\Ovalbox{\theSelectNo #1}}%
       \vphantom{\box0}%
        \makebox[\wd0][c]{\currentbox{\theSelectNo #1}}%
	 \expandafter\makeAnsListBody%
   \fi%
}
\newcommand{\SelectAnsRem}{%
\\\ifHMSol {\upshape(答を選択するところは正解のほうが囲まれています。)}\fi\par}

%  丸付き数字記述マクロ(出典は忘れてしまいました。多分 LaTex Companion?)
%   Ding という ps フォントを使うこともできますが 0 の丸付きがないので
%   変更しました。
%
\newcommand{\Ding}[1]{{%
\setbox0=\hbox{\bfseries\large{$\bigcirc$}}%
\makebox[\wd0][c]{\textup{#1}}\hspace{-\wd0}\box0
}}
\newcommand{\Newpage}{\vfill\hfill{\bfseries 裏にもあります。}\newpage%
}
\newcommand{\StackShow}[4]{
 \begin{minipage}{2.7em}
\begin{center}
 \begin{tabular}[t]{|p{2em}|}
 \hline
 \StackItem{#1}
 \StackItem{#2}
 \StackItem{#3}
 \StackItem{#4}
\end{tabular}
\end{center}
 \end{minipage}
}
\newcommand{\StackShowN}[6]{
 \begin{tabular}{|@{}c@{}|}
\multicolumn{1}{c}{\MBoxSA{%
   \ifx#1d\Push{#2}\else%
      \ifx#1u\relax\Pop\else%
         \ifx#1o\relax\Operator{#2}%
         \fi%
      \fi%
   \fi}{\hspace*{3.em}}}\\%
 \hline
 \MBoxSA{#3}{3}\\\hline\MBoxSA{#4}{3}\\\hline
 \MBoxSA{#5}{3}\\\hline\MBoxSA{#6}{3}\\\hline
\end{tabular}
}
\newcommand{\StackOp}[2]{
\begin{minipage}{4.5em}
\makebox[4.5em][c]{%
   \ifx#1d\Push{#2}\else%
      \ifx#1u\relax\Pop\else%
         \ifx#1o\relax\Operator{#2}%
         \fi%
      \fi%
   \fi}\\%
\hspace*{\fill}$\longrightarrow$\hspace*{\fill}
\end{minipage}
}
\newcommand{\Operator}[1]{%
#1}
\newcommand{\Push}[1]{%
#1を\texttt{push}}
\newcommand{\Pop}{\texttt{pop}}
\newcommand{\StackItem}[1]{%
\hspace*{\fill}\texttt{#1}\hspace*{\fill}\rule{0em}{0em}\\\hline}
\newcommand{\Ls}[1]{\hbox{\Large #1}}
%\renewcommand{\labelstyle}{\textstyle}
\newcommand{\StateTable}[3]{%
\begin{center}%\Large
\begin{tabular}{|c|c|c|c|c|}
\hline
\smash{\raisebox{-0.8em}{状態}}& \multicolumn{2}{c|}{入力}&
\smash{\raisebox{-0.8em}{受理状態}}&\smash{\raisebox{-0.8em}{コメント}}\\
\cline{2-3}
& \makebox[5em]{#1} &  \makebox[5em]{#2}&&\\\hline
#3%\relax\relax\relax\relax\relax\relax
\end{tabular}\end{center}
}
\newtheorem{Th}{定理}[section]
\newcommand{\CSpace}{\rule{10mm}{0mm}}
\newcommand{\RSpace}{\rule{0mm}{10mm}}
\newcommand{\Node}[1]{*[o]=<8mm>[F]{#1}}
\newcommand{\NodeA}[1]{*[o]=<8mm>[F=]{#1}}

%%%%%%
\newcommand{\SetHeight}[2]{\rule[#2]{0mm}{#1}}
%
%
\newif\ifKaiNo
\newcounter{Kai}
\newcommand{\Kai}[2]{%
\setcounter{Kai}{0}%
\gdef\ifKaiNo{\iftrue}%
\KaiNo{#1}{#2}%
}
\newcommand{\KaiNonum}[2]{%
\setcounter{Kai}{0}%
\gdef\ifKaiNo{\iffalse}%
\KaiNo{#1}{#2}%
}
\newcommand{\KaiNo}[2]{%
\stepcounter{Kai}%
\ifLt{\value{Kai}}{#1}%
  {&\makebox[#2em][c]{\ifKaiNo\arabic{Kai}\fi}\KaiNo{#1}{#2}}%
  {\rule[-0.7em]{0em}{2em}}%
}
\newcommand{\VRULE}{\vrule height 0.4pt width 6mm}
\newcommand{\VRULET}{\VRULE &\VRULE &\VRULE }
\newcommand{\VRULEV}{\raisebox{-3mm}{\vrule height 9mm width 0.4pt}}
\newcommand{\MM}[1]{\MV&\makebox[0mm][c]{#1}&\makebox[6mm][c]{}}
\newcommand{\MV}{\VRULEV\makebox[6mm][c]{}}
\newcommand{\MVS}[1]{\VRULEV\makebox[6mm][c]{\ifHMSol#1\fi}}
\newcommand{\MVA}[2]{\VRULEV\makebox[6mm][c]{#1\ifHMSol#2\fi}}
\newcommand{\MA}[1]{\makebox[6mm][c]{\ifHMSol#1\fi}}
\newcommand{\Explanation}[1]{%
{\ifHMSol\bfseries 解説:} #1\fi}
%
%
%
\newcommand{\StateTableHead}[2]{%
\begin{center}%\Large
\begin{tabular}{|c|c|c|c|c|}
\hline
\smash{\raisebox{-0.8em}{状態}}& \multicolumn{2}{c|}{入力}&
受理&\smash{\raisebox{-0.8em}{コメント}}\\
\cline{2-3}%
& \makebox[4em]{#1} & \makebox[4em]{#2}&状態&\\\hline
}
\newcommand{\NFAStateTable}[3]{%
\StateTableHead{#1}{#2}%
\SetNFAStateTable#3\hline\relax\relax\relax\relax\relax%
\end{center}
}
\newcommand{\eNFAStateTable}[4]{%
\begin{center}%\Large
\begin{tabular}{|c|c|c|c|c|c|}
\hline
\smash{\raisebox{-0.8em}{状態}}& \multicolumn{3}{c|}{入力}&
受理&\smash{\raisebox{-0.8em}{コメント}}\\
\cline{2-4}%
& \makebox[5em]{#1} &  \makebox[5em]{#2}&\makebox[5em]{#3}&状態&\\\hline
\SeteNFAStateTable#4\hline\relax\relax\relax\relax\relax\relax%
\end{center}
}
\newcommand{\NFAtoDFAStateTable}[3]{%
\StateTableHead{#1}{#2}%
\SetStateSet#3\hline\relax\relax\relax\relax\relax\relax%
\end{center}
}
\newcommand{\SetL}[1]{\ifx#1-$-$\else\Ls{$\{#1\}$}\fi}
\newcommand{\SetStateSet}[7]{%
\ifx#1\hline\hline\end{tabular}\else%
\ifx#1-\relax\else\Ls{$#1=$}\fi
\hspace*{\fill}\MBoxA{\SetL{#2}}\hspace*{\fill}&
\MBoxA{\SetL{#3}}&\MBoxA{\SetL{#4}}&
\ifHMSol{\ifx#5y\relax$\bigcirc$\fi}\fi&\ifHMSol{#6}\fi\\#7%
\expandafter\SetStateSet\fi}
%
%
%
\newcommand{\SetNFAStateTable}[6]{%
\ifx#1\hline\hline\end{tabular}\else%
\ifx#1-\relax\else\Ls{$#1$}\fi
&
\MBoxA{\SetL{#2}}&\MBoxA{\SetL{#3}}&
\ifHMSol{\ifx#4y\relax$\bigcirc$\fi}\fi&\ifHMSol{#5}\fi\\#6%
\expandafter\SetNFAStateTable\fi}
%
\newcommand{\SeteNFAStateTable}[7]{%
\ifx#1\hline\hline\end{tabular}\else%
\ifx#1-\relax\else\Ls{$#1$}\fi
&
\MBoxA{\SetL{#2}}&\MBoxA{\SetL{#3}}&\MBoxA{\SetL{#4}}&
\MBoxA{\ifx#5y\relax$\bigcirc$\fi}&\MBoxA{#6}\\#7%
\expandafter\SeteNFAStateTable\fi}
\newcommand{\MinmizeDFA}[3]{%
$\begin{array}[t]{|c|c|c|c|c|}
\hline
 \smash{\raisebox{-1em}{\small 同値類}}& \smash{\raisebox{-1em}{状態}}&
\multicolumn{2}{c|}{入力}&受理 \\\cline{3-4}
 & &#1 &#2 &{状態}\\\hline
\ShowMinimizeDFA#3\hline\relax\relax\relax\relax\relax\relax
\end{array}$
}
\newcommand{\ShowMinimizeDFA}[7]{%
\ifx#1\hline\hline\else%
\smash{\raisebox{-#1em}{\MBoxA{$ #2$}}}&
\MBoxA{$ #3$}&\MBoxA{$ #4$}&\MBoxA{$ #5$}%
&\MBoxA{\ifx#6y\relax$\bigcirc$\fi}\\#7
\expandafter\ShowMinimizeDFA\fi%
}
\newcommand{\ProbS}[2]{\theProbsub #1\vspace*{#2\textheight}\\}
