\input ProgramDevHead.tex
\begin{document}
\paragraph{訂正}

実行例5.7(配布資料49ページ)の\Verb+Student+を次のように変えてもよい。

\LISTN{05-01prototype-test2.html}{47}{52,firstnumber=1}{\normalsize}
\begin{itemize}
 \item \Verb+Person2+内で定義されているメンバー変数をそのまま利用するた
			 めに、2行目で\Verb+Person2+を呼び出している。
 \item 単純に呼び出すのでは\Verb+Person2+内での\Verb+this+が
			 \Verb+Person2+になる。
 \item そこで\Verb+Object+のメソッド\Verb+call+を用いると、第1番目の引数
			 をそのオブジェクト内での\Verb+this+を書き直すことが可能となる。
\end{itemize}
このリストでは\Verb+Person2+内で\Verb+this+を出力している。

\paragraph{\texttt{class}について}

ECMAScriptは 6 から \Verb+class+によりクラスの宣言が可能となった。
今までの例を\Verb+class+で書き直すと次のようになる。
\LISTN{05-01prototype-class.html}{8}{39,firstnumber=1}{\normalsize}
\begin{itemize}
 \item クラスの宣言は\Verb+class+を用いる(1行目)。
 \item コンストラクターは\Verb+constructor+関数で定義される(2行目から7行
			 目)。
 \item \Verb+class+内ではメソッドを宣言する。ここでは3つのメソッドが定義
			 されている。
 \item 27行目から32行目は継承の例である。
 \item クラス名の後にキーワード\Verb+extends+をつけて親クラスを指定する。
			 親クラスを複数指定する多重継承はできない。
 \item 29行目では親クラスのコンストラクタを呼び出すために\Verb+super+を
			 利用している。
% \item いわゆるクラス変数は定義できない。
\end{itemize}
\end{document}