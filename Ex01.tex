\ProblemNN{
 \begin{Prob}\upshape\Must
  次の問いに答えよ。
  \begin{enumerate}
   \item 使用するブラウザの名称とバージョン\\[0.05\textheight]
   \item そのブラウザでの開発者ツールほ開き方とタブの種類
         \\[0.1\textheight]
   \item 簡単な式を実行させたときのキャプチャ画面を印刷してこの提出用紙
				 とまとめて提出する。%\\[0.2\textheight]
  \end{enumerate}
 \end{Prob}
 \Newpage
 \begin{Prob}\upshape\Must
  課題1.2の実行結果の報告をしなさい。自分で内容を変更してもかまわない
  (ここまでが{\bfseries 必須})。内容を変更したり、独自のコードを付け加えてもよい。そ
  の場合にはプログラムリストや考察もつけること。
 \end{Prob}
 }

\RubricC{復習の目的は次のとおりである。
\begin{itemize}
 \item 通常使用しているブラウザにおいてJavaScriptのプログラムがデバッグ
       できるツールになっていることを理解し、使いこなせるようにする。
 \item 簡単なHTML文書からデバッグをする方法を身に着ける。
\end{itemize}
}
{
{\RProbNoM{1.1}{0}}{5}{
	{使用したブラウザに関する情報を示すキャプチャ画面がある。}
  {キャプチャ画面内で名称、バージョンが読み取れる。}
  {キャプチャ画面に基づいて正しく記述している。}
	}
	{
	{使用したブラウザのバージョンを表示するページのキャプチャ画面がある}
  {キャプチャ画面がブラウザ画面全体になっている。}
  {キャプチャ画面内のブラウザの名称、バージョンが読みとれる。}
  {ブラウザ名とバージョンをすべて記述している。}
	{使用したブラウザの名称がある。}
  {使用したブラウザのバージョンがある。}
  {}
	}
	{
	{使用したブラウザのバージョンを表示するキャプチャ画面がない。}
  {キャプチャ画面がブラウザ画面全体になっていない。}
  {キャプチャ画面内のブラウザの名称、バージョンが読みにくいか全く読めな
  い。}
  {ブラウザ名とバージョンがないか、一部しか記述していない。}
	{使用したブラウザの名称がないか、正確ではない。}
  {使用したブラウザのバージョンがない。}
	}
  {\RProbNoM{1.2}{0}}{5}{
	{使用したブラウザでの開発者ツールの開き方が書いてある。}
  {開発者ツールにあるタブの種類がほとんどすべて書いてあり、重要なものに
  ついて目的が書いてある。}
  }
  {
	{使用したブラウザでの開発者ツールほ開き方についての記述がある。}
  {開発者ツールにあるタブの種類のうち、「Elements」、「console」、
  「Sources」、「Application」について説明がある。(Chromeの場合)}
  {開発者ツールにあるタブの残りの種類について説明がある。}
  }
  {
	{使用したブラウザでの開発者ツールほ開き方について説明がない。}
  {開発者ツールにあるタブの種類で重要なものが足りない。}
  {開発者ツールにあるタブの説明がほとんどないか全くない。}
  }
  {\RProbNoM{1.3}{0}}{10}{
  {開発者ツールで実行した結果が十分にあり、それらの種類が多い。}
  {開発者ツールで実行した結果が分かるような大きさのフォントでキャプチャ
  されている。}
  {充分な考察がある。}
  }
  {
  {開発者ツールで実行した結果が10つ以上ある。}
  {開発者ツールで実行した結果の種類が3つ以上ある。}
  {開発者ツールで実行した結果のキャプチャ画面がブラウザ全体になっている。}
  {開発者ツールで実行した結果のキャプチャ画面が見やすい。}
  {考察がある。}
  }
  {
  {開発者ツールで実行した結果が少ない。}
  {開発者ツールで実行した結果の種類が少ない。}
  {開発者ツールで実行した結果のキャプチャ画面の範囲が狭いか、ない。}
  {開発者ツールで実行した結果のキャプチャ画面で結果の確認ができない。}
  {考察がない。}
  {キャプチャ画面の用紙が張り付けてないか、貼り付けかたに問題がある。}
  }
	{\RProbNoM{2}{0.5}}{10}
	{
  {実行結果のキャプチャが実行結果が見やすい。}
  {実行結果に関する考察が十分にある。}
  {Strict モードでの実行結果が十分に読み取れる。}
  {Strictモードでの実行について十分な解説と考察がある。}
  }
	{
  {\texttt{foo();}を実行した結果と、その後に\texttt{i;}を実行した結果が
  ある。}
  {変数\texttt{i}の宣言を省略したときの実行結果がある。}
  {変数の宣言の有無の違いの考察がある。}
  {実行結果に関する考察が十分である。}
  {Strict モードでの実行結果の内容が読みとれる。}
  {Strictモードでの実行結果がある。}
  {Strictモードでの実行結果の解説と考察がある。}
  {実行結果のキャプチャ内の内容が読み取れる。}
	}
	{
  {変数\texttt{i}の宣言を省略したときの実行結果がないか、間違っている。}
  {変数の宣言の有無の違いの考察がないか、間違っている。}
  {実行結果に関する考察がないか、不十分である。}
  {Strict モードでの実行結果が見にくい。}
  {Strictモードでの実行結果がないか、不十分である。}
  {Strictモードでの実行結果の解説と考察がないか、不十分である。}
  {実行結果のキャプチャ内の内容が小さすぎて読みくいか、読めない。}
	}
	{問題2}{10}
	{
  {HTMLファイルのJavaScriptの部分に独自のプログラムを付け加えていて、
  十分な解説と考察がある。}
  {実行結果のキャプチャ内実行結果が十分に読み取れる。}
  }
	{
  {HTMLファイルのJavaScriptの部分に独自のプログラムを付け加えている。}
  {付け加えたコードに独創性がある。}
  {付け加えたコードの十分な解説がある。}
  {実行結果のキャプチャ内の実行結果が読みとれる。}
  {実行結果に関する考察がある。}
	}
	{
  {HTMLファイルのJavaScriptの部分に独自のプログラムを付け加えていない。}
  {付け加えたコードに独創性がない。}
  {付け加えたコードの十分な解説が少ないか、不十分である。}
  {実行結果のキャプチャ内の実行結果が読みとれない。}
  {実行結果に関する考察が少ないかない。}
	}
}
