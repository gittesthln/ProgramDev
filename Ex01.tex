\ProblemNN{
 \begin{Prob}\upshape
  次の問いに答えよ。
  \begin{enumerate}
   \item 使用するブラウザの名称とバージョン\\[0.05\textheight]
   \item そのブラウザでの開発者ツールほ開き方とタブの種類
         \\[0.1\textheight]
   \item 簡単な式を実行させたときのキャプチャ画面を印刷してこの提出用紙
				 に張り付ける。貼り付ける用紙の裏面には学籍番号と名前、問題の番
				 号を書いておくこと(はがれたとき、誰のかわかるようにするため)\\[0.2\textheight]
  \end{enumerate}
 \end{Prob}
 \Newpage
 \begin{Prob}\upshape
  課題1.2の実行結果のキャプチャ画面を張り付けなさい。貼り付ける用紙の裏
  面には学籍番号と名前、問題の番号を書いておくこと。出力の形式を変える
  なり、自分で内容を変更してもかまわない。内容を変更した場合にはプログラ
  ムリストもつけること。
 \end{Prob}
 }

\Rubric{復習の目的は次のとおりである。
\begin{itemize}
 \item 通常使用しているブラウザにおいてJavaScriptのプログラムがデバッグ
       できるツールになっていること
 \item 簡単なHTML文書からデバッグをする方法を身に着ける。
\end{itemize}
}
{{問題1.1}{2}{
	{使用したブラウザのバージョンを表示するページ全体ののキャプチャ画面がある。}
  {キャプチャ画面内の名称、バージョンが読み取れ、それに基づいて正しく記
  述している。}
	}
	{
	{使用したブラウザのバージョンを表示するページのキャプチャ画面があるが、
  ブラウザ画面全体になっていない。}
  {キャプチャ画面内のブラウザの名称、バージョンが読みにくい。}
  {ブラウザ名とバージョンの一部を省略して記述している。}
	{使用したブラウザの名称はあるが、バージョンの一部ががない。}
	}
	{
	{使用したブラウザのバージョンを表示するページのキャプチャ画面がない。}
  {ブラウザ名とバージョンのかなりの部分を省略して記述している。または、
  記述がない。}
	}
  {問題1.2}{3}{
	{使用したブラウザでの開発者ツールの開き方が書いてある。}
  {開発者ツールにあるタブの種類がほとんどすべて書いてあり、重要なものに
  ついて目的が書いてある。}
  }
  {
	{使用したブラウザでの開発者ツールほ開き方について不十分な点がある。}
  {開発者ツールにあるタブの種類が十分にあるが説明が少し足りない。}
  }
  {
	{使用したブラウザでの開発者ツールほ開き方について説明がない。}
  {開発者ツールにあるタブの種類で重要なものが足りない。また、それぞれの
  タブの説明がほとんどないか全くない。}
  }
  {問題1.3}{5}{
  {開発者ツールで実行した結果が十分にある。}
  {開発者ツールで実行した結果が分かるような大きさのフォントでキャプチャ
  されている。}
  {充分な考察がある。}
  {キャプチャ画面の用紙がしっかり張り付けてある。}
  }
  {
  {開発者ツールで実行した結果が少し足りない。}
  {開発者ツールで実行した結果のキャプチャ画面がブラウザ全体になっていな
  い。}
  {開発者ツールで実行した結果のキャプチャ画面が少し見にくい。}
  {考察が足りない。}
  {キャプチャ画面の用紙が張り付けが少し不十分てある。}
  }
  {
  {開発者ツールで実行した結果が足りない。}
  {開発者ツールで実行した結果のキャプチャ画面の範囲が少ないか、ない。}
  {開発者ツールで実行した結果のキャプチャ画面で結果の確認ができない。}
  {考察がない。}
  {キャプチャ画面の用紙が張り付けてないか、貼り付けかたに問題がある。}
  }
	{問題2}{10}
	{
  {HTMLファイルのJavaScriptの部分に独自のプログラムを付け加えている。}
  {独自のプログラム部分の十分な解説がある。}
  {実行結果のキャプチャ内の文字が十分な大きさなので実行結果が見やすい。}
  {実行結果に関する考察が十分にある。}
  {独自のプログラムについて、リストとその解説が十分にある。}
  {Strict モードでの実行結果のキャプチャの文字が十分な大きさなので内容が読みやすい。}
  {Strictモードでの実行について十分な解説と考察がある。}
  }
	{
  {HTMLファイルのJavaScriptの部分に独自のプログラムを付け加えていない。}
  {実行結果のキャプチャ内の文字が少し小さくて実行結果が少し見にくい。}
  {実行結果に関する考察が少し不十分である。}
  {リストとその解説が少しある。}
  {Strict モードでの実行結果のキャプチャの文字が小さくて内容が読みにくい。}
  {Strictモードでの実行について解説と考察が少し不十分である。}
	}
	{
  {HTMLファイルのJavaScriptの使用したプログラムに言及がない。}
  {実行結果のキャプチャ内の文字が小さすぎて実行結果が読めない。}
  {実行結果に関する考察が不十分であるか全くない。}
  {リストとその解説が少なすぎるか全くない。}
  {Strict モードでの実行結果のキャプチャの文字が小さすぎて内容が読めない。}
  {Strictモードでの実行について解説と考察がないか不十分である。}
	}
}
