\begin{Prob}\upshape\Must
次の文字列にマッチする正規表現を作れ。
\begin{enumerate}
 \item C言語の変数名の命名規則に合う文字列\ifText 。
			 対象とする文字にいわゆる全角文字は考慮しないでよい。\\[0.05\textheight]\fi
 \item 符号付小数。符号はなくてもよい。整数の場合は小数点はなくてもよい。また、小数
       点はあっても小数部はなくてもよい。整数部分には数字が少なくとも一
       つはあること。たとえば\Verb+-1.+にはマッチするが、\Verb+-.0+には
       整数部分がないのでマッチしない。\Verb+.+のエスケープを忘
       れないようにすること。\ifText\\[0.05\textheight]\fi
 \item 前問の正規表現を拡張して、指数部が付いた浮動小数にマッチするもの
       を作れ。指数部は\Verb+E+または\Verb+e+で始まり、符号付き(なくても
       よい)整数とする。\ifText\\[0.05\textheight]\fi
 \item 24時間生の時刻の表し方。時、分、秒はすべて2桁とし、それらの区切り
       は\Verb+:+とする。たとえば午後1時10分6秒は\Verb+13:10:06+である。
  また、\Verb+13:10:66+ は秒数が60以上になっているのでマッチしてはいけな
       い。
       \ifText\\[0.05\textheight]\fi
 \item ファイルの拡張子が\texttt{.html}であるファイル名
			 \ifText 。ファイル名には使用できない文字やいわゆる全角文字の使用
			 に関しては考慮しなくてよい。\\[0.05\textheight]\fi
\end{enumerate}
\end{Prob}
