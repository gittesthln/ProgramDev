\section{配列とオブジェクト}
配列はいくつかのデータをまとめて一つの変数に格納している。各データを利用
するためには \verb+foo[1]+ のように数による添え字を使う。これに対し、
オブジェクトでは添え字に任意の文字列を使うことができる。
\begin{Exec}\label{Execconstructor}\upshape
次の例はあるオブジェクトを定義してそれの各データにアクセスする方法を示している。
\begin{Verbatim}
let person = {
  name : "foo",
  birthday :{
    year : 2001,
    month : 4,
    day : 1
  },
  "hometown" : "神奈川",
}
\end{Verbatim}
\begin{itemize}
 \item オブジェクトは全体を \verb+{}+ で囲む。
 \item 各要素はキーと値の組で表される。両者の間は \verb+:+ で区切る。
 \item キーは任意の文字列でよい。
 \item キーの文字列が変数名の約束事に合わないときはキー全体を \verb+""+
			 で囲う必要がある。
 \item 値はJavaScriptで取り扱えるデータなあらば何でもよい。上の例ではキー
       \verb+birthday+ の値がまたオブジェクトとなっている。値が関数であ
       るキーはそのオブジェクトのメソッドと呼ばれる。
 \item 各要素の値を取り出す方法は2通りある。
\begin{itemize}
 \item \verb+.+演算子を用いてオブジェクトのキーをそのあとに書く。
\begin{Verbatim}
>person.name;
"foo"
\end{Verbatim}
 \item 配列と同様に\verb+[]+内にキーを文字列として指定する。
\begin{Verbatim}
>person["name"];
"foo"
\end{Verbatim}
\end{itemize}

 \item オブジェクトの中にあるキーをすべて網羅するようなループを書く場合や変数名
       として利用できないキーを参照する場合には後者
       の方法が利用される。
 \item キーの値が再びオブジェクトであれば、前と同様の方法で値を取り出せ
       る。
\begin{Verbatim}
>person.birthday;
Object {year: 2001, month: 4, day: 1}
>person.birthday.year;
2001
>person.birthday["year"];
2001
\end{Verbatim}
この例のように取り出し方は混在してもよい。
 \item キーの値は代入して変更できる。
\begin{Verbatim}
>person.hometown;
"神奈川"
>person.hometown="北海道";
"北海道"
>person.hometown;
"北海道"
\end{Verbatim}
 \item 存在しないキーを指定すると値として\verb+undefined+が返る。
\begin{Verbatim}
>person.mother;
undefined
\end{Verbatim}
 \item 存在しないキーに値を代入すると、キーが自動で生成される。
\begin{Verbatim}
>person.mother = "aaa";
"aaa"
>person.mother;
"aaa"
\end{Verbatim}
 \item オブジェクトのキーをすべて渡るループは \verb+for-in+で実現できる。
\begin{itemize}
 \item \verb+for( v in obj)+ の形で使用する。変数 \verb+v+ はループ内で
       キーの値が代入される変数、\verb+obj+ はキーが走査されるオブジェク
       トである。\footnote{\pageref{ES2016}ページの[\ref{ES2016}162ペー
       ジ]にはオブジェクトの独自プロパティだけを見たい場合には
       \ElmJ{hasOwnProperty()}を使うことが推奨されている。}
 \item キーの値は \verb+obj[v]+ で得られる。
\end{itemize}
\begin{Verbatim}
>for(i in person) { console.log(`i ${person[i]}`);};
name foo
birthday [object Object]
hometown 北海道
mother aaa
undefined
\end{Verbatim}
最後の\verb+undefined+は\verb+for+ループの戻り値である。
\end{itemize}
\end{Exec}
オブジェクトを\verb+{}+の形式で表したものを\ElmJ{オブジェクトリテラル}と
よぶ。
\begin{Prob}\upshape
\verb+window+ オブジェクトにはどのようなプロパティがあるか調べよ。2つ以
 上のブラウザで実行し、比較すること。
\end{Prob}
