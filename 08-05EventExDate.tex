%-*- coding: utf-8 -*-
\begin{Exec}\upshape\label{pulldownDate}
次の例は3つのプルダウンメニューが順に年、月、日を選択できる
 ものである。年や月が変化(\texttt{change}イベントが発生)すると日
 のプルダウンメニューの日付のメニューがその年月にある日までに変わるよう
 になっている。
 \LISTN{09-03pulldown.html}{1}{16}{\normalsize}
 このプログラムは8行目から始まる
       \texttt{window.onload}のイベントハンドラーの中にある。

 9行目から16行目では指定された範囲の値を選択できるプルダウンメニュー
       を作成する関数\texttt{makeSelectNumber()}を定義している。
\begin{itemize}
 \item 引数は順に、下限値(\texttt{from})、上限値(\texttt{to})、数字の前
       後に付ける文字列(\texttt{prefix}と\texttt{suffix})、プルダウンメ
       ニュ-の\texttt{id}属性の属性値(\texttt{id})と親要素
       (\texttt{parent})である。
 \item 10行目で\texttt{select}要素を作成している。作成のために
       \texttt{makeElm}関数(18行目から26行目で定義)を呼び出している。
 \item 11行目から14行目で\texttt{option}要素を順に作成している。
 \item 13行目でプルダウンメニューに表示される文字列をテキストノードを
       作成する関数\texttt{makeTextNode()} (25行目から27行目)を呼び出している。
\end{itemize}
 \LISTN{09-03pulldown.html}{17}{24}{\normalsize}
17行目から24行目で与えられた要素を作成する関数\texttt{makeElm()}を定義し
 ている。
\begin{itemize}
 \item 引数は順に、要素名(\texttt{name})、属性値のリスト
       (\texttt{attribs})と親要素の指定である。
 \item 指定された要素を作成し(18行目)て、与えられた属
			 性とその属性値がメンバーになっているオブジェクトの値を順に選んで
			 属性値を設定する(19行目から21行目)。
 \item 親要素が指定されている場合には、作成した要素
       を子要素として付け加える(22行目)。
 \item 作成した要素を戻り値にする(23行目)。
\end{itemize}
 \LISTN{09-03pulldown.html}{25}{27}{\normalsize}
関数\texttt{makeTextNode()}は指定された文字列を基にテキストノードを作成
 し、指定された親要素の子要素に設定している。
 \LISTN{09-03pulldown.html}{28}{34}{\normalsize}
56行目の\texttt{form}要素内にプルダウンメニューを作成する。プルダウンメ
 ニューで表示される日付は実行当日になるように設定される。
\begin{itemize}
 \item 28行目では\texttt{form}要素を得ている。
 \item 29行目と30行目ではそれぞれ年、月のプルダウンメニューを作成して
       \texttt{form}要素の子要素にしている。
 \item 28日から31日まであるプルダウンメニューを4つ作成して配
       列に格納している(32行目から34行目)。ここでの関数呼び出しは最後の
       親要素が省略されてので、別の要素の子要素としては登録されない。
\end{itemize}
 \LISTN{09-03pulldown.html}{35}{40}{\normalsize}
ここでは、実行時の日付がプルダウンメニューの初期値として表示されるように
 している。
\begin{itemize}
 \item \texttt{Date()}オブジェクトを引数なしでよぶと、実行時の時間が得ら
       れる(35行目)。\footnote{この時間は1970年1月1日午前0時(GMT)この時刻をUNIX
       エポックという。 からの経過時間であり、単位はミリ秒である。}
 \item 36行目で年を得ている。\texttt{Date}オブジェクトの
       \texttt{getYear()}メソッドは西暦の下2桁し
       か返さないので使わないこと。
 \item 37行目で月を得ている。\texttt{getMonth()}メソッドの戻り値は
       $0$(1月)から $11$(12月)である。
 \item 39行目と40行目では、得られた年と月をそれぞれのプルダウンメニュー
       に設定している。月の値を1増やしていることに注意すること。
 \item \texttt{Date()}オブジェクトを年、月、日(さらに、時、分、秒もオプ
       ションの引数として与えられる)を与えて、インスタンスを作成できる。
       実行当日の翌月の0日を指定することで翌月の1日の1日前の日
       付に設定できる。その日付から\texttt{getDate()}メソッドで現在の月
       の最後の日が得られる\footnote{ネットの情報を参考にした。
       \texttt{Date}オブジェクトのコンストラクタは範囲外の日時を指定して
       も正しい日時に解釈してくれる。}(38行目)。
 % \item
       
			 なお、\texttt{Date}オブジェクトの\texttt{Day()}
       は曜日を得るメソッドで$0$(日曜日)から$6$(土曜日)の値が返
       る。
 \item 39行目で、この日数を持つプルダウンメニューを子要素として追加し、
			 40行目で、日の選択する値を設定している。
\end{itemize}
 \LISTN{09-03pulldown.html}{41}{51}{\normalsize}
 ここではプルダウンメニューの値が変化したイベント(\texttt{change})ハンド
 ラーを登録している。
\begin{itemize}
 \item イベントハンドラーは\texttt{form}要素につけている。
 \item 42行目で、現在表示されている日を保存している。
 \item 43行目で、現在、プルダウンメニューで設定されている年月の日数を求
       めている。
 \item この日数と現在表示されている日数のプルダウンメニューの日数が異な
       る場合には(44行目)子ノードを入れ替える(45行目)。さらに、初期値を
       現在の値と月の最終日の小さいほうに設定する(46行目)。
\end{itemize}
 \LISTN{09-03pulldown.html}{53}{last}{\normalsize}
ここではHTML文書内で表示される要素を記述している。
ここでは\texttt{body}要素内に\texttt{form}要素があるだけである。
\end{Exec}
