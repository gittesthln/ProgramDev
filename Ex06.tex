\ProblemNN{
\input 06-01makeRegEx.tex
\input 06-01getDate.tex
\Newpage
\input 06-02checkRegEx.tex
\input 06-03checkPromptValue.tex
}
\RubricC{
復習の目的は次の項目を理解することである。
\begin{itemize}
 \item 簡単な正規表現を作成できる。
 \item 正規表現を用いて文字列処理を簡単にする方法を理解する。
 \item 正規表現における文字列のマッチを理解する。
 \item 正規表現を用いて文字列が与えられた条件を満たすかどうかをチェック
			 する。
\end{itemize}
}
{
{\RProbNoM{1.1}{0.2}}{3}
	{
	{C言語の変数名の命名規則の正規表現が正しい}
	}
	{
	{C言語の変数名の命名規則の先頭文字の範囲が正しい}
	{C言語の変数名の命名規則の2文字目以降の範囲が正しい}
	{C言語の変数名の命名規則の長さの指定が正しい}
	}
	{
	{C言語の変数名の命名規則の先頭文字の範囲に間違いがある}
	{C言語の変数名の命名規則の2文字目以降の範囲に間違いがある}
	{C言語の変数名の命名規則の長さの指定に間違いがある}
	}
	{\RProbNoM{1.2}{0.2}}{4}
	{
	{符号付小数を表す正規表現が正しい}
	{十分なチェックの報告がある。}
	}
	{
	{符号がないか先頭文字になっている。}
	{整数部分の正規表現が正しい。}
	{小数点のエスケープをしている。}
	{小数点がない場合に対処している。}
	{小数部分が0文字以上の正規表現である。}
	{\texttt{*}と\texttt{?}の区別ができている。}
	{マッチする場合のチェックの報告がある。}
	{マッチしない場合のチェックの報告がある。}
	}
	{
	{符号の指定が間違っている。}
	{整数部分の正規表現の部分が間違っている。}
	{小数点のエスケープをしていない。}
	{小数部分の正規表現が間違っている。}
	{\texttt{*}と\texttt{?}の区別ができていない。}
	{マッチする場合のチェックの報告がない。}
	{マッチしない場合のチェックの報告がない。}
	}
	{\RProbNoM{1.3}{0.2}}{3}
	{
	{仮数部の部分が正しい。}
	{指数部の部分が正しい。}
	{十分なチェックの報告がある。}
	}
	{
	{仮数部の符号の部分が正しい。}
	{仮数部の小数点の部分が正しい。}
	{仮数部の小数部の部分が正しい。}
	{指数部の正規表現が正しい。}
	{指数部の符号がない場合も含めている。}
	{指数部がない場合も含めている。}
	{マッチする場合としない場合のチェックの報告がある。}
	}
	{
	{仮数部の符号の指定が間違っている。}
	{仮数部の小数点の指定が間違っている。}
	{仮数部の小数部の指定が間違っている。}
	{指数部の指定が間違っている。}
	{\texttt{*}と\texttt{?}の区別ができていない。}
	{マッチする場合としない場合のチェックの報告がない。}
	}
	{\RProbNoM{1.4}{0.2}}{3}
	{
	{時間の部分の正規表現が正しい。}
	{分と秒の部分の正規表現が正しい。}
	{十分なチェックの報告がある。}
	}
	{
	{時間の部分の正規表現が正しい。}
	{分と秒の部分の正規表現が正しい。}
	{マッチする場合としない場合のチェックの報告がある。}
	}
	{
  {文字列としてマッチする表現になっていない。}
	{時が24以上の値にマッチする。}
	{分、秒が60以上の値にマッチする。}
	{マッチする場合としない場合のチェックの報告がない。}
	}
	{\RProbNoM{1.5}{0.2}}{3}
	{
	{拡張子の前部分の正規表現が正しい。}
	{拡張子の部分の正規表現が正しい。}
	{十分なチェックの報告がある。}
	}
	{
	{拡張子の前部分の正規表現が正しい}
	{拡張子の部分の正規表現が正しい}
	{{\bfseries\textdollar}の使い方を理解している。}
	{マッチする場合としない場合のチェックの報告がある。}
	}
	{
	{拡張子の前部分の正規表現が間違っている。}
	{拡張子の部分の正規表現が間違っている。}
	{{\bfseries\textdollar}の使い方を理解していない。}
	{マッチする場合としない場合のチェックの報告がない。}
	}
	{\RProbNoM{2}{0.6}}{5}
	{
	{一つの文で書かれていて正しく動作する。}
	{考察が正しい。}
	}
	{
	{\texttt{split}メソッドまたは\texttt{match}メソッドを用いている。}
	{正規表現を利用している。}
	{分割代入を利用している。}
	{考察が少し足りない。}
	}
	{
	{\texttt{split}メソッドまたは\texttt{match}メソッドを用いていない。}
	{正規表現を利用していない。}
	{分割代入を利用していない。}
	{考察がないか、足りない。}
	}
	{\RProbNoMM{3.1~3.4}{0.95}{0.1}}{4}
	{
	{結果がすべて正しい。}
  {考察が適切である。}
	}
	{
	{\texttt{g}フラグを理解している。}
  {貪欲なマッチと非貪欲なマッチの区別を理解している。}
	}
	{
	{\texttt{g}フラグを理解していない。}
  {貪欲なマッチと非貪欲なマッチの区別を理解していない。}
	}
	{\RProbNoMM{3.5~3.8}{0.95}{0.1}}{4}
	{
	{結果がすべて正しい。}
  {考察が適切である。}
	}
	{
	{\texttt{g}フラグ有無で結果が異なることを正しく理解している。}
  {貪欲なマッチと非貪欲なマッチの区別を正しく考察している。}
  {前方参照を理解している。}
  {戻り値がどのように決まるのかを正しく考察している。}
	}
	{
	{\texttt{g}フラグ有無で結果が異なることを考察していない。}
  {貪欲なマッチと非貪欲なマッチの区別を正しく考察していない。}
  {前方参照を理解していない。}
  {戻り値がどのように決まるのかを考察していないか、間違って理解している。}
	}
	{\RProbNoMM{3.9~3.12}{0.95}{-0.3}}{6}
	{
	{結果がすべて正しい。}
  {考察が適切である。}
	}
	{
	{\texttt{g}フラグ有無で結果が異なることを正しく理解している。}
  {貪欲なマッチと非貪欲なマッチの区別を正しく考察している。}
  {前方参照を理解している。}
  {戻り値がどのように決まるのかを正しく考察している。}
	}
	{
	{\texttt{g}フラグ有無で結果が異なることを考察していない。}
  {貪欲なマッチと非貪欲なマッチの区別を正しく考察していない。}
  {前方参照を理解していない。}
  {戻り値がどのように決まるのかを考察していないか、間違って理解している。}
	}
	{問題4}{10}
	{
	{年、月、日の入力のチェックの第1段階として正規表現を用いて行っている。}
  {考察が適切である。}
	}
	{
	{年の入力のチェックの第1段階として正規表現を用いて行っている。}
	{月の入力のチェックの第1段階として正規表現を用いて行っている。}
	{日の入力のチェックの第1段階として正規表現を用いて行っている。}
  {考察が少し足りない。}
	}
	{
	{年の入力のチェックを正規表現を用いて行っていない。}
	{月の入力のチェックを正規表現を用いて行っていない。}
	{日の入力のチェックを正規表現を用いて行っていない。}
  {考察が足りないか全くない。}
	}
}