\ProblemNN{
\input 06-01makeRegEx.tex
\input 06-01getDate.tex
\Newpage
\input 06-02checkRegEx.tex
\input 06-03checkPromptValue.tex
}
\RubricC{
復習の目的は次の項目を理解することである。
\begin{itemize}
 \item 簡単な正規表現を作成できる。
 \item 正規表現を用いて文字列処理を簡単にする方法を理解する。
 \item 正規表現における文字列のマッチを理解する。
 \item 正規表現を用いて文字列が与えられた条件を満たすかどうかをチェック
			 する。
\end{itemize}
}
{
{\RProbNoM{1.1}{0.2}}{3}
	{
	{C言語の変数名の命名規則の正規表現が正しい}
	}
	{
	{C言語の変数名の命名規則の先頭文字の範囲が正しい}
	{C言語の変数名の命名規則の2文字目以降の範囲が正しい}
	{C言語の変数名の命名規則の長さの指定が正しい}
	}
	{
	{C言語の変数名の命名規則の先頭文字の範囲に間違いがある}
	{C言語の変数名の命名規則の2文字目以降の範囲に間違いがある}
	{C言語の変数名の命名規則の長さの指定に間違いがある}
	}
	{\RProbNoM{1.2}{0.2}}{4}
	{
	{符号付小数を表す正規表現が正しい}
	{十分なチェックの報告がある。}
	}
	{
	{符号が必ず必要になっているところが間違っている。}
	{整数部分の正規表現の部分が正しい。}
	{小数点のエスケープをしている。}
	{小数点が必ず必要になっているので間違っている。}
	{小数部分の正規表現が0文字以上で正しい。}
	{\texttt{*}と\texttt{?}の区別ができている。}
	{マッチする場合のチェックの報告がある。}
	{マッチしない場合のチェックの報告がある。}
	}
	{
	{符号の部分が間違っている。}
	{整数部分の正規表現の部分が間違っている。}
	{小数点のエスケープをしていない。}
	{小数部分の正規表現が間違っている。}
	{\texttt{*}と\texttt{?}の区別ができていない。}
	{マッチする場合のチェックの報告がない。}
	{マッチしない場合のチェックの報告がない。}
	}
	{\RProbNoM{1.3}{0.2}}{3}
	{
	{仮数部の部分が正しい。}
	{指数部の部分が正しい。}
	{十分なチェックの報告がある。}
	}
	{
	{仮数部の符号の部分が正しい。}
	{仮数部の小数点の部分が正しい。}
	{仮数部の小数部の部分が正しい。}
	{指数部の正規表現が正しい。}
	{指数部の符号が必ず必要になっている点が間違っている}
	{指数部が必ず必要になっている点が間違っている。}
	{マッチする場合のチェックの報告がある。}
	{マッチしない場合のチェックの報告がある。}
	}
	{
	{仮数部の符号の部分が間違っている。}
	{仮数部の小数点の部分が間違っている。}
	{仮数部の小数部の部分が間違っている正しい。}
	{指数部の正規表現が間違っている。}
	{\texttt{*}と\texttt{?}の区別ができていない。}
	{マッチする場合のチェックの報告がない。}
	{マッチしない場合のチェックの報告がない。}
	}
	{\RProbNoM{1.4}{0.2}}{3}
	{
	{時間の部分の正規表現が正しい。}
	{分と秒の部分の正規表現が正しい。}
	{十分なチェックの報告がある。}
	}
	{
	{時間の部分の正規表現が正しい。}
	{分と秒の部分の正規表現が正しい。}
	{マッチする場合のチェックの報告がある。}
	{マッチしない場合のチェックの報告がある。}
	}
	{
	{時、分、秒が2桁の整数すべてにマッチする。}
	{時が24,25などにマッチする。}
	{分、秒が60、61などにマッチする。}
	{マッチする場合のチェックの報告がない。}
	{マッチしない場合のチェックの報告がない。}
	}
	{\RProbNoM{1.5}{0.2}}{3}
	{
	{拡張子の前部分の正規表現が正しい。}
	{拡張子の部分の正規表現が正しい。}
	{十分なチェックの報告がある。}
	}
	{
	{拡張子の前部分の正規表現が正しい}
	{拡張子の部分の正規表現が正しい}
	{\texttt{\textdollar}の使い方を理解している。}
	{マッチする場合のチェックの報告がある。}
	{マッチしない場合のチェックの報告がある。}
	{}
	}
	{
	{拡張子の前部分の正規表現が間違っている。}
	{拡張子の部分の正規表現が間違っている。}
	{\texttt{textdollar}の使い方を理解していない。}
	{マッチする場合のチェックの報告がない。}
	{マッチしない場合のチェックの報告がない。}
	}
	{\RProbNoM{2}{0.6}}{5}
	{
	{一つの文で書かれていて正しく動作する。}
	{考察が正しい。}
	}
	{
	{\texttt{split}メソッドを用いて書いている。}
	{正規表現を利用している。}
	{分割代入を利用している。}
	{考察が少し足りない。}
	}
	{
	{\texttt{split}メソッドを用いていない。}
	{正規表現を利用していない。}
	{分割代入を利用していない。}
	{考察がないか、足りない。}
	}
	{\RProbNoMM{3.1~3.4}{0.95}{0.1}}{4}
	{
	{}
	}
	{
	{}
	}
	{
	}
	{\RProbNoMM{3.5~3.8}{0.95}{0.1}}{4}
	{
	{}
	}
	{
	{}
	}
	{
	}
	{\RProbNoMM{3.9~3.12}{0.95}{-0.3}}{6}
	{
	{}
	}
	{
	{}
	}
	{
	}
	{問題4}{10}
	{
	{}
	}
	{
	{}
	}
	{
	}
}