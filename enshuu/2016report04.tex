\documentclass[a4j]{jarticle}
\title{プログラム開発レポート第1回}
\author{}
\date{}
\newtheorem{Prob}{}
\begin{document}
\maketitle
次の課題をレポートで提出のこと。

{\bfseries 締め切りは10月21日授業時間まで}
\begin{Prob}\upshape
 コンソールに次のような九九の表を出力するJavaScriptのプログラムを作成し
 なさい。
 \input 04res
\end{Prob}
\begin{Prob}
 \upshape
\verb+window+ オブジェクトにはどのようなプロパティがあるか調べよ。2つ以
 上のブラウザで実行し、比較すること(課題4.1)。

 なお、ブラウザで開くページは次のような何も表示しないページで行うことが望ましい。
\begin{verbatim}
<!DOCTYPE HTML>
  <html>
    <head></head>
  <body></body>
  </html>
\end{verbatim}
 {\bfseries 考察を必ずつけること。}
\end{Prob}
\begin{Prob}
 \upshape
 4.2.3項の\texttt{Person2}にオブジェクト属性を付け加えて、
 \texttt{ID}プロパティが外部から変更できないように直しなさい(課題4.4)。
\end{Prob}
\end{document}