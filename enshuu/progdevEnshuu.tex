%-*- coding: utf-8 -*-
\documentclass[a4j,12pt]{jarticle}
\newif\ifHMSol
\newcommand{\LecName}{ソフトウェア開発}
\input enshuuhead.tex
\input softdevhead.tex
\HMSolfalse%%%
%\input automatonrev.tex
\begin{document}
\pagestyle{enshuu}
\iffalse
\ProblemREP{
\begin{Prob}\upshape
 コンソールに次のような九九の表を出力するJavaScriptのプログラムを作成し
 なさい。
 \input 04res
\end{Prob}
%
\begin{Prob}
 \upshape
\verb+window+ オブジェクトにはどのようなプロパティがあるか調べよ。2つ以
 上のブラウザで実行し、比較すること。(課題4.1)
 {\bfseries 考察を必ずつけること。}
\end{Prob}
}
\fi
%\end{document}
\Problem{
\begin{Prob}
\upshape
次の文字列にマッチする正規表現を作れ。(細かい仕様は配布資料課題6.1を見よ)。
\begin{enumerate}
 \item C言語の変数名の命名規則に合う文字列\vspace{\baselineskip}
 \item 符号付小数。符号はなくてもよい。小数点はなくてもよい。また、小数
       点はあっても小数部はなくてもよい。整数部分には数字が少なくても一
       つはあること。\vspace{\baselineskip}
 \item 前問の正規表現を拡張して、指数部が付いた浮動小数にマッチするもの
       を作れ。\vspace{\baselineskip}
 \item 24時間生の時刻の表し方。時、分、秒はすべて2桁とし、それらの区切り
       は\texttt{:}とする。
\vspace{\baselineskip}
\end{enumerate}
\end{Prob}
\Newpage
\begin{Prob}
次の実行結果がどうなるか答えよ。
\begin{enumerate}\upshape
 \item \texttt{"aaaabaaabb".match(/.*b/);}\vspace{\baselineskip}
%["aaaabaaabb"]
 \item \texttt{"aaaabaaabb".match(/.*b/g);}\vspace{\baselineskip}
%["aaaabaaabb"]
 \item \texttt{"aaaabaaabb".match(/.*?b/);}\vspace{\baselineskip}
%["aaaab"]
 \item \texttt{"aaaabaaabb".match(/.*?b/g);}\vspace{\baselineskip}
%["aaaab", "aaab", "b"]
 \item \texttt{"aaaabaaabb".match(/.*?b\textbackslash B/);}\vspace{\baselineskip}
%["aaaab"]
 \item \texttt{"aaaabaaabb".match(/.*?b\textbackslash B/g);}\vspace{\baselineskip}
%["aaaab", "aaab"]
 \item \texttt{"aaaabaaabb".match(/.*(?=b)/);}\vspace{\baselineskip}
%["aaaabaaab"]\vspace{\baselineskip}
 \item \texttt{"aaaabaaabb".match(/.*?(?=b)/);}
\vspace{\baselineskip}
%["aaaa"]
 \item \texttt{"aaaabaaabb".match(/.*?(?=b)/g);}
\vspace{\baselineskip}
%\Newpage
%["aaaa", "", "aaa", "", ""]
%"aaaabaaabb".match(/.*?(?!b)/);
%[""]
%"aaaabaaabb".match(/.*?(?!b)/g);
%["", "", "", "", "b", "", "", "", "bb", ""]
 \item \texttt{"abccbcckkccaaMMaacc".match(/((.)\textbackslash2).*\textbackslash1/);}\vspace{\baselineskip}
%["ccbcckkccaaMMaacc", "cc", "c"]
 \item \texttt{"abccbcckkccaaMMaacc".match(/((.)\textbackslash2).*\textbackslash1/g);}\vspace{\baselineskip}
%["ccbcckkccaaMMaacc"]
 \item \texttt{"abccbcckkccaaMMaacc".match(/((.)\textbackslash2).*?\textbackslash1/);}\vspace{\baselineskip}
%["ccbcc", "cc", "c"]
 \item \texttt{"abccbcckkccaaMMaacc".match(/((.)\textbackslash2).*?\textbackslash1/g);}\vspace{\baselineskip}
%["ccbcc", "ccaaMMaacc"]
 \item \texttt{"abccbcckkccaaMMaa".match(/((.)\textbackslash2).*\textbackslash1/);}\vspace{\baselineskip}
%["ccbcckkcc", "cc", "c"]
 \item \texttt{"abccbcckkccaaMMaa".match(/((.)\textbackslash2).*\textbackslash1/g);}\vspace{\baselineskip}
%["ccbcckkcc", "aaMMaa"]
 \item \texttt{"abccbcckkccaaMMccaa".match(/((.)\textbackslash2).*\textbackslash1/g);}\vspace{\baselineskip}
%["ccbcckkccaaMMcc"]
 \item \texttt{"abccbcckkccaaMMccaa".match(/((.)\textbackslash2).*?\textbackslash1/g);}\vspace{\baselineskip}
%["ccbcc", "ccaaMMcc"]
% \item \texttt{+
%
\end{enumerate}
\end{Prob}
}
\end{document}
\Problem{
\begin{Problem}
 課題8.1、8.2を行いなさい。\vspace*{5cm}
\end{Problem}
}
