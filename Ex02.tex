\ProblemNN{
\input prob02-01string.tex
\input prob02-02asign.tex
\input prob02-03.tex
\input prob02-04Date.tex
\Texttrue
\begin{Prob}\upshape
\input prob02-05Check.tex
\end{Prob}
}
\Rubric{復習の目的は次のとおりである。
\begin{itemize}
 \item 文字列ののメソッドを用いた文字列の取り扱いに慣れる。
 \item 分割代入を理解する。
 \item 配列のプロパティとメソッドの利用法を理解する。
 \item \ElmJ{Date}オブジェクトの利用方法を学ぶ。
 \item JavaScriptのプログラミングで他の言語と異なる点を理解する。できれ
       ば自分で整理することが望ましい。
\end{itemize}
}
{{問題1}{6}{
  {与えられた課題の解答がすべて正しい。}
	}
	{
	{\ElmJ{indexOf()}メソッドの解答がすべて正しい。}
	{\ElmJ{split()}メソッドの解答がすべて正しい。}
	{\ElmJ{substring()}メソッドの解答がすべて正しい。}
	{\ElmJ{slice()}メソッドの解答がすべて正しい。}
	}
	{
	{\ElmJ{indexOf()}メソッドの解答に間違いがかなりある。}
	{\ElmJ{split()}メソッドの解答に間違いがかなりあるが。}
	{\ElmJ{substring()}メソッドの解答に間違いがかなりある。}
	{\ElmJ{slice()}メソッドの解答に間違いがかなりある。}
	}
  {問題2}{4}{
	{分割代入の解答がすべて正しい。また、適切な考察がある。}
  }
  {
	{分割代入の1の解答が正しい。}
	{分割代入の2の解答が正しい。}
	{分割代入の2に関する考察がある。}
  }
  {
	{分割代入の1の解答が間違っている。}
	{分割代入の2の解答が間違っている。}
	{分割代入の2に関する考察がない。}
  }
  {問題3}{5}{
  {配列のメソッドに関する解答がすべて正しい。}
  }
  {
  {配列のメソッド\ElmJ{length}の解答が正しい。}
  {配列のメソッド\ElmJ{pop}の解答が正しい。}
  {配列のメソッド\ElmJ{push}の解答が正しい。}
  {配列のメソッド\ElmJ{shift}の解答が正しい。}
  {配列のメソッド\ElmJ{pop}、\ElmJ{push}、\ElmJ{shift}の利用法について考察がある。}
  {配列のメソッド\ElmJ{join}の解答が正しい。}
  {配列のメソッド\ElmJ{slice}の解答が正しい。}
  {配列のメソッド\ElmJ{splice}の解答が正しい。}
  {配列のメソッド\ElmJ{slice}と\ElmJ{splice}の違いについて考察がある。}
  {配列のメソッド\ElmJ{indexOf}の解答が正しい。}
  {配列のメソッド\ElmJ{lastIndexOf}の解答が正しい。}
  {配列のメソッド\ElmJ{indexOf}と\ElmJ{lastIndexOf}の違いについて考察がある。}
  }
  {
  {配列のメソッド\ElmJ{length}の解答が間違っている。}
  {配列のメソッド\ElmJ{pop}の解答が間違っている。}
  {配列のメソッド\ElmJ{push}の解答が間違っている。}
  {配列のメソッド\ElmJ{shift}の解答が間違っている。}
  {配列のメソッド\ElmJ{pop}、\ElmJ{push}、\ElmJ{shift}の利用法について考
  察がない。}
  {配列のメソッド\ElmJ{join}の解答が間違っている。}
  {配列のメソッド\ElmJ{slice}の解答が間違っている。}
  {配列のメソッド\ElmJ{splice}の解答が間違っている。}
  {配列のメソッド\ElmJ{slice}と\ElmJ{splice}の違いについて考察がない。}
  {配列のメソッド\ElmJ{indexOf}の解答が間違っている。}
  {配列のメソッド\ElmJ{lastIndexOf}の解答が間違っている。}
  {配列のメソッド\ElmJ{indexOf}と\ElmJ{lastIndexOf}の違いについて考察が
  ない。}
  }
	{問題3}{8}
	{
  {解答がすべて目的にかなっている。}
  {コードの質すべて良い。}
  {充分なデバッグを行った報告がある。}
  }
	{
  {1週間後の日時のコードが正しく動作する。}
  {1週間後の日時のコードに改良点がある。}
  {翌月の1日のコードが正しく動作する。}
  {翌月の1日ののコードに改良点がある。}
  {前の月の最終日のコードが正しく動作する。}
  {前の月の最終日のコードに改良点がある。}
  {月の第1日曜日のコードが正しく動作する。}
  {月の第1日曜日のコードに改良点がある。}
	{\ElmJ{new}を用いて新しい\ElmJ{Date}オブジェクトを作成していない。}
  {動作確認の報告がない。}
	}
	{
  {1週間後の日時のコードが正しく動作しない。}
  {翌月の1日のコードが正しく動作しない。}
  {前の月の最終日のコードが正しく動作しない。}
  {月の第1日曜日のコードが正しく動作しない。}
	}
	{問題4}{12}
	{
	{ほとんどすべての結果が正しい。}
	{ほとんどすべての項目の説明が正しい。}
  }
	{
	{文字列に対する\texttt{+}演算子の結果が正しい。}
	{文字列に対する\texttt{+}演算子の結果の理由が正しい。}
	{組み込み関数の\texttt{parseInt()}と\texttt{parseFloat()}の結果が正しい。}
	{組み込み関数の\texttt{parseInt()}と\texttt{parseFloat()}の結果の理由
	が正しい。}
	{文字列に対する\texttt{*}や\texttt{/}の演算子の結果が正しい。}
	{文字列に対する\texttt{*}や\texttt{/}の演算子の結果の理由が正しい。}
	{配列の\ElmJ{length}プロパティの結果が正しい。}
	{配列の\ElmJ{length}プロパティの結果の理由が正しい。}
	{\ElmJ{==}演算子の結果が正しい。}
	{\ElmJ{==}演算子の結果の理由が正しい。}
	{配列に対する\ElmJ{==}演算子の結果が正しい。}
	{配列に対する\ElmJ{==}演算子の結果の理由が正しい。}
	}
	{
	{文字列に対する\texttt{+}演算子の結果に間違いが多い。}
	{文字列に対する\texttt{+}演算子の結果の理由に間違いが多い。}
	{組み込み関数の\texttt{parseInt()}と\texttt{parseFloat()}の結果に間違いが多い。}
	{組み込み関数の\texttt{parseInt()}と\texttt{parseFloat()}の結果の理由
	に間違いが多い。}
	{文字列に対する\texttt{*}や\texttt{/}の演算子の結果に間違いが多い。}
	{文字列に対する\texttt{*}や\texttt{/}の演算子の結果の正しい。}
	{配列の\ElmJ{length}プロパティの結果に間違いが多い。}
	{配列の\ElmJ{length}プロパティの結果の理由に間違いが多い。}
	{\ElmJ{==}演算子の結果に間違いが多い。}
	{\ElmJ{==}演算子の結果の理由に間違いが多い。}
	{配列に対する\ElmJ{==}演算子の結果が正しい。}
	{配列に対する\ElmJ{==}演算子の結果の理由が正しい。}
	}
}
