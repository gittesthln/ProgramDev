%-*- coding: utf-8 -*-
\chapter{オブジェクトの詳細}
\section{\protect\texttt{class}の実体}
\ES ではキーワード\ElmJ{class}が導入されたが、その実体はこれまでの
ECMASript に新しい機能を追加したわけではなく、従来のオブジェクトや継承に
関して記述を簡単にしたものに過ぎない\footnote{このことを syntactic
sugar(糖衣構文)と呼ぶ。}。実行例\ref{ExecconstructorClass}に
おいて\texttt{typeof Person}の結果は次のとおりである。
\begin{Verbatim}
>typeof Person;
"function"
\end{Verbatim}
従来の EcmaSCript では関数を用いてクラスが定義されていて、それの手法が使
われていることを示す。
\begin{Exec}\label{constructor}\upshape
次の例はコンストラクタ関数を用いて、前の例と同じオブジェクト(インスタン
 ス)を構成している。

 実行例\ref{ExecconstructorClass}の例で確認する。
\begin{Verbatim}
function PersonF(name, year, month, day, hometown = "神奈川"){
    this.name = name;
    this.birthday = {
      year : year,
      month : month,
      day : day
    };
    this["hometown"] = hometown;
}
\end{Verbatim}
 この定義では\ElmJ{constructor}の内容をそのまま関数にしているに過ぎない。
 なお、ここではメソッドなどが定義されていないが、それについては後で解説
 をする。
 \end{Exec}
\begin{Verbatim}
>p = new PersonF("foo",2001,4,1);
PersonF {name: "foo", birthday: {…}, hometown: "神奈川"}
\end{Verbatim}
\section{オブジェクト属性}
JavaScriptの関数オブジェクトには\ElmJ{prototype}、\ElmJ{class}と
\ElmJ{extensible}という3つの属性がある。\ElmJ{prototype}属性はオブジェ
クトの継承に関係するので、最後に説明をする。
\subsection{\protect\ElmJ{class}属性}
オブジェクトの\ElmJ{class}属性はオブジェクトの型情報を表す文字列である。
最新の \JS でもこの属性を設定する方法はない。直接値を取得する方
法もない。クラス属性は間接的にしか得られない。
組み込みのコンストラクタで生成されたオブジェクトではそのクラス名が間接的
に得られるが、独自のコンストラクタ関数では、\texttt{"Object"}しか得られ
ない。
\begin{Exec}\upshape\label{}
\ElmJ{Object}から継承した\ElmJ{toString()}を直接呼び出すと次のような結果
になる。
\begin{Verbatim}
>p = new PersonF("foo",2001,4,1);
PersonF {name: "foo", birthday: {…}, hometown: "神奈川"}
>`${p}`;
"[object Object]"
\end{Verbatim}
ここで、\texttt{Person()}コンストラクタには\ElmJ{toString()}メソッドが定
義されていないので、もともとの\ElmJ{Object}で定義されている
 \texttt{toString()}が呼び出されている。
\end{Exec}

\subsection{\protect\ElmJ{extensible}属性}
\ElmJ{extensible}はオブジェクトに対してプロパティの追加ができるか
どうかを指定する。\JS ではこの属性の取得や設定ができる関数が用意され
ている。

この属性の取得は\ElmJ{Object.isExtensible()}に調べたいオブジェクトを引数
にして渡す。オブジェクトのプロパティを拡張できなくするためには
\ElmJ{Object.preventExtension()}に引数として設定したいオブジェクトを渡す。
 \begin{Exec}\upshape\label{ExecconstructorExtensible}
  実行例\ref{Execconstructor}の例で確認する。
\begin{Verbatim}
>p = new PersonF("foo",2001,4,1);
PersonF {name: "foo", birthday: {…}, hometown: "神奈川"}
PersonF
>p.mother = "Alice";
Alice
>p.grandmother = "Old Alice";
Old Alice
>Object.preventExtensions(p);
PersonF
>p.father = "Bob";
Bob
>p.father;
undefined
>delete p.mother;
true
>p.mother;
undefined
\end{Verbatim}
\begin{itemize}
 \item \texttt{Object.preventExtensions(p)}を実行する前では存在しな
       い属性の追加ができている(\texttt{p.mother})。
 \item \texttt{Object.preventExtensions(p)}を実行後は、新しい属性が定義
			 できていない(\texttt{p.father})。
 \item \texttt{Object.preventExtensions(p)}では属性の削除までは禁止でき
       ない(\texttt{delet p.mother}が成功していて、
			 値を参照すると\ElmJ{undefined}が返る)。
\end{itemize}
属性の削除まで禁止したい場合には\ElmJ{Object.seal()}を用いる。一度この関
数を実行されたオブジェクトは解除できない。
	
引き続いて実行すると次のようになる。
\begin{Verbatim}
>Object.seal(p);
PersonF
>Object.isSealed(p);
true
>delete p.grandmother;
false
>p.grandmother;
Old Alice
>p.grandmother = "Very Old Alice";
Very Old Alice
>p.grandmother;;
Very Old Alice
\end{Verbatim}
\begin{itemize}
	 \item すでに\texttt{seal}された状態になっているかどうかは、
				 \ElmJ{Object.isSealed()}で調べられる。
 \item \texttt{seal}された状態ではプロパティを削除できない
			 (\texttt{delete p.grandmother}の結果が\ElmJ{false})。
 \item \texttt{seal}された状態ではプロパティの値を変えることができる
			 (\texttt{p.grandmother}の値の書き換えができている)。
\end{itemize}
最も強く、オブジェクトを拘束するためには、\ElmJ{Object.freeze()}を用いる。
また、この状態を確認するためには\ElmJ{Object.isFrozen()}を用いる。

   実行例\ref{ExecContined}に引き続いて\ElmJ{Object.freeze()}を実行した
   結果である
\begin{Verbatim}
>Object.freeze(p);
PersonF
>p.grandmother = "Very Very Old Alice";
Very Very Old Alice
>p.grandmother;
Very Old Alice
\end{Verbatim}
 \texttt{p.grandmother}の値が設定できていないことがわかる。
このような状態で属性値を変えたい場合には、属性にたいするセッターメソッド
  を定義することになる。

このオブジェクトではプロパティ\texttt{birthday}の値がオブジェクトになっ
ている。このオブジェクトは\ElmJ{seal}されていない。
\begin{Verbatim}
>p.birthday;
{year: 2001, month: 4, day: 1}
>delete p.birthday;
false
>p.birthday = {};
{}
>p.birthday;
{year: 2001, month: 4, day: 1}
>delete p.birthday.year;
true
>p.birthday;
{month: 4, day: 1}
\end{Verbatim}
  \begin{itemize}
   \item \texttt{p.birthday}は削除できない。また、値の変更もできない。
   \item 一方で、\texttt{p.birthday.year}は削除できる。
  \end{itemize}
 \end{Exec}
 \input 05-01extensibleClass.tex
 このように JavaScript におけるクラスとそれから生成されるインスタンスは
 通常のオブジェクト指向言語のように構造を固定化された状態で存在させるた
 めにはインスタンスごとに行う必要があることが分かる。
 \subsection{\protect\ElmJ{prototye}属性}
 オブジェクトの\ElmJ{prototype}属性の値は、同じコンストラクタ関数で生成
 された間で共通のものになっている。オブジェクトリテラルで生成されたオブ
 ジェクトは\ElmJ{Object.prototype}で参照できる。\ElmJ{new}を用いて生成さ
 れたオブジェクトはそのコンストラクタ関数の\ElmJ{prototype}を参照する。
 このコンストラクタ関数の\ElmJ{prototype}もオブジェクトであるから、それ
 の\ElmJ{prototype}も存在する。この一連の\ElmJ{prototype}オブジェクトを
 プロトタイプチェーンとよぶ。

 \JS で定義されている \ElmJ{Object.create()} メソッドは引数で与
 えられたオブジェクトの\ElmJ{prototype}を\ElmJ{prototype}に持つオブジェ
 クトを生成する。このメソッドに対して2番目の引数を与えて、新たに生成され
 たオブジェクトのプロパティを指定できる。
 \subsection{\protect\texttt{prototype}の使用例}
 \begin{Exec}\upshape\label{ExecConst2}
  次の例は\ref{Execconstructor}の属性を簡略し、いくつかのメソッ
 ドを追加したものである。
\begin{Verbatim}[numbers=left]
PersonF.prototype = {
  toString:function() {
    return `${this.name}さんは`+
      `${this.birthday.year}年${this.birthday.month}月${this.birthday.day}日に` +
   `${this.hometown}で生まれました。`;
  },
  get age(){
    let today = new Date();console.log(x);
    let age = today.getFullYear() - this.birthday.year;
    if(today.getTime() <
         new Date(today.getFullYear(),
                  this.birthday.month-1,
                  this.birthday.day).getTime()) age--;
    return age;
  },
  constructor: PersonF
}
\end{Verbatim}
 \end{Exec}
これらの定義は前と同じである。オブジェクトリテラルの形式で
\ElmJ{prototype}に代入している。異なるところは\ElmJ{constructor}プロパティ
に自分自身を定義しているところである。

なお、\ElmJ{get}キーワードの代わりに\ElmJ{set}キーワードを用いるとセッター
を定義できる。
\input 05-02checkPrototype.tex
\section{関数によるオブジェクトの継承}
\JS では\ElmJ{prototype}を用
いることでメソッドの継承が可能となっている。
\begin{Exec}\upshape\label{Execconst4}
次のリストは実行例\ref{Execconst3}の\texttt{Person}を継承して、学籍番号
 を追加のプロパティとする\texttt{Student}オブジェクトのコンストラクタ関
 数である。
\begin{Verbatim}[numbers=left]
const StudentF = (function(){
  let id = 10000;
  return function(name, year, month, day, hometown = "神奈川") {
    this.name = name
    this.birthday = {
      year : year,
      month : month,
      day : day
    };
    this["hometown"] = hometown;
    this.id = id++;
    }
  })();
StudentF.prototype = new PersonF();
StudentF.prototype.constructor = StudentF;
\end{Verbatim}
\end{Exec}
\begin{itemize}
 \item 関数を用いた継承では\ElmJ{super}が使えないので、継承元のクラスで
       定義されたプロパティ(\texttt{name}など)は継承先で定義する必要があ
       る(4行目から10行目)。
 \item 一方、メソッドはインスタンスで共有されるので継承元で定義をする必
       要はないが、参照できるようにする必要がある。そのために14行目で継
       承元のクラスを作成して継承するクラスの\ElmJ{prototype}を書き直し
       ている(14行目)。
 \item このままでは\ElmJ{constructor}プロパティが\texttt{Person}になって
       しまうので、\texttt{Student.prototype.constructor}を
       \texttt{Student}に戻して\texttt{Student.prototype.constructor}を
       継承するクラスに置き換える(15行目)
\end{itemize}

 \section{エラーオブジェクトについて}
エラーオブジェクトとはエラーが発生したことを知らせるオブジェクトである。
通常は計算の継続ができなくなったときにエラーオブジェクトをシステムに送る
操作が必要である。これを通常、エラーを投げる(\ElmJ{throw}する)という。
エラーオブジェクトには表\ref{ErrorProp}のようなプロパティがある。
\begin{table}
 \caption{エラーオブジェクトのプロパティ}\label{ErrorProp}
 \begin{center}
	 \begin{tabular}{|c|m{30zw}|}\hline
		プロパティ&\multicolumn{1}{c|}{説明}\\ \hline
		\texttt{message}&エラーに関する詳細なメッセージ。コンストラクタで渡
				された文字列か、デフォルトの文字列\\ \hline
		\texttt{name}&エラーの名前。エラーを作成したコンストラクタ名になる\\ \hline

	\end{tabular}
 \end{center}
\end{table}
\subsection{エラー処理の例}
エラー処理の例としてインスタンス作成時に引数の妥当性をチェックする例を挙げる。
 \begin{Exec}\upshape\label{throwError}
	次のリストは、実行例\ref{Execconst3}において、コンストラクタに与えられ
	た引数をチェックして不正な値の場合にはエラーを投げるように書き直したも
	のである。

	なお、このリストでは\texttt{Person}の\texttt{toString()}と\texttt{age}、
	継承するクラス\texttt{Student}の内容が以前のものと
	同じなので \texttt{...} で省略している。
 \end{Exec}
\begin{Verbatim}[numbers=left]
class Person{
  static checkName(name) {
    if(name === "") throw new Error("名前がありません");
    return name;
  }
  static checkDate(y, m, d) {
    if(m<1 || m>12) throw new Error("月が不正です");
    let date = new Date(y,m,0);
    if(d<1 || d>date.getDate()) throw new Error("日が不正です");
    return {year: y, month: m, day: d};
  }
  constructor(name, year, month, day, hometown="神奈川"){
    this.name = Person.checkName(name);
    this.birthday = Person.checkDate(year, month, day);
    this["hometown"] = hometown;
  }
  toString() {
  ...
  }
  get age(){
    ...
  }
}
const Student = (function(){
  ...
})();
\end{Verbatim}
\begin{itemize}
 \item 2行目から5行目で、\texttt{name} が空文字であればエラーを発生させ、
       正しければ値をそのまま返すクラスメソッドを定義している。
 \item 6行目から11行目で正しい日付でないとエラーを発生させ、正しいときは
       日付のオブジェクトを返すクラスメソッドを定義している。
       \begin{itemize}
        \item 7行目は月の値の範囲をチェックしている。
 \item 8行目では、与えられた年と月からその月の最終の日を求めている。
			 \ElmJ{Date.getMonth()}の戻り値が0(1月)から11(12月)になっているの
			 で、\texttt{new Date(y,m,0)}により翌月の1日の1日前、つまり、問題
			 としている月の最終日が設定できる\footnote{課題\ref{DateProb}
			 の3番目の問題の解答である。}。
 \item 9行目で与えられた範囲に日が含まれていなければエラーを発生させてい
			 る。
        \item 10行目fで日付のオブジェクトを作成し、戻り値として返している。
       \end{itemize}
 \item 13行目と14行目で、名前と日付を指定したプロパティにクラスメソッド
       からの戻り値で設定している。
% \item 残りの部分は前のリストと同じである。
\end{itemize}
これをいくつかのデータで実行した結果は次のようになる。
\begin{itemize}
 \item 通常の日時ならば問題なく、オブジェクトが構成される。
\begin{Verbatim}
>p = new Person("foo",1995,4,1);
Person {name: "foo", year: 1995, month: 4, day: 1}
\end{Verbatim}
 \item 1996年はうるう年なので2月29日が存在する。したがって、エラーは起こ
			 らず正しくオブジェクトが作成できる。
\begin{Verbatim}
>p = new Person("foo",1996,2,29);
Person {name: "foo", year: 1996, month: 2, day: 29}
\end{Verbatim}
 \item 1995年はうるう年ではないので2月29日がない。したがって、エラーが起
			 きる。
\begin{Verbatim}
>p = new Person("foo",1995,2,29);
 Uncaught Error: 日が不正です(…)
\end{Verbatim}
 \item 不正な月や日では当然、エラーが起こる。
\begin{Verbatim}
>p = new Person("foo",1995,13,29);
 Uncaught Error: 月が不正です(…)
>p = new Person("foo",1995,12,0);
 Uncaught Error: 日が不正です(…)
\end{Verbatim}
\end{itemize}
\input 05-03errorCheck.tex
\input 05-04errorCheckMoreStrict.tex
\subsection{エラーからの復帰}
前節の例ではエラーが発生するとそこでプログラムの実行が止まってしまう。エ
ラーが発生したときに、投げられた(\ElmJ{throw}された)エラーを捕まえる
(\ElmJ{catch}する)ことが必要である。このためには\ElmJ{try\{...\}catch\{...\}}構文
を使用する。

\begin{itemize}
 \item \ElmJ{try}のブロック内にエラーが発生する可能性があるコードを記述
       する。
 \item エラーが発生したときは\ElmJ{throw}を用いてエラーを発生させる。
 \item エラーが発生すると\ElmJ{catch(e)}ブロックが実行される。
       \ElmJ{catch}の後にある\texttt{e}は投げられたエラーオブジェクトが
       渡される。この変数のスコープは\ElmJ{catch}内である。
 \item \ElmJ{finally\{\}}を付けることも
できる。\texttt{try\{...\}catch\{...\}}内の部分は\texttt{try}や
\texttt{catch}の処理がが実行後、必ず呼び出される。これは\ElmJ{try}や
       \ElmJ{catch}の部分が\ElmJ{return}文、\ElmJ{break}文、
       \ElmJ{continue}文、\ElmJ{return}文や新しい例外を投げたとしても呼
       び出される。
 \item \ElmJ{try\{...\}catch\{...\}}構文は入れ子にできる。投げられたエラー
			 に一番近い\texttt{catch}にエラーが捕まえられる。
\end{itemize}
 \begin{Exec}\upshape
	次のリストは実行例\ref{throwError}において
  \ElmJ{try\{...\}catch\{...\}}構文を用いてオブジェクトが正しくでき
	るまで繰り返すようにしたものである。
 なお、このリストはブラウザで実行することを想定している。
 \end{Exec}
\begin{Verbatim}[numbers=left]
function test() {
  for(;;) {
    try {
      let y = Number(prompt("生まれた年を西暦で入力してください"));
      console.log(`年:${y}`)
      let m = Number(prompt("生まれた月を入力してください"));
      console.log(`月:${m}`)
      let d = Number(prompt("生まれた日を入力してください"));
      console.log(`日:${d}`)
      return new Person("foo", y, m, d);
    } catch(e) {
      console.log(e.name+":"+e.message);
    } finally{
      console.log("finnally");
    }
  }
}
\end{Verbatim}
\begin{itemize}
 \item テストを繰り返す関数\texttt{test()}が定義されている。
 \item 2行目では無限ループが定義されている。正しいパラメータが与えられた
			 ときに7行目で作成されたオブジェクトを戻り値にして関数の実行が終了
			 する。
 \item \ElmJ{try\{\}}内にはエラーが発生するかもしれないコードを中に含め
			 る。
	\begin{itemize}
	 \item ここでは年、月、日の入力を\ElmJ{prompt}用いてダイアログボックス
				 を表示させ、そこに入力させている。
	 \item 戻り値は文字列なので、\ElmJ{Number}で数に直している。
   \item 代入された値が正当であればその値をコンソールに出力している。
	\end{itemize}
 \item 与えられた入力が正しくなければエラーが投げられ、
			 \ElmJ{catch}\texttt{(e)}の中に制御が移る。
 \item \ElmJ{catch}\texttt{(e)}における\texttt{e}には発生したエラーオブ
			 ジェクトが渡されるので、コンソールにその情報を出力する(12行目)。
\end{itemize}
\begin{Verbatim}
>p = test();
 年:2010
 月:4
 日:31
 Error:日が不正です
 finnally
 年:2010
 月:4
 日:30
finnally
Person {name: "foo", birthday: {…}, hometown: "神奈川"}
\end{Verbatim}
\begin{itemize}
 \item 200年4月31日は不正な日付なので、9行目の\ElmJ{console.log()}の出力
       はない。
 \item その代り、エラーの内容が出力されている。
 \item また、エラー出力後は\ElmJ{finally}ブロックが実行されていることも
       分かる。
 \item 入力データが正しい場合でも\ElmJ{finally}ブロックが実行されている
       (下から2行目)。
\end{itemize}