 \begin{Prob}\upshape\Must\label{nth-child}
  \ifText 課題\ref{nth-child}の問に答えなさい。
  \else
 次のHTML文書において以下の問いに答えなさい。
 \LISTN{07-01nth-child.html}{1}{last}{\normalsize}
 ここで\texttt{<ol>}は箇条書きの開始を示す要素で
 あり、\texttt{<li>}は箇条書きの各項目を示す要素である。\fi
\begin{enumerate}
\item \texttt{nth-child}の\texttt{()}内に次の式を入れたとき、各項目の背
      景色がどうなるか報告しなさい。
\begin{enumerate}
 \item \texttt{n}(ここでのリストの設定)\AnsLine{0.9}
 \item \texttt{2n}\AnsLine{0.9}
 \item \texttt{n+3}\AnsLine{0.9}
 \item \texttt{-n+2}\AnsLine{0.9}
\end{enumerate}
 \item 背景色が次のようになるようにそれぞれCSSを設定しなさい。
\begin{enumerate}
 \item 偶数番目が黄色、奇数番目がオレンジ色\AnsS{0.3}
 \item 1番目、4番目、\dots のように$3$で割ったとき、$1$ 余る位置が明るい
       灰色\AnsS{0.3}
 \item 4番目以下がピンク\AnsS{0.3}
 \item 下から2番目以下が緑色\AnsS{0.3}
\end{enumerate}
\end{enumerate}
\end{Prob}
