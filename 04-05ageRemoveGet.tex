\begin{Prob}\upshape\Must
 実行例\ifText4.5\else\ref{PersonWidthGetter}\fi
 のリストにあるプロトタイプメソッド
 \texttt{age()}の前にある\ElmJ{get}を省略して通常のメソッドとして定義し
 ときの実行方法について報告せよ。
 \ifText また、\ElmJ{get}がある場合とない場合の違いを述べよ。\fi
\end{Prob}
\ifText %リストは印刷したものを貼り付けてもよい。
{\bfseries \ElmJ{get}がある場合とない場合の実行方法の違い}\\[1\baselineskip]
  \underline{\makebox[0.95\textwidth]{}}\\[1\baselineskip]
  \underline{\makebox[0.95\textwidth]{}}\\[0.3\baselineskip]
	{\bfseries \ElmJ{get}がある場合とない場合のメソッドの定義の違い}\\[1\baselineskip]
  \underline{\makebox[0.95\textwidth]{}}\\[1\baselineskip]
  \underline{\makebox[0.95\textwidth]{}}\\[1\baselineskip]
  \underline{\makebox[0.95\textwidth]{}}\\[1\baselineskip]
  \underline{\makebox[0.95\textwidth]{}}
	%\vspace{0.2\textheight}
	\fi
