 \begin{Prob}\upshape
  \ifText
  課題3.4の%
  \else
  次のリストは上記の例のコールバック関数を引数付にしたもの
    である。
\begin{Verbatim}
let T = new Date();
window.setTimeout(
  function callMe(L){
    let NT = new Date();
    if(NT.getTime()-T.getTime()<10000) {
      console.log(Math.floor((NT.getTime()-T.getTime())/1000));
      L +=1000;
      window.setTimeout(callMe,L,L);
    }
   },1000,1000);
\end{Verbatim}
\fi
  コンソールの出力結果と動作を確認しなさい。
  \ifText

  {\bfseries コンソールの出力結果}\\[0.1\textheight]
  {\bfseries 動作の違いの説明}\\[1\baselineskip]
  \underline{\makebox[0.95\textwidth]{}}\\[1\baselineskip]
  \underline{\makebox[0.95\textwidth]{}}
  \fi
\end{Prob}
