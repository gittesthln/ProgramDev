\section{クラス}
\subsection{クラスの宣言とインスタンスの生成}
前節で同じキーを持つオブジェクトを複数作成したい場合に同じようなコードを
繰り返すのは問題である。また、キーの追加をするときの修正の手間がかかりすぎてプ
ログラムのメンテナンスが面倒になる。

そこで、オブジェクトのひな形を作成し、それをもとにオブジェクトを構成する
ことが考えられる。このオブジェクトのひな形は通常クラスと呼ばれる。\ES で
はクラスの宣言ができる。クラスから作成されたオブジェクトはこのクラスのイ
ンスタンスと呼ばれる。
\begin{Exec}\upshape\label{ExecconstructorClass}
実行例\ref{Execconstructor}をクラスを用いて書き直すと次のようになる。
\begin{Verbatim}[numbers=left]
class Person{
  constructor(name, year, month, day, hometown="神奈川"){
    this.name = name;
    this.birthday = {
      year : year,
      month : month,
      day : day
    };
    this["hometown"] = hometown;
  }
\end{Verbatim}
\begin{itemize}
 \item 1行目がクラスの宣言である。キーワード\ElmJ{class}の後にクラス名
       \texttt{Person}を記述する。通常、クラス名は大文字で始まる。関数の
       時と同様にクラスの宣言を変数に代入することもできる。
 \item その後の\Verb+{+と\Verb+}+内にクラスの記述を行う。
 \item クラスの定義には初期化を行う\ElmJ{constructor()}が必須である。こ
       こでは5つの引数を取るコンストラクタが定義されている。5番目の引数
       はデフォルトの値が指定されている。
 \item キーワード\ElmJ{this}はクラスから作成されたインスタンスを指す。
 \item 3行目から9行目で、そのインスタンスのオブジェクトのキーはプロパティ
			 と呼ばれる。プロパティの名前は前のオブジェクトと同じである。
 \item 9行目の定義はコンストラクタの最後の引数はデフォルトの値が与えられ
			 ている。
% \item この関数を用いてオブジェクトを作成するためには、
%       \verb+new Person(...)+ とする。
 \item クラス内のコードは\Strict で実行される。
\begin{Verbatim}
>p = new Person("foo",1900,4,1);
Person {name: "foo", birthday: {…}, hometown: "神奈川"}
\end{Verbatim}
\end{itemize}

\end{Exec}
クラスの定義からインスタンスを作成するためにはキーワード
\ElmJ{new}をつけてクラスを呼び出す。上の例ではコンストラクタが引数を必要
としているので次のようになる。
\begin{Verbatim}
>p = new Person("foo",2001,8,18);
Person {name: "foo", birthday: {…}, hometown: "神奈川"}
>p.birthday
{year: 2001, month: 8, day: 18}
>p.hometown;
"神奈川"
\end{Verbatim}
コンストラクタの最後の引数にはデフォルト値が指定されているので、この実行例の
ように値が指定されていない場合にはデフォルト値に設定されていることが分か
る。
\input 04-03ChangeProperties.tex
\subsection{クラスメソッド}
クラス内では\ElmJ{constructor}のほかにメソッドと呼ばれる関数で定義された
プロパティがある。また、メソッドにはアクセッサプロパティと呼ばれるオブジェ
クトに値を渡すセッター、値を得るゲッターの2種類を指定することもできる。\ES では
ゲッターやセッターにはキーワード\ElmJ{get}と\ElmJ{set}を用いる。
\begin{Exec}\upshape\label{PersonWidthGetter}
次のリストは\texttt{Person}に文字列に変換する\ElmJ{toString()}と、実行時に
 おける年令を返すゲッター\texttt{age}を追加したものである。
\begin{Verbatim}[numbers=left]
class Person{
  constructor(name, year, month, day, hometown = "神奈川"){
    this.name = "foo";
    this.birthday = {
      year : year,
      month : month,
      day : day
    };
    this["hometown"] = hometown;
  }
  toString() {
    return `${this.name}さんは`+
      `${this.birthday.year}年${this.birthday.month}月${this.birthday.day}日に` +
     `${this.hometown}で生まれました。`;
  }
  get age() {
    let today = new Date();
    let age = today.getFullYear() - this.birthday.year;
    if(today.getTime() <
         new Date(today.getFullYear(),
                  this.birthday.month-1,
                  this.birthday.day).getTime()) age--;
    return age;
  }
}
\end{Verbatim}
 11行目から15行目でメソッド\ElmJ{toString()}が定義されている。

 一般に\ElmJ{toString()}メソッドはすべてのオブジェクトに定義されていて、
 文字列が必要な時に呼び出される。ここでは「~さんは~年~月~日に~
        で生まれました。」という表示を返す。

実行例は次のとおりである。文字列が必要なときに呼び出されることがわかる。
\begin{Verbatim}
p.toString();//toString()の明示的呼び出し
"fooさんは2001年4月1日に}神奈川で生まれました。"
`${p}`;      //暗黙のtoString()呼び出し
"fooさんは2001年4月1日に}神奈川で生まれました。"
\end{Verbatim}
 16行目から24行目でゲッター\texttt{age}が定義されている。
 \begin{itemize}
 \item 16行目でキーワード\ElmJ{get}をつけて、ゲッター\texttt{age()}を宣
       言している。メソッドは関数なので\texttt{()}をつける必要がある。た
       だし、ゲッターに仮引数をつけることはできない。
 \item 17行目でアクセス時の時間を変数\texttt{today}に保存している。
 \item 18行目でアクセス時の年から誕生日の年を引いている。
 \item 19行目から22行目で、今年の誕生日が過ぎているかどうかをチェックし
       ている。アクセス時の年と誕生日の月を日をもとに日付を作成し、その
       時間(\ElmJ{getTime()}を利用)を比較することでチェックができる。
 \item 誕生日前ならば18行目で求めた値を$1$減少させる。
 \end{itemize}
\end{Exec}
 実行例は次のとおりである。
\begin{Verbatim}
>p.age;
16
>p.age = 50;
50
>p.age;
16
\end{Verbatim}
\begin{itemize}
 \item 定義の方はにはメソッドであることを示す\texttt{()}があるが、利用す
       るときはプロパティと同じようになる。\texttt{()}を付けるとエラーに
       なる。
 \item 代入の式はエラーがなく実行できるが、ゲッターの機能は
       変わらない。代入はセッターの呼び出しが行われる。
\end{itemize}
\begin{Verbatim}
p.age(10);
VM87:1 Uncaught TypeError: p.age is not a function
    at <anonymous>:1:3
\end{Verbatim}
\input 04-04ageAtToday.tex
\input 04-05ageRemoveGet.tex
\input 04-06ageSetter.tex
\subsection{継承}
すでにあるクラスをもとに機能の追加や変更を加えた新しいクラスを作ることが
できる。新しいクラスはもとになるクラスを継承しているという。もとのクラス
は新しいクラスのスーパークラス、新しいクラスはもとになるクラスのサブクラ
スという。JavaScript では複数のクラスから同時に継承する多重継承はサポー
トされていない。
 \begin{Exec}\upshape\label{Inheritance}
  次の例は実行例\ref{PersonWidthGetter}のクラス\texttt{Person}を継承して
  新しいクラス\texttt{Student}を作成するものである。
\begin{Verbatim}[numbers=left]
class Student extends Person {
  constructor(name, id, year, month, day, hometown) {
    super(name, year, month, day, hometown);
    this.id = id;
  }
}
\end{Verbatim}
  \begin{itemize}
   \item クラスを継承するためにはクラス名の後に、キーワード
         \ElmJ{extends}を付けて継承するクラス名を書く(1行目)。
   \item \texttt{Student}クラスのコンストラクタには\texttt{Person}クラス
         で必要なパラメタのほかに\texttt{id}が付け加えている(2行目)。
   \item 親クラス(スーパークラス)のコンストラクタを呼び出すために
         \ElmJ{super()}を実行する(3行目)
   \item 4行目で追加のプロパティの設定を行っている。
  \end{itemize}
 \end{Exec}
 実行例は次のとおりである。
\begin{Verbatim}
>s = new Student("foo",1523999,2001,4,1);//Person のデフォルト引数値が設定されている
Student {name: "foo", birthday: {…}, hometown: "神奈川", id: 1523999}
> `${s}` //toString()もPersonで定義されたものが使われる
"fooさんは2001年4月1日に神奈川で生まれました。"
>s2 = new Student("foo",1523999,2001,4,1,"厚木");//デフォルト以外の設定もできる
Student {name: "foo", birthday: {…}, hometown: "厚木", id: 1523999}
>`${s2}`;
"fooさんは2001年4月1日に厚木で生まれました。"
\end{Verbatim}
\begin{Prob}\upshape
\texttt{Student}クラスにプロパティ\texttt{id}も表記するような
 \texttt{toString()}メソッドを定義すると、\texttt{Person}の
 \ElmJ{toString()}が上書きされることを確認しなさい。
\end{Prob}
\subsection{\protect\texttt{instanceof}演算子}
\texttt{instanceof}演算子はオブジェクトを生成したコンストラクタ関数が指
定されたものかを判定する。
\begin{Exec}\upshape
次の結果は Chrome で、実行例\ref{PersonWidthGetter}の例である。
\begin{Verbatim}
>p instanceof Person
true
>p instanceof Object;
true
>p instanceof Date;
false
\end{Verbatim}
\verb+Person+ オブジェクトが \verb+Object+ を継承している
 ので\texttt{p instanceof Object} が\texttt{true}となっている。
% 継承については後の授業で解説する。
\end{Exec}
\subsection{静的メソッド}
クラスに対して、インスタンスを作成しないで使用できる静的メソッドを定義で
きる。静的メソッドはキーワード\ElmJ{static}を付ける。インスタ
ンス化されたオブジェクトからは使用できない。

\ElmJ{Math}オブジェクトにある各関数が\ElmJ{Math}オブジェクトの静的メソッ
ドの例になっている。
%\subsection{同じクラスのインスタンスに通し番号をつける}

実行例\ref{Inheritance}のプロパティ\texttt{id}を重複がなくかつ連続な値に
自動で設定しようとするとクラスにおいて変数を管理する変数(クラス変数)が必要となる。
%[\ref{ES2016}、174ページ]にはクラスメソッドを用いてクラス変数
%を実現する方法の説明がある。
\begin{Exec}\upshape
クラスメソッドを用いて連続した\texttt{id}を作成する方法は次のようになる。
\begin{Verbatim}[numbers=left]
class Student extends Person {
  static getNextId(){
    return Student.nextId++;
  }
  constructor(name, year, month, day, hometown) {
    super(name, year, month, day, hometown);
    this.id = Student.getNextId();
  }
}
Student.nextId = 10000;
\end{Verbatim}
\begin{itemize}
 \item 2行目から4行目で次の\texttt{id}を求めるクラスメソッドを定義してい
       る。
 \item ここではクラス変数\texttt{nextId}を$1$増加させている(3行目)
 \item \texttt{nextId}の初期化は10行目で行っている。
 \item 初期化の方法からも推察されるように、この値は途中で変更が可能であ
       る。
\end{itemize}
実行例は次のとおりである。
\begin{Verbatim}
>s1 = new Student("foo1",2001,4,1);
Student {name: "foo", birthday: {…}, hometown: "神奈川", id: 10000}
>s2 = new Student("foo2",2001,5,1);
Student {name: "foo", birthday: {…}, hometown: "神奈川", id: 10001}
Student.nextId = -100;
-100
>s3 = new Student("foo2",2001,6,1);
Student {name: "foo", birthday: {…}, hometown: "神奈川", id: -100}
\end{Verbatim}
3つ目のインスタンスを作成する前にこの値を上書きしているので、最後のイン
スタンスは前の2つと異なるものに設定できていることが分かる。

\end{Exec}

この欠点を解消するには
クラスメソッドで定義する代わりにクラスをクロージャの中で定義して、その中
に変数を置く方法がある。

\begin{Exec}\upshape
次の例は即時実行関数内に\texttt{id}を管理する変数を用意して、その関数が
 クラス式を返すようにしたものである。
\begin{Verbatim}[numbers=left]
const Student = (function(){
  let id = 10000;
  return class extends Person {
    constructor(name, year, month, day, hometown) {
      super(name, year, month, day, hometown);
      this.id = id++;
    }
  }
})();
\end{Verbatim}
 \begin{itemize}
  \item 即時実行関数の戻り値はクラス式であり、それを変数\texttt{Student}
        に代入している。同じ名前のクラスが再定義されないようにするため変
        数を\ElmJ{const}で宣言している。
  \item 2行目で次に設定する\texttt{id}を保存する変数を初期化している。
  \item 関数の戻り値としてクラス式を返す。クラスの定義は以前のものとほと
        んど同じである。
  \item 違いはインスタンスに設定した後で\texttt{id}を管理する変数を
        増加させているところ(6行目の右辺)。
 \end{itemize}
 実行例は次のとおりである。
\begin{Verbatim}
>s = new Student("foo",2001,4,1);  //初めのインスタンスの id は 10000
{name: "foo", birthday: {…}, hometown: "神奈川", id: 10000}
>`${s}`;
"fooさんは2001年4月1日に}神奈川で生まれました。"
>s2 = new Student("foo",2001,4,1,"厚木");  //次のインスタンスの id は 10001
{name: "foo", birthday: {…}, hometown: "厚木", id: 10001}
>`${s2}`;
"fooさんは2001年4月1日に}厚木で生まれました。"
\end{Verbatim}
\end{Exec}
\iffalse
\section{コンストラクタ関数}
オブジェクトを定義する方法としてはコンストラクタ関数を使う方法がある。
\begin{Exec}\label{constructor}\upshape
次の例はコンストラクタ関数を用いて、前の例と同じオブジェクト(インスタン
 ス)を構成している。
\begin{Verbatim}
function Person(){
  this.name = "foo";
  this.birthday = {
    year : 2001,
    month : 4,
    day : 1
  };
  this["hometown"] = "神奈川";
}
\end{Verbatim}
\begin{itemize}
 \item 通常、コンストラクタ関数は大文字で始まる名前を付ける。
 \item そのオブジェクト内にメンバーを定義するために、\verb+this+をつけて
       定義する。ここでは、前の例と同じメンバー名で同じ値を設定してい
       る。
 \item この関数には\verb+return+がないことに注意すること。
 \item この関数を用いてオブジェクトを作成するためには、\verb+new+ をつけ
       て関数を呼び出す。
>var person = new Person();
undefined
\end{Verbatim}
 \item 元来、戻り値がないので\verb+undefined+が表示されているが、オブジェ
       クトは作成されている。
 \item 前と同じ文を実行すれば同じ結果が得られる。
\end{itemize}
\end{Exec}
ここの例はコンストラクタ関数に引数がないが、引数を持つコンストラクタ関数
 も定義が可能である。これにより同じメンバーを持つオブジェクトをいくつか
 作る必要がある場合にプログラムが簡単になる。

\begin{Exec}\upshape
次の例はコンストラクタ関数を \verb+new+ を用いないで実行した場合である。
\begin{Verbatim}
>p = Person();
undefined
>p.name
VM88:2 Uncaught TypeError: Cannot read property 'name' of undefined(…)
>p;
undefined
>name;
"foo"
>window.name;
"foo"
>name === window.name
true
\end{Verbatim}
\begin{itemize}\upshape
 \item この関数は戻り値がないので、\verb+undefined+ が変数 \verb+p+ に代
       入される。
 \item したがって、\texttt{p.name} も当然定義されていない。
 \item このとき、キーワード \verb+this+ が指すのはグローバルオブジェクト
       となる。
 \item 現在の実行環境はブラウザ上なので、このときのグローバルオブジェク
       トは \verb+window+ である。
 \begin{itemize}
  \item このとき、グローバル変数はすべてグローバルオブジェクトのメンバー
	としてアクセス可能である。この例では \verb+this.name+ に値を代入
	した時点で変数 \verb+name+ が定義されている。
  \item 最後の例からも、\verb+name+ と \verb+window.name+ が同じものであ
	ることがわかる。
 \end{itemize}
\end{itemize}
\end{Exec}
\iffalse
\begin{Prob}
実行例\ref{constructor}を次のように変える。
\begin{Verbatim}
function Person(D){
  this.name = "foo";
  this.birthday = {
    year : 2001,
    month : 4,
    day : 1
  };
  this["hometown"] = "神奈川";
  return D;
}
\end{Verbatim}
これに対して次のように実行したとき、作成されるオブジェクトは
 何か答えよ。
\begin{enumerate}
 \item \verb+p = new Person(1);+
%Person {name: "foo", birthday: Object, hometown: "神奈川"}
 \item \verb+p = new Person([1,2,3]);+
%[1, 2, 3]
 \item \verb+p = new Person({o:"1"});+
%Object {o: "1"}
%p.o;
%"1" 
 \item \verb+p = new Person(function(){return 2;});+
%function (){return 2;}
%p;
%function (){return 2;}
%p();
%2
% \item \verb+p = new Person(new function(){return 2;});+
%Object {}
%p;
%Object {}
 \item \verb+p = new Person(new function(){this.a = "a"});+
%Object {a: "a"}
%p.a
%"a"
\end{enumerate}
\end{Prob}
\fi
\subsection{\protect\texttt{constructor}プロパティ}\label{constructorProp}
オブジェクトが作成されると、\verb+constructor+ プロパティとよばれる特殊
なプロパティも設定される。
\begin{itemize}
 \item このプロパティはオブジェクトを作成したときに使
われたコンストラクタ関数を返す。実行例\ref{constructor}で確認すると次の
       ようになる。
\begin{Verbatim}
>var p = new Person();
undefined
>p.name;
"foo"
>p.constructor;
function Person(){
  this.name = "foo";
  this.birthday = {
    year : 2001,
    month : 4,
    day : 1
  };
  this["hometown"] = "神奈川";
}
\end{Verbatim}
Opera では定義全体が表示される。
 \item このプロパティに含まれるものは関数なので、コンストラクタの名前を
       知らなくても、元と同じオブジェクトのコピーが作成できる。
\begin{Verbatim}
>np = new p.constructor();
Person {name: "foo", birthday: Object, hometown: "神奈川"}
>np.constructor;
function Person(){
  this.name = "foo";
  this.birthday = {
    year : 2001,
    month : 4,
    day : 1
  };
  this["hometown"] = "神奈川";
}
\end{Verbatim}
 \item オブジェクトリテラルを使ってオブジェクトを作ると、組み込み関数の
       \verb+Object()+ コンストラクタ関数がセットされる。
\begin{Verbatim}
>o = {}
Object {}
>o.constructor;
function Object() { [native code] }
\end{Verbatim}
 \item なお、実行例\ref{Execconstructor}でみたように、このプロパティは
       \verb+for-in+ループ内では表示されない。
\end{itemize}
\fi
\iffalse
\section{\protect\texttt{Function}オブジェクト}
\paragraph{\protect\texttt{Function}オブジェクトのプロパティ}
関数は \texttt{function} を用いるほかに \texttt{Function()} コンストラク
タ関数を用いても作成できます。ほとんどメリットがないので、ここでは省略す
る。代表的なプロパティとして、コンストラクタ関数を表す
\texttt{constructor} と定義された関数の仮引数の個数を示す
\texttt{length} がある。
\begin{Verbatim}
>Person.constructor;
function Function() { [native code] }
>Person.length
0
\end{Verbatim}
このほかに重要な\texttt{prototype} プロパティがある。これについては次回
の授業で解説をする。
%\paragraph{\protect\texttt{Function}オブジェクトのメソッド}
\fi
\iffalse
\subsection{同じコンストラクタ関数からの生成されたオブジェクトに連番号を付ける}\label{SetID}
オブジェクト指向言語では同じコンストラクタ関数(クラス)から生成されたオブ
ジェクトに重複のない番号を付けることがある。\JS でそのようなことを実現す
るためには工夫が必要である。

次のように単純にメンバーを追加しても、コンストラクタ関数が呼ばれるごとに、
変数\texttt{ID}が初期化されてしまい、目的を果たすことができない。
\begin{Verbatim}
function Person(){
  var ID = 0;
  this.ID = ID++;
  this.name = "foo";
}
\end{Verbatim}
これから2つオブジェクトを作成しても、\texttt{ID} がともに \texttt{0} となって
いることがわかる。
\begin{Verbatim}
>p1 = new Person();
Person {ID: 0, name: "foo"}
>p2 = new Person();
Person {ID: 0, name: "foo"}
\end{Verbatim}

クロージャを用いて関数ににすると一見、うまくいくように見える。
\begin{Verbatim}
var Person2 = (function (){
  var ID = 0;
  return function(){
    this.ID = ID++;
    this.name = "foo";
  }
})();
\end{Verbatim}
これを実行すると次のようになる。
\begin{Verbatim}
>p1 = new Person2();
Object {ID: 0, name: "foo"}
>p2 = new Person2();
Object {ID: 1, name: "foo"}
>p1.ID;
0
>p2.ID;
1
\end{Verbatim}
しかし、このプロパティは外部から変更が可能となってしまっている。
\begin{Verbatim}
>p1.ID=10;
10
>p1.ID
10
\end{Verbatim}
これを避けるためにはクロージャ内の変数を参照するようなメソッドに変更すれ
ばよい。
\begin{Verbatim}
var Person3 = (function (){
  var ID = 0;
  return function(){
    this.getID = function() {
      return ID++;
    }
    this.name = "foo";
  }
})();
\end{Verbatim}
このコードでは\texttt{getID()}メソッドを実行するごとに、クロージャ内の変
数\texttt{ID}が増加してしまい、失敗である。
\begin{Verbatim}
>p1 = new Person3();
Object {name: "foo"}
>p1.getID();
0
>p2 = new Person3();
Object {name: "foo"}
>p2.getID();
1
>p1.getID();
2
>p1.getID();
3
\end{Verbatim}
呼ばれた時点での\texttt{ID}の値を保存するためには、いったん、スコープチェ
インを切る必要がある。
\begin{Verbatim}
var Person4 = (function (){
  var ID = 0;
  return function(){
    this.getID = (function(x) {
      return function(){ return x;}
    })(ID++);
    this.name = "foo";
  }
})();
\end{Verbatim}
これで目的は達成されている。
\begin{Verbatim}
>p1 = new Person4();
Object {name: "foo"}
>p1.getID();
0
>p2 = new Person4();
Object {name: "foo"}
>p2.getID();
1
>p1.getID();
0
\end{Verbatim}
しかし、これでも\texttt{getID()}メソッドは書き直すことが可能である。
\begin{Prob}\upshape
 \texttt{p1.getID()}が設定しなおせることを確認しなさい。
\end{Prob}
\input ObjectProperty.tex
\begin{Prob}\upshape
 \ref{SetID}項の\texttt{Person2}の\texttt{ID}プロパティにプロパティ属性
 を付けて、外部から変更できないようにしなさい。(ヒント:\texttt{ID}の値を
 参照するために別の変数が必要になるかもしれない。)
\end{Prob}
\fi
