\ProblemNN{
 \begin{Prob}\upshape
  次のプログラムを実行したときのコンソールの出力を記せ。また、その理由も
  述べよ。
  \input prob03-01.tex
 \end{Prob}
\underline{\makebox[10zw][l]{(1)}}\hfill
\underline{\makebox[10zw][l]{(2)}}\hfill
\underline{\makebox[10zw][l]{(3)}}\hfill
\underline{\makebox[10zw][l]{(4)}}\\[\baselineskip]
\underline{\makebox[\textwidth][l]{\bfseries 理由:}}
%\input 03-01OrderOfDefFunction.tex
\input 03-02FunctionWithArbituraryArgs.tex
\input prob03-03declairVAriablesByVar.tex
\Newpage
\input 03-04setTimeout.tex
\begin{Prob}\upshape
 実行例3.5から3.8における\texttt{f1()}、\texttt{foo()}、\texttt{f2()}と
 \texttt{f3()}の動作を確認しなさい。コンソール画面のキャプチャを貼り付け
 ること。
\end{Prob}
}
\Rubric{復習の目的は次のとおりである。
\begin{itemize}
 \item 関数の定義方法、仮引数の取り扱いを理解する。
 \item 変数のスコープとクロージャの使い方を理解する。
 \item 変数の宣言における\ElmJ{var}を使わない理由を正しく説明できる。
 \item コールバック関数を理解する。
 \item 即時実行関数の利用法を理解する。
\end{itemize}
}
{{問題1}{5}{
  {コンソールへの出力結果がすべて正しい。}
	{結果に対する理由が正しく述べられている。}
	}
	{
  {コンソールへの出力結果(1)が正しい。}
  {コンソールへの出力結果(2)が正しい。}
  {コンソールへの出力結果(3)が正しい。}
  {コンソールへの出力結果(4)が正しい。}
	{関数の定義位置についての理由が正しい。}
	{不足する関数の仮引数の説明が正しい。}
	{関数の戻り値の計算についての説明が正しい。}
	}
	{
	{関数の定義位置についての理由がほとんど間違っているか正しくない。}
	{不足する関数の仮引数の説明がほとんど間違っているか正しくない。}
	{関数の戻り値の計算についての説明がほとんど間違っているか正しくない。}
	}
{問題2}{5}{
  {十分な場合について\texttt{sumN()}を実行していて、結果に対する考察も正し
	い。}
	{配列を使って\texttt{sumN()}を正しく実行させる方法を説明している}
	}
	{
  {\texttt{sumN()}の実行に関して引数の数を変えて実行している。}
  {\texttt{sumN()}の実行結果に対する考察が正しい。}
	{配列を使って\texttt{sumN()}を正しく実行させる方法をほとんど正しく説明している}
	}
	{
	{}
	}
{問題3}{6}{
  {}
	}
	{
	{}
	}
	{
	{}
	}
{問題4}{6}{
  {}
	}
	{
	{}
	}
	{
	{}
	}
{問題5}{6}{
  {}
	}
	{
	{}
	}
	{
	{}
	}
}