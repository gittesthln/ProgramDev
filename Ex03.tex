\ProblemNN{
 \begin{Prob}\upshape\Must
  次のプログラムを実行したときのコンソールの出力を記せ。また、その理由も
  述べよ。
  \input prob03-01.tex
 \end{Prob}
\underline{\makebox[10zw][l]{(1)}}\hfill
\underline{\makebox[10zw][l]{(2)}}\hfill
\underline{\makebox[10zw][l]{(3)}}\hfill
\underline{\makebox[10zw][l]{(4)}}\\[\baselineskip]
\underline{\makebox[\textwidth][l]{\bfseries 理由:}}
%\input 03-01OrderOfDefFunction.tex
\input 03-02FunctionWithArbituraryArgs.tex
\input prob03-03declairVAriablesByVar.tex
\Newpage
\input 03-04setTimeout.tex
\newcommand{\AnsList}[1]{%
\item 実行例#1の実行結果の確認\\[0.04\textheight]
\item 実行例#1の実行結果の考察\\[0.04\textheight]
}
\begin{Prob}\upshape\Must
 実行例3.5から3.8における\texttt{f1()}、\texttt{foo()}、\texttt{f2()}と
 \texttt{f3()}の動作を確認しなさい。コンソール画面のキャプチャを貼り付け
 るか別紙で添付のこと。
 \begin{itemize}
	\AnsList{3.5}
	\AnsList{3.6}
	\AnsList{3.7}
	\AnsList{3.8}
 \end{itemize}
\end{Prob}
}
\Rubric{復習の目的は次のとおりである。
\begin{itemize}
 \item 関数の定義方法、仮引数の取り扱いを理解する。
 \item 変数のスコープとクロージャの使い方を理解する。
 \item 変数の宣言における\ElmJ{var}を使わない理由を正しく説明できる。
 \item コールバック関数の概念や即時実行関数の利用法を理解する。
\end{itemize}
}
{{問題1}{5}{
  {コンソールへの出力結果がすべて正しい。}
	{結果に対する理由が正しく述べられている。}
	}
	{
  {コンソールへの出力結果(1)が正しい。}
  {コンソールへの出力結果(2)が正しい。}
  {コンソールへの出力結果(3)が正しい。}
  {コンソールへの出力結果(4)が正しい。}
	{関数の定義位置についての理由が正しい。}
	{不足する関数の仮引数の説明が正しい。}
	{関数の戻り値の計算についての説明が正しい。}
	}
	{
	{関数の定義位置についての理由がほとんど間違っているか正しくない。}
	{不足する関数の仮引数の説明がほとんど間違っているか正しくない。}
	{関数の戻り値の計算についての説明がほとんど間違っているか正しくない。}
	}
{問題2}{5}{
  {十分な場合について\texttt{sumN()}を実行していて、結果に対する考察も正し
	い。}
	{配列を使って\texttt{sumN()}を正しく実行させる方法を説明している}
	}
	{
  {\texttt{sumN()}の実行に関して引数の数を変えて実行している場合が十分あ
	る。}
  {\texttt{sumN()}の実行結果に対する考察が正しい。}
	{配列を使って\texttt{sumN()}を正しく実行させる方法をほとんど正しく説明している}
	}
	{
  {\texttt{sumN()}の実行に関して引数の数を変えて実行していないか、その数が
	足りない。}
  {\texttt{sumN()}の実行結果に対する考察がないか、ほとんど間違っている。}
	{配列を使って\texttt{sumN()}を正しく実行させる方法がないか、説明が間違っ
  ている。}
	}
{問題3}{5}{
  {変数の宣言をすべて\ElmJ{var}に変更して実行している。}
  {実行された結果がすべて正しい。}
  {実行結果に対して十分な考察がある。}
  {\ElmJ{var}と\ElmJ{let}による変数の宣言の違いに関して考察がある。}
	}
	{
	{\texttt{func1()}の実行結果が正しい。}
	{\texttt{func1()}の実行結果の考察が正しい。}
	{\texttt{func2()}の実行結果が正しい。}
	{\texttt{func2()}の実行結果の考察が正しい。}
	{\texttt{func3()}の実行結果が正しい。}
	{\texttt{func3()}の実行結果の考察が正しい。}
	{\texttt{func4()}の実行結果が正しい。}
	{\texttt{func4()}の実行結果の考察が正しい。}
	{\texttt{func5()}の実行結果が正しい。}
	{\texttt{func5()}の実行結果の考察が正しい。}
  {\ElmJ{var}と\ElmJ{let}による変数の宣言の違いに関して考察が少し足りな
  い。}
	}
	{
	{\texttt{func1()}の実行結果が間違っている%か、
  %\texttt{func1()}の変数の宣言を変更していない
  。}
	{\texttt{func1()}の実行結果の考察がないか間違っている。}
	{\texttt{func2()}の実行結果が間違っている。}
	{\texttt{func2()}の実行結果の考察がないか間違っている。}
	{\texttt{func3()}の実行結果が間違っている。}
	{\texttt{func3()}の実行結果の考察がないか間違っている。}
	{\texttt{func4()}の実行結果が間違っている。}
	{\texttt{func4()}の実行結果の考察がないか間違っている。}
	{\texttt{func5()}の実行結果が間違っている。}
	{\texttt{func5()}の実行結果の考察がないか間違っている。}
	}
{問題4}{5}{
  {コンソールの出力がキャプチャで正しく表示されている。}
	{独自のコードの十分な解説がある。}
	{動作の説明が的確である。}
	}
	{
  {コンソールの出力結果が正しく答えている。}
	{独自のコードの解説がある。一部不足や間違いが見受けられる。}
	{動作の説明に一部不足や間違いが見受けられる。}
	}
	{
  {コンソールの出力結果が間違っている。}
	{独自のコードの解説がないか、不足や間違いが多く見受けられる。}
	{動作の説明がないか、一部不足や間違いが多く見受けられる。}
	}
{問題5}{10}{
  {実行例3.5の実行結果をテキストよりも場合を増やして確認している。}
  {実行例3.5のリストを改良している。}
  {実行例3.5に関する考察が的確である。}
  {実行例3.6の実行結果をテキストよりも場合を増やして確認している。}
  {実行例3.6のリストを改良している。}
  {実行例3.6に関する考察が的確である。}
  {実行例3.7の実行結果をテキストよりも場合を増やして確認している。}
  {実行例3.7のリストを改良している。}
  {実行例3.7に関する考察が的確である。}
  {実行例3.8の実行結果をテキストよりも場合を増やして確認している。}
  {実行例3.8のリストを改良している。}
  {実行例3.8に関する考察が的確である。}
  {実行例3.7と3.8の結果などを比較検討している。}
	}
	{
  {実行例3.5の実行結果をテキストの場合で確認している。}
  {実行例3.5のリストをそのまま実行している。}
  {実行例3.5に関する考察がほぼ正しい。}
  {実行例3.6の実行結果をテキストの場合で確認している。}
  {実行例3.6のリストをそのまま実行している。}
  {実行例3.6に関する考察がほぼ正しい。}
  {実行例3.7の実行結果をテキストの場合で確認している。}
  {実行例3.7のリストをそのまま実行している。}
  {実行例3.7に関する考察がほぼ正しい。}
  {実行例3.8の実行結果をテキストの場合で確認している。}
  {実行例3.8のリストをそのまま実行している。}
  {実行例3.8に関する考察がほぼ正しい。}
  {実行例3.7と3.8の結果などの比較がない。}
	}
	{
  {実行例3.5の実行結果をテキストの場合より少ない場合で確認している。}
  {実行例3.5のリストのまま実行していない。}
  {実行例3.5に関する考察がないか間違っている。}
  {実行例3.6の実行結果をテキストの場合より少ない場合で確認している。}
  {実行例3.6のリストのまま実行していない。}
  {実行例3.6に関する考察がないか間違っている。}
  {実行例3.7の実行結果をテキストの場合より少ない場合で確認している。}
  {実行例3.7のリストのまま実行していない。}
  {実行例3.7に関する考察がないか間違っている。}
  {実行例3.8の実行結果をテキストの場合より少ない場合で確認している。}
  {実行例3.8のリストのまま実行していない。}
  {実行例3.8に関する考察がないか間違っている。}
	}
}