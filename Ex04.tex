\ProblemNN{
\input 04-01windowProperty.tex
\input 04-02JSONProb.tex
\ \vspace{0.15\textheight}
\input 04-03ChangeProperties.tex
\input 04-04ageAtToday.tex
\Newpage
\input 04-05ageRemoveGet.tex
\input 04-06ageSetter.tex
}
\Rubric{復習の目的は次のとおりである。
\begin{itemize}
 \item \texttt{window}オブジェクトの内容を理解する。
 \item \texttt{JSON}の構造と取り扱いを理解する。
 \item \texttt{class}によるオブジェクトの作成方法とプロパティの取り扱い
       を他のオブジェクト指向言語との比較で理解する。
 \item メソッドとプロパティの宣言と使用法の違いを理解する。
\end{itemize}
}
{{問題1}{8}{
 {\ElmJ{window}オブジェクトのプロパティの十分なリストがある。}
 {\ElmJ{window}オブジェクトのプロパティに関して十分な考察がある。}
 {\ElmJ{window}オブジェクトをページを何も表示しないページで行っている。}
 {\ElmJ{window}オブジェクトを調べるためにページを開いた直後にコンソール
 で行っているか、JavaSçriptのプログラムで行っている。}
 }
 {
 {\ElmJ{window}オブジェクトのプロパティのリストがそれなりにある。}
 {\ElmJ{window}オブジェクトのプロパティに関して考察が少しある。}
 {\ElmJ{window}オブジェクトを調べるページが既存のものになっている。}
 {\ElmJ{window}オブジェクトを調べるためにページを開いた直後に行っていない。}
 }
 {
 {\ElmJ{window}オブジェクトのプロパティのリストの数が少なすぎる。}
 {\ElmJ{window}オブジェクトのプロパティに関して考察がない。}
 {\ElmJ{window}オブジェクトを調べるページが既存のものになっている。}
 {\ElmJ{window}オブジェクトを調べるためにページを開いた直後に行っていない。}
 }
 {問題2}{3}{
 {実行結果が正しい。}
 {設問以外のメンバーについても実行している。}
 {十分な考察がある。}
 }
 {
 {実行結果に一部誤りがある。}
 {考察が少し足りない。}
 }
 {
 {実行結果がほとんど間違っている。}
 {考察がない。}
 }
 {問題3}{5}{
 {クラスで作成されたインスタンスについて十分な実行結果がある。}
 {十分な考察がある。}
 }
 {
 {クラスで作成されたインスタンスのプロパティの値の変更の実行結果がある。}
 {クラスで作成されたインスタンスのプロパティの追加の実行結果がある。}
 {クラスで作成されたインスタンスのプロパティ列挙のの実行結果がある。}
 {考察が少し足りない。}
 }
 {
 {クラスで作成されたインスタンスのプロパティの値の変更の実行結果がないか
 不十分である。}
 {クラスで作成されたインスタンスのプロパティの追加の実行結果がないか
 不十分である。}
 {クラスで作成されたインスタンスのプロパティ列挙のの実行結果がないか
 不十分である。}
 {考察が少し足りないか全くない。}
 }
 {問題4}{8}{
 {問題で指定された機能を持つメソッドが作成されている。}
 {リストがあり、解説が十分にある。}
 {十分な実行結果とそれに関する考察がある。}
 }
 {
 {リストがあるが、解説が十分ではない。}
 {条件を満たす引数の数が異なる複数のメソッドを作成している。}
 {メソッドの定義が正しい。}
 {引数がない場合には\texttt{age}と同じに動作する。}
 {年しかない場合にはその年の1月1日現在の年令が正しく求まる。}
 {年と月しかない場合にはその年月の1日現在の年令が正しく求まる。}
 {引数にデフォルトの値を与えている。}
 {動作の確認が少し足りない。}
 }
 {
 {リストがない。}
 {リストの解説がない。}
 {メソッドの定義に間違いがある。}
 {引数がない場合には\texttt{age}と同じに動作しない。}
 {年しかない場合にはその年の1月1日現在の年令が正しく求まっていない。}
 {年と月しかない場合にはその年月の1日現在の年令が正しく求まっていない。}
 {引数にデフォルトの値を与えていない。}
 {動作の確認の場合が足りない。}
 }
 {問5}{3}{
 {\ElmJ{get}を省略して通常のメソッドとしたときの実行方法が正しい。}
 {考察が十分にあり、正しい。}
 }
 {
 {\ElmJ{get}を省略して通常のメソッドとしたときの実行方法に勘違いがある。}
 {考察があり、正しい。}
 }
 {
 {\ElmJ{get}を省略して通常のメソッドとしたときの実行方法が間違っている。}
 {考察がほとんどないか間違っている。}
 }
 {問題6}{8}{
 {問題で指定された機能を持つセッターが作成されている。}
 {リストがあり、解説が十分にある。}
 {十分な実行結果とそれに関する考察がある。}
 }
 {
 {セッターの定義が正しい。}
 {リストがあるが、解説が十分ではない。}
 {動作の確認が少し足りない。}
 }
 {
 {リストがない。}
 {リストの解説がない。}
 {セッターの定義で値を変更している。}
 {エラーメッセージが出力されていない。}
 {動作の確認の場合が足りない。}
 }
 }

 {\bfseries レポートに関してもう一度解説してほしいところがあれば下に書いてく
 ださい。}