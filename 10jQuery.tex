\chapter{jQuery}
\section{jQueryとは}
jQuery の開発元\footnote{\texttt{https://jQuery.com}参照日:2017/11/26}に
は次のように書かれている。
\begin{quotation}
jQuery is a fast, small, and feature-rich JavaScript library. It makes
 things like HTML document traversal and manipulation, event handling,
 animation, and Ajax much simpler with an easy-to-use API that works
 across a multitude of browsers. With a combination of versatility and
 extensibility, jQuery has changed the way that millions of people write
 JavaScript.
 \end{quotation}
jQuery はJavaScriptのライブラリーであり、これまでに解説してきた
JavaScriptの使い方を簡単にすることを目的としている。2018年12月12日現在、
jQuery は ver. 3.3.1 が最新のものとなっている。
%ver. 3.0 は2016年6月9日に%
%リリースされた。それまで提供されていた ver. 1(最終は 1.12.4) と
%ver. 2(最終は2.2.4)は重大な訂正(critical support patch)しか行わないと記さ
%れている。

\section{jQueryの利用方法}
jQuery は\href{jquery.com}{jquery.com}のページの右上にある「Download
jQuery」をクリックすることでダウンロードのページに行くと次の2つの最新版
のダウンロードが可能となる。

Download the compressed, production jQuery 3.3.1

Download the uncompressed, development jQuery 3.3.1

これらのファイルの違いはコメントなどがそのまま入っているもの(uncompressed)と、
コメントを取り除き、変数名などを短いものに置き換えて、ファイルサイズを小
さくしたもの(compressed)の2種類がある。通常は compressed バージョンを用
いる。ダウンロードしたファイルを取り込む\Elm{script}要素を記述すれば利用できる。

jQuery のファイルをローカルにダウンロードしないでCDN(Contents Deliverly
Network)を利用する方法がある。CDNとはライブラリーをネットワークからダウ
ンロードできるようにしてい
るサイトのことである。jQuery では\href{code.jquery.com}{code.jquery.com}
を開いて必要とされるバージョンのところをクリックすることで CDN を安全に利用する
\Elm{script}要素の記述がクリップボードにコピーされる\footnote{コピーがうまくいかな
かったら手作業で行うこと。}。jQuery の compressed の最新
版は minified をクリックすることで得られる。
次のようなテキストがコピーされる。
\begin{Verbatim}
<script
  src="https://code.jquery.com/jquery-3.3.1.min.js"
  integrity="sha256-FgpCb/KJQlLNfOu91ta32o/NMZxltwRo8QtmkMRdAu8="
  crossorigin="anonymous"></script>
\end{Verbatim}
\begin{Prob}
 \upshape
 CDNを利用する利点について述べなさい。また、CDN を利用して配布されている
 ライブラリーにどのようなものがあるか調べなさい。
\end{Prob}
\newcounter{Func}
\newcommand{\FuncRef}[1]{%
\refstepcounter{Func}\label{#1}({\bfseries 機能\arabic{Func}})}
%\section{jQueryの基本}
\section{jQuery()関数とjQueryオブジェクト}
jQueryライブラリーは\texttt{jQuery()}というグローバル関数を一つだけ定義
する。この関数の短縮形として、\texttt{\$}というグローバル関数も定義する。
jQuery() 関数はコンストラクタではない。
jQuery関数は引数の与え方により動作が異なるが、戻り値としてjQuery
オブジェクトとと呼ばれるDOMの要素を拡張した集まりを返す。このオブジェクトには多数のメ
ソッドが定義されていて、これらの要素群を操作できる。

jQuery() 関数の呼び出し方は引数の型により返される要素群が異なる。
\begin{itemize}
 \item 引数が(拡張された)CSSセレクタ形式の場合\FuncRef{CSSSelector}

 CSSセレクタにマッチした要素群を返す。省略可能な2番目の引数として要素やjQueryオブ
       ジェクトを指定した場合には、その要素の子要素からマッチしたものを
       返す。この形はすでに解説したDOMのメソッド
       \texttt{querySelectorAll()}と似ている。
 \item 引数が要素やDocumentオブジェクトなどの場合\FuncRef{ObjSelector}

       与えられた要素をjQueryオブジェクトに変換する。
 \item 引数としてHTMLテキストを渡す場合\FuncRef{HTMLText}

       テキストで表される要素を作成し、この要素を表すjQueryオブジェクト
       を返す。\texttt{createElement()}メソッドに相当する。

       省略可能な2番目の引数は属性を定義するものであり、オブジェクトリテ
       ラルの形式で与える。
 \item 引数として関数を渡す場合\FuncRef{Function}

       引数の関数はドキュメントがロードされ、DOMが操作で可能になった時に実行される。
\end{itemize}
\section{jQueryオブジェクトのメソッド}
jQueryオブジェクトに対する多くの処理はHTMLの属性やCSSスタイルの値を設定
したり、読み出したりすることである。
\begin{itemize}
 \item これらのメソッドに対してセッターと
ゲッターを同じメソッドを使う。メソッドに値を与えるとセッターになり、ない
とゲッターになる。
 \item セッターとして使った場合は戻り値がjQueryオブジェクトとなるので、
       メソッドチェインが使える。
 \item ゲッターとして使った場合は要素群の最初の要素だけ問い合わせる。
\end{itemize}
\paragraph{HTML属性の取得と設定}
\texttt{attr()}メソッドはHTML属性用のjQueryのゲッター/セッターである。
\begin{itemize}
 \item 属性名だけを引数に与えるとゲッターとなる。
 \item 属性名と値の2つを与えるとセッターになる。
 \item 引数にオブジェクトリテラルを与えると複数の属性を一時に設定できる。
 \item 属性を取り除く\texttt{removeAttr()}もある。
\end{itemize}
\paragraph{CSS属性の取得と設定}
\texttt{css()}メソッドはCSSのスタイルを設定する。
\begin{itemize}
 \item CSSスタイル名は元来のCSSスタイル名(ハイフン付)でもJavaScriptの
       キャメル形式でも問い合わせ、設定が可能である。
 \item 戻り値は単位を含めて文字列で返される。
\end{itemize}
\paragraph{HTMLフォーム要素の値の取得と設定}
\texttt{val()}はHTMLフォーム要素の\texttt{value}属性の値の設定や取得がで
きる。これにより、\texttt{<select>}要素の選択された値を得ることなどがで
きる。
\paragraph{要素のコンテンツの取得と設定}
\texttt{text()}と\texttt{html()}メソッドはそれぞれ要素のコンテンツを通常
のテキストまたはHTML形式で返す。引数がある場合には、既存のコンテンツを置
き換える。
\section{ドキュメントの構造の変更}
表\ref{Insertreplace}は挿入や置換を行う基本的なメソッドをまとめたもので
ある。これらのメソッドには対になるメソッドがある。
\begin{table}[ht]
 \caption{ドキュメントの構造の変更するメソッド}\label{Insertreplace}
\begin{tabular}{|c|c|l|}
 \hline
 \texttt{\$(T).method(C)}&\texttt{\$(C).method(T)}&
    \multicolumn{1}{c|}{機能}\\\hline
 \texttt{append}&\texttt{appendTo}&
	 要素\texttt{T}の最後の子要素として\texttt{C}を付け加える\\ \hline
 \texttt{prepend}&\texttt{prependTo}&
	 要素\texttt{T}の初めの子要素として\texttt{C}を付け加える\\ \hline
 \texttt{before}&\texttt{insertBefore}&
	 要素\texttt{T}の直前の要素として\texttt{C}を付け加える\\ \hline
 \texttt{after}&\texttt{insertAfter}&
	 要素\texttt{T}の直後の要素として\texttt{C}を付け加える\\ \hline
 \texttt{replaceWith}&\texttt{replaceAll}&
	 要素\texttt{T}と\texttt{C}を置き換える\\ \hline
\end{tabular}\end{table}

このほかに、要素をコピーする\texttt{clone()}、要素の子要素をすべて消す
\texttt{empty()}と選択された要素(とその子要素すべて)を削除する\texttt{remove()}もある。

\section{イベントハンドラーの取り扱い}
jQueryのイベントハンドラーの登録はイベントの種類ごとにメソッドが定義され
ていてる。指定されたjQueryオブジェクトが複数の場合にはそれぞれに対してイ
ベントハンドラーが登録される。次のようなイベントハンドラ-登録メソッドが
ある。
\begin{center}
 \begin{tabular}{llllll}
  \texttt{blur()}& \texttt{error()}&\texttt{keypress()} &\texttt{mouseup()} &\texttt{mouseover()}&\texttt{select()} \\
  \texttt{change()}& \texttt{focus()}&\texttt{keyup()} &\texttt{mouseenter()} &\texttt{mouseup()}&\texttt{submit()} \\
  \texttt{click()}& \texttt{focusin()}&\texttt{load()} &\texttt{mouseleave()} &\texttt{resize()}&\texttt{unload()} \\
  \texttt{dbleclick()}& \texttt{keydown()}&\texttt{mousedown()} &\texttt{mousemove()} &\texttt{scroll()} \\
 \end{tabular}
\end{center}
このほかに、特殊なメソッドとして\texttt{hover()}がある。これは
\texttt{mouseenter}イベントと\texttt{mouseleave}イベントに対するハンドラ
-を同時に登録できる。また、\texttt{toggle()}はクリックイベントに複数の
イベントハンドラ-を登録し、イベントが発生するごとに順番に呼び出す。

イベントハンドラ-の登録解除には\texttt{unbind()}メソッドがある。このメ
ソッドの呼び出しはいろいろな方法があるが、\texttt{removeEvenListener()}
と同様な形式として1番目の引数にイベントタイプ(文字列で与える)、2番目の引
数に登録した関数を与えるものがある。この場合に、登録したイベントハンドラ
-には名前が必要となる。

イベントに関してはこのほかにも便利な事項が多くある。
\section{Ajaxの処理}
jQueryではAjaxの処理に関するいろいろな方法を提供している。ここではもっと
も簡単な処理を提供する\texttt{jQuery.ajax()}関数を紹介する。

この関数は引数にオブジェクトリテラルをとる。このオブジェクトリテラルの属
性名として代表的なものを表\ref{jQueryAjax}に挙げる。
\begin{table}[ht]
 \caption{jQuery.ajax()関数で利用できる属性(一部))}\label{jQueryAjax}
\begin{center}
 \begin{tabular}{|c|m{28zw}|}\hline
属性名  &\multicolumn{1}{c|}{説明} \\\hline
  \texttt{type}&通信の種類。通常は\texttt{"POST"}または\texttt{"GET"}を指定\\
  \hline
  \texttt{url}&サーバーのアドレス\\ \hline
  \texttt{data}&\texttt{"GET"}のときはURLの後に続けるデータ。
%      \texttt{"POST"}のときは\texttt{null}
      \\ \hline
  \texttt{dataType}&戻り値のデータの型を指定する。\texttt{"text"}、
      \texttt{"html"}、\texttt{"script"}、      
\texttt{"json"}、\texttt{"xml"}などがある\\ \hline
\texttt{success}&通信が正常終了したときに呼び出されるコールバック関数
      \\ \hline
\texttt{error}&通信が成功しなかったときに呼び出されるコールバック関数
      \\ \hline
  \texttt{timeout}&タイムアウト時間をミリ秒単位で指定\\ \hline
 \end{tabular}
\end{center}
\end{table}
\section{jQuery のサンプル}
 \begin{Exec}\upshape\label{ExjQuery}
  次の例は実行例\ref{WhatDay}をjQueryを用いて書き直したもので
 ある。
 \LISTN{09-01whatday-jquery.html}{1}{last}{\normalsize}
 \begin{itemize}
  \item 7行目でローカルに保存した\texttt{jQuery}のライブラリーを読み込ん
        でいる。
  \item 10行目の\texttt{\$(window).redy()}はドキュメントが解釈されたとき
	に引数の関数が呼び出されるイベントである。ここでの要素の参照は機
	能\ref{ObjSelector}を用いている。
  \item 11行目から21行目は連続した番号を値に持つプルダウンメニューを作成
	する関数である。仕様は以前と同じである。
 \LISTN{09-01whatday-jquery.html}{11}{20}{\normalsize}
	\begin{itemize}
	 \item 12行目では\texttt{<select>}要素を作成し、同時に属性も設定
	       している(機能\ref{HTMLText})。
	 \item 14行目から18行目で\texttt{<option>}要素を定義し、
	       \texttt{<select>}要素の子要素にしている。なお、15行目と16
	       行目は\texttt{appendTo()}メソッドを用いると17行目のコメン
	       トアウトしてあるように記述することができる。
	\end{itemize}
  \item 21行目から34行目も依然とほとんど同じである。24行目では機能
	\ref{HTMLText}を用いて要素を参照している。
  \item 31行目、32行目と35行目はプルダウンメニューの初期値を本日に設定し
	ている。
  \item 36行目から53行目はプルダウンメニュが変化したときに呼び出される関
	数を定義している。
	\begin{itemize}
	 \item jQueryのメソッドを用いていることを除けば37行目から42行
	       目までは以前と同じである。39行目のセレクタは\texttt{id}が
	       \texttt{day}(\texttt{\#day})の下にある
	       \texttt{<option>}要素を選択し、その数を\texttt{length}で
	       調べている。
	 \item 43行目から52行目はAjaxの処理を定義している。
 \LISTN{09-01whatday-jquery.html}{43}{53}{\normalsize}
\begin{itemize}
 \item 44行目では通信の方法を\texttt{GET}に指定している。
 \item 45行目では通信の相手のファイルを指定している。呼び出されるファイ
       ルは106ページのものがそのまま使える。
 \item 46行目では送信のデータを定義している。通信方法が\texttt{GET}の場
       合でも jQuery ではこのように分けて書いてよい。
 \item 47行目ではサーバーから返されるデータの形式を指定している。
 \item 49行目から50行目ではサーバーとの通信が成功した場合の処理を指定し
       ている。引数がサーバーから送られたデータとなる。\texttt{json}を指定す
       るとこの値はオブジェクトになっている。
 \item 51行目では通信にエラーが発生したときの処理を指定している。
\end{itemize}
         %ブラウザによる違いに対する対処や、
         XMLHTTPRequestの処理部分が大幅に減っ
	       ていて見やすいコードになっていることがわかる。
	\end{itemize}
 \end{itemize}
 \end{Exec}
 \begin{Prob}\upshape
  実行例\ref{ExjQuery}に関して次の問いに答えよ。
\begin{enumerate}
   \item 送信メソッドを\texttt{POST}に変更して実行しなさい。
   \item メニューを作成終了後に変数\texttt{Form}をコンソールに出力させて
         その構造を調べなさい。変数\texttt{Form}の値と
         \texttt{document.getElementById("menu")}で得られるものが同じか
         どうかに注意すること。
   \item   実行例\ref{ExGoogleMaps}をjQueryを用いて書き直しなさい。
\end{enumerate}
 \end{Prob}
